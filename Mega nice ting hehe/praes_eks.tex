\documentclass{beamer}
\usetheme{Boadilla}
\title{Lineære programeringsproblemer}
\subtitle{Gemetri og simplex}
\author{B303c}
\institute{Aalborg Universitet}
\date{\today}

\begin{document}
\begin{frame}
\frametitle{Læseguide}
Hvert item på slides er umiddelbart tænkt som 1-2 underslides hvor relevante sætninger osv indsættes.
Det jeg har lavet er således oversigten over det indhold jeg umiddelbart tænker der bør være fokus på til eksamen.
Forestiller mig umiddelbart der er flest på der fremligger under løsning(hvor jeg har også smidt den geometrisk fremstilling ind da de er pænt relaterede)
\end{frame}

\section{Introduktion til lineær algebra}

\begin{frame}
\frametitle{Lineær Algebra}
\begin{itemize}
\item \textbf{$nr. 1$}
\item Matrixsum og skalering.
\item Transponering.
\item Matrix-vektorprodukt og matrixmultiplikation.
\item Linearkombinationer.
\item Span.
\item Lineær uafhængighed.
\item Underrum.
%Kort om Matrixsum og skalering, Transponering Matrix-vektorprodukt og matrixmultiplakation.
\end{itemize}
\end{frame}
%%

\begin{frame}
\frametitle{Lineære ligningssystemer}
\begin{itemize}
\item \textbf{$nr. 2$}
\item Lineære ligningssystemer
\item Rækkeoperationer (evt. elemtentærmaticer)
\item Trappeform
\item Gauss-elimination
\item Rang og Nullitet
%Kort om Rang og Nullitet
\end{itemize}
\end{frame}
%
\section{Løsninger og grafisk repræsentation}
%
\begin{frame}
\frametitle{Løsning af lineære programeringsproblemer}
\begin{itemize}
\item \textbf{$nr. 3$}
\item Intro til løsninger fra afsnit 2
%De fire punkter fra raporten (angående løsninger)
\item Standardform.
\item Eventuelt dualproblemer som sidebemærkning. Hvis vi skal have det med måske en bemærkning om spilteori og nulsum(slackvariable løsning for dual).
\item Polyedre og repræsentation(herunder standardform)
\item Det konvekse hylster.
\item Injektiv og surjektiv (bijektiv)
\item Lokal $\rightarrow$ Global (omkring fig 3.9)
\end{itemize}
\end{frame}
%
\begin{frame}
\frametitle{Hjørner}
\begin{itemize}
\item \textbf{$nr. 4$}
\item Ekstremumspunkter, hjørnepunkter og basale mulige løsninger
\item Sæt 3.4 ækvivalens mellem hjørnepunkterne
\item Bevis for løsninger i hjørnepunkter (3.8)
\item \textbf{$nr. 5$}
\item Bevis for løsninger i hjørnepunkter (3.9)
\item Outro $\rightarrow$ hjørnepunkter og simplex
\end{itemize}
\end{frame}
%
\begin{frame}
\frametitle{Simplex}
\begin{itemize}
\item Hvordan simplex afsøger hjørnepunkter.
\item \textbf{$nr. 6$}
\item Implementationer af simplex(tænker kort gennemgang af forskelle)
\item Gennemgang af fuld tabelimplementering (mere udførligt (eksempel)).
\item Lexicografi til valg af pivoteringsmetode.
\item Kompleksitet som udgangspunkt for valg af metode.
\item Afrunding / konklusion?!
\end{itemize}
\end{frame}


\begin{frame}
\frametitle{Proces(hvis vi gider)}
\begin{itemize}
\item Er der mere tid? Har Horia lyst til at høre på det? Nej
\item Projektarbejde i cornaens tidsalder.
\item En flot tegning
\end{itemize}
\end{frame}


\end{document}