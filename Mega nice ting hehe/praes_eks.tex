\documentclass{beamer}
\usetheme{Boadilla}
\title{Lineære programeringsproblemer}
\subtitle{Gemetri og simplex}
\author{c303c}
\institute{Aalborg Universitet}
\date{\today}

\begin{document}
\begin{frame}
\frametitle{Læseguide}
Hvert item på slides er umiddelbart tænkt som 1-2 underslides hvor relevante sætninger osv indsættes.
Det jeg har lavet er således oversigten over det indhold jeg umiddelbart tænker der bør være fokus på til eksamen.
Forestiller mig umiddelbart der er flest på der fremligger under løsning(hvor jeg har også smidt den geometrisk fremstilling ind da de er pænt relaterede)
\end{frame}

\section{Introduktion til lineær algebra}

\begin{frame}
\frametitle{Lineær Algebra}
\begin{itemize}
\item Matrixsum og skalering.
\item Transponering.
\item Linearkombinationer.
\item matrix-vektorprodukt og matrixmultiplakation.
\end{itemize}
\end{frame}

\section{Løsninger og grafisk repræsentation}

\begin{frame}
\frametitle{Løsning af lineære programeringsproblemer}
\begin{itemize}
\item Intro til løsninger fra afsnit 2
\item Polyedre og repræsentation(herunder standardform)
\item Bevis for løsninger i hjørnepunkter
\item Eventuelt dualproblemer som sidebemærkning. Hvis vi skal have det med måske en bemærkning om spilteori og nulsum(slackvariable løsning for dual).
\item Det konvekse hylster.
\end{itemize}
\end{frame}


\begin{frame}
\frametitle{Simplex}
\begin{itemize}
\item Hvordan simplex afsøger hjørnepunkter.
\item implementationer af simplex(tænker kort gennemgang af forskelle)
\item gennemgang af fuld tabelimplementering (mere udførligt).
\item Lexicografi til valg af pivoteringsmetode.
\item Kompleksitet som udgangspunkt for valg af metode.
\end{itemize}
\end{frame}


\begin{frame}
\frametitle{Proces(hvis vi gider)}
\begin{itemize}
\item Projektarbejde i cornaens tidsalder.
\end{itemize}
\end{frame}


\end{document}