\documentclass{beamer}

\usepackage[utf8]{inputenc}
\usepackage{amsmath}


%Information to be included in the title page:
\title{Optimering: Lineær programmering}
\author{B303c}
\institute{Aalborg Universitet}
\date{2020}



\begin{document}

\frame{\titlepage}


\begin{frame}
\frametitle{Foreløbig problemformulering}
Hvordan kan lineære programmeringsproblemer anskues geometrisk, og hvordan kan disse løses ved hjælp af simplexmetoden?
\end{frame}
%tænker umiddelbart vi forklare hvad hvert afsnit handler om og hvorfor det er med, så er den skid vel slået?
\begin{frame}
\frametitle{Introduktion til lineær algebra}
\begin{itemize}
\item Matricer og vektorer.
\item Lineære ligningssystemer.
\item Løsning af Liniære ligningssystemer.
\item Span, lineær uafhængighed, afbilding, invers og underrum.
\end{itemize}
\end{frame}


\begin{frame}
\frametitle{Lineære optimeringsproblemer}
\begin{itemize}
\item Løsninger.
\item Standardform.
\item Dualproblemer.
\end{itemize}
\end{frame}


\begin{frame}
\frametitle{Grafisk løsning}
\begin{itemize}
\item Konvekse mængder.
\item Ektremer, hjørnepunkter og basale løsninger.
\item Polyeder på standardform.
\item Degenerering.
\item Eksistens af ekstremumspunkter. %her skal det nok påpeges at det her det for alvor går nuts :P
\item Ekstremumpunkter som optimal løsning.
\item Repræsentation af begrænsede polyedre.
\end{itemize}
\end{frame}

\begin{frame}
\frametitle{Simplex}
\begin{itemize}
\item Gennemgang af algoritmen.
\item Pivoterings-løkker og lexicografi.
\item Kompleksitet.
\item Overvejelser
\begin{itemize}
\item Bevis for at simplex finder optimal løsning.
\item Python-implementering.
\end{itemize}
\end{itemize}
\end{frame}


\begin{frame}
\frametitle{Proces}
\begin{itemize}
\item Godt med.
\item Milepælsplanlægning.
\item Coronaproblematikker.
\end{itemize}
\end{frame}

\end{document}