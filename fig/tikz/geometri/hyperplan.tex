\begin{center}
\begin{tikzpicture}
% Den øverste linje
\draw[name path=b,-, white, thick] (0,3.5) -- (6,3.5);
%
% Den nederste linje
\draw[name path=c,-, white, thick] (0,-1) -- (6,-1);
%
% Den mellemste linje
\draw[name path=a,-, white, thick] (0,0) -- (6,3);
%
% Farvning 
\tikzfillbetween [of=a and b]{myblue!15}
\tikzfillbetween [of=a and c]{myred!15}
%
% Linjen mellem 
\draw[-, black, very thick] (0,0) -- (6,3);
\draw[->, black, thick] (2.4,1.2) -- (2,2);
%
% Navn på hyperplanet
\filldraw[black] (0.2,-0.3) circle (0pt) node[anchor=west] {$\mathbf{a}^T \mathbf{x}=b$};
%
% Navnet på den øvre halvrum - blå
\filldraw[black] (3,2.6) circle (0pt) node[anchor=west] {$\mathbf{a}^T \mathbf{x} > b$};
%
% Navnet på den nedre halvrum - rød
\filldraw[black] (3,1.3) circle (0pt) node[anchor=west] {$\mathbf{a}^T \mathbf{x} < b$};
% 
% Navnet på vektorern - a
\filldraw[black] (1.6,2.2) circle (0pt) node[anchor=west] {$\mathbf{a}$};
%
\end{tikzpicture}
  \captionof{figure}{Et hyperplan $\mathbf{a}^T \mathbf{x}=b$, og to halvrum, markeret med blå for den øvre halvrum $\mathbf{a}^T \mathbf{x} < b$ og rød for den nedre halvrum $\mathbf{a}^T \mathbf{x} > b$.}
  \label{fig:Graf123}
\end{center}