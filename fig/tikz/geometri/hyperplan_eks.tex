\begin{center}
\begin{tikzpicture}[scale=8]
% Koordinater
% -------------------------------------------------------
\coordinate (c) at (0.7,1.2,1.2);
\coordinate (d) at (1.2,1.2,1.2);
\coordinate (g) at (1.2,0.7,1.2);
\coordinate (h) at (0.7,0.7,1.2);
%
% Tegning af figur og kommentarer
% -------------------------------------------------------
%
% Farve 
% -------------------------------------------------------
\filldraw[fill=myblue,opacity=0.3, thick](c)--(d)--(g)--(h)--(c);
%
% Linjer og navne
% -------------------------------------------------------
% a_1
\draw[gray,thin,dashed](0.7,0.5,1.2)--(0.7,1.41,1.2);
\draw[thick](h)--node[left]{$\mathbf{a}_1^T \mathbf{x}=b_1$} (c);
\draw[thick,->](0.7,0.95,1.2)--(0.8,0.95,1.2) node[right]{$a_1$};
%
% a_2
\draw[gray,thin,dashed](1.4,1.2,1.2)--(0.5,1.2,1.2);
\draw[thick](d)--node[above]{$\mathbf{a}_2^T \mathbf{x}=b_2$} (c);
\draw[thick,->](0.95,1.2,1.2)--(0.95,1.1,1.2) node[below]{$a_2$};
%
% a_3
\draw[gray,thin,dashed](1.2,1.4,1.2)--(1.2,0.5,1.2);
\draw[thick](d)--node[right] {$\mathbf{a}_3^T \mathbf{x}=b_3$} (g);
\draw[thick,->](1.2,0.95,1.2)--(1.1,0.95,1.2) node[left]{$a_3$};
%
% a_4  
\draw[gray,thin,dashed](1.4,0.7,1.2)--(0.5,0.7,1.2);
\draw[thick](g)--node[below]{$\mathbf{a}_4^T \mathbf{x}=b_4$} (h);
\draw[thick,->](0.95,0.7,1.2)--(0.95,0.8,1.2) node[above]{$a_4$};
%
% Navn til polyeden
% -------------------------------------------------------
\draw[black] (0.91,0.96,1.2) circle (0pt) node[anchor=west] {$\mathcal{P}$};
%
\end{tikzpicture}
  \captionof{figure}{Fællesmængden af de fire hyperplaners øvre halvrum udgør polyederet $\mathcal{P} \in \R^2$ markeret med blå.}
  \label{fig:nej5}
\end{center}