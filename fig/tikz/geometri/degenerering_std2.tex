\begin{minipage}[b]{0.45\textwidth}
%
%%%%%%%%%%%%%%%%%%%%%%%%%%%%%%%%
%%% Flot graf alla Julie     %%%
%%%%%%%%%%%%%%%%%%%%%%%%%%%%%%%%
%
%
\begin{center}
\begin{tikzpicture}[scale=6]
%
% Koordinater 
% ------------------------------------------------------
\coordinate (a) at (0.1,0.1,-0.1);
\coordinate (b) at (0.5,0.1,-0.1);
\coordinate (aa) at (0.35,0.6,-0.1);
\coordinate (bb) at (0.25,0.6,-0.1);
\coordinate (a1) at (0.075,0.05,-0.1);
\coordinate (b1) at (0.525,0.05,-0.1);
\coordinate (a2) at (0.05,0.1,-0.1);
\coordinate (b2) at (0.55,0.1,-0.1);
\coordinate (c) at (0.17,0.46,-0.1);
\coordinate (d) at (0.43,0.46,-0.1);
\coordinate (cc) at (0.28,0.46,-0.1);
\coordinate (dd) at (0.32,0.46,-0.1);
\coordinate (e) at (0.3,0.5,-0.1);
%
% Polyeden
% -------------------------------------------------------
\filldraw [fill=myblue,opacity=0.5] 
         (a) -- (b) -- (dd) -- (cc) -- (a);
%        
% Streger 
% -------------------------------------------------------
  \draw[thick](d)--(c);
  \draw[thick](a)--(b)--(e)--(a);
  \draw[thick](e)--(aa);
  \draw[thick](e)--(bb);
  \draw[thick](b)--(b1);
  \draw[thick](a)--(a1);
  \draw[thick](b)--(b2);
  \draw[thick](a)--(a2);
%
%
% Punkt 
% -------------------------------------------------------
\filldraw [black] (e) circle (0.2pt);
\node at (0.3,0.6,-0.1) (){$\mathbf{x}$};
\node at (0.6 ,0.5,-0.1) (){$\mathbf{a}_1 \mathbf{x} = b_1 - \epsilon$};
\filldraw [black] (0,0,0.5) circle (0.pt);
% 
% 
%
% Koordinatsystemet 
% -------------------------------------------------------
\draw[thick,->] (0,0,0) -- (0.9,0,0) node[anchor=south east]{$x$};
\draw[thick] (0,0,0) -- (-0.1,0,0);
\draw[thick,->] (0,0,0) -- (0,0.7,0) node[anchor=north west]{$y$};
\draw[thick] (0,0,0) -- (0,-0.1,0);
%
\end{tikzpicture}
  \captionof{figure}{En lille ændring i den aktive betingelse $\mathbf{a}_1 \mathbf{x} = b_1- \epsilon$, der gør at $\mathbf{x}$ ikke er en degenereret basal løsning.}
  \label{fig:mmmjegerikkesyg}
\end{center}
%
\end{minipage}