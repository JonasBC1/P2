\begin{figure}[H]
\centering
\begin{tikzpicture}
\begin{axis}[
	 axis x line=center,
  axis y line=center,
  xtick={-5,-4,...,5},
  ytick={-5,-4,...,5},
  xlabel={$x_1$},
  ylabel={$x_2$},
  xlabel style={below right},
  ylabel style={above left},
  xmin=-1,
  xmax=4,
  ymin=-1,
  ymax=4]
	legend style={at={(0.45,0.025)},anchor=south west},
]



\addplot[name path=b, domain={0:6}, color=myblue, style=very thick,samples=2]{6-2*x};
	\addlegendentry{$2x_1+x_2 \leq 6$}
	\addplot[name path=a, domain={0:6}, color=myred, style=very thick, samples=2]{3-0.5*x};
	\addlegendentry{$x_1+2x_2 \leq 6$}
	

	\addplot[domain=0:6,samples=2,color=myred, style=very thick]{((6-2*x};
	\addplot[domain=0:6,samples=2,color=myblue, style=very thick]{3-0.5*x};

	
\addplot[domain=-1:6, samples=2, color=black, dotted, thick]{-x};
\addplot[domain=-1:6, samples=2, color=black, dotted, thick]{2-x};
\addplot[domain=0:6, samples=2, color=black, dotted, thick]{4-x};
\addplot[mark=*] coordinates {(2,2)};


\end{axis}
\end{tikzpicture}
\caption{Grafisk løsning til problemet i \ref{eks:min_loes}}
\label{fig:min_loes}
\end{figure}
\noindent
Løsningsmængden er det skraverede område på figur \ref{fig:min_loes}. For at finde den optimale løsning findes enhver given skalar $k$ og overvejer løsningsmængden med alle punkter, hvor objektfunktionen $\textbf{c}^T\textbf{x}$ er lig med $k$. Dette illustreres ved den linje, der er beskrevet af ligningen $-x_l-x_2=k$. Bemærk, at denne linje er vinkelret på vektoren $\textbf{v}$ med koordinaterne $\textbf{v}=(-1,-1)$.
Forskellige værdier for $k$ fører til forskellige linjer, hvor  alle linjer er parallelle med hinanden samt vinkelret med vektoren. Hvis man øger værdien af $k$ svarer til at bevæge linjen $z=x_l-x_2$ langs retningen af vektoren $\textbf{v}$. For at finde den optimale løsning er det interessante at minimere k, og dermed bevæge linjen så meget som muligt i retning af $-\textbf{v}$ uden at forlade løsningsmængden. Det mest optimale er derfor $k = -4$ og dette gør, at vektor $\textbf{x}=(2,2)$ er den mest optimale løsning i løsningsmængden. 