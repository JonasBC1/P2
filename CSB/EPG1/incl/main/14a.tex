\subsection*{(a) Newtons metode og bisektionsmetoden}
% 
%
\begin{color}{AAUblue2}
Bestem en nummerisk approksimation til værdien af $\lambda$ ud fra 
\begin{align*}
\lambda \text{cosh}\left( \frac{75}{\lambda} \right) = \lambda + 15,
\end{align*}
ved både bisektionsmetoden og ved Newtons metode. 
Argument ud fra teorien i Turner for anvendelsen af disse to metoder.
\end{color}
\\\\
%
%
Både Newtons metode og bisektionsmetoden er metoder til numerisk bestemmelse af nulpunkter.
Derfor omskrives ligning (bingbong) til
%
\begin{align*}
\cosh( \frac{L}{2\lambda}) - \lambda - s &= 0,
\end{align*}
%
hvorefter metoderne kan bruges til at bestemme et nulpunkt.
Bisektionsmetoden når en tolerance på $1\cdot 10^{-12}$ efter $47$ iterationer.
Med samme tolerance afsluttes Newtons metode efter kun $7$ iterationer.
$$
\begin{array}{l|c|c}
\text{Metode} & \text{Resultat} & \text{Iterationer}\\
\hline
\text{Bisektion}	& 189.94865 & 46\\
\text{Newton}		& 189.94865 & 7\\
\end{array}
$$
I forhold til anvendelsen af de to metoder kan man i forbindelse med Newtons metode ikke på forhånd se om den giver et resultat, opfører sig kaotisk eller ikke giver er resultat. 
Men i de tilfælde hvor den giver er resultat findes den eksakte værdi ved få iterationer og er derfor rigtig brugbar. 
Den er nem og arbejde med og hurtig til at få et eksakt resultat. 
%
Modsat virker bisektionsmetoden altid, og den er god og stabil. Dog har den mange iterationer og er tung at arbejde med.

I sammenligning med resultaterne ses det også her at Newtons metode bruger 7 iterationer, hvorimod bisektionsmetoden bruger 46. 
%
Man vil derfor kunne argumentere for at newtons metode er den bedste i de tilfælde den virker, hvorimod bisektionsmetoden altid giver et resultat. 
%