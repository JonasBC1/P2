\chapter{Det hængende kabel}
% God arbejdslyst xD
\begin{color}{AAUblue2}
%
Denne delopgave giver en grundlæggende forståelse for kvadraturregler.
Givet et kabel ophængt i to pæle af samme højde vil under påvirkning af tyngdekraften antage en form, der kan beskrives ved en kurve, som på dansk kaldes en kædelinje, der tager formen 
\begin{align*}
f(x)=h_0 + \lambda \text{cosh} \left( \frac{x}{\lambda} \right).
\end{align*}
Kablets massetæthed betegnes med $\rho$ og spændingen i kablet i dets laveste punkt med $T$, så $\lambda = \frac{T}{\rho}$. Afstanden som kablet synker i midten, betegnes $s$.
% 
\end{color}
\section*{1. Interpolerende kvadraturregler}
% 
%
\begin{color}{AAUblue2}
%
fisk
% 
\end{color}
\\\\
% 
%
%
%%%%%%%%%%%%%%%%%%%%%%%%%%%%%%%%
%%% Flot graf alla Julie     %%%
%%%%%%%%%%%%%%%%%%%%%%%%%%%%%%%%
%
\begin{figure}[h!]
\begin{center}
\begin{tikzpicture}[scale=10]
%
% Koordinater 
% ------------------------------------------------------
\coordinate (a) at (0.1,0.5);
\coordinate (b) at (0.7,0.4);
\coordinate (c) at (0.4,0.483);
%
% Funktion
% ------------------------------------------------------
\filldraw[thick,color=AAUred!15] (0.1,0.483) -- (0.7,0.483) -- (0.7,0.3) -- (0.1,0.3) -- (0.1,0.483);
\draw[very thick, color=AAUblue2] (a) .. controls (0.3,0.8) and (0.6,0.1) .. (b);  
%
%
\draw[thick](0.1,0.28) -- (0.1,0.32); % (a)
\draw[thick](0.7,0.28) -- (0.7,0.32); % (b)
\draw[thick](0.4,0.28) -- (0.4,0.32); % (m_1)
%
% Punkt 
% -------------------------------------------------------
\filldraw [black] (a) circle (0.1pt);  
\filldraw [black] (b) circle (0.1pt); 
\filldraw [black] (c) circle (0.1pt); 
%
\node at (0.1,0.22) (){$a$};
\node at (0.7,0.22) (){$b$};
\node at (0.4,0.22) (){$\frac{a+b}{2}$};
%
% 
%
% Koordinatsystemet 
% -------------------------------------------------------
\draw[thick,->] (-0.05,0.3) -- (0.85,0.3) node[anchor=south east]{$x$};
\draw[thick,->] (0,0.2) -- (0,0.7) node[anchor=north west]{$y$};
%
%
\end{tikzpicture}
  \caption{Illustration af Midtpunktsreglen.}
  \label{fig:midtpunkt}
\end{center}
\end{figure}



%
%%%%%%%%%%%%%%%%%%%%%%%%%%%%%%%%
%%% Flot graf alla Julie     %%%
%%%%%%%%%%%%%%%%%%%%%%%%%%%%%%%%
%
\begin{figure}[h!]
\begin{center}
\begin{tikzpicture}[scale=10]
%
% Koordinater 
% ------------------------------------------------------
\coordinate (a) at (0.1,0.5);
\coordinate (b) at (0.7,0.4);
\coordinate (c) at (0.4,0.483);
%
% Funktion
% ------------------------------------------------------
\filldraw[thick,color=AAUred!15] (a) -- (b) -- (0.7,0.3) -- (0.1,0.3) -- (a);
\draw[very thick, color=AAUblue2] (a) .. controls (0.3,0.8) and (0.6,0.1) .. (b);  
%
%
\draw[thick](0.1,0.28) -- (0.1,0.32); % (a)
\draw[thick](0.7,0.28) -- (0.7,0.32); % (b)
%\draw[thick](0.4,0.28) -- (0.4,0.32); % (m)
%
% Punkt 
% -------------------------------------------------------
\filldraw [black] (a) circle (0.1pt);  
\filldraw [black] (b) circle (0.1pt); 
%\filldraw [black] (c) circle (0.1pt); 
%
\node at (0.1,0.22) (){$a$};
\node at (0.7,0.22) (){$b$};
%\node at (0.4,0.22) (){$\frac{a+b}{2}$};
%
% 
%
% Koordinatsystemet 
% -------------------------------------------------------
\draw[thick,->] (-0.05,0.3) -- (0.85,0.3) node[anchor=south east]{$x$};
\draw[thick,->] (0,0.2) -- (0,0.7) node[anchor=north west]{$y$};
%
%
\end{tikzpicture}
  \caption{Illustration af Trapezreglen.}
  \label{fig:trapez}
\end{center}
\end{figure}


%
%%%%%%%%%%%%%%%%%%%%%%%%%%%%%%%%
%%% Flot graf alla Julie     %%%
%%%%%%%%%%%%%%%%%%%%%%%%%%%%%%%%
%
\begin{figure}[h!]
\begin{center}
\begin{tikzpicture}[scale=10]
%
% Koordinater 
% ------------------------------------------------------
\coordinate (a) at (0.1,0.5);
\coordinate (b) at (0.7,0.4);
\coordinate (c) at (0.4,0.483);
%
% Funktion
% ------------------------------------------------------
\filldraw[thick,color=AAUred!15] (a) -- (c) -- (b) -- (0.7,0.3) -- (0.1,0.3) -- (a);
\draw[very thick, color=AAUblue2] (a) .. controls (0.3,0.8) and (0.6,0.1) .. (b);  
%
%
\draw[thick](0.1,0.28) -- (0.1,0.32); % (a)
\draw[thick](0.7,0.28) -- (0.7,0.32); % (b)
%\draw[thick](0.4,0.28) -- (0.4,0.32); % (m)
%
% Punkt 
% -------------------------------------------------------
\filldraw [black] (a) circle (0.1pt);  
\filldraw [black] (b) circle (0.1pt); 
\filldraw [black] (c) circle (0.1pt); 
%
\node at (0.1,0.22) (){$a$};
\node at (0.7,0.22) (){$b$};
\node at (0.4,0.22) (){$\frac{a+b}{2}$};
%
% 
%
% Koordinatsystemet 
% -------------------------------------------------------
\draw[thick,->] (-0.05,0.3) -- (0.85,0.3) node[anchor=south east]{$x$};
\draw[thick,->] (0,0.2) -- (0,0.7) node[anchor=north west]{$y$};
%
%
\end{tikzpicture}
  \caption{Illustration af Simpsonsreglen.}
  \label{fig:simpsons}
\end{center}
\end{figure}
%
\section*{2. Formler}
% 
\begin{color}{AAUblue2}
%
Vis, at følgende gælder:
\begin{align*}
f \left( \frac{L}{2} \right) = h_0 + \lambda \text{cosh} \left( \frac{L}{2 \lambda} \right) = h, \\
f(0) = h_0 + \lambda = h - s.
\end{align*}
Vis, at disse formler giver ligningen 
\begin{align*}
\lambda \text{cosh} \left( \frac{L}{2 \lambda} \right) = \lambda + s.
\end{align*}
%
\end{color}
\\
% 
%
Da søjlerne er placeret i $x = \pm \frac{L}{2}$ og søjlerne har højden $h$ gælder følgende
\begin{align}
f\left(\frac{L}{2}\right)	= h_0 + \lambda \cosh \left(\frac{L}{2 \lambda}\right) = h
\end{align}
Søjlerne stå således symetrisk omkring $x = 0$ og da $s$ er afstanden som kæden flader, haves følgende
\begin{align}
f(0) =	 h_0 + \lambda 	\cosh (0)	= h_0 + \lambda = h-s
\end{align}
%
Ligning (1.1) og (1.2) kan give anledning til en ny ligning:
%
\begin{align}
h_0 + \lambda \cosh \left(\frac{L}{2 \lambda}\right) -\left(h_0 + \lambda \cosh \left(\frac{L}{2 \lambda}\right)\right) &= h_0 + \lambda -(h_0 + \lambda)\\
h_0 + \lambda \cosh \left(\frac{L}{2 \lambda}\right) - h &= h_0 + \lambda - (h - s)\\
\lambda \cosh \left(\frac{L}{2 \lambda}\right) - h &= \lambda - h + s\\
\lambda \cosh \left(\frac{L}{2 \lambda}\right) &= \lambda + s
\end{align}
\section*{3. Adaptiv kvadratturregler}
%
% 
%
\begin{color}{AAUblue2}
%
\textbf{Opgavebeskrivelse:} 
Forklar idéen i en sammensat adaptiv kvadraturregel.
% 
\end{color}
\\\\
% 
\textbf{Besvarelse:} 
Idéen i de adaptive kvadraturregler omhandler, hvordan en funktion kan behandles, når der ikke kan udvælges punkter til underinddeling på forhånd. Eksempelvis hvis funktionen ikke er kendt, men funktionsværdien kendes for en mængde af punkter.
Der findes to primære måder at lave disse underinddelinger: \\
1. Der vælges et antal $N$ underinddelinger af det oprindelige interval af formen $\left [  x_k, x_{k+1} \right ]$, hvor $a=x_0<x_1<x_2<\ldots<x_N$ herefter anvendes den sammensatte Simpsons regel beskrevet i opgave 3 hvor antallet af underintervaller i de oprindelige $N$ gennemgår kontinuerlig fordobling.
(af uransagelige årsager siger bogen at 5 eller 20 er de mest normale inddellinger, dette bliver ikke uddybet) \\
2. Den anden måde benytter ligeledes den sammensatte Simpsons regel her på intervallet, denne udføres igen på de nye delintervaller samt på begge halvdele af disse, såfremt man er inden for den ønskede nøjagtighed for et givet interval accepteres dette som delresultat og der arbejdes nu med næste delinterval.
Dette gentages indtil alle delintervaller er opdelt i et antal intervaller der bringer det inden for den ønskede fejl.
(side 75 bunden)
%

\section*{4.}
% 
\section*{(a) }

\section*{(b) Python script - Sekantmetoden}
%Skriv et Python script der implementerer sekantmetoden. Det kan gøres ved at modicere scriptet beregnet til Newtons metode. Sammenlig sekantmetoden med Newtons metode.

Forskellen mellem Sekantmetoden og Newtons metode er at fiiiiiiiiiiiiiiiiiiiiiiiiiiiiiiiiiiiiiiiiiiiiiiiiiiiiiiiiiiiiiiiiiiisk 

\section*{(c) Fixpunktligningen}
% Ligningen (6) kan omskrives til en xpunktligning. Gr rede for de to omskrivninger her


\section*{(d) }