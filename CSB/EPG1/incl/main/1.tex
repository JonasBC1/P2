\chapter{Det hængende kabel}
% God arbejdslyst xD
\begin{color}{AAUblue2}
%
Denne delopgave giver en grundlæggende forståelse for kvadraturregler.
Givet et kabel ophængt i to pæle af samme højde vil under påvirkning af tyngdekraften antage en form, der kan beskrives ved en kurve, som på dansk kaldes en kædelinje, der tager formen 
\begin{align*}
f(x)=h_0 + \lambda \text{cosh} \left( \frac{x}{\lambda} \right).
\end{align*}
Kablets massetæthed betegnes med $\rho$ og spændingen i kablet i dets laveste punkt med $T$, så $\lambda = \frac{T}{\rho}$. Afstanden som kablet synker i midten, betegnes $s$.
% 
\end{color}
\section*{1. Interpolerende kvadraturregler}
% 
%
\begin{color}{AAUblue2}
%
fisk
% 
\end{color}
\\\\
% 
%
%
%%%%%%%%%%%%%%%%%%%%%%%%%%%%%%%%
%%% Flot graf alla Julie     %%%
%%%%%%%%%%%%%%%%%%%%%%%%%%%%%%%%
%
\begin{figure}[h!]
\begin{center}
\begin{tikzpicture}[scale=10]
%
% Koordinater 
% ------------------------------------------------------
\coordinate (a) at (0.1,0.5);
\coordinate (b) at (0.7,0.5);
\coordinate (c) at (0.4,0.463);
%
% Funktion
% ------------------------------------------------------
\draw[very thick, color=AAUblue2] (a) .. controls (0.3,0.8) and (0.5,0.1) .. (b);    
\filldraw[thick,color=AAUred!15] (0.1,0.463) -- (0.7,0.463)-- (0.7,0.3) -- (0.1,0.3) -- (0.1,0.463);
%
%
\draw[thick](0.1,0.28) -- (0.1,0.32); % (a)
\draw[thick](0.7,0.28) -- (0.7,0.32); % (b)
\draw[thick](0.4,0.28) -- (0.4,0.32); % (m_1)
%
% Punkt 
% -------------------------------------------------------
\filldraw [black] (a) circle (0.1pt);  
\filldraw [black] (b) circle (0.1pt); 
\filldraw [black] (c) circle (0.1pt); 
%
\node at (0.1,0.22) (){$a$};
\node at (0.7,0.22) (){$b$};
\node at (0.4,0.22) (){$\frac{a+b}{2}$};
%
% 
%
% Koordinatsystemet 
% -------------------------------------------------------
\draw[thick,->] (-0.05,0.3) -- (0.85,0.3) node[anchor=south east]{$x$};
\draw[thick,->] (0,0.2) -- (0,0.7) node[anchor=north west]{$y$};
%
%
\end{tikzpicture}
  \caption{Illustration af Midtpunktsreglen.}
  \label{fig:midtpunkt}
\end{center}
\end{figure}





%
%%%%%%%%%%%%%%%%%%%%%%%%%%%%%%%%
%%% Flot graf alla Julie     %%%
%%%%%%%%%%%%%%%%%%%%%%%%%%%%%%%%
%
\begin{figure}[h!]
\begin{center}
\begin{tikzpicture}[scale=10]
%
% Koordinater 
% ------------------------------------------------------
\coordinate (a) at (0.1,0.3);
\coordinate (b) at (0.7,0.3);
\coordinate (k2) at (0.7,0.862);
%
% Funktion
% ------------------------------------------------------
\draw[very thick, color=AAUblue2] (a) .. controls (0.3,0.5) and (0.45,0.5) .. (b);   
%
% K1 
% ------------------------------------------------------
\draw[dashed, very thick, color=AAUred] (a) -- (0.74,0.9);
%
\draw[thick](0.1,0.18) -- (0.1,0.22); % (a)
\draw[thick](0.7,0.18) -- (0.7,0.22); % (b)
%
% Punkt 
% -------------------------------------------------------
\filldraw [black] (k2) circle (0.1pt); 
\filldraw [black] (a) circle (0.1pt);  
\filldraw [black] (b) circle (0.1pt); 
%
\node at (0.1,0.12) (){$x_0$};
\node at (0.7,0.12) (){$x_1$};
\node at (0.39,0.12) (){$x_0 + \frac{h}{2}$};
%
%\node at (0.55,0.3) (){$m_2$};
%\node at (0.475,0.3) (){$m_3$};
% 
%
% Koordinatsystemet 
% -------------------------------------------------------
\draw[thick,->] (-0.05,0.2) -- (0.85,0.2) node[anchor=south east]{$x$};
\draw[thick,->] (0,0.1) -- (0,1) node[anchor=north west]{$y$};
%
%
\end{tikzpicture}
  \caption{Illustration af et skridt i Eulers metode, hvor  tangentvektoren er markeret med rød.}
  \label{fig:midtpunkt}
\end{center}
\end{figure}
%
\section*{2. Sammensat kvadraturregel}
% 
%
\begin{color}{AAUblue2}
%
fisk
% 
\end{color}
\\\\
% 
%
\section*{3. Adaptiv kvadratturregler}
%
% 
%
\begin{color}{AAUblue2}
%
\textbf{Opgavebeskrivelse:} 
Forklar idéen i en sammensat adaptiv kvadraturregel.
% 
\end{color}
\\\\
% 
\textbf{Besvarelse:} 
Idéen i de adaptive kvadraturregler omhandler, hvordan en funktion kan behandles, når der ikke kan udvælges punkter til underinddeling på forhånd. Eksempelvis hvis funktionen ikke er kendt, men funktionsværdien kendes for en mængde af punkter.
Der findes to primære måder at lave disse underinddelinger: \\
1. Der vælges et antal $N$ underinddelinger af det oprindelige interval af formen $\left [  x_k, x_{k+1} \right ]$, hvor $a=x_0<x_1<x_2<\ldots<x_N$ herefter anvendes den sammensatte Simpsons regel beskrevet i opgave 3 hvor antallet af underintervaller i de oprindelige $N$ gennemgår kontinuerlig fordobling.
(af uransagelige årsager siger bogen at 5 eller 20 er de mest normale inddellinger, dette bliver ikke uddybet) \\
2. Den anden måde benytter ligeledes den sammensatte Simpsons regel her på intervallet, denne udføres igen på de nye delintervaller samt på begge halvdele af disse, såfremt man er inden for den ønskede nøjagtighed for et givet interval accepteres dette som delresultat og der arbejdes nu med næste delinterval.
Dette gentages indtil alle delintervaller er opdelt i et antal intervaller der bringer det inden for den ønskede fejl.
(side 75 bunden)
%

\section*{4. Konkret eksempel}
%
\begin{color}{AAUblue2}
Konkret eksempel, hvor $L=150$ og $s=15$.
\end{color}
%
% 
\subsection*{(a) Newtons metode og bisektionsmetoden}
% 
%
\begin{color}{AAUblue2}
Bestem en nummerisk approksimation til værdien af $\lambda$ ud fra 
\begin{align*}
\lambda \text{cosh}\left( \frac{75}{\lambda} \right) = \lambda + 15,
\end{align*}
ved både bisektionsmetoden og ved Newtons metode. 
Argument ud fra teorien i Turner for anvendelsen af disse to metoder.
\end{color}
\\\\
%
%
Både Newtons metode og bisektionsmetoden er metoder til numerisk bestemmelse af nulpunkter.
Derfor omskrives ligning (bingbong) til
%
\begin{align*}
\cosh( \frac{L}{2\lambda}) - \lambda - s &= 0,
\end{align*}
%
hvorefter metoderne kan bruges til at bestemme et nulpunkt.
Bisektionsmetoden når en tolerance på $1\cdot 10^{-12}$ efter $47$ iterationer.
Med samme tolerance afsluttes Newtons metode efter kun $7$ iterationer.
$$
\begin{array}{l|c|c}
\text{Metode} & \text{Resultat} & \text{Iterationer}\\
\hline
\text{Bisektion}	& 189.94865 & 46\\
\text{Newton}		& 189.94865 & 7\\
\end{array}
$$
I forhold til anvendelsen af de to metoder kan man i forbindelse med Newtons metode ikke på forhånd se om den giver et resultat, opfører sig kaotisk eller ikke giver er resultat. 
Men i de tilfælde hvor den giver er resultat findes den eksakte værdi ved få iterationer og er derfor rigtig brugbar. 
Den er nem og arbejde med og hurtig til at få et eksakt resultat. 
%
Modsat virker bisektionsmetoden altid, og den er god og stabil. Dog har den mange iterationer og er tung at arbejde med.

I sammenligning med resultaterne ses det også her at Newtons metode bruger 7 iterationer, hvorimod bisektionsmetoden bruger 46. 
%
Man vil derfor kunne argumentere for at newtons metode er den bedste i de tilfælde den virker, hvorimod bisektionsmetoden altid giver et resultat. 
%
\subsection*{(b) Sekantmetoden}
%
\begin{color}{AAUblue2}
Skriv et Python script der implementerer sekantmetoden. 
Sammenlig sekantmetoden med Newtons metode.
\end{color}
\\\\
%
%
Forskellen mellem Sekantmetoden og Newtons metode er at Newtons metode bruger den afledede i et punkt, hvorimod Sekantmetoden bruger sekanten mellem to punkter. \\
I sammenligning med eksemplet på en tolerance $1\cdot10^{-12}$, ses forskellen i på 2 iterationer. 
%
$$\begin{array}{l|c|c}
\text{Metode} & \text{Resultat} & \text{Iterationer}\\
\hline
\text{Newton}		& 189.94865 & 7\\
\text{Sekant}		& 189.94865 & 9
\end{array}$$
%
\subsubsection*{Python script - Sekantmetoden}
%
\lstset{style=mystyle}
\lstinputlisting[language=Python]{code/sekant.py}
\phantom{matematik}
\subsection*{(c) Fixpunktligningen}
% Ligningen (6) kan omskrives til en xpunktligning. Gr rede for de to omskrivninger her

\subsection*{(d) Sammenligning}
%
\begin{color}{AAUblue2}
Sammenlig de forskellige metoder til bestemmelse af $\lambda$ ud fra 
\begin{align*}
\lambda \text{cosh} \left( \frac{75}{\lambda} \right) = \lambda + 15.
\end{align*}
Sammenligningen skal omfatte både de konkrete tal og konvergenshastighed af metoderne.
\end{color}
\\\\
%
%
For at give et bedre sammenligningsgrundlag er alle metoderne kørt med tolerancen $1\cdot10^{-12}$.
Dette betyder, at alle metoderne finder frem til grundlæggende samme resultat og at antallet af iterationer er det interessante.
%
$$\begin{array}{l|c|c}
\text{Metode} & \text{Resultat} & \text{Iterationer}\\
\hline
\text{Bisektion}	& 189.94865 & 46\\
\text{Newton}		& 189.94865 & 7\\
\text{Sekant}		& 189.94865 & 9\\
\text{Integrations}	& 189.94865 & 345
\end{array}$$
