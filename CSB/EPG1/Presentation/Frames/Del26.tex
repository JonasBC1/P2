%\begin{frame}{Opgave 6: Størrelsen af $N$}
%    Hvor stor skal $N$ være, for at fejlen er mindre end $10^{-10}$? 
%    Brug den oprindelige version af \textsc{Interpolation bin.py} til at sammenligne teori og numerisk eksperiment. 
%    Hvad er konklusionen? 
%    Hvor kommer problemerne fra? 
%    I den forbindelse kan I inddrage den grafiske undersøgelse fra punkt 4.
%\end{frame}
\begin{frame}{Teoretisk vurdering af fejlen}
Det skal nu vises, at $f^{(N+1)}(\xi)$ er givet ved $M=10^{(N+1)}$ for funktionen $f(x)=x^2-\sin(10x).$ \\
  For $N=0$ gælder, at $f'(x)=2x-10cos(10x)$.\\
  For $N=1$ gælder, at $f''(x)=2+100sin(10x)$.\\
  For $N=2$ gælder, at $f'''(x)=1000cos(10x)$.\\
  De efterfølgende afledede vil herefter skifte mellem sinus og cosinus.
  Maksimalværdien for disse vil altid være $1$ og den maksimale værdi for den $N$'te afledede bliver derfor $10^{N+1}$. 
\\
  Den teoretiske maksimale fejl er derfor givet ved
  $$\frac{1}{4}h^{(N+1)}10^{(N+1)}$$ for $N \geq 2$. 
\end{frame}
\begin{frame}{Størrelsen af $N$}
    Den teoretiske fejl (max) bliver $10^{-10}$ ved $N=48$, men i det numeriske eksperiment bliver fejlen aldrig mindre eller lig $10^{-10}$. \\ 
    En mulig årsag til forandringer ved fejlen er divergens i yderpunkterne, hvilket kan gøre fejlen betydeligt større end den teoretiske, da de kan gå mod $ \infty $ og $ - \infty $. Det kan også skyldes afrundingsfejl. 
  
\end{frame}