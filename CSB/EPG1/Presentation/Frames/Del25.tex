%\begin{frame}{Opgave 5: Teoretisk vurdering af fejlen}
%    Brug Theorem 16 i afsnit 6.2 i Turner til at lave en teoretisk vurdering af fejlen
%    \begin{align*}
%    \underset{x \in \left [0,3 \right ]}{\text{max}} \lvert f(x)-p_N(x) \rvert.
%    \end{align*}
%    %% SE HJÆLP I OPGAVEFORMULERING %% JA TAK %%
%    %
%    %%%%% THEREOM 16 %&%%%%
%    \begin{align*}
%    f(x)-p(x) = \frac{(x-x_0)(x-x_1)\cdots(x-x_N)}{(N+1)!}f^{(N+1)}(\xi)
%    \end{align*}
%    \text{for ethvert} $\xi$ $\in$ $\left [a,b \right ]$.
%    %%%%%%%%%%%%%%%%%%%%%%%
%    \\
%    %%%%% HINT 1/2 %&%%%%
%    Lad
%    \begin{align*}
%    L(x)=(x-x_0)(x-x_1)\cdots(x-x_N).
%    \end{align*}
%    %%%%%%%%%%%%%%%%%%%%%%%
%    \\
%    For ækvidistante punkter $x_{j+1} - x_j = h$ kan man vise, at der gælder
%    %%%%% HINT 2/2 %&%%%%
%    \begin{align*}
%    \lvert L(x) \rvert \leq \frac{1}{4}(N+1)!h^{N+1}, \text{   } x \in \left [x_0, x_N \right ].
%    \end{align*}
%\end{frame}

\begin{frame}{Teoretisk vurdering af fejlen}
    Udtrykket for maksimal fejl kan derfor omskrives til
    \begin{align*}
    \underset{x \in \left [0,3 \right ]}{\text{max}} \lvert f(x)-p_N(x) \rvert \leq \frac{\frac{1}{4}(N+1)!h^{N+1}}{(N+1)!}f^{(N+1)}(\xi)=\frac{1}{4}h^{N+1}f^{(N+1)}(\xi).
    \end{align*}
    Det skal nu vises, at denne er lig $\frac{1}{4}h^{(N+1)}M$ hvor $M=10^{(N+1)}$ ved funktionen $f(x)=x^2-\sin(10x), x \in \mathbb{R}$ i intervallet $a=0$ til $b=3$, hvor $h$ er antallet af steplength, bliver $h=abs(b-a)/N$. 
    Det skal nu vises, at $f(x)^{(N+1)}(\xi)$ kan omskrives til $M=10^{(N+1)}$.
    For $N=0$ gælder, at $f'(x)=2x-10cos(10x)$.
    For $N=1$ gælder, at $f''(x)=2+100sin(10x)$.
    For $N=2$ gælder, at $f'''(x)=1000cos(10x)$. 
    Den afledede vil herefter skifte mellem sinus og cosinus.
    Maksimalværdien for disse vil altid være $1$ og den maksismale værdi for $N$'te afledede bliver derfor $10^{N+1}$.
    Den teoretiske maksimale fejl er derfor $\frac{1}{4}h^{(N+1)}10^{(N+1)}$ for $N \geq 2$
\end{frame}