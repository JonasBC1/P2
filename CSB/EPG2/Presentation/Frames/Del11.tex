%\begin{frame}{Opgave 1: Interpolerende kvadraturregler}
%    Redegør for interpolerende kvadraturregler. 
%    Beskriv de tre regler, midtpunktsreglen, trapezreglen, og Simpsons regel. 
%    Udled mindst en af reglerne. Forklar begrebet ‘grad af præcision’ for en interpolerende kvadraturregel. 
%    Angiv grad af præcision for de tre ovennævnte regler. 
%    Svarene skal begrundes.
%\end{frame}

\begin{frame}{Interpolerende kvadraturregler}
    Interpolerende kvadraturregler approksimerer en funktion $f(x)$ med et polynomie $p(x)$ over et interval for at beregne det bestemte integrale for $p(x)$, således at 
    \begin{align}
        \int_{a}^{b} f(x) dx \approx \int_{a}^{b} p(x) dx.
    \end{align}
    På de følgende slides er henholdsvis midtpunktsreglen, trapezreglen og Simpsons regel beskrevet, der alle tre er typer af kvadraturregler. 
\end{frame}


\begin{frame}{Interpolerende kvadraturregler:Midtpunktsreglen}
    Midtpunktsreglen tager udgangspunkt i funktionsværdien i midtpunktet mellem to punkter $a$ og $b$, og denne multipliceres med $(b-a)$.
    Således bestemmes et rektangel fra $a$ til $b$ under funktionen, hvor højden svarer til funktionsværdien imellem de to punkter. 
    Rektanglet har til hensigt at approksimere integralet for funktionen. 
    %
%%%%%%%%%%%%%%%%%%%%%%%%%%%%%%%%
%%% Flot graf alla Julie     %%%
%%%%%%%%%%%%%%%%%%%%%%%%%%%%%%%%
%
\begin{figure}[h!]
\begin{center}
\begin{tikzpicture}[scale=10]
%
% Koordinater 
% ------------------------------------------------------
\coordinate (a) at (0.1,0.5);
\coordinate (b) at (0.7,0.4);
\coordinate (c) at (0.4,0.483);
%
% Funktion
% ------------------------------------------------------
\filldraw[thick,color=AAUred!15] (0.1,0.483) -- (0.7,0.483) -- (0.7,0.3) -- (0.1,0.3) -- (0.1,0.483);
\draw[very thick, color=AAUblue2] (a) .. controls (0.3,0.8) and (0.6,0.1) .. (b);  
%
%
\draw[thick](0.1,0.28) -- (0.1,0.32); % (a)
\draw[thick](0.7,0.28) -- (0.7,0.32); % (b)
\draw[thick](0.4,0.28) -- (0.4,0.32); % (m_1)
%
% Punkt 
% -------------------------------------------------------
\filldraw [black] (a) circle (0.1pt);  
\filldraw [black] (b) circle (0.1pt); 
\filldraw [black] (c) circle (0.1pt); 
%
\node at (0.1,0.22) (){$a$};
\node at (0.7,0.22) (){$b$};
\node at (0.4,0.22) (){$\frac{a+b}{2}$};
%
% 
%
% Koordinatsystemet 
% -------------------------------------------------------
\draw[thick,->] (-0.05,0.3) -- (0.85,0.3) node[anchor=south east]{$x$};
\draw[thick,->] (0,0.2) -- (0,0.7) node[anchor=north west]{$y$};
%
%
\end{tikzpicture}
  \caption{Illustration af Midtpunktsreglen.}
  \label{fig:midtpunkt}
\end{center}
\end{figure}
\end{frame}


\begin{frame}{Interpolerende kvadraturregler}
\begin{itemize}
    \item Midtpunktsreglen er givet ved formlen
    \begin{align*}
        M_1 = (b-a)f(\frac{a+b}{2})=hf(\frac{a+b}{2}).    
    \end{align*}
    \\
    \item Polynomiet dannet af midtpunktsreglen er givet ved 
    \begin{align*}
        P_{M}(x)=f(\frac{a+b}{2}) \cdot 1, x \in \left [a,b  \right ].
    \end{align*}
    \\
    \item Fejlleddet for midtpunktsreglen er givet ved 
    \begin{align*}
        \int_{a}^{b}f(x)dx-(b-a)f(\frac{(a+b)}{2})=\frac{(b-a)h^2}{24}{f}''(\xi_M)
    \end{align*}
\end{itemize}
\end{frame}


\begin{frame}{Interpolerende kvadraturregler:Trapezreglen}
    Trapezreglen benytter Lagrange-interpolation med to punkter i stedet for et. 
    Arealet af en trapez med hjørnerne $A, B, C$ og $D$ er givet ved formlen $h(\frac{|AB| + |CD|}{2})$. 
    Heraf reglens navn, da formlen danner et trapez under funktionen, der har til hensigt at approksimere integralet for den oprindelige funktion. \\
    Trapezreglen er givet ved formlen 
    \begin{align*}
    T_1 = \frac{(b-a)}{2}(f(a)+f(b))=h (\frac{f(a)+f(b)}{2}).
    \end{align*}
    Genkald, at 
    \begin{align}
        \int_{a}^{b} f(x) dx = \Bar{f}(b-a),
    \end{align}
    hvor $\Bar{f}$ er den gennemsnitlige funktionsværdi i intervallet $[a,b]$.
    Bemærk, at trapezreglen tager udgangspunkt i, at gennemsnittet for funktionsværdien er gennemsnittet $a$ og $b$, og approksimerer dermed integralet for $f(x)$. 
    % Midtpunktsreglen gør det samme, men med ét punkt i stedet for. 
\end{frame}


\begin{frame}{Interpolerende kvadraturregler}
\begin{itemize}
    \item Det interpolerende polynomium fundet via trapezreglen er givet ved
    \begin{align*}
    P_{T}(x)=f(a)\frac{x-b}{a-b}+f(b)\frac{x-a}{b-a}       
    \end{align*}
    \item Fejlleddet for trapezreglen er givet ved 
    \begin{align*}
    \int_{a}^{b}f(x)dx-\frac{b-a}{2}\left [ f(a)+f(b) \right ]=-\frac{(b-a)h^2}{12}{f}''(\xi_T)
    \end{align*}
\end{itemize}
        %
%%%%%%%%%%%%%%%%%%%%%%%%%%%%%%%%
%%% Flot graf alla Julie     %%%
%%%%%%%%%%%%%%%%%%%%%%%%%%%%%%%%
%
\begin{figure}[h!]
\begin{center}
\begin{tikzpicture}[scale=10]
%
% Koordinater 
% ------------------------------------------------------
\coordinate (a) at (0.1,0.5);
\coordinate (b) at (0.7,0.4);
\coordinate (c) at (0.4,0.483);
%
% Funktion
% ------------------------------------------------------
\filldraw[thick,color=AAUred!15] (a) -- (b) -- (0.7,0.3) -- (0.1,0.3) -- (a);
\draw[very thick, color=AAUblue2] (a) .. controls (0.3,0.8) and (0.6,0.1) .. (b);  
%
%
\draw[thick](0.1,0.28) -- (0.1,0.32); % (a)
\draw[thick](0.7,0.28) -- (0.7,0.32); % (b)
%\draw[thick](0.4,0.28) -- (0.4,0.32); % (m)
%
% Punkt 
% -------------------------------------------------------
\filldraw [black] (a) circle (0.1pt);  
\filldraw [black] (b) circle (0.1pt); 
%\filldraw [black] (c) circle (0.1pt); 
%
\node at (0.1,0.22) (){$a$};
\node at (0.7,0.22) (){$b$};
%\node at (0.4,0.22) (){$\frac{a+b}{2}$};
%
% 
%
% Koordinatsystemet 
% -------------------------------------------------------
\draw[thick,->] (-0.05,0.3) -- (0.85,0.3) node[anchor=south east]{$x$};
\draw[thick,->] (0,0.2) -- (0,0.7) node[anchor=north west]{$y$};
%
%
\end{tikzpicture}
  \caption{Illustration af Trapezreglen.}
  \label{fig:trapez}
\end{center}
\end{figure}
\end{frame}


\begin{frame}{Interpolerende kvadraturregler}
    Simpsons regel benytter Lagrange-interpolation med tre punkter og udregner integralet under en parabel, modsat de førnævnte to regler, der udregner integraler under lige linjer. 
    Simpsons regel er givet ved formlen 
    \begin{align*}
    S_2 = \frac{(b-a)}{6}(f(a)+4f(\frac{a+b}{2})+f(b)=\frac{\widetilde{h}}{3}(f(a)+4f(\frac{a+b}{2})+f(b))
    \end{align*}
    Det interpolerende polynomium fundet via Simpsons regel er givet ved formlen
    \begin{align*}
    P_{S}(x)=f(a)\frac{(x-m)(x-b)}{(a-m)(a-b)}+f(m)\frac{(x-a)(x-b)}{(m-a)(m-b)}+f(b)\frac{(x-a)(x-m)}{(b-a)(b-m)}.
    \end{align*}
\end{frame}


\begin{frame}{Interpolerende kvadraturregler:Fejlled Simpson}
    Fejlleddet for Simpsons regel er givet ved 
    \begin{align*}
    \int_{a}^{b}f(x)dx-\frac{\Tilde{h}}{3}\left [ f(a)+4f(a+h) + f(b) \right ]=-\frac{(b-a)\Tilde{h}^4}{180}{f}^{(4)}(\xi_S)
    \end{align*}
    %
%%%%%%%%%%%%%%%%%%%%%%%%%%%%%%%%
%%% Flot graf alla Julie     %%%
%%%%%%%%%%%%%%%%%%%%%%%%%%%%%%%%
%
\begin{figure}[h!]
\begin{center}
\begin{tikzpicture}[scale=8]
%
% Koordinater 
% ------------------------------------------------------
\coordinate (a) at (0.1,0.5);
\coordinate (b) at (0.7,0.4);
\coordinate (c) at (0.4,0.483);
%
% Funktion
% ------------------------------------------------------
\filldraw[thick,color=AAUred!15] (a) .. controls (0.26,0.55) .. (c) .. controls (0.55,0.39) .. (b) -- (0.7,0.3) -- (0.1,0.3) -- (a);
\draw[very thick, color=AAUblue2] (a) .. controls (0.3,0.8) and (0.6,0.1) .. (b);  
%
%
\draw[thick](0.1,0.28) -- (0.1,0.32); % (a)
\draw[thick](0.7,0.28) -- (0.7,0.32); % (b)
%\draw[thick](0.4,0.28) -- (0.4,0.32); % (m)
%
% Punkt 
% -------------------------------------------------------
\filldraw [black] (a) circle (0.1pt);  
\filldraw [black] (b) circle (0.1pt); 
\filldraw [black] (c) circle (0.1pt); 
%
\node at (0.1,0.22) (){$a$};
\node at (0.7,0.22) (){$b$};
\node at (0.4,0.22) (){$\frac{a+b}{2}$};
%
% 
%
% Koordinatsystemet 
% -------------------------------------------------------
\draw[thick,->] (-0.05,0.3) -- (0.85,0.3) node[anchor=south east]{$x$};
\draw[thick,->] (0,0.2) -- (0,0.7) node[anchor=north west]{$y$};
%
%
\end{tikzpicture}
  \caption{Illustration af Simpsonsreglen.}
  \label{fig:simpsons}
\end{center}
\end{figure}
\end{frame}


\begin{frame}{Interpolerende kvadraturregler:Graden}
    Graden af præcision $m$ for en interpolerende kvadraturregel er det største $n \in \mathbb{Z^+}$, således at formlen er præcis for $x^k$, for alle $k = 0, 1, \ldots , n$, men ikke for $n+1$.
    Ved at betragte fejlleddet for hver regel kan graden af præcision udledes trivielt. 
    Midtpunktsreglen og trapezreglen har begge grad af præcision $m=1$, da der i fejlleddet differentieres $2$ gange, hvormed fejlleddene for ethvert førstegradspolynomium er $0$. 
    En tilsvarende analyse af fejlleddet i Simpsons regel viser, at den har graden af præcision $m=3$.
\end{frame}