%\begin{frame}
%\frametitle{Overskrift}
%\begin{itemize}
%\item Fisk
%\end{itemize}
%\end{frame}
%
% JULIE
\subsection{Runge-Kutta metoder}
\begin{frame}
\frametitle{Eulers metode } 
\begin{minipage}[b]{0.31\textwidth}
\begin{align*}
x_{n+1} & = x_n+h \\
y_{n+1} & = y_n + h f(x_n,y_n) \\
\phantom{x_1}\\
\\
\\
\end{align*}
\end{minipage}
\phantom{hejmeddig}
%
%
%%%%%%%%%%%%%%%%%%%%%%%%%%%%%%%%
%%% Flot graf alla Julie     %%%
%%%%%%%%%%%%%%%%%%%%%%%%%%%%%%%%
%
\begin{figure}[h!]
\begin{center}
\begin{tikzpicture}[scale=10]
%
% Koordinater 
% ------------------------------------------------------
\coordinate (a) at (0.1,0.3);
\coordinate (b) at (0.7,0.3);
\coordinate (k2) at (0.7,0.862);
%
% Funktion
% ------------------------------------------------------
\draw[very thick, color=AAUblue2] (a) .. controls (0.3,0.5) and (0.45,0.5) .. (b);   
%
% K1 
% ------------------------------------------------------
\draw[dashed, very thick, color=AAUred] (a) -- (0.74,0.9);
%
\draw[thick](0.1,0.18) -- (0.1,0.22); % (a)
\draw[thick](0.7,0.18) -- (0.7,0.22); % (b)
%
% Punkt 
% -------------------------------------------------------
\filldraw [black] (k2) circle (0.1pt); 
\filldraw [black] (a) circle (0.1pt);  
\filldraw [black] (b) circle (0.1pt); 
%
\node at (0.1,0.12) (){$x_0$};
\node at (0.7,0.12) (){$x_1$};
%
%\node at (0.55,0.3) (){$m_2$};
%\node at (0.475,0.3) (){$m_3$};
% 
%
% Koordinatsystemet 
% -------------------------------------------------------
\draw[thick,->] (-0.05,0.2) -- (0.85,0.2) node[anchor=south east]{$x$};
\draw[thick,->] (0,0.1) -- (0,1) node[anchor=north west]{$y$};
%
%
\end{tikzpicture}
  \caption{Illustration af et skridt i Eulers metode, hvor  tangentvektoren er markeret med rød.}
  \label{fig:euler}
\end{center}
\end{figure}
\\
\phantom{H}
\\
%
\end{frame}
%
%
%
%
\begin{frame}
\frametitle{Korrigeret Eulersmetode $\alpha = \frac{1}{2}$ } 
\begin{minipage}[t]{0.31\textwidth}
\begin{align*}
k_1 & = f(x_n , y_n) \\
k_2 & = f \left( x_n + \frac{h}{2} , y_n + \frac{h}{2} k_1 \right)  \\
\end{align*}
\end{minipage} 
\phantom{Hej}
\begin{minipage}[t]{0.31\textwidth}
\begin{align*}
x_{n+1} & = x_n+h \\
y_{n+1} & = y_n + h k_2 \\
\end{align*}
\end{minipage} 
\\
%
%
%%%%%%%%%%%%%%%%%%%%%%%%%%%%%%%%
%%% Flot graf alla Julie     %%%
%%%%%%%%%%%%%%%%%%%%%%%%%%%%%%%%
%
\begin{figure}[h!]
\begin{center}
\begin{minipage}[b]{0.45\textwidth}
\begin{tikzpicture}[scale=7.3]
%
% Koordinater 
% ------------------------------------------------------
\coordinate (a) at (0.1,0.3);
\coordinate (b) at (0.7,0.3);
\coordinate (k2) at (0.4,0.577);
%
% Funktion
% ------------------------------------------------------
\draw[very thick, color=AAUblue2] (a) .. controls (0.3,0.5) and (0.45,0.5) .. (b);   
%
% K1 
% ------------------------------------------------------
\draw[dashed, very thick, color=AAUred] (a) -- (0.47,0.64);
% K2
% ------------------------------------------------------
\draw[dashed, very thick, color=AAUgreen] (0.3,0.585) -- (0.5,0.568);

%
\draw[thick](0.1,0.18) -- (0.1,0.22); % (a)
\draw[thick](0.7,0.18) -- (0.7,0.22); % (b)
\draw[thick](0.4,0.18) -- (0.4,0.22); % (m_1)
%  \draw[thick](0.55,0.23) -- (0.55,0.27); % (m_2)
%  \draw[thick](0.475,0.23) -- (0.475,0.27); % (m_3)
%
% Punkt 
% -------------------------------------------------------
\filldraw [black] (k2) circle (0.1pt); 
\filldraw [black] (a) circle (0.1pt);  
\filldraw [black] (b) circle (0.1pt); 
%
\node at (0.1,0.12) (){$x_0$};
\node at (0.7,0.12) (){$x_1$};
\node at (0.39,0.12) (){$x_0 + \frac{h}{2}$};
%
%\node at (0.55,0.3) (){$m_2$};
%\node at (0.475,0.3) (){$m_3$};
% 
%
% Koordinatsystemet 
% -------------------------------------------------------
\draw[thick,->] (-0.05,0.2) -- (0.85,0.2) node[anchor=south east]{$x$};
\draw[thick,->] (0,0.1) -- (0,0.7) node[anchor=north west]{$y$};
%
%
\end{tikzpicture}
\end{minipage}
%
%
\begin{minipage}[b]{0.45\textwidth}
\begin{tikzpicture}[scale=7.3]
%
% Koordinater 
% ------------------------------------------------------
\coordinate (a) at (0.1,0.3);
\coordinate (b) at (0.7,0.3);
\coordinate (k2) at (0.4,0.577);
%
% Funktion
% ------------------------------------------------------
\draw[very thick, color=AAUblue2] (a) .. controls (0.3,0.5) and (0.45,0.5) .. (b);   
%
% K1 
% ------------------------------------------------------
%\draw[dashed, very thick, color=AAUred] (a) -- (0.47,0.64);
% K2
% ------------------------------------------------------
%\draw[dashed, very thick, color=AAUgreen] (0.3,0.585) -- (0.5,0.568);
%\draw[dashed, very thick, color=AAUgreen] (0.3,0.457) -- (0.5,0.44); 
\draw[dashed, very thick, color=AAUgreen] (a) -- (0.7,0.249); 

%
\draw[thick](0.1,0.18) -- (0.1,0.22); % (a)
\draw[thick](0.7,0.18) -- (0.7,0.22); % (b)
\draw[thick](0.4,0.18) -- (0.4,0.22); % (m_1)
%  \draw[thick](0.55,0.23) -- (0.55,0.27); % (m_2)
%  \draw[thick](0.475,0.23) -- (0.475,0.27); % (m_3)
%
% Punkt 
% -------------------------------------------------------
%\filldraw [black] (k2) circle (0.1pt); 
\filldraw [black] (a) circle (0.1pt);  
\filldraw [black] (b) circle (0.1pt); 
\filldraw [black] (0.7,0.249) circle (0.1pt); 
%
\node at (0.1,0.12) (){$x_0$};
\node at (0.7,0.12) (){$x_1$};
\node at (0.39,0.12) (){$x_0 + \frac{h}{2}$};
%
%\node at (0.55,0.3) (){$m_2$};
%\node at (0.475,0.3) (){$m_3$};
% 
%
% Koordinatsystemet 
% -------------------------------------------------------
\draw[thick,->] (-0.05,0.2) -- (0.85,0.2) node[anchor=south east]{$x$};
\draw[thick,->] (0,0.1) -- (0,0.7) node[anchor=north west]{$y$};
%
\end{tikzpicture}
\end{minipage}
  \caption{Illustration af et skridt i Korrigeret Eulers metode, hvor  hældningen $k_1$ er markeret med rød, og hældningen $k_2$ er markeret med grøn.}
  \label{fig:kori}
\end{center}
\end{figure}
\\
\phantom{H}
\\
%
\end{frame}
%
%
%
\begin{frame}
\frametitle{Modificeret Eulersmetode $\alpha = 1$}  
\begin{minipage}[t]{0.31\textwidth}
\begin{align*}
k_1 & = f(x_n , y_n) \\
k_2 & = f( x_n + h , y_n  + h k_1 ) \\
\end{align*}
%
\end{minipage} 
\phantom{Hej}
\begin{minipage}[t]{0.31\textwidth}
\begin{align*}
x_{n+1} & = x_n+h \\
y_{n+1} & = y_n + \frac{h}{2} (k_1 + k_2 ) \\
\end{align*}
%
\end{minipage} 
\\
%
%%%%%%%%%%%%%%%%%%%%%%%%%%%%%%%%
%%% Flot graf alla Julie     %%%
%%%%%%%%%%%%%%%%%%%%%%%%%%%%%%%%
%
{ \small
\begin{minipage}[b]{0.45\textwidth}
\begin{tikzpicture}[scale=4]
%
% Koordinater 
% ------------------------------------------------------
\coordinate (a) at (0.1,0.3);
\coordinate (b) at (0.7,0.3);
\coordinate (k2) at (0.7,0.862);
%
% Funktion
% ------------------------------------------------------
\draw[very thick, color=AAUblue2] (a) .. controls (0.3,0.5) and (0.45,0.5) .. (b);   
%
% K1 
% ------------------------------------------------------
\draw[dashed, very thick, color=AAUred] (a) -- (0.74,0.9);
% K2
% ------------------------------------------------------
\draw[dashed, very thick, color=AAUgreen] (0.65,0.9) -- (0.75,0.823);
%\draw[dashed, very thick, color=AAUgreen] (0.65,0.338) -- (0.75,0.261);
%
\draw[thick](0.1,0.18) -- (0.1,0.22); % (a)
\draw[thick](0.7,0.18) -- (0.7,0.22); % (b)
%\draw[thick](0.4,0.18) -- (0.4,0.22); % (m_1)
%  \draw[thick](0.55,0.23) -- (0.55,0.27); % (m_2)
%  \draw[thick](0.475,0.23) -- (0.475,0.27); % (m_3)
%
% Punkt 
% -------------------------------------------------------
\filldraw [black] (k2) circle (0.3pt); 
\filldraw [black] (a) circle (0.3pt);  
\filldraw [black] (b) circle (0.3pt); 
%
\node at (0.1,0.12) (){$x_0$};
\node at (0.7,0.12) (){$x_1$};
%\node at (0.39,0.12) (){$x_0 + \frac{h}{2}$};
%
%\node at (0.55,0.3) (){$m_2$};
%\node at (0.475,0.3) (){$m_3$};
% 
%
% Koordinatsystemet 
% -------------------------------------------------------
\draw[thick,->] (-0.05,0.2) -- (0.85,0.2) node[anchor=south east]{$x$};
\draw[thick,->] (0,0.1) -- (0,1) node[anchor=north west]{$y$};
%
%
\end{tikzpicture}
\end{minipage}
%
%
\begin{minipage}[b]{0.45\textwidth}
\begin{tikzpicture}[scale=4]
%
% Koordinater 
% ------------------------------------------------------
\coordinate (a) at (0.1,0.3);
\coordinate (b) at (0.7,0.3);
\coordinate (k2) at (0.7,0.369);
%
% Funktion
% ------------------------------------------------------
\draw[very thick, color=AAUblue2] (a) .. controls (0.3,0.5) and (0.45,0.5) .. (b);   
%
\draw[dashed, very thick, color=yellow] (a) -- (k2);
%
\draw[thick](0.1,0.18) -- (0.1,0.22); % (a)
\draw[thick](0.7,0.18) -- (0.7,0.22); % (b)
%\draw[thick](0.4,0.18) -- (0.4,0.22); % (m_1)
%  \draw[thick](0.55,0.23) -- (0.55,0.27); % (m_2)
%  \draw[thick](0.475,0.23) -- (0.475,0.27); % (m_3)
%
% Punkt 
% -------------------------------------------------------
\filldraw [black] (k2) circle (0.3pt); 
\filldraw [black] (a) circle (0.3pt);  
\filldraw [black] (b) circle (0.3pt); 
%
\node at (0.1,0.12) (){$x_0$};
\node at (0.7,0.12) (){$x_1$};
%\node at (0.39,0.12) (){$x_0 + \frac{h}{2}$};
%
%\node at (0.55,0.3) (){$m_2$};
%\node at (0.475,0.3) (){$m_3$};
% 
%
% Koordinatsystemet 
% -------------------------------------------------------
\draw[thick,->] (-0.05,0.2) -- (0.85,0.2) node[anchor=south east]{$x$};
\draw[thick,->] (0,0.1) -- (0,1) node[anchor=north west]{$y$};
%
% rød stiger 100 pr 0.1 
% grøn falder 77 pr 0.1 
% Gennemsnit: 100 + (-77) / 2 = 11.5 
% Gange med skridtlængen ( = 6)
% = 69
%
\end{tikzpicture}
\end{minipage}
}
\\
\phantom{H}
\\
%
\end{frame}
%
%
% MADS
\begin{frame}
\frametitle{Heuns metode $\alpha = \frac{2}{3}$ }
\begin{minipage}[t]{0.31\textwidth}
\begin{align*}
k_1 & = f(x_n , y_n) \\
k_2 & = f \left( x_n + \frac{2}{3} h , y_n  + \frac{2}{3} h k_1 \right) \\
\end{align*}
%
\end{minipage} 
\phantom{Hej}
\begin{minipage}[t]{0.31\textwidth}
\begin{align*}
x_{n+1} & = x_n+h \\
y_{n+1} & = y_n + \frac{h}{4} (k_1 + k_2 )  \\
\end{align*}
%
\end{minipage} 
\\
%
%%%%%%%%%%%%%%%%%%%%%%%%%%%%%%%%
%%% Flot graf alla Julie     %%%
%%%%%%%%%%%%%%%%%%%%%%%%%%%%%%%%
%
{ \small
\begin{minipage}[b]{0.45\textwidth}
\begin{tikzpicture}[scale=4]
%
% Koordinater 
% ------------------------------------------------------
\coordinate (a) at (0.1,0.3);
\coordinate (b) at (0.7,0.3);
\coordinate (k2) at (0.5,0.673);
%
% Funktion
% ------------------------------------------------------
\draw[very thick, color=AAUblue2] (a) .. controls (0.3,0.5) and (0.45,0.5) .. (b);   
%
% K1 
% ------------------------------------------------------
\draw[dashed, very thick, color=AAUred] (a) -- (0.57,0.74);

% K2
% ------------------------------------------------------
\draw[dashed, very thick, color=AAUgreen] (0.4,0.714) -- (0.6,0.631);
%\draw[dashed, very thick, color=AAUgreen] (0.4,0.465) -- (0.6,0.382);
% 83
%
\draw[thick](0.1,-0.02) -- (0.1,0.02); % (a)
\draw[thick](0.7,-0.02) -- (0.7,0.02); % (b)
\draw[thick](0.5,-0.02) -- (0.5,0.02); % (m_1)
%
%
% Punkt 
% -------------------------------------------------------
\filldraw [black] (k2) circle (0.3pt); 
\filldraw [black] (a) circle (0.3pt);  
\filldraw [black] (b) circle (0.3pt); 
%
\node at (0.1,-0.08) (){$x_0$};
\node at (0.7,-0.08) (){$x_1$};
\node at (0.51,-0.08) (){$x_0 + \frac{2}{3} h$};
%
%\node at (0.55,0.3) (){$m_2$};
%\node at (0.475,0.3) (){$m_3$};
% 
%
% Koordinatsystemet 
% -------------------------------------------------------
\draw[thick,->] (-0.05,0) -- (0.85,0) node[anchor=south east]{$x$};
\draw[thick,->] (0,-0.1) -- (0,0.8) node[anchor=north west]{$y$};
%
%
\end{tikzpicture}
\end{minipage}
%
%
\begin{minipage}[b]{0.45\textwidth}
\begin{tikzpicture}[scale=4]
%
% Koordinater 
% ------------------------------------------------------
\coordinate (a) at (0.1,0.3);
\coordinate (b) at (0.7,0.3);
\coordinate (k2) at (0.7,0.0765);
%
% Funktion
% ------------------------------------------------------
\draw[very thick, color=AAUblue2] (a) .. controls (0.3,0.5) and (0.45,0.5) .. (b);   
%
% K1 
% ------------------------------------------------------
%\draw[dashed, very thick, color=AAUred] (a) -- (0.57,0.74);

% K2
% ------------------------------------------------------
%\draw[dashed, very thick, color=AAUgreen] (0.4,0.714) -- (0.6,0.631);
%\draw[dashed, very thick, color=AAUgreen] (0.4,0.465) -- (0.6,0.382);
% 83
%
\draw[thick](0.1,-0.02) -- (0.1,0.02); % (a)
\draw[thick](0.7,-0.02) -- (0.7,0.02); % (b)
\draw[thick](0.5,-0.02) -- (0.5,0.02); % (m_1)
%  \draw[thick](0.55,0.23) -- (0.55,0.27); % (m_2)
%  \draw[thick](0.475,0.23) -- (0.475,0.27); % (m_3)
%
\draw[dashed, very thick, color=yellow] (a) -- (k2);
%
% Punkt 
% -------------------------------------------------------
\filldraw [black] (k2) circle (0.3pt); 
\filldraw [black] (a) circle (0.3pt);  
\filldraw [black] (b) circle (0.3pt); 
%
\node at (0.1,-0.08) (){$x_0$};
\node at (0.7,-0.08) (){$x_1$};
\node at (0.51,-0.08) (){$x_0 + \frac{2}{3} h$};
%
%\node at (0.55,0.3) (){$m_2$};
%\node at (0.475,0.3) (){$m_3$};
% 
%
% Koordinatsystemet 
% -------------------------------------------------------
\draw[thick,->] (-0.05,0) -- (0.85,0) node[anchor=south east]{$x$};
\draw[thick,->] (0,-0.1) -- (0,0.8) node[anchor=north west]{$y$};
%
% rød stiger 100 pr 0.1 
% grøn falder 81 pr 0.1 
% Gennemsnit: 100 + 3*(-83) / 4 = -37,25
% Gange med skridtlængen ( = 6)
% = -223.5
%
%
\end{tikzpicture}
\end{minipage}
}
\\
\phantom{H}
\\
\end{frame}