\begin{frame}{Opgave 2: Sammensat kvadraturregel}
    Forklar idéen i en sammensat kvadraturregel. Udled den sammensatte kvadraturregel
    $T_N$.
\end{frame}

\begin{frame}{Sammensat kvadraturregel}
    Ideen ved en sammensat kvadraturregel er, at man tager sit interval fra a til b og inddeler i ækvidistante inddelinger, hvor $x_0$ angiver a, og $x_n$ angiver b og så gælder der ydermere følgende
    \begin{align*}
    a = x_0 < x_1 < … x_{n-1} < x_n = b
    \end{align*}
    $h$ angiver længden intervallerne mellem fx $x_0$ og $x_1$ , og er givet ved $\frac{b-a}{n}$, som betyder, at intervallet derfor inddeles i $n$ såkaldte delintervaller eller underinddelinger. 
    Ideen er så for en sådanne sammensatte kvadraturregel, at der kan anvendes midtpunkt-, trapez- og Simpsons regel på hver af de mindre delintervaller fremfor hele intervallet fra a til b. 
\end{frame}


\begin{frame}{Sammensat kvadraturregel}
    For at udlede den sammensatte kvadraturregel $T_N$ for et integrale, hvor intervallet opdeles i underinddelinger beskrives i følgende
    Trapez-reglen er som sagt, som følger
    \begin{align*}
    T = \frac{h}{2}(f(a)+f(b)),
    \end{align*}
    og ved at omskrive trapez-reglen, så den gælder for to underinddelinger samt indføre notationen for de tilsvarende venstre og højre Riemann summer $h(f(a)+f(a+h)$ og $h(f(a+h)+f(b))$, så får man følgende 
    \begin{align*}
    T_2=\frac{h}{2}(f(a)+f(a+h))+\frac{h}{2}(f(a+h)+f(b)= \frac{h}{2}(f(a)+2f(a+h)+f(b),
    \end{align*}
    hvor $h=\frac{b-a}{2}$.
\end{frame}


\begin{frame}{Sammensat kvadraturregel}
    Bemærk her, at man har evalueringer i det to endepunkter a og b, samt 2 gange evaluering i de indre punkter. Dette er givet, da man benytter trapez-reglen fra intervallet $x_0$ til $x_1$ samt fra $x_1$ til $x_2$, som vil sige, at man har en evaluering i punktet $x_1$ to gange, og derfor divideres med $\frac{1}{2}$. Derved er det videre muligt, at opskrive den generelle formel for den sammensatte kvadraturregel $T_n$ givet ved
    \begin{align*}
    T_n = \frac{1}{2}\left (  h\sum_{k=0}^{N-1}f(a+kh)+h\sum_{k=1}^{N}f(a+kh)\right )
    \end{align*}
    som kan reduceres og omskrives til 
    \begin{align*}
    T_n =\frac{h}{2}\left ((f(a)+2\sum_{k=1}^{N-1}f(a+kh)+f(b) \right )
    \end{align*}
\end{frame}

%%%%