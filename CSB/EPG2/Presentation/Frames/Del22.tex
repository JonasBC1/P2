% JONAS
\begin{frame}
\frametitle{Runge-Kutta metode}
RK4:\\
\begin{align*}
k_1&=f(x_n,y_n)\\
k_2&=f(x_n+\frac{h}{2},y_n+\frac{h}{2}k_1)\\
k_3&=f(x_n+\frac{h}{2},y_n+\frac{h}{2}k_2)\\
k_4&=f(x_n+h,y_n+h k_3)\\
x_{n+1}&=x_n+h\\
y_{n+1}&=y_n+\frac{h}{6}(k_1+2k_2+2k_3+k_4)
\end{align*}
\end{frame}

\begin{frame}
\frametitle{Simpsons regel}
%Hvis funktionen afhænger af en koordinat så svarer $k_1$ til startpunktet, $k_4$ til endepunktet og $k_2$ og $k_3$ bliver i midtpunktet
\begin{align*}
\int_a^bf(x)dx &\approx \frac{b-a}{6} \left( f(a)+4f \left( \frac{a+b}{2} \right) f(a+b)+f(b) \right)\\
h&=b-a\\
y_0+\int_a^bf(x)dx &\approx y_0+ \frac{h}{6} \left( f(a)+4f \left( \frac{h}{2} \right) f(h)+f(b) \right)\\
y_{n+1}&=y_0+(k_1+2k_2+2k_3+k_4)
\end{align*}
\end{frame}

