\subsection*{(b) }
% 
%
\begin{color}{AAUblue2}
%
fisk
% 
\end{color}
\\\\
% 
%