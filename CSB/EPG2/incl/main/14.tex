\section*{4. Adaptiv metode}
% 
%
\begin{color}{AAUblue2}
%
\textbf{Opgavebeskrivelse:} 
Vi skal nu se på en anvendelse. Vi definerer en funktion ved
%\begin{align}
%f(x)=\left \{ \substack{-(x-\sqrt{2})^2 \text{  } \text{for} \text{  } x \leq \sqrt{2} \\ (x-\sqrt{2})^2\text{  } \text{for} \text{  } x > \sqrt{2}}
%\end{align}
%
%Noget galt i ovenstående
%
Vi vil se på approksimation af integralet 
\begin{align}
\int_{0}^{3}f(x)dx    
\end{align}
\\
(a) Lav en Python funktion til beregning af funktionen defineret i (1).
% 
\end{color}
\\\\
% 
\textbf{Besvarelse:} \\\\
(b) Vis, at den eksakte værdi af integralet er $15-\frac{31\sqrt{2}}{3}$\\
\textbf{Besvarelse:} \\ 
\begin{align*}
\int^{\sqrt{2}}_0-(x-\sqrt{2})^2+\int^3_{\sqrt{2}}(x-\sqrt{2})^2=15-\frac{29\sqrt{2}}{3}-\frac{2^{\frac{3}{2}}}{3}=15-\frac{31\sqrt{2}}{3} 
\end{align*} \\
Ved at tage integralet for begge funktioner, og derefter plusse dem sammen, fåes den eksakte værdi $15-\frac{31\sqrt{2}}{3}$\\\\
(c) Bestem integralet numerisk ved hjælp af de tre sammensatte kvadraturregler
(midtpunkt, trapez og Simpson). Hvor mange ækvidistante inddelinger skal der
anvendes i hvert af de tre tilfælde for at opnå en fejl på mindre end $10^{-8}$?\\
Ved midtpunkt er fejlen mindre end $10^{-8}$ ved cirka $4096$ indelinger.\\
Ved trapez sker dette først ved cirka $8192$, altså senere end ved midtpunkt.\\ 
Ved simpsonreglen sker dette ved ca $512$ og denne opnår dermed højere præcision ved færre iterationer end de foregående to.
\textbf{Besvarelse:} \\
(d) Lav i de tre tilfælde numeriske eksperimenter, baseret på fordobling af antal delepunkter, og brug dem til en eksperimentel bestemmelse af ordenen af de tre
metoder. Stemmer resultaterne med teorien? Hvis der er afvigelser, hvordan kan
de så forklares?\\
\textbf{spørgsmål}:Hvorfor springer orden så meget i Simpson efter 8192? \\
% Igen noget med fejlformler, og defirentier den funktion vi arbejder med, i forhold til fejlformlerne. Derefter kan vi konkludere hvorfor det begynder at afvige. --
\textbf{Besvarelse:}For både trapez samt midtpunktreglen stabliseres der omkring en orden på 4 efter et passende antal inddelinger hvilket stemmer overens med teorien, dette er dog ikke tilfældet for Simpson. Den opnår de 16, men falder drastisk efterfølgende \\\\
(e) Prøv at anvende en simpel adaptiv metode til at bestemme integralet (2). Hvad
sker der hvis man inddeler i to delintervaller af samme længde. Hvordan bidrager de hver til fejlen i approksimationen? 
Angiv resultaterne for både trapezreglen og Simpsons regel. \\
\textbf{spørgsmål}: Vi har svært ved at skulle dele i to delintervaller og alt det Python-fis
% Intervallet deles til 0-1.5 og 1.5-3 og husk at modificere Iexact, så skulle vi kunne konkludere fejlen. 
\textbf{Besvarelse:} \\
Integralet af funktionen fra 1.5 til 3 er præcist idet antallet af indelinger er ligegyldigt da det er et tredjegradspolynomium og funktionen forbliver stabil i hele intervallet, derimod sker der fra 0 til 1.5 det at funktionen skifter og derfor behov for et større antal indelinger og det biddrager hermed til at denne del af aproksimationen giver en større fejl med behov for et større antal indelinger.
For Trapez: her biddrager begge intervaller lige meget til fejlen og kræver samme antal underindelinger(512 hvis mit bud om faktor 3 er korrekt) dette skyldes at præcisionen er mindre her hvorfor approskismationen ikke er lig funktionsværdien i intervallet fra 1.5 til 3 heller
%%%%%%%%% DONZO GONZO %%%%%%%%%%
%
