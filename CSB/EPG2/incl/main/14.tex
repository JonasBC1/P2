\section*{4. Adaptiv metode}
% 
%
\begin{color}{AAUblue2}
%
\textbf{Opgavebeskrivelse:} 
Vi skal nu se på en anvendelse. Vi definerer en funktion ved
%\begin{align}
%f(x)=\left \{ \substack{-(x-\sqrt{2})^2 \text{  } \text{for} \text{  } x \leq \sqrt{2} \\ (x-\sqrt{2})^2\text{  } \text{for} \text{  } x > \sqrt{2}}
%\end{align}
%
%Noget galt i ovenstående
%
Vi vil se på approksimation af integralet 
\begin{align}
\int_{0}^{3}f(x)dx    
\end{align}
\\
(a) Lav en Python funktion til beregning af funktionen defineret i (1).
% 
\end{color}
\\\\
% 
\textbf{Besvarelse:} \\\\
(b) Vis, at den eksakte værdi af integralet er $15-\frac{31\sqrt{2}}{3}$\\
\textbf{Besvarelse:} \\ 
\begin{align*}
\int^{\sqrt{2}}_0-(x-\sqrt{2})^2+\int^3_{\sqrt{2}}(x-\sqrt{2})^2=15-\frac{29\sqrt{2}}{3}-\frac{2^{\frac{3}{2}}}{3}=15-\frac{31\sqrt{2}}{3} 
\end{align*} \\
Ved at tage integralet for begge funktioner, og derefter plusse dem sammen, fåes den eksakte værdi $15-\frac{31\sqrt{2}}{3}$\\\\
(c) Bestem integralet numerisk ved hjælp af de tre sammensatte kvadraturregler
(midtpunkt, trapez og Simpson). Hvor mange ækvidistante inddelinger skal der
anvendes i hvert af de tre tilfælde for at opnå en fejl på mindre end $10^{-8}$?\\
Ved midtpunkt er fejlen mindre end $10^{-8}$ ved cirka $4096$ indelinger.\\
ved trapez sker dette først ved cirka 8192, altså senere end ved midtpunkt.\\ 
Ved Simpsons regel sker dette ved ca $512$ og denne opnår dermed højere præcision ved færre iterationer end de foregående to.
%(Har i andre forslag til hvor mange punkter og iterationer der skal med frem for fordoblingen
\textbf{spørgsmål}:Hvad menes der med numerisk?\\
\textbf{Besvarelse:} \\
(d) Lav i de tre tilfælde numeriske eksperimenter, baseret på fordobling af antal delepunkter, og brug dem til en eksperimentel bestemmelse af ordenen af de tre
metoder. Stemmer resultaterne med teorien? Hvis der er afvigelser, hvordan kan
de så forklares?\\
\textbf{spørgsmål}:Hvorfor springer orden så meget i Simpson efter 8192? \\
\textbf{Besvarelse:}For både trapez samt midtpunktreglen stabliseres der omkring en orden på 4 efter et passende antal indellinger hvilket stemmer overens med teorien, dette er dog ikke tilfældet for Simpson. Den opnår de 16, men falder drastisk efterfølgende \\\\
(e) Prøv at anvende en simpel adaptiv metode til at bestemme integralet (2). Hvad
sker der hvis man inddeler i to delintervaller af samme længde. Hvordan bidrager
de hver til fejlen i approksimationen? Angiv resultaterne for både trapezreglen og
Simpsons regel. \\
\textbf{spørgsmål}: Vi har svært ved at skulle dele i to delintervaller og alt det Python-fis
\textbf{Besvarelse:} \\
%%%%%%%%% DONZO GONZO %%%%%%%%%%
%
