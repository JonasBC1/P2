\section*{3. Numerisk løsning af systemer af differentialligninger}
% 
%
\begin{color}{AAUblue2}
%
%DEN HER GIDER JEG I HVERT FALDE IKKE OG SKRIVE IND XD SÅÅÅÅ EN ELLER ANDDEN SOM GODT KUNNE LAVE ANDET END AT SPILLE COMPUTERSPIL FÅR ÆREN AF DETTE XD HEHEHEHEHEHEHE
%\\\\
%meh 
%\\\\
RK4 metoden og de andre metoder kan anvendes til numerisk løsning af systemer af differentialligninger. 
Følgende tager udgangspunkt i et konkret eksempel. Der er givet et system af differentialligninger, som anvendes til modellering af en ufarlig infektion i population. 
Populationen består af $N$ individer og antages konstant i tid. Den opdeles i tre kategorier. $x_1(t)$ er antallet af individer
der er modtagelige for smitte på tidspunktet $t$. $x_2(t)$ er antallet af smittede individer.
Disse kan smitte de individer der er modtagelige. 
$x_3(t)$ betegner individer der enten er immune, eller som er blevet immune efter at have været smittet.
%
Modellen er beskrevet ved følgende tre differentialligninger: 
%
\begin{align}
\frac{dx_1}{dt} & = - \alpha x_1 x_2, \\
\frac{dx_2}{dt} & = \alpha x_1 x_2 - \beta x_2 , \\
\frac{dx_3}{dt} & = \beta x_2.
\end{align}
%
Vi kan skrive systemet og begyndelsesbetingelsen på vektorform som
%
\begin{align*}
\frac{d}{dt} 
\begin{bmatrix}
x_1 (t) \\
x_2 (t) 
\end{bmatrix} = 
\begin{bmatrix}
- \alpha x_1 (t) x_2 (t) \\
\alpha x_1 (t) x_2 (t) - \beta x_2 (t) 
\end{bmatrix}
,
\begin{bmatrix}
x_1 (0) \\
x_2 (0) 
\end{bmatrix}
= 
\begin{bmatrix}
x_{10} \\
x_{20}
\end{bmatrix}
\end{align*}
%
Parameteren $\alpha$ beskriver den rate hvormed de modtagelige individer bliver smittet.
Parameteren $\beta$ beskriver den rate hvormed smittede individer overstår infektionen. 
% 
\end{color}
%
\subsection*{(a) }
%
%
\begin{color}{AAUblue2}
%
Implementér Euler metoden og Runge-Kutta RK4 metoden for vektorfunktioner. 
% 
\end{color}
\\\\
% 
Se \textsc{system-example.py}.
%
\subsection*{(b) }
% 
%
\begin{color}{AAUblue2}
%
fisk
% 
\end{color}
\\\\
% 
%
Stigning i $\alpha$. 
Flere smittede - Stiger kurven 
Hvor hurtigt folk der kan smittes bliver smittes 

Stigning i $\beta$. 
Så der ikke så mange syge 
Kurven til at falde 
Hvor hurtigt folk bliver raske igen

Ændring i gamma 
Gamme = vaccine 
Sort falder hurtigere 
og den røde falder hurtigere
Hvor mange der er syge falder og hvor mange der kan blive syge falder 


Begyndelsesværdi: 

$x_2$ antal syge 
Stigning så topper kurven hurtigere
 
$x_1$ Kan blive syge 
Fald så bliver kurven ikke så høj 

%
\newpage
\subsection*{(c)}
%
\begin{color}{AAUblue2}
%
Modificer \textsc{opg2-system.py} til et system hvor $\gamma$ kan ændres med udgangspunkt i at modelles første ser ud som følgende 
\begin{align}
\frac{dx_1}{dt} &= - \alpha x_1 x_2 - \gamma x_1, \\
\frac{dx_2}{dt} & = \alpha x_1 x_2 - \beta x_2 , \\
\frac{dx_3}{dt} & = \beta x_2 + \gamma x_1.
\end{align}
%
Beskriv hvad der sker med en ændring i gamma.  
%
\end{color}
\\\\
% 
Se \textsc{system-example-modi.py}. Implementeringerne er ved
\textbf{\textit{def f(t, x)}}, linje $60$.
\\\\
%
Introduceres en vaccine, hvor $\gamma$ beskriver, hvor hurtigt der kan vaccineres, vil dette have en effekt på smittespredningen.
Værdien gamma forøger dermed, hvor hurtigt befolkningen bliver immun uden at have været smittet. Jo højere værdi af gamma jo hurtigere bliver befolkningen immun.
Først ses på udgangspunktet, dog med vaccine.
Dette ses på figur \ref{fig:a1_b35_g5}.
%
\begin{figure}[!ht]
\centering
$
\begin{matrix}
\includegraphics[scale=0.5]{fig/img/a1_b35_g5.png}&
\includegraphics[scale=0.5]{fig/img/t_a1_b35_g5.png}
\end{matrix}
$
\caption{Smitten når $\alpha = 0.01$, $\beta = 0.35$ og $\gamma = 0.05$.}
\label{fig:a1_b35_g5}
\end{figure}
\\\\
%
Her ses det, at vaccinen ikke har effekt på hvornår smitten topper, men reducere toppen.
Naturligvis øger det også raten hvormed folk bliver immune.
%
\begin{figure}[!ht]
\centering
$
\begin{matrix}
\includegraphics[scale=0.5]{fig/img/a2_b35_g5.png}&
\includegraphics[scale=0.5]{fig/img/t_a2_b35_g5.png}
\end{matrix}
$
\caption{Smitten når $\alpha = 0.02$, $\beta = 0.35$ og $\gamma = 0.05$.}
\label{fig:a2_b35_g5}
\end{figure}
\\\\
%
Det samme lader til at være tilfældet selv når $\alpha$ er blevet fordoblet, hvilket kan ses på figur \ref{fig:a2_b35_g5}.
Det kan ses på figur \ref{fig:a1_b7_g5}, at når $\beta$ er fordoblet og der er tilføjet en vaccine er der dog ikke en betydelig forskel i antallet af smittede, men tilgengæld flatliner antallet af immune ikke med samme rate.
%
\begin{figure}[!ht]
\centering
$
\begin{matrix}
\includegraphics[scale=0.5]{fig/img/a1_b7_g5.png}&
\includegraphics[scale=0.5]{fig/img/t_a1_b7_g5.png}
\end{matrix}
$
\caption{Smitten når $\alpha = 0.01$, $\beta = 0.7$ og $\gamma = 0.05$.}
\label{fig:a1_b7_g5}
\end{figure}
\\\\
%
Dermed falder det maksimale antal, der er smittet på en gang, og hvis gamma ikke er nul så er antallet, som bliver inficeret, lavere.
\subsection*{(d) }
%
\begin{color}{AAUblue2}
fisk
\end{color}
\\\\
% 
