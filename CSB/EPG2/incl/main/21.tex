\section*{1. Løsning af en første ordens differentialligning}
% 
%
\begin{color}{AAUblue2}
%
Beskriv et trin i de fire metoder 
% 
\begin{itemize}
\item Eulers metode 
\item Korrigeret Euler metode 
\item Modificeret Euler metode 
\item Heuns metode 
\end{itemize}
% 
til numerisk løsning af en første ordens differentialligning. 
Lav tegninger, der angiver valgene, som man har gjort for at få metoderne.  
\\\\
Forklar, at den modificerede Euler metode anvendt på problemet 
%
\begin{align*}
\frac{dy}{dx} = f(x), \phantom{heeej} y(a)= y_0,
\end{align*}
% 
til approksimation af $y(b)$, er det samme som trapezmetoden anvendt til at approksimere 
%
\begin{align*}
y_0 + fisk jeg gider ikke skrive dette ind :P
\end{align*}
% 
\end{color}
% 
% 
%
\subsection*{Eulers metode}

%
\begin{align*}
x_{n+1} & = x_n+h \\
y_{n+1} & = y_n + h f(x_n,y_n)
\end{align*}
% 

%
%%%%%%%%%%%%%%%%%%%%%%%%%%%%%%%%
%%% Flot graf alla Julie     %%%
%%%%%%%%%%%%%%%%%%%%%%%%%%%%%%%%
%
\begin{center}
\begin{tikzpicture}[scale=12]
%
% Koordinater 
% ------------------------------------------------------
\coordinate (a) at (0.1,0.35);
\coordinate (b) at (0.4,0.3);
\coordinate (c) at (0.7,0.1);
%
%        
% Streger 
% -------------------------------------------------------
  \draw[densely dotted](0.1,0.25) -- (a);
  \draw[densely dotted](0.4,0.25) -- (b);
  \draw[densely dotted](0.55,0.23) -- (0.55,0.15);
  \draw[densely dotted](0.475,0.23) -- (0.475,0.21);
  \draw[very thick, color=AAUblue2](a) parabola (b);
 % \draw[very thick, color=AAUblue2](a) parabola[bend at end] (b);
  \draw[thick](0.1,0.23) -- (0.1,0.27); % (a)
  \draw[thick](0.7,0.23) -- (0.7,0.27); % (b)
  \draw[thick](0.4,0.23) -- (0.4,0.27); % (m_1)
  \draw[thick](0.55,0.23) -- (0.55,0.27); % (m_2)
  \draw[thick](0.475,0.23) -- (0.475,0.27); % (m_3)
%
%
% Punkt 
% -------------------------------------------------------
\node at (0.1,0.2) (){$a$};
\node at (0.7,0.3) (){$b$};
\node at (0.4,0.2) (){$m_1$};
\node at (0.55,0.3) (){$m_2$};
\node at (0.475,0.3) (){$m_3$};
% 
%
% Koordinatsystemet 
% -------------------------------------------------------
\draw[thick,->] (-0.1,0.25) -- (0.8,0.25) node[anchor=south east]{$x$};
\draw[thick,->] (0,0.1) -- (0,0.6) node[anchor=north west]{$y$};
%
\end{tikzpicture}
  \captionof{figure}{Illustration af bisektionsmetoden, hvor $m_1=\frac{a+b}{2}$, $m_2=\frac{m_1+b}{2}$, $m_3=\frac{m_1+m_2}{2}$.}
  \label{fig:bis}
\end{center}
%



%
\subsection*{Korrigeret Eulers metode}
% 
først findes hældningen i begyndelsespunktet $x_0$
så følger man langs tangentvektoren til et halt Eulerskridt $\frac{h}{2}$
Her evalueres funktionen i punktet, hvor y dog er $ y_n = + \frac{h}{2} k_1$
og  hældningen $k_2$ findes 
så starter man i begyndelsespunktet $x_0$ og følger langs tangenvektoren med hældning $k_2$ hen til et fuldt Eulerskridt



Et skridt i den korrigerede Eulers metode bruger hældningen $k_1$ i begyndelsespunktet $x_0$ og hældningen $k_2$ i punktet $x_0 + \frac{h}{2}$, hvor $h$ er skridtlængen mellem $x_0$ og $x_1$. Punktet $x_1$ findes nu ved at følge hældningen $k_2$
%
\begin{align*}
k_1 & = f(x_n , y_n) \\
k_2 & = f(x_n + \frac{h}{2} , y_n = + \frac{h}{2} k_1 )  \\
x_{n+1} & = x_n+h \\
y_{n+1} & = y_n + h k_2 
\end{align*}
% 




%
\subsection*{Modificeret Eulers metode}

Først findes hældningen i begyndelsespunktet $x_0$ 
så følger man langs tangetvektoren til et helt Eulerskridt
Her evalueres funktionen i punktet, hvor y dog er $y_n = + h k_1$ og hældningen $k_2$ findes 
Så findes gennemsnittet mellem $k_1 + k_2$ og følger langs denne tangentvektor til et fuldt eulerskridt 


%
\begin{align*}
k_1 & = f(x_n , y_n) \\
k_2 & = f(x_n + h , y_n = + h k_1 ) \\
x_{n+1} & = x_n+h \\
y_{n+1} & = y_n + \frac{h}{2} (k_1 + k_2 )
\end{align*}
% 


\subsection*{Heuns metode}
Først findes hældningen i begyndelsespunktet $x_0$ 
så følger man langt tangentvektoren til en $\frac{2}{3}$ Eulerskridt 
Her evalueres funktionen i puntket, hvor $y$ dog er $y_n = + \frac{2}{3} h k_1$ og hældningen $k_2$ findes 
Så findes en ny hældning med udgangspunkt i 
$\frac{h}{4} (k_1 + k_2 ) $ 

%
\begin{align*}
k_1 & = f(x_n , y_n) \\
k_2 & = f(x_n + \frac{2}{3} h , y_n = + \frac{2}{3} h k_1 ) \\
x_{n+1} & = x_n+h \\
y_{n+1} & = y_n + \frac{h}{4} (k_1 + k_2 )
\end{align*}


