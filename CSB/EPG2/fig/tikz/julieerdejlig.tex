%
%%%%%%%%%%%%%%%%%%%%%%%%%%%%%%%%
%%% Flot graf alla Julie     %%%
%%%%%%%%%%%%%%%%%%%%%%%%%%%%%%%%
%
\begin{figure}[h!]
\begin{center}
\begin{tikzpicture}[scale=10]
%
% Koordinater 
% ------------------------------------------------------
\coordinate (a) at (0.1,0.5);
\coordinate (b) at (0.7,0.5);
\coordinate (c) at (0.4,0.463);
%
% Funktion
% ------------------------------------------------------
\draw[very thick, color=AAUblue2] (a) .. controls (0.3,0.8) and (0.5,0.1) .. (b);    
\filldraw[thick,color=AAUred!15] (0.1,0.463) -- (0.7,0.463)-- (0.7,0.3) -- (0.1,0.3) -- (0.1,0.463);
%
%
\draw[thick](0.1,0.28) -- (0.1,0.32); % (a)
\draw[thick](0.7,0.28) -- (0.7,0.32); % (b)
\draw[thick](0.4,0.28) -- (0.4,0.32); % (m_1)
%
% Punkt 
% -------------------------------------------------------
\filldraw [black] (a) circle (0.1pt);  
\filldraw [black] (b) circle (0.1pt); 
\filldraw [black] (c) circle (0.1pt); 
%
\node at (0.1,0.22) (){$a$};
\node at (0.7,0.22) (){$b$};
\node at (0.4,0.22) (){$\frac{a+b}{2}$};
%
% 
%
% Koordinatsystemet 
% -------------------------------------------------------
\draw[thick,->] (-0.05,0.3) -- (0.85,0.3) node[anchor=south east]{$x$};
\draw[thick,->] (0,0.2) -- (0,0.7) node[anchor=north west]{$y$};
%
%
\end{tikzpicture}
  \caption{Illustration af Midtpunktsreglen.}
  \label{fig:midtpunkt}
\end{center}
\end{figure}





%
%%%%%%%%%%%%%%%%%%%%%%%%%%%%%%%%
%%% Flot graf alla Julie     %%%
%%%%%%%%%%%%%%%%%%%%%%%%%%%%%%%%
%
\begin{figure}[h!]
\begin{center}
\begin{tikzpicture}[scale=10]
%
% Koordinater 
% ------------------------------------------------------
\coordinate (a) at (0.1,0.3);
\coordinate (b) at (0.7,0.3);
\coordinate (k2) at (0.7,0.862);
%
% Funktion
% ------------------------------------------------------
\draw[very thick, color=AAUblue2] (a) .. controls (0.3,0.5) and (0.45,0.5) .. (b);   
%
% K1 
% ------------------------------------------------------
\draw[dashed, very thick, color=AAUred] (a) -- (0.74,0.9);
%
\draw[thick](0.1,0.18) -- (0.1,0.22); % (a)
\draw[thick](0.7,0.18) -- (0.7,0.22); % (b)
%
% Punkt 
% -------------------------------------------------------
\filldraw [black] (k2) circle (0.1pt); 
\filldraw [black] (a) circle (0.1pt);  
\filldraw [black] (b) circle (0.1pt); 
%
\node at (0.1,0.12) (){$x_0$};
\node at (0.7,0.12) (){$x_1$};
\node at (0.39,0.12) (){$x_0 + \frac{h}{2}$};
%
%\node at (0.55,0.3) (){$m_2$};
%\node at (0.475,0.3) (){$m_3$};
% 
%
% Koordinatsystemet 
% -------------------------------------------------------
\draw[thick,->] (-0.05,0.2) -- (0.85,0.2) node[anchor=south east]{$x$};
\draw[thick,->] (0,0.1) -- (0,1) node[anchor=north west]{$y$};
%
%
\end{tikzpicture}
  \caption{Illustration af et skridt i Eulers metode, hvor  tangentvektoren er markeret med rød.}
  \label{fig:midtpunkt}
\end{center}
\end{figure}