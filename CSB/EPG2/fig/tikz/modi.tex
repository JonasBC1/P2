%
%%%%%%%%%%%%%%%%%%%%%%%%%%%%%%%%
%%% Flot graf alla Julie     %%%
%%%%%%%%%%%%%%%%%%%%%%%%%%%%%%%%
%
\begin{figure}[h!]
\begin{center}
\begin{minipage}[b]{0.45\textwidth}
\begin{tikzpicture}[scale=7.3]
%
% Koordinater 
% ------------------------------------------------------
\coordinate (a) at (0.1,0.3);
\coordinate (b) at (0.7,0.3);
\coordinate (k2) at (0.7,0.862);
%
% Funktion
% ------------------------------------------------------
\draw[very thick, color=AAUblue2] (a) .. controls (0.3,0.5) and (0.45,0.5) .. (b);   
%
% K1 
% ------------------------------------------------------
\draw[dashed, very thick, color=AAUred] (a) -- (0.74,0.9);
% K2
% ------------------------------------------------------
\draw[dashed, very thick, color=AAUgreen] (0.65,0.9) -- (0.75,0.823);
%\draw[dashed, very thick, color=AAUgreen] (0.65,0.338) -- (0.75,0.261);
%
\draw[thick](0.1,0.18) -- (0.1,0.22); % (a)
\draw[thick](0.7,0.18) -- (0.7,0.22); % (b)
%\draw[thick](0.4,0.18) -- (0.4,0.22); % (m_1)
%  \draw[thick](0.55,0.23) -- (0.55,0.27); % (m_2)
%  \draw[thick](0.475,0.23) -- (0.475,0.27); % (m_3)
%
% Punkt 
% -------------------------------------------------------
\filldraw [black] (k2) circle (0.1pt); 
\filldraw [black] (a) circle (0.1pt);  
\filldraw [black] (b) circle (0.1pt); 
%
\node at (0.1,0.12) (){$x_0$};
\node at (0.7,0.12) (){$x_1$};
%\node at (0.39,0.12) (){$x_0 + \frac{h}{2}$};
%
%\node at (0.55,0.3) (){$m_2$};
%\node at (0.475,0.3) (){$m_3$};
% 
%
% Koordinatsystemet 
% -------------------------------------------------------
\draw[thick,->] (-0.05,0.2) -- (0.85,0.2) node[anchor=south east]{$x$};
\draw[thick,->] (0,0.1) -- (0,1) node[anchor=north west]{$y$};
%
%
\end{tikzpicture}
\end{minipage}
%
%
\begin{minipage}[b]{0.45\textwidth}
\begin{tikzpicture}[scale=7.3]
%
% Koordinater 
% ------------------------------------------------------
\coordinate (a) at (0.1,0.3);
\coordinate (b) at (0.7,0.3);
\coordinate (k2) at (0.7,0.369);
%
% Funktion
% ------------------------------------------------------
\draw[very thick, color=AAUblue2] (a) .. controls (0.3,0.5) and (0.45,0.5) .. (b);   
%
\draw[dashed, very thick, color=yellow] (a) -- (k2);
%
\draw[thick](0.1,0.18) -- (0.1,0.22); % (a)
\draw[thick](0.7,0.18) -- (0.7,0.22); % (b)
%\draw[thick](0.4,0.18) -- (0.4,0.22); % (m_1)
%  \draw[thick](0.55,0.23) -- (0.55,0.27); % (m_2)
%  \draw[thick](0.475,0.23) -- (0.475,0.27); % (m_3)
%
% Punkt 
% -------------------------------------------------------
\filldraw [black] (k2) circle (0.1pt); 
\filldraw [black] (a) circle (0.1pt);  
\filldraw [black] (b) circle (0.1pt); 
%
\node at (0.1,0.12) (){$x_0$};
\node at (0.7,0.12) (){$x_1$};
%\node at (0.39,0.12) (){$x_0 + \frac{h}{2}$};
%
%\node at (0.55,0.3) (){$m_2$};
%\node at (0.475,0.3) (){$m_3$};
% 
%
% Koordinatsystemet 
% -------------------------------------------------------
\draw[thick,->] (-0.05,0.2) -- (0.85,0.2) node[anchor=south east]{$x$};
\draw[thick,->] (0,0.1) -- (0,1) node[anchor=north west]{$y$};
%
% rød stiger 100 pr 0.1 
% grøn falder 77 pr 0.1 
% Gennemsnit: 100 + (-77) / 2 = 11.5 
% Gange med skridtlængen ( = 6)
% = 69
%
\end{tikzpicture}
\end{minipage}
  \caption{Illustration af et skridt i Modificeret Eulers metode, hvor hældningen $k_1$ er markeret med rød, og hældningen $k_2$ er markeret med grøn. Gennemsnittet mellem $k_1$ og $k_2$ er markeret med gul.}
  \label{fig:modi}
\end{center}
\end{figure}