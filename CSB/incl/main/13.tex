\section*{3. Nulpunktbestemmelse}
% Gennemgå bisektionsmetoden, Newtons metode og sekantmetoden til bestemmelse af nulpunkt af en funktion. 
% Gennemgå iterationsmetoden til bestemmelse af x-punkt for en funktion.
%
Følgende metoder bestemmer nulpunktet for en funktion $f$.
For $f:I\rightarrow\R$ kontinuert og $f(a)f(b)<0$, hvor $I=[a,b]$, er der jævnfør mellemværdi-sætningen mindst en løsning til $f(s)=0$.
% 
%%%%%%%%%%%%%%%%%%%%%%%%%%%%%%%%%%%%%%%%%%%%%%%%%%%%%%%%%%%%%%%%
%%%%%%%%%%%%%%%%%%%%%%%%%%%%%%%%%%%%%%%%%%%%%%%%%%%%%%%%%%%%%%%%
%%%%%%%%%%%%%%%%%%%%%%%%%%%%%%%%%%%%%%%%%%%%%%%%%%%%%%%%%%%%%%%%
%%%%%%%%%%%%%%%%%%%%%%%%%%%%%%%%%%%%%%%%%%%%%%%%%%%%%%%%%%%%%%%%
%
\subsection*{Bisektionsmetoden}
\textbf{Bisektionsmetoden} bestemmer midtpunktet $m=\frac{a+b}{2}$ og sætter $b=m$, hvis $f(a)f(m)<0$, ellers sættes $a=m$. Denne procedure gentages indtil indervallet $b-a$ er mindre en en given tolerance. % Fx 10^-4
Herefter bestemmes midtpunktet for dette interval og sættes lig $s$, for $f(s)=0$.
På figur \ref{fig:bis} ses de første $3$ midtpunkter og funktionen $f(x)$.
Her ses det at $m_1$ sættes lig $a$, hvorefter $m_2$ sættes lig $b$, og $m_3$ sættes lig $b$.
%
%%%%%%%%%%%%%%%%%%%%%%%%%%%%%%%%
%%% Flot graf alla Julie     %%%
%%%%%%%%%%%%%%%%%%%%%%%%%%%%%%%%
%
\begin{center}
\begin{tikzpicture}[scale=12]
%
% Koordinater 
% ------------------------------------------------------
\coordinate (a) at (0.1,0.5);
\coordinate (b) at (0.4,0.3);
\coordinate (c) at (0.7,0.1);
%
%        
% Streger 
% -------------------------------------------------------
  \draw[densely dotted](0.1,0.25) -- (a);
  \draw[densely dotted](0.4,0.25) -- (b);
  \draw[densely dotted](0.7,0.25) -- (c);
  \draw[densely dotted](0.55,0.23) -- (0.55,0.15);
  \draw[densely dotted](0.475,0.23) -- (0.475,0.21);
  \draw[very thick, color=AAUblue2](a) parabola (b);
  \draw[very thick, color=AAUblue2](b) parabola[bend at end] (c);
  \draw[thick](0.1,0.23) -- (0.1,0.27); % (a)
  \draw[thick](0.7,0.23) -- (0.7,0.27); % (b)
  \draw[thick](0.4,0.23) -- (0.4,0.27); % (m_1)
  \draw[thick](0.55,0.23) -- (0.55,0.27); % (m_2)
  \draw[thick](0.475,0.23) -- (0.475,0.27); % (m_3)
%
%
% Punkt 
% -------------------------------------------------------
\node at (0.1,0.2) (){$a$};
\node at (0.7,0.3) (){$b$};
\node at (0.4,0.2) (){$m_1$};
\node at (0.55,0.3) (){$m_2$};
\node at (0.475,0.3) (){$m_3$};
% 
%
% Koordinatsystemet 
% -------------------------------------------------------
\draw[thick,->] (-0.1,0.25) -- (0.8,0.25) node[anchor=south east]{$x$};
\draw[thick,->] (0,0.1) -- (0,0.6) node[anchor=north west]{$y$};
%
\end{tikzpicture}
  \captionof{figure}{Illustration af bisektionsmetoden, hvor $m_1=\frac{a+b}{2}$, $m_2=\frac{m_1+b}{2}$, $m_3=\frac{m_1+m_2}{2}$.}
  \label{fig:bis}
\end{center}
%

%%%%%%%%%%%%%%%%%%%%%%%%%%%%%%%%%%%%%%%%%%%%%%%%%%%%%%%%%%%%%%%%
%%%%%%%%%%%%%%%%%%%%%%%%%%%%%%%%%%%%%%%%%%%%%%%%%%%%%%%%%%%%%%%%
%%%%%%%%%%%%%%%%%%%%%%%%%%%%%%%%%%%%%%%%%%%%%%%%%%%%%%%%%%%%%%%%
%%%%%%%%%%%%%%%%%%%%%%%%%%%%%%%%%%%%%%%%%%%%%%%%%%%%%%%%%%%%%%%%
\subsection*{Integrationsmetoden}
\textbf{Integrationsmetoden} omskriver, i modsætningen til de andre metoder, $f(x)=0$ til en \textbf{fixpunkt-ligning} $g(x)=x$, ved at isolere $x$ i udtrykket. Nulpunktet $f(s)=0$ bestemmes dermed ved en løsning $s$, der opfylder $g(s)=s$.

Et vilkårlig $x_0$ vælges indenfor intervallet $I$, og $$x_n=g(x_{n-1})$$
beregnes for de følgende $x$-værdier indtil skæringen mellem fixpunkt-ligningen og $y=x$ findes. 
Hvis $x_n=s$ så gælder $g(x_{n-1})=s$, altså vil $s=g(s)$.
På figur \ref{fig:iter} ses dette gjort for en fixpunkt-ligning $g(x)$, hvor de første fem $x$-værdier er markeret. Dette er illustreret i et cobweb diagram. 
%
%%%%%%%%%%%%%%%%%%%%%%%%%%%%%%%%
%%% Flot graf alla Julie     %%%
%%%%%%%%%%%%%%%%%%%%%%%%%%%%%%%%
%
\begin{center}
\begin{tikzpicture}[scale=12]
%
% Koordinater 
% ------------------------------------------------------
\coordinate (a) at (0.05,0.5);
\coordinate (b) at (0.65,0.05);
\coordinate (c) at (0.1,0);			% 1
\coordinate (d) at (0.1,0.497);		% 2
\coordinate (e) at (0.497,0.497); 	% 3
\coordinate (f) at (0.497,0.25);
\coordinate (g) at (0.25,0.25);
\coordinate (h) at (0.25,0.45);
\coordinate (i) at (0.45,0.45);
\coordinate (j) at (0.45,0.299);
\coordinate (k) at (0.299,0.299);
\coordinate (l) at (0.299,0.423);
\coordinate (m) at (0.423,0.423);
\coordinate (n) at (0.423,0.327);
\coordinate (o) at (0.327,0.327);
\coordinate (p) at (0.327,0.405);
\coordinate (q) at (0.405,0.405);
\coordinate (r) at (0.405,0.344);
\coordinate (s) at (0.344,0.344);
\coordinate (t) at (0.344,0.392);
\coordinate (u) at (0.392,0.392);
\coordinate (v) at (0.392,0.354);
\coordinate (x) at (0.354,0.354);
\coordinate (y) at (0.354,0.384);
\coordinate (z) at (0.384,0.384);
\coordinate (z1) at (0.384,0.363);
\coordinate (z2) at (0.363,0.363);
\coordinate (z3) at (0.363,0.376);
\coordinate (z4) at (0.376,0.376);
\coordinate (z5) at (0.376,0.368);
\coordinate (z6) at (0.368,0.368);
\coordinate (z7) at (0.372,0.372);
%
%        
% Streger 
% -------------------------------------------------------
  \draw[very thick, color=AAUblue2](a) parabola (b);
  \draw[very thick, color=AAUred] (-0.08,-0.08) -- (0.6,0.6);
  
  \draw[thick, color = AAUgreen](c) -- (d); 
  \draw[thick, color = AAUgreen](d) -- (e); 
  \draw[thick, color = AAUgreen](e) -- (f);
  \draw[thick, color = AAUgreen](f) -- (g);
  \draw[thick, color = AAUgreen](g) -- (h); 
  \draw[thick, color = AAUgreen](h) -- (i); 
  \draw[thick, color = AAUgreen](i) -- (j); 
  \draw[thick, color = AAUgreen](j) -- (k); 
  \draw[thick, color = AAUgreen](k) -- (l); 
  \draw[thick, color = AAUgreen](l) -- (m); 
  \draw[thick, color = AAUgreen](m) -- (n); 
  \draw[thick, color = AAUgreen](n) -- (o); 
  \draw[thick, color = AAUgreen](o) -- (p); 
  \draw[thick, color = AAUgreen](p) -- (q); 
  \draw[thick, color = AAUgreen](q) -- (r); 
  \draw[thick, color = AAUgreen](r) -- (s);
  \draw[thick, color = AAUgreen](s) -- (t); 
  \draw[thick, color = AAUgreen](t) -- (u);   
  \draw[thick, color = AAUgreen](u) -- (v);   
  \draw[thick, color = AAUgreen](v) -- (x);   
  \draw[thick, color = AAUgreen](x) -- (y);   
  \draw[thick, color = AAUgreen](y) -- (z);
  \draw[thick, color = AAUgreen](z) -- (z1);
  \draw[thick, color = AAUgreen](z1) -- (z2);
  \draw[thick, color = AAUgreen](z2) -- (z3);
  \draw[thick, color = AAUgreen](z3) -- (z4);
  \draw[thick, color = AAUgreen](z4) -- (z5);   
  \draw[thick, color = AAUgreen](z5) -- (z6);   
  \draw[thick, color = AAUgreen](z6) -- (z7);       
  
%
% Punkter
% -------------------------------------------------------
%
  \node at (0.7,0.05) (){$g(x)$};
  \node at (0.67,0.6) (){$y=x$};
  
  \node at (0.1,-0.05) (){$x_1$};
  \draw[densely dotted](0.497,0) -- (f);
  \node at (0.497,-0.05) (){$x_2$}; 
  \draw[densely dotted](0.25,0) -- (g);
  \node at (0.25,-0.05) (){$x_3$}; 
  \draw[densely dotted](0.45,0) -- (j);
  \node at (0.45,-0.05) (){$x_4$};
  \draw[densely dotted](0.299,0) -- (k);
  \node at (0.299,-0.05) (){$x_5$};
%  \draw[densely dotted](0.423,0) -- (n);
%  \node at (0.423,-0.05) (){$x_6$};
%  \draw[densely dotted](0.327,0) -- (o);
%  \node at (0.327,-0.05) (){$x_7$};
%
%
% Koordinatsystemet 
% -------------------------------------------------------
\draw[thick,->] (-0.1,0) -- (0.8,0) node[anchor=south east]{$x$};
\draw[thick,->] (0,-0.1) -- (0,0.65) node[anchor=north west]{$y$};
%
\end{tikzpicture}
  \captionof{figure}{Cobweb diagram for fixpunkt-ligningen $g(x)=x$, markeret med blå, og $y=x$ markeret med rød, hvor de første fem $x$-værdier er markeret.}
  \label{fig:iter}
\end{center}


%%%%%%%%%%%%%%%%%%%%%%%%%%%%%%%%%%%%%%%%%%%%%%%%%%%%%%%%%%%%%%%%
%%%%%%%%%%%%%%%%%%%%%%%%%%%%%%%%%%%%%%%%%%%%%%%%%%%%%%%%%%%%%%%%
%%%%%                                                       %%%%
%%%%%   %%  %                                               %%%%
%%%%%   % % %                                               %%%%
%%%%%   %  %%                                               %%%%
%%%%%                                                       %%%%
%%%%%%%%%%%%%%%%%%%%%%%%%%%%%%%%%%%%%%%%%%%%%%%%%%%%%%%%%%%%%%%%
%%%%%%%%%%%%%%%%%%%%%%%%%%%%%%%%%%%%%%%%%%%%%%%%%%%%%%%%%%%%%%%%
\subsection*{Newtons metode}
Lad $f$ være differentialbel. For \textbf{newtons metode} vælges et vilkårlig $x_1$, og der gøres derefter brug af \textbf{Newtons iterationsformel}:
\begin{align*}
x_{n+1}=x_n- \frac{f(x_n)}{f'(x_n)},
\end{align*} 
der beregnes for de efterfølgende $x$-værdier indtil nulpunktet er fundet.  
% måske noget om taylors formel xD
På figur \ref{fig:new} ses de første tre $x$-værdier og tilhørende første ordens taylorudvikling. 
%
\begin{center}
\begin{tikzpicture}[scale=12]
%
% Koordinater 
% ------------------------------------------------------
\coordinate (a) at (0.05,-0.05);
\coordinate (b) at (0.65,0.58);
\coordinate (c) at (0.6,0.48);		% 1
\coordinate (d) at (0.33,-0.05);	% a
\coordinate (e) at (0.66,0.59); 	% b
\coordinate (f) at (0.408,0.173);	% 2
\coordinate (g) at (0.47,0.252);	% a
\coordinate (h) at (0.238,-0.05);	% b
\coordinate (i) at (0.353,0.11);	% 3
\coordinate (j) at (0.32,0.07);		% a
\coordinate (k) at (0.16,-0.05);	% b
\coordinate (l) at (0.299,0.423);
%
%        
% Streger 
% -------------------------------------------------------
  \draw[very thick, color=AAUblue2](a) parabola (b);
   
  \draw[densely dashed, thick, color = AAUgreen](d) -- (e); 
  \draw[densely dashed, thick, color = AAUgreen](g) -- (h);   
%
% Punkter
% -------------------------------------------------------
%
  \node at (0.6,0.6) (){$f(x)$};
  
  \node at (0.6,0.07) (){$x_1$};
  \draw[densely dotted](0.6,0.1) -- (c);
  \node at (0.408,0.07) (){$x_2$}; 
  \draw[densely dotted](0.408,0.1) -- (f);
  \node at (0.353,0.07) (){$x_3$}; 
  \draw[densely dotted](0.353,0.1) -- (i);
%
%
% Koordinatsystemet 
% -------------------------------------------------------
\draw[thick,->] (-0.1,0.1) -- (0.8,0.1) node[anchor=south east]{$x$};
\draw[thick,->] (0,-0.1) -- (0,0.65) node[anchor=north west]{$y$};
%
\end{tikzpicture}
  \captionof{figure}{Illustration af newtons metode, hvor de første tre $x$-værdier er markeret.}
  \label{fig:new}
\end{center}


%%%%%%%%%%%%%%%%%%%%%%%%%%%%%%%%%%%%%%%%%%%%%%%%%%%%%%%%%%%%%%%%
%%%%%%%%%%%%%%%%%%%%%%%%%%%%%%%%%%%%%%%%%%%%%%%%%%%%%%%%%%%%%%%%
%%%%%%%%%%%%%%%%%%%%%%%%%%%%%%%%%%%%%%%%%%%%%%%%%%%%%%%%%%%%%%%%
%%%%%%%%%%%%%%%%%%%%%%%%%%%%%%%%%%%%%%%%%%%%%%%%%%%%%%%%%%%%%%%%
\subsection*{Sekantmetoden}
\textbf{Sekantmetoden} kan bruges, hvis man ikke kender $f'(x)$ eller i tilfælde af at $f'(x)=0$. 
To vilkårlig punkter $x_0$ og $x_1$ vælges, hvor $x_0x_1<0$, og \textbf{sekantmetodens iterationsformel}
\begin{align*}
x_{n+1}=x_n- \frac{x_n-x_{n-1}}{f(x_n)-f(x_{n-1})}f(x_n) ,
\end{align*}
beregnes for de efterfølgende $x$-værdier indtil nulpunktet er fundet. 
På figur \ref{fig:sek} ses de første fem $x$-værdier markeret og tilhørende sekanthældning.
%
%%%%%%%%%%%%%%%%%%%%%%%%%%%%%%%%
%%% Flot graf alla Julie     %%%
%%%%%%%%%%%%%%%%%%%%%%%%%%%%%%%%
%
\begin{center}
\begin{tikzpicture}[scale=12]
%
% Koordinater 
% ------------------------------------------------------
\coordinate (a) at (0.05,-0.05);
\coordinate (b) at (0.65,0.58);
\coordinate (c) at (0.6,0.48);		% x_0
\coordinate (d) at (0.07,-0.07);	% a
\coordinate (e) at (0.66,0.54); 	% b
\coordinate (f) at (0.092,-0.05);	% x_1
\coordinate (g) at (0.236,0.01);	% x_2
\coordinate (h) at (0.05,-0.065);	% a
\coordinate (i) at (0.5,0.12);		% b
\coordinate (j) at (0.45,0.228);	% x_3
\coordinate (k) at (0.47,0.25);		% a
\coordinate (l) at (0.218,-0.005);  % b
\coordinate (m) at (0.323,0.083);   % x_4
\coordinate (n) at (0.47,0.25);		% a
\coordinate (o) at (0.31,0.065);  % b
\coordinate (p) at (0.323,0.083);   % x_5
%
%        
% Streger 
% -------------------------------------------------------
  \draw[very thick, color=AAUblue2](a) parabola (b);
   
  \draw[densely dashed, thick, color = AAUgreen](d) -- (e); 
  \draw[densely dashed, thick, color = AAUgreen](h) -- (i); 
  \draw[densely dashed, thick, color = AAUgreen](k) -- (l);
  \draw[densely dashed, thick, color = AAUgreen](n) -- (o);   
%
% Punkter
% -------------------------------------------------------
%
  \node at (0.6,0.6) (){$f(x)$};
  
  \node at (0.6,0.08) (){$x_0$};
  \draw[densely dotted](0.6,0.1) -- (c);
  \node at (0.092,0.12) (){$x_1$}; 
  \draw[densely dotted](0.092,0.1) -- (f);
  \node at (0.236,0.12) (){$x_2$}; 
  \draw[densely dotted](0.236,0.1) -- (g);
  \node at (0.45,0.08) (){$x_3$}; 
  \draw[densely dotted](0.45,0.1) -- (j);
  \node at (0.323,0.12) (){$x_4$}; 
  \draw[densely dotted](0.323,0.1) -- (m);
%
%
% Koordinatsystemet 
% -------------------------------------------------------
\draw[thick,->] (-0.1,0.1) -- (0.8,0.1) node[anchor=south east]{$x$};
\draw[thick,->] (0,-0.1) -- (0,0.65) node[anchor=north west]{$y$};
%
\end{tikzpicture}
  \captionof{figure}{Illustration af sekantmetoden, hvor de første fem $x$-værdier er markeret.}
  \label{fig:sek}
\end{center}


% EKSTRA  
%%
%\begin{center}
%\begin{tikzpicture}[scale=12]
%%
%% Koordinater 
%% ------------------------------------------------------
%\coordinate (a) at (0.05,-0.05);
%\coordinate (b) at (0.65,0.58);
%\coordinate (c) at (0.6,0.48);		% 1
%\coordinate (d) at (0.33,-0.05);	% a
%\coordinate (e) at (0.66,0.59); 	% b
%\coordinate (f) at (0.355,0.115);	% 2
%\coordinate (g) at (0.48,0.25);		% a
%\coordinate (h) at (0.21,-0.05);	% b
%\coordinate (i) at (0.254,0.022);	% 3
%\coordinate (j) at (0.32,0.07);		% a
%\coordinate (k) at (0.16,-0.05);	% b
%\coordinate (l) at (0.299,0.423);
%%
%%        
%% Streger 
%% -------------------------------------------------------
%  \draw[very thick, color=AAUblue2](a) parabola (b);
%   
%  \draw[thick, color = AAUgreen](d) -- (e); 
%  \draw[thick, color = AAUgreen](g) -- (h); 
%  \draw[thick, color = AAUgreen](j) -- (k);   
%%
%% Punkter
%% -------------------------------------------------------
%%
%  \node at (0.6,0.6) (){$f(x)$};
%  
%  \node at (0.6,-0.05) (){$x_1$};
%  \draw[densely dotted](0.6,0) -- (c);
%  \node at (0.355,-0.05) (){$x_2$}; 
%  \draw[densely dotted](0.355,0) -- (f);
%  \node at (0.254,-0.05) (){$x_3$}; 
%  \draw[densely dotted](0.254,0) -- (i);
%%
%%
%% Koordinatsystemet 
%% -------------------------------------------------------
%\draw[thick,->] (-0.1,0) -- (0.8,0) node[anchor=south east]{$x$};
%\draw[thick,->] (0,-0.1) -- (0,0.65) node[anchor=north west]{$y$};
%%
%\end{tikzpicture}
%  \captionof{figure}{Illustration af newtons metode.}
%  \label{fig:new}
%\end{center}
