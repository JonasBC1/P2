%
%     %%%%%%%%%%%%%%%%%%%%%%%%%%%%%%%%%%%%%
%     %%%%%                           %%%%%
%     %%%%%      Person nummer 6      %%%%%
%     %%%%%                           %%%%%
%     %%%%%%%%%%%%%%%%%%%%%%%%%%%%%%%%%%%%%
%
%\begin{frame}
%\frametitle{Simplexmetoden}
%\begin{itemize}
%\item Hvordan simplex afsøger hjørnepunkter.
%\item \textbf{$nr. 6$}
%\item Implementationer af simplex(tænker kort gennemgang af forskelle)
%\item Gennemgang af fuld tabelimplementering (mere udførligt (eksempel)).
%\item Lexicografi til valg af pivoteringsmetode.
%\item Kompleksitet som udgangspunkt for valg af metode.
%\item Afrunding / konklusion?!
%\end{itemize}
%\end{frame}
%
\begin{frame}
\frametitle{Implementeringer}
\begin{itemize}
\item Den naive implementering 
	\begin{itemize}
	% \item Høj tidspkompleksitet 
	\item Løser to lineær ligningssystemer
	\end{itemize}
\item Den reviderede implementering 
	\begin{itemize}
	\item $B^{-1}$ opdateres ved hver iteration
	\end{itemize}
\item Fuld-tabel implementeringen
	\begin{itemize}
	\item Simplextabel opdateres ved hver iteration
	\end{itemize} 
\end{itemize}
\end{frame}

\begin{frame}
\frametitle{Fuld-tabel implementeringen}
\begin{itemize}
\item Simplextabel 
\item Pivotering
\end{itemize}
\begin{align*}
\begin{blockarray}{ccccccccccc}
x_1 & x_2 & \cdots & x_n & \textcolor{blue}{s_1} & \textcolor{blue}{s_2} &  \textcolor{blue}{\cdots} & \textcolor{blue}{s_m} & z & b \\
\begin{block}{[cccc|ccccc|c]c}
a_{1,1} & a_{1,2} & \cdots & a_{1,n} & 1 & 0 & \cdots & 0 & 0 & b_1 \\
a_{2,1} & a_{2,2} & \cdots & a_{2,n} & 0 & 1 & \cdots & 0 & 0 & b_2 \\
\vdots & \vdots & \ddots & \vdots & \vdots & \vdots & \ddots & \vdots & \vdots & \vdots \\
a_{m,1} & a_{m,2} & \cdots & a_{m,n} & 0 & 0 & \cdots  & 1  & 0 & b_{m}\\
\cline{1-10}
-c_1 & -c_2 & \cdots & -c_n & 0 & 0 & \cdots & 0 & 1 & z\\
\end{block}
\end{blockarray}.
\end{align*}
\end{frame}

\begin{frame}
\frametitle{Pivoteringsløkker}
\begin{itemize}
\item Lexicografi
	\begin{itemize}
	\item $\mathbf{u}$ er lexicografisk større $\textbf{u} >^L \textbf{v}$
	\item $\mathbf{u}$ er lexicografisk mindre $ \textbf{u} <^L \textbf{v}$
	\end{itemize}
\item Lexicografiske metode 
\end{itemize}
\end{frame}

\begin{frame}
\frametitle{Kompleksitet}
%som udgangspunkt for valg af metode.
\begin{itemize}
\item Den naive implementering 
	\begin{itemize}
	% \item Høj tidspkompleksitet 
	\item $O(m^3)$ 
	\end{itemize}
\item Den reviderede implementering 
	\begin{itemize}
	\item $O(mn)$
	\end{itemize}
\item Fuld-tabel implementeringen
	\begin{itemize}
	\item $O(mn)$
	\end{itemize} 
\end{itemize}
\end{frame}