\section{Lineære ligningssystemer}
\begin{frame}
\centering
\Huge
Lineære ligningssystemer
\end{frame}
%
\begin{frame}
\frametitle{Lineære ligningssystemer}
\begin{itemize}
\item Et system af $m$ lineære ligninger med de samme $n$ variable. 
\item Eksempel:
\begin{align*}
x_1-x_3-2x_4-8x_5&=-3 \\
-2x_1+x_3+2x_4+9x_5&=5 \\
3x_1-2x_3-3x_4-15x_5&=-9.
\end{align*}
\item Ingen, én eller uendeligt mange løsninger. 
\item Hvis systemet ingen løsning har, er det inkonsistent. Ellers er det konsistent. 
\end{itemize}
\end{frame}
%%
\begin{frame}
\frametitle{Lineære ligningssystemer}
\begin{itemize}
\item Lineære ligningssystemer kan opstilles i en totalmatrix. 
\item Totalmatricen er koefficientmatricen udvidet. 
\item Eksempel:
\begin{equation*}
  [A \mid \mathbf{b}] =
\begin{blockarray}{ccccccc}
x_1 & x_2 & x_3 & x_4 & x_5 & b \\
\begin{block}{[ccccc|c]c}
  1 & 0 & -1 & -2 & -8 & -3 \\
  -2 & 0 & 1 & 2 & 9 & 5 \\
  3 & 0 & -2 & -3 & -15 & -9 \\
\end{block}
\end{blockarray}.
\end{equation*}
\item Løsningen er en vektor \mathbf{v}, således at ligningerne er opfyldt, når $x_i = v_i$.
\end{itemize}
\end{frame}
%%
\begin{frame}
\frametitle{Rækkeoperationer}
\begin{itemize}
\item Totalmatricen gør det let at udføre elementære rækkeoperationer. 
\item Ombytning: $A \xrightarrow{R_i \leftrightarrow R_j} B$. 
\item Skalering: $A \xrightarrow{R_i \rightarrow cR_i} B$.
\item Udskiftning: $A \xrightarrow{R_i \rightarrow R_i + cR_h} B$.
\end{itemize}
\end{frame}
%%
\begin{frame}
\frametitle{Trappeform}
\begin{itemize}
\item Kan opnås via elementære rækkeoperationer. 
\item Ikke-nulrækker over nulrækker. 
\item Første ikke-nulindgang ligger i en søjle til højre for ikke-nulindgangen i forrige række. 
\item Hvis en søjle har den første ikke-nulindgang i en række, er alle understående indgange i søjlen nulindgange. 
\item Eksempel: 
\begin{align*}
A=
\begin{blockarray}{cccc}
\begin{block}{[cccc]}
2 & 4 & 8 & 2\\
0 & 5 & -2 & 5\\
0 & 0 & 2 & 7\\
0 & 0 & 0 & 0\\
\end{block}
\end{blockarray}
\text{ og }
B=
\begin{blockarray}{cccc}
\begin{block}{[cccc]}
2 & 4 & 8 & 2\\
0 & 0 & -2 & 5\\
0 & 4 & 2 & 7\\
0 & 0 & 0 & 4\\
\end{block}
\end{blockarray}.
\end{align*}
\end{itemize}
\end{frame}
%%
\begin{frame}
\frametitle{Reduceret trappeform}
\begin{itemize}
\item Kan opnås via elementære rækkeoperationer. 
\item Matricen er på trappeform.
\item Hvis en søjle har den første ikke-nulindgang i en række, så er alle øvrige indgange i søjlen $0$.
\item Den første ikke-nulindgang i hver ikke-nulrække er lig $1$. 
\item Eksempel: 
\begin{align*}
A_R=
\begin{blockarray}{ccccc}
\begin{block}{[ccccc]}
1 & 0 & 5 & 0 & 0\\
0 & 1 & 3 & 0 & 0\\
0 & 0 & 0 & 1 & 0\\
0 & 0 & 0 & 0 & 1\\
\end{block}
\end{blockarray}
\end{align*}
\item Konsistens og pivot. 
\end{itemize}
\end{frame}
%%
\begin{frame}
\frametitle{Gauss-elimination}
\begin{itemize}
\item En algoritme, der benytter elementære rækkeoperationer. 
\item Reducerer en matrix til reduceret trappeform. 
\end{itemize}
\begin{enumerate}
\item Find første ikke-nulsøjle fra venstre i matricen.
\item Ved rækkeombytning placeres en ikke-nulindgang øverst i pivotsøjlen.
\item Skab nulindgange under pivotindgangen ved hjælp af rækkeudskiftning.
\item Øverste række markeres som afsluttet og trin $1-4$, hvor afsluttede rækker ignoreres.
Dette gentages, indtil alle rækker er markeret som afsluttet.
\item Alle rækker med pivotindgange skaleres, så alle pivotindgange er lig $1$.
\item Ved rækkeudskiftning sikres nu nulindgange over og under pivotindgangene.
\end{enumerate}
\end{frame}
%%
\begin{frame}
\frametitle{Rang og nullitet}
\begin{itemize}
\item Rang$(A)$ er antallet af pivotsøjler i en matrix $A$. 
\item Null$(A)$ er antallet er ikke-pivotsøjler i en matrix $A$. 
\item I en $m \cross n$ matrix er Rang$(A)$ $+$ Null$(A) = n$. 
% Kort!!
\end{itemize}
\end{frame}