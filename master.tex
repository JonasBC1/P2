% master.tex : master-fil for projektet
% ------------------------------------------------------------------------------
% Dette er hovedfilen for projektet, hvori indhold fra alle input-filer (tekst,
% billeder, litteraturdatabaser, osv.) samles

% Dokumenttypen 'book' er valgt pga. dens mange fleksible indstillinger
% Se https://tex.stackexchange.com/a/36989/118167
\documentclass[11pt,a4paper,twoside,openright,danish]{book}

% Variabler, som bruges til automatisk at indsætte titel, forfattere, osv. på
% forsiden og titelbladet.
\def \projecttitle       {Optimering}
\def \projectsubtitle    {Lineær programmering}
\def \projecttheme       {Lineær algebra}
\def \projectdegree      {Matematik}
\def \projectperiod      {Efterårssemesteret 2020}
\def \projectnumber      {P2}
\def \projectgroup       {B303c}
\def \projectauthors     {
	Jens Koldkur\\
	Jonas Bach Christensen\\
	Julie Havbo Lund\\
	Mads Bjerregaard Kjær\\
	Mathias Gammelgaard\\
	Rasmus Jespersgaard
}
\def \projectsupervisors {
  Horia Cornean
}

% Preamblet indeholder alle de indstillinger og makroer, som skal indsættes for
% hovedindholdet, og i denne skabelon samles det i filen aaumath.sty, som
% definerer en pakke, der kan indlæses med \usepackage.
\usepackage{aaumath}

% Dokumentets indhold indsættes mellem \begin- og \end-makroerne for
% 'document'-blokken
\begin{document}

% Dokumentets 'front matter' tælles ikke med ifm. antal sider og nummereres med
% romerske tal. Herunder hører f.eks. forsiden, titelbladet, forordet og
% indholdsfortegnelsen.
\frontmatter
\include{incl/misc/frontpage}
\include{incl/misc/titlepage}
\include{incl/misc/contents}

% Dokumentets 'main matter' (hovedindhold) er der, hvor det meste indhold skal
% sættes ind. Sider og overskrifter er nummererede med arabiske tal.
\mainmatter

% Input-filer bør opdeles således, at hver fil svarer til et kapitel. Makroen
% \include indsætter et sideskift og indholdet fra den givne stil.

% INDLEDNING
	\chapter*{Forord}
\addcontentsline{toc}{chapter}{\phantom{lol.}Forord}
Nærværende rapport er udarbejdet af seks studerende på andet semester  på bacheloruddannelsen Matematik på Aalborg Universitet. 
Rapporten er udarbejdet fra 3. februar til 27. maj 2020.
% 
Temaet for projektet er optimering, herunder mere specifikt lineær programmering.
%
Formålet med rapporten er at få en forståelse for optimeringsmetoder, samt at formidle mere matematisk præcist.
% 	
Rapporten er nummereret i afsnit, definitioner, figurer m.m.
Beviser afsluttes med $\square$, der indikerer, at beviset er gennemført.
% \square skulle have været \qed!
%
Der skal lyde en tak til vores vejleder Horia Cornean, og vores bi-vejleder Maiken Winther, for at være behjælpelige gennem projektperioden.
%
\\\\
\section*{Underskrifter}
\begin{center}
\begin{minipage}[b]{0.45\textwidth}
\begin{center}
\begin{tabular}{l}
\phantom{Julie er virkelig dejlig, sød og smuk} \\
\\
\\
\hline
Jens Koldkur \\
\\
\\
\\
\\
\hline
Jonas Bach Christensen \\
\\
\\
\\
\\
\hline
Julie Havbo Lund \\         
\end{tabular}
\end{center}
\end{minipage}
%
\begin{minipage}[b]{0.038\textwidth}
\phantom{xD}
\end{minipage}
\begin{minipage}[b]{0.45\textwidth}
\begin{center}
\begin{tabular}{l}
\phantom{Julieervirkeligdejligogsødogsmukogs} \\
\\
\\
\hline
Mads Bjerregaard Kjær \\
\\
\\
\\
\\
\hline
Mathias Gammelgaard \\
\\
\\
\\
\\
\hline
Rasmus Jespersgaard \\         
\end{tabular}
\end{center}
\end{minipage}
\end{center}
	\chapter{Indledning}
%To til tre sider 
%Vi skal argumentere det er interessant at kigge på lineær programmering
%Kort beskrivelse af hvad vi har lavet 
%Eventuelt 
%   Kapitel 1: bla bla 
%   Kapitel 2: bla bla 
%   m.m.
Optimeringsproblemer er en fundamental del af matematikken, og i den anvendte matematik udgøre disse en vægtig del af matematikkeres virke.
I denne rapport vil lineære optimeringsproblemer undersøges. 
Dele af forskningsfeltet betragter disse som havende ophav ved Danzig og dennes udarbejdelse af simplexmetoden \citep[side 107]{refa} i 1947.
Der er dog tilstadighed diskussion om, hvor lineære programmeringsproblemer som matematisk felt stammer fra og andre peger på Sovjetunionen og behovet for at optimere produktionen i en planøkonomisk kontekst som ophav \citep[side 155]{refb}.
Forudsætningen for, at Danzig arbejdede med lineære programmeringsproblemer var hans ansættelse ved Royal Airforce \citep[side 107]{refa} og en applikation blev i forbindelse med den vestligt kontrollerede sektor af Berlin.
Vestberlin lå dybt inde i den sovjetisk kontrollerede sektor af Tyskland og som følge af spændingerne mellem Vesteuropa og Sovjet besluttede Josef Stalin at lukke for forsyninger via jorden til Berlin.
Det blev derfor nødvendigt for vestmagterne at etablere en luftbro til Berlin, der gjorde det muligt, trods de sovjetiske restriktioner, at indføre basale fornødenheder til de $2,5$ millioner mennesker der levede i byen.
Da en sådan måde at indføre vare på er ressourceintensiv, blev der derfor behov for at gøre dette mest effektivt underlagt de logistiske restriktioner.
Det er derfor et matematisk felt, som i sin brug er nært bundet op på praktiske problemstillinger dette værende økonomiske og logistiske.
Det er derfor også et felt, der har været nært knyttet op til militære problemstillinger som Danzigs ansættelsesforhold også antyder.
Danzigs bidrag til forskningsfeltet, altså simplexmetoden, er dog stadigt en fundamental del af hvordan løsningen af lineære programmeringsproblemer behandles i dag.
\\\\
Projektets primære arbejdsområde vil derfor omhandle simplexmetoden samt den geometriske fremstilling af lineære programmeringsproblemer.
Det skal dog nævnes, at der ligeledes findes optimeringsproblemer, som ikke kan beskrives igennem lineære sammenhænge, en afgrænsning bliver her at disse ikke vil beskrives i dette indeværende projekt.
Problemformuleringen bliver derfor som følger:
\begin{col}{}{}
Hvordan kan lineære programmeringsproblemer anskues geometrisk og hvordan kan disse løses ved hjælp af simplexmetoden?
\end{col}
Med henblik på at besvare denne vil følgende emneområder bliver inddraget:
\begin{itemize}
\item En introduktion til lineær algebra: Afsnittet har til hensigt at indføre passende notation, samt introducere væsentlige metoder og begreber til løsning af lineære programmeringsproblemer.
\item Lineære programmeringsproblemer: Projektets primære emneområde. Det vil her introduceres hvordan disse fremstilles matematiske samt interessante egenskaber herved.
\item Geometrisk fremstilling: her skal der stå et eller andet når vi ved hvad der er med.
\item Simplexmetoden: Beskrivelse og bevis af Simplexmetoden som løsningsalgoritme.
\end{itemize}
	
% LINEÆRE ALGEBRA 
	\chapter{Lineær algebra}
%
\section{Matricer og vektorer}
% God arbejdslyst 

%Noget indledningsvist - hvorfor er matricer/vektorer vigtige? 

I forbindelse med løsning af lineære ligningssystemer og optimeringsproblemer er det nødvendigt at redegøre for \textit{matricer} og tilhørende notation. 

\begin{defn}{}{}
En matrix er en rektangulær tabel, hvis elementer er \textit{skalarer}, der er $k \in \R$. 
En matrix med $m$ rækker og $n$ søjler har størrelsen $m \times n$.
Hvis $m=n$, er matricen kvadratisk. 
Elementet $a_{i,j}$ i $i'te$ række og $j'te$ søjle kaldes $(i,j)-indgangen$ i matricen. 
\end{defn}

%HVORFOR DUR REFERENCER IKKE HVAD FUCK HJÆLP FOR SATAN

På figur \ref{fig:matrix_gen_eks} ses et generelt eksempel på en matrix $A = m \times n$.

\begin{figure}[H]
	\begin{center}
$$
A=
\begin{bmatrix}
a_{1,1} & a_{1,2} & \ldots & a_{1,j} \\
a_{2,1} & a_{2,2} & \ldots & a_{2,j} \\
\vdots  & \ldots  & \ddots & \vdots \\
a_{i,1} & a_{i,2} & \ldots & a_{i,j} \\
\end{bmatrix}
$$
	\end{center}
	\caption{En generel matrix $A=m \times n$.}
	\label{fig:matrix_gen_eks}
\end{figure}

Enhver række $i$ eller søjle $j$ er en \textit{vektor}. En matrix med én række kaldes en \textit{rækkevektor}, og en matrix med én søjle kaldes en \textit{søjlevektor}. I denne rapport noteres vektorer med små, fede bogstaver, såsom $\textbf{a}$ og $\textbf{b}$. Søjlevektorer noteres desuden ofte med et indeks tilsvarende søjlen i matricen, således 
$$
\textbf{a}_j= 
\begin{bmatrix}
a_{1,j} \\
a_{2,} \\
\vdots \\
a_{i,j} \\
\end{bmatrix}
$$. 











	\subsection{Udvalgte matricer og vektorer} 
% 
Hvis alle indgange i en matrix er nul, kaldes dette for en \textit{nulmatrix}, noteret $O$. 
En $m \times n$ nulmatrix noteres $O_{m,n}$.
%
Et andet særtilfælde er \textit{standardvektorerne} i $\R^n$, som er defineret ved  
% Jeg er lidt i tvivl om om der skal være et ekstra nul i længen så der ikke står 1 ... ? - Julie 
$$
\textbf{e}_1=
\begin{bmatrix}
1 \\ 
0 \\
0 \\
\vdots \\
0
\end{bmatrix}
\text{, }
\textbf{e}_2=
\begin{bmatrix}
0 \\ 
1 \\
0 \\
\vdots \\
0
\end{bmatrix}
\text{, }
\ldots
\text{, }
\textbf{e}_n=
\begin{bmatrix}
0 \\ 
0 \\
0 \\
\vdots \\
1
\end{bmatrix}
\text{. }
$$
%
Sammensættes standardvektorer i en matrix opnås en \textit{identitetsmatrix}.
%
\begin{defn}{}{}
%
En $n \times n$ \textbf{identitetsmatrix} $I_n$, hvor $n \in \Z^+$, består af alle standardvektorer $\textbf{e}_1, \textbf{e}_2, \ldots, \textbf{e}_n$ i $\R^n$, således at
$$
I_n=
\begin{bmatrix}
\textbf{e}_1 & \textbf{e}_2 & \ldots & \textbf{e}_n
\end{bmatrix}.
$$ 
\end{defn}
\noindent
%

	\subsection{Matrixsum og skalarmultiplikation}
%Blev vi ikke enige om skalering? Skal der ikke stå det samme i underafsnitstitlen som i definitionen, eller er det ligemeget? 
\begin{defn}{}{mxsum}
Lad $A$ og $B$ være $m \times n$ matricer.
\textbf{Summen} af $A$ og $B$ er $m \times n$ matricen $A + B$, hvor de $(i,j)$'te indgange er $a_{i,j} + b_{i,j}$.
Ligeledes er subtraktion mellem matricerne $A$ og $B$ muligt og indgangene i den resulterende matrix er $a_{i,j} - b_{i,j}$.
Lad nu $k$ være en skalar.
\textbf{Skalering} af $A$, noteret $kA$, er en $m \times n$ matrix, hvor de $(i,j)$'te indgange er $ka_{i,j}$.
Bemærk, at $1A = A$, $-1A = -A$ og $0A = O$.
\end{defn}
%
\begin{eks}
Lad
\begin{align*}
A= 
\begin{bmatrix}
3	&	-3	&	1\\
2	&	0	&	4
\end{bmatrix}
\text{\phantom{---}og\phantom{---}}
B= 
\begin{bmatrix}
-1	&	4	&	1\\
5	&	3	&	2
\end{bmatrix}.
\end{align*}
Jævnfør definition \ref{defn:mxsum} haves
\begin{align*}
3A= 
\begin{bmatrix}
9	&	-9	&	3\\
6	&	0	&	12
\end{bmatrix}
\text{,\phantom{---}}
-B= 
\begin{bmatrix}
1	&	-4	&	-1\\
-5	&	-3	&	-2
\end{bmatrix}
\end{align*}
og
\begin{align*}
3A+B= 
\begin{bmatrix}
9	&	-9	&	3\\
6	&	0	&	12
\end{bmatrix}
+ 
\begin{bmatrix}
-1	&	4	&	1\\
5	&	3	&	2
\end{bmatrix}
=
\begin{bmatrix}
8	&	-5	&	4\\
11	&	3	&	14
\end{bmatrix}.
\end{align*}
\end{eks}
%
Ud fra disse operationer har matricer en række egenskaber, som ses i sætning \ref{thm:mxprop}.
%Nedenstående definition er ordret den samme som i bogen (s. 6), men vi kan jo nærmest ikke skrive den og lignende sætninger anderledes; skal vi ikke bare lige vende det med Horia, tror I? 
\begin{thm}{}{mxprop}
Lad $A$, $B$ og $C$ være $m \times n$ matricer, og lad $k$ og $s$ være skalarer.
Så har matricer følgende egenskaber:
\begin{enumerate}[label=(\alph*)]
\item $A + B = B + A$.
\item $(A + B) + C = A + (B + C)$.
\item $A + O = A$.
\item $A + (-A) = O$.
\item $(ks)A = s(kA)$.
\item $k(A + B) = kA + kB$.
\item $(k + s)A = kA + sA$.
\end{enumerate}
\end{thm}
%
\begin{proof}
Lad $A$, $B$ og $C$ være $m \times n$ matricer, og lad $k$ og $s$ være skalarer.
Betragt $A + B$ og $B + A$ og bemærk jævnfør \ref{defn:mxsum}, at de $(i,j)$'te indgange bliver henholdsvis $a_{i,j} + b_{i,j}$ og $b_{i,j} + a_{i,j}$, hvilket beviser (a).
Ligeledes ses det, at $(a_{i,j} + b_{i,j}) + c_{i,j} = a_{i,j} + (b_{i,j} + c_{i,j})$, hvilket beviser (b).
Endvidere kan $0$ isoleres i $a_{i,j} + 0 = a_{i,j}$, således $a_{i,j} + (-a_{i,j}) = 0$, hvilket beviser (c) og (d), da alle indgange i en nulmatrix er $0$.
\\\\%I linje 83: Kan man godt bare sige "... og så $s$ ..."? :-)
(e) Betragt skaleringen af $A$ med skalaren $ks$, som er et produkt af skalarene $k$ og $s$, og skaleringen af $A$ med $k$ og så $s$.
For indgangene i disse skaleringer findes ligheden $(ks)a_{i,j} = s(ka_{i,j})$, hvilket beviser (e).
Betragt ligeledes (f) og bemærk, at $k(a_{i,j} + b_{i,j}) = ka_{i,j} + kb_{i,j}$, hvilket beviser (f).
Det samme gøres for ligheden i (g), hvor ligheden $(k + s)a_{i,j} = ka_{i,j} + sa_{i,j}$ opstilles, hvilket beviser (g).
\end{proof}
\noindent
\\
%
Da vektorer er matricer med enten én række eller én søjle er samme operationer mulige og opfylder samme egenskaber.
	\subsection{Matrix transponering}
\begin{defn}{}{mxtrans}
Lad $A$ være en $m \times n$ matrix. \textbf{Transponeringen} til $A$, er en $n \times m$ matrix, noteret $A^T$, hvor $A$'s $(i,j)$'te indgang er den $(j,i)$'te indgang i $A^T$.
\end{defn}
En transponering af en matrice resulterer i en ny matrice, hvor rækkerne i den originale matrice kommer til at være søjlerne i den transponerede matrice.

\begin{eks}\label{eks:trans}
Betragt nu $A^T$ og $(A+B)^T$, hvor 
\begin{align*}
A= 
\begin{bmatrix}
3	&	-3	&	1\\
2	&	0	&	4
\end{bmatrix}
\text{\phantom{---}og\phantom{---}}
B= 
\begin{bmatrix}
-1	&	4	&	1\\
5	&	3	&	2
\end{bmatrix}.
\end{align*}
Jævnfør definition \ref{defn:mxtrans} haves
\begin{align*}
A^T =
\begin{bmatrix}
3	&	2\\
-3	&	0\\
1	&	4
\end{bmatrix}
\text{\phantom{---}og\phantom{---}}
(A+B)^T=
\begin{bmatrix}
2	&	1	&	2\\
7	&	3	&	6
\end{bmatrix}^T
=
\begin{bmatrix}
2	&	7\\
1	&	3\\
2	&	6
\end{bmatrix}.
\end{align*}
\end{eks}

Følgende sætning viser, at transponering bevarer operationerne matrix summering og skalering.
\begin{thm}{}{}
Lad $A$ og $B$ være $m \times n$ matricer, og lad $k$ være en skalar.
Så har matricerne følgende egenskaber:
\begin{enumerate}[label=(\alph*)]
\item $(A + B)^T = A^T + B^T$.
\item $(kA)^T = kA^T$.
\item $(A^T)^T = A$.
\end{enumerate}
\end{thm}

\begin{proof}
Lad $A$ og $B$ være $m \times n$ matricer og lad $k$ være en skalar.
Det vides at indgangene i $A+B$ er $a_{i,j} + b_{i,j}$ og transponeringen af dette resulterer $n \times m$ matrix med indgangende $a_{j,i} + b_{j,i}$.
Det ses at  $A^T + B^T$ også er en $n \times m$ matrix med indgangene $a_{i,j} + b_{i,j}$, hvilket beviser (a).
Ved skalering og transponering af en matrix ses, at forskellig rækkefølge på operationerne er uden betydning for slutresultatet.
Dette tydeliggøres ved at betragte $kA^T$ og $(kA)^T$.
For $kA^T$ er indgangende $ka_{j,i}$ og indgangende i $kA$ er $ka_{i,j}$, som efter transponering er $ka_{j,i}$, hvilket beviser (b).
Transponeres $A$ to gange vil resultatet være $A$.
Første transponering resulterer i en $n \times m$ matrix med indgangene $a_{j,i}$ og anden transponering resulterer i en $m \times n$ matrix med indgangene $a_{i,j}$, hvilket beviser (c).
\end{proof}
	\subsection{Linearkombination}

.... metatekst

\begin{defn}{}{}
Givet matricerne $A_1, A_2, \ldots, A_k$ med dimensionerne $m \times n$, så er \textit{linearkombinationen} af $A_1, A_2, \ldots, A_k$ en matrix $B$ på formen 

$$c_1A_1+c_2A_2+\ldots+c_kA_k=\sum\limits_{i=1}^k c_iA_i=B,$$

med dimensionerne $m \times n$. 
Linearkombinationen er mulig, hvis og kun hvis skalarerne $c_1, c_2, \ldots, c_k$ eksisterer. 
Skalarerne kaldes linearkombinationens \textit{koefficienter}.
\end{defn}
%KILDE: https://www.statlect.com/matrix-algebra/linear-combinations

I lineær algebra er det oftest række- eller søjlevektorer, der indgår i linearkombinationer. 







\subsection{Matrix-vektorprodukt}

\begin{defn}{}{}

\end{defn}
	\subsection{Matrixmultiplikation}

Det er ofte nødvendigt at multiplicere matricer med andre matricer. 
\begin{defn}{}{mxmulti}
Matrixproduktet $AB$ af en $m \times n$ matrix $A$ og en $n \times p$ matrix $B$ er en $m \times p$ matrix $C$, 
$$
C=
\begin{bmatrix}
A\textbf{b}_1 & A\textbf{b}_2 & \ldots & A\textbf{b}_p
\end{bmatrix}\text{,}
$$
hvor $A\textbf{b}_j$ er den $j$'te række i $C$.
\end{defn}
\noindent
%
Som det fremgår af definition \ref{def:mxmulti}, skal antallet af søjler i $A$ tilsvare antallet af rækker i $B$, for at $AB$ er defineret og matrixmultiplikation dermed er muligt. 
Bemærk, at selvom $AB$ er defineret, er $BA$ ikke nødvendigvis defineret. 
Hvis både $AB$ og $BA$ er defineret, er de desuden ikke nødvendigvis lig hinanden. 
For at beregne værdien af komponenterne i matrixproduktet $AB$ kan følgende sammenhæng mellem den $i$'te række i $A$ og den $j$'te række i $B$ betragtes: 
$$
a_{i1}b_{1j} + a_{i2}b_{2j} + \cdots + a_{in}b_{nj}
\text{. }$$
\\\\
%
\begin{eks}
Lad 
$$
A=
\begin{bmatrix}
7 & 9 & 1 \\
4 & 0 & 0 \\
8 & 5 & 5
\end{bmatrix}
\text{og }
B=
\begin{bmatrix}
2 & 3 \\
5 & 1 \\
0 & 1 
\end{bmatrix}
\text{.}
$$
Ved at betragte $A$ og $B$ vides det, $AB$ er defineret og er en $3 \times 2$ matrix. 
Først findes søjlerne $A\textbf{b}_1$ og $A\textbf{b}_2$:
$$
A\textbf{b}_1=
\begin{bmatrix}
7 & 9 & 1 \\
4 & 0 & 0 \\
8 & 5 & 5
\end{bmatrix}
\begin{bmatrix}
2 \\
5 \\
0
\end{bmatrix}
=
\begin{bmatrix}
14 + 45 + 0 \\
8 + 0 + 0 \\
16 + 25 + 0
\end{bmatrix}
=
\begin{bmatrix}
59 \\
8 \\
41
\end{bmatrix}
$$
og
$$
A\textbf{b}_1=
\begin{bmatrix}
7 & 9 & 1 \\
4 & 0 & 0 \\
8 & 5 & 5
\end{bmatrix}
\begin{bmatrix}
3 \\
1 \\
1
\end{bmatrix}
=
\begin{bmatrix}
21 + 9 + 1 \\
12 + 0 + 0 \\
24 + 5 + 5
\end{bmatrix}
=
\begin{bmatrix}
31 \\
12 \\
34
\end{bmatrix}
\text{.}
$$
Dermed er 
$$
AB=
\begin{bmatrix}
A\textbf{b}_1 & A\textbf{b}_2
\end{bmatrix}
=
\begin{bmatrix}
59 & 31 \\
8 & 12 \\
41 & 34
\end{bmatrix}
$$
\end{eks}
Da matricen $B$ inddeles i sine søjlevektorer, er grundprincippet i matrixmultiplikation at finde matrix-vektorprodukter.\\\\
En række egenskaber gælder for matrixmultiplikation, som summeret i særning \ref{thm:mxmulti}.

\begin{thm}{}{mxmulti}
Lad $A$ og $B$ være $k \times m$ matricer, $C$ være en $m \times n$ matrix, og $P$ og $Q$ være $n \times p$ matricer. Følgende udsagn er dermed sande:
\begin{enumerate}[label=(\alph*)]
\item $s(AC)=(sA)C=A(sC)$ for en vilkårlig skalar $s$.
\item $A(CP)=(AC)P$.
\item $(A+B)C=AC+BC$.
\item $C(P+Q)=CP+CQ$.
\item $I_kA=A=AI_m$.
\item Produktet af en nulmatrice og an vilkårlig matrice er altid $0$.
\item $(AC)^T=C^TA^T$.
\end{enumerate}
\end{thm}
\begin{proof}
Lad $A$ og $B$ være $k \times m$ matricer, $C$ være en $m \times n$ matrix, og $P$ og $Q$ være $n \times p$ matricer. 
(a) Den $(i,j)$'te indgang i matrixprodukterne $s(AC)$ og $(sA)C$ har formen $sa_{i,1}c_{1,j} + sa_{i,2}c_{2,j} + \cdots + sa_{i,m}c_{m,j}$, hvilket er det samme som $a_{i,1}sc_{1,j} + a_{i,2}sc_{2,j} + \cdots + a_{i,m}sc_{m,j}$, der er en indgang i $A(sC)$. 
Da indgangene er ens, er matrixproduktet det samme i de tre tilfælde.
(b) Bemærk, at $A$ ikke kan multipliceres med $P$, mens $C$ kan multipliceres med begge. 
Derfor kan matrixprodukterne $AC$ og $CP$ betragtes som to matricer, således at både $A(CP)$ og $(AC)P$ er $k \times p$ matricer. 
At komponenten i den $(i,j)$'te indgang i $A(CP)$ er ækvivalent med komponenten i den $(i,j)$'te indgang $(AC)P$ fås ved samme fremgangsmåde som vist i (a). 
%Måske er (b) ikke helt nok? 
(c) Lad $\textbf{c}_j$ være en vilkårlig søjle i $C$. 
Det følger deraf, at $(A+B)\textbf{c}_j=A\textbf{c}_j+B\textbf{c}_j$ jævnfør sætning \ref{thm:mxvpro}. 
(d) Lad $\textbf{p}_j$ og $\textbf{q}_j$ være søjler i henholdvis $P$ og $Q$. Det følger deraf, at $C(\textbf{p}_j+\textbf{q}_j)=C\textbf{p}_j+C\textbf{q}_j$ jævnfør sætning \ref{thm:mxvpro}. 
(e) Da $I_k$ og $I_m$ henholdsvis korresponderer til antallet af rækker og søjler i $A$, er matrixprodukterne $I_kA$ og $AI_m$ defineret. 
Da $I_k\textbf{a}_j=\textbf{a}_j$ og $I_m\textbf{a}_i=I_m\textbf{a}_i$, er $I_kA=A$ og $AI_m=A$, hvormed $I_kA=AI_m$.
(f) Er triviel, da enhver komponent i en vilkårlig matrix multipliceret med $0$ er $0$. 
(g) Den $(i,j)$'te indgang i $(AC)^T$ er den $(j,i)$'te indgang i $AC$. De to udtryks ækvivalens vises ved sammen fremgangsmåde som i (a).
\end{proof}
	


	\section{Lineære ligningsystemer }
% xD

	%Matrixvektorproduct
\begin{defn}{}{}
\phantom{gdfs}\\Lad $A$ være en $m$ $x$ $n$ matrix og \textbf{v} være en $n$ $x$ $1$ vektor. \textbf{Matrix-vektor-produktet} af A og \textbf{v} defineres som den lineære kombination af søjlerne i A hvis koefficienter er de tilsvarende komponenter af \textbf{v}, og noteres A\textbf{v}. Det vil sige
\begin{align*}
A\textbf{v} =v_1\textbf{a}_1 + v_2\textbf{a}_2 + \cdots + v_n\textbf{a}_n
\end{align*}
\end{defn}
\noindent
For at $A\textbf{v}$ eksisterer, skal der være lige mange komponenter i \textbf{v} som søjler i $A$. Et eksempel på et matrix-vektor-produkt mellem en matrix $B$ og en vektor \textbf{u} ses i eksempel \ref{Matrix-vektor}.
\begin{eks}\label{Matrix-vektor}
\begin{center} 
$$B=
\begin{blockarray}{c c c c}
\begin{block}{[c c c c]}
2 & 5 & 6 \\
4 & 7 & 3\\
1 & 4 & 1\\
2 & 3 & 8\\
\end{block}
\end{blockarray}
%
\text{ og }
%
\textbf{u}=
\begin{bmatrix}
4 \\
2 \\
3 \\ 
\end{bmatrix}
$$
Eftersom $B$ har 3 søjler og \textbf{u} har 3 komponenter, så eksisterer $B\textbf{u}$ og findes ved
$$
B\textbf{u}=
\begin{bmatrix}
2 & 5 & 6 \\
4 & 7 & 3\\
1 & 4 & 1\\
2 & 3 & 8\\
\end{bmatrix}
\begin{bmatrix}
4 \\
2 \\
3 \\ 
\end{bmatrix}
=4
\begin{bmatrix}
2\\
4\\
1\\
2\\
\end{bmatrix}
+2
\begin{bmatrix}
5\\
7\\
4\\
3\\
\end{bmatrix}
+3
\begin{bmatrix}
6\\
3\\
1\\
8\\
\end{bmatrix}
=
\begin{bmatrix}
8\\
16\\
4\\
8\\
\end{bmatrix}
+
\begin{bmatrix}
10\\
14\\
8\\
6\\
\end{bmatrix}
+
\begin{bmatrix}
18\\
9\\
3\\
24\\
\end{bmatrix}
=
\begin{bmatrix}
36\\
39\\
15\\
38\\
\end{bmatrix}
$$
\end{center}
\end{eks}


\begin{eks}
123
test for noget
\end{eks}	
	\section{Løsning til lineær ligningssystemer}


\subsection{Elementære rækkeopperationer}
Der eksisterer tre opperationer som kan udføres på rækker i et lineært ligningssystem.
Dette gøres mest når systemet er skrevet som en matrice.
De er ombytning, udskiftning og skalaring.

\begin{defn}{}{element}
\textbf{Ombytning} af rækker, bytter rundt på pladserne af to rækker og noteres
\begin{align*}
A \xrightarrow{R_i \leftrightarrow R_h} B, 
\end{align*}
hvor $A$ er matricen før opperationen, $B$ er matricen efter opperationen og $R_h$ og $R_i$ er rækker i $A$.\\
\textbf{Skalaring} betyder at en bestemt række skaleres med en ikke nul værdi, og noteres
\begin{align*}
A \xrightarrow{R_i \rightarrow cR_i} B, 
\end{align*}
\textbf{Udskiftning} af rækker betyder at en række udskiftes med rækken selv plus en skalaring af en anden række og noteres
\begin{align*}
A \xrightarrow{R_i \rightarrow R_i + cR_h} B
\end{align*}

\end{defn}

Som eksempel på elementære rækkeoperationer ses eksempler på dem i eksempel \thechapter .\ref{eks1}

\begin{eks}\label{eks1}
\begin{align*}
\begin{bmatrix}
5 & 5 & 3 \\
2 & 2 & 1\\
3 & 4 & 8
\end{bmatrix}
&\xrightarrow{R_1 \leftrightarrow R_3}
\begin{bmatrix}
3 & 4 & 8\\
2 & 2 & 1\\
5 & 5 & 3
\end{bmatrix}\\
%
%
\begin{bmatrix}
3 & 4 & 8\\
2 & 2 & 1\\
5 & 5 & 3
\end{bmatrix}
&\xrightarrow{R_2 \rightarrow -2R_2}
\begin{bmatrix}
3 & 4 & 8\\
-4 & -4 & -2\\
5 & 5 & 3
\end{bmatrix}\\
\begin{bmatrix}
3 & 4 & 8\\
-4 & -4 & -2\\
5 & 5 & 3
\end{bmatrix}
&\xrightarrow{R_3 \rightarrow 3R_1+R_3}
\begin{bmatrix}
3 & 4 & 8\\
-4 & -4 & -2\\
14 & 17 & 27
\end{bmatrix}
\end{align*}
\end{eks}


\subsection{Trappeform og reduceret trappeform}
En række kaldes for en nulrække hvis alle indgange er nul, og en ikke nul række hvis der bare er en indgang der er forskellig fra nul.
Dette bruges til definitionen af trappeform.
\begin{defn}{}{}
En matrix er på \textbf{trappeform}, hvis den opfylder følgende krav.
\itemize
\item Enhver ikke nul række ligger over alle nul rækker
\item Den første ikke nul indgang i en ikke nul række ligger i en søjle til højre for første ikke nul indgange i forrige række.
\item Hvis en søjle har den første ikke nul indgang i en række, så er alle indgangene i søjlen under den nul.
\end{defn}

\begin{eks}
Det kan ses på de to følgende matricer at $A$ er på trappeform, mens $B$ ikke er på trappeform, da den første indgang i række fire ikke er en nul indgang. Hvis der laves en række ombytning på række to og fire, så ville matrix $B$ også være på trappeform
\begin{align*}
A=
\begin{bmatrix}
2 & 4 & 8 & 2\\
0 & 5 & -2 & 5\\
0 & 0 & 2 & 7\\
0 & 0 & 0 & 0
\end{bmatrix}
\text{ }
B=
\begin{bmatrix}
2 & 4 & 8 & 2\\
0 & 5 & -2 & 5\\
0 & 4 & 2 & 7\\
2 & 0 & 0 & 0
\end{bmatrix}
\end{align*}

\end{eks}

\begin{defn}{}{}
En matrix er på \textbf{reduceret trappeform}, hvis den opfylder følgende krav.
\itemize
\item Hvis en søjle har den første ikke nul indgang i en række, så er alle indgange i søjlen nul.
\item Den første indgang i hver ikke nul række er 1. 
\end{defn}

\begin{eks}
Det kan ses på følgende matrix, at $A$ er på reduceret trappeform.
\begin{align*}
A=
\begin{bmatrix}
1 & 0 & 5 & 0 & 0\\
0 & 1 & 3 & 0 & 0\\
0 & 0 & 0 & 1 & 0\\
0 & 0 & 0 & 0 & 1
\end{bmatrix}
\end{align*}
\end{eks}

\subsection{Pivotindgange}
\begin{defn}{}{}
Pivotindgange betegnes som den første ikke nul indgang i enhver række i en matrix på trappeform. 
\end{defn}

\begin{align*}
A=
\begin{blockarray}{ccccc}
x_1 & x_2 & x_3 & x_4 & b \\
\begin{block}{[cccc|c]}
1 & 0 & 5 & 0 & 0\\
0 & 1 & 3 & 0 & 0\\
0 & 0 & 0 & 1 & 0\\
0 & 0 & 0 & 0 & 1\\
\end{block}
\end{blockarray}
\end{align*}

\begin{align*}
x_1+5x_3&=0\\
x_2+3x_3&=0\\
x_4&=0\\
0&=1
\end{align*}
	\section{Gauss elimination}
%sætning 1.4 kan vi bevise den
En algoritme til løsningen af linære ligningssystemer er  rækkereduktionsalgoritmen, som omdanner en matrix fra en totalmatrice til reduceret trappeform.
Når den bruges til at løse linære ligningssystemer, kaldes denne process for Gauss elimination.
I algoritmen bruges de elemæntere rækkeoperationer for at omdanne en matrix $A$ til trappeform. 
Herefter benyttes rækkeoperationerne igen til at omdanne matricen på trappeform til reduceret trappeform.
Algortimen gennemgår således følgende skridt:
%
\begin{enumerate}
\item Find første ikke-nul søjlse i $A$ fra venstre.
\item Ved rækkeombytning placeres en ikke-nul indgang øverst i pivot-søjlen.
\item Skab nuller under pivot-indgangen øverst i pivot-søjlen ved hjælp af rækkeudskiftning.
\item Øverste række markeres som afsluttet og trin $1-3$ gennemføres nu på den næste række.
\item Alle rækker med pivot-indgange skaleres så alle pivot-indgange er lig $1$.
\item Ved rækkeudskiftning sikres nu $0$er over og under pivot-indgangene.
\end{enumerate}
% Eksempel
Givet ligningssystemet:
\begin{align*}
x_1-x_3-2x_4-8x_5&=-3 \\
-2x_1+x_3+2x_4+9x_5&=5 \\
3x_1-2x_3-3x_4-15x_5&=-9 \\
\end{align*}
\[
\begin{blockarray}{cccccc}
a & b & c & d & e \\
\begin{block}{[ccccc]c}
  1 & 1 & 1 & 1 & 1 & f \\
  0 & 1 & 0 & 0 & 1 & g \\
  0 & 0 & 1 & 0 & 1 & h \\
  0 & 0 & 0 & 1 & 1 & i \\
  0 & 0 & 0 & 0 & 1 & j \\
\end{block}
\end{blockarray}
 \]

\end{document}

\begin{thm}{}{konsitens}
\end{thm}

% Appendicer indsættes inde i en appendices-blok og bliver nummereret med
% bogstaver i stedet for tal
\begin{appendices}

\end{appendices}

% Dokumentets 'back matter' er til ekstra ting som f.eks. litteraturlisten.
% Overskrifter bliver ikke nummereret her.
\backmatter

% Automatisk litteraturliste baseret på, hvilke kilder, der er blevet refereret
% til i løbet af rapporten.
\bibliographystyle{apalike}
\bibliography{
  incl/bib/books,
  incl/bib/articles,
  incl/bib/software
}

\end{document}
