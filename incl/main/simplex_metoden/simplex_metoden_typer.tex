%
\section{Implementering}
\label{julieergudesmuk}
\textcolor{red}{Metoden, hvor der bevæges fra en basal mulig løsning til en ny basal mulig løsning, hvor omkostningen minimeres ved hjælp af basale retninger i polyederet, kaldes simplexmetoden.}
Der findes tre forskellige implementeringer af simplexmetoden, henholdsvis den \textit{naive}, den \textit{reviderede}, og \textit{fuld-tabel} implementeringen.
%
\subsection{Den naive implementering}
I den naive implementering af simplexmetoden overgår ingen af de bærende elementer fra en iteration til den næste iteration. 
For den indledende iteration, haves indekserne
$B(1),\ldots,B(m)$ for de givne basale variable. 
Derudfra kan der dannes en basismatrix $B$ og beregnes $\mathbf{p}^T=\mathbf{c}_{\text{B}}^T B^{-1}$ ved at løse det lineære ligningssystem $\mathbf{p}^T B=\mathbf{c}_{\text{B}}^T$ for den ukendte vektor $\mathbf{p}$.
Den reducerede omkostning $c_j^* = c_j - \mathbf{c}_B^T B^{-1}A_j$ for enhver variabel $x_j$, er derfor givet ved formelen
%
\begin{align*}
c_j^* = c_j - \mathbf{p}^T A_j.
\end{align*}
%
Afhængigt af hvilken pivoteringsregel benyttes, er det nødvendigt at udregne alle de mulige reducerede omkostninger, \textcolor{red}{eller i så fald udregne indtil en negativ objektfunktion opstår.}
Når en søjle $\mathbf{A}_j$ er valgt til at være en del af basen, løses det lineære ligningssystem $B\mathbf{u}=\mathbf{A}_j$ for at bestemme vektoren $\mathbf{u}=B^{-1}\mathbf{A}_j$.
Ved hjælp af dette vil der kunne konstrueres den basal retning, hvor man bevæger sig væk fra de nuværende basale mulige løsninger.
Endeligt kan $$\theta^* = \max \{ \theta \geq 0 \mid \textbf{x} + \theta^*\textbf{d} \in \mathcal{P} \}$$ bestemmes, samt den variabel der gør, at basen forlades, og derved kan en ny basal mulig løsning konstrueres. 
En iteration af den naive simpleximplementering gennemløber følgende punkter: 
%
\begin{tcolorbox}[
title=Den naive implementering,
colback		= myblue!15,
colframe	= myblue!15,
coltitle	= black,
before skip	= 20pt plus 2pt,
after skip	= 20pt plus 2pt,
fonttitle	= \bfseries]
\begin{enumerate}
\item En normal iteration starter med et basis bestående af de basale rækker $\textbf{A}_{B(1)},\ldots,\textbf{A}_{B(m)}$ og den tilhørende basale mulige løsning $\textbf{v}$.
\item Beregn den reducerede omkostning $c_j^* = c_j - \mathbf{c}_B^T \textbf{B}^{-1}A_j$ for alle ikke-basale indekser $j$.
Hvis alle er ikke-negative, så er den nuværende basale mulige løsning optimal; ellers vælges et indeks $j$, hvorom det gælder, at $c^*_j<0.$
\item Udregn $\textbf{u}=\textbf{B}^{-1}\textbf{A}_j$.
Hvis ingen komponent i $\textbf{u}$, er positiv, så er $\theta ^*=\infty$, den optimale løsningsværdi er $-\infty$, og algorithmen stopper.
\item Hvis en komponent i $\textbf{u}$ er positiv, så lad 
$$\theta^*= \min_{ \{i=1,\ldots,m|u_i>0 \} }        \dfrac{x_{B(i)}}{u_i}.$$
\item Lad $l$ være indekset for $\theta^*=  \dfrac{x_{B(l)}}{u_l}$.
Dan en nyt base ved at udskifte $\textbf{A}_{B(l)}$ med $\textbf{A}_j$.
Den nye basale mulige løsning, $\textbf{w}$, har så de nye basale variable $w_j=\theta^*$ og $w_{B(i)}=v_{B(i)}-\theta^*u_i$ for $i\neq l.$
%Hvis $\textbf{w}$ er den nye basale mulige løsning, så er værdierne af de nye basale variable $y_j=\theta^*$ og $y_{B(i)}=x_{B(i)}-\theta^*u_i,i\neq l.$
\end{enumerate}
\end{tcolorbox}
%
\subsection{Den reviderede implementering}
Problemet ved den naive implementering er dens potentielt høje tidskompleksitet, da den er nødsaget til at løse to lineære ligningssystemer.
En alternativ implementering til dette er den revidere implementering, som benytter matricen $B^{-1}$ ved hver iteration, samt er vektorerne $\mathbf{c}_{\text{B}}^T B^{-1}$ og $B^{-1} \mathbf{A}_j$ udregnet ved hjælp af matrix-vektor multiplikation.
En iteration af den reviderede simpleximplementering gennemløber følgende punkter: 
% 
\begin{tcolorbox}[
title=Den reviderede implementering,
colback		= myblue!15,
colframe	= myblue!15,
coltitle	= black,
before skip	= 20pt plus 2pt,
after skip	= 20pt plus 2pt,
fonttitle	= \bfseries]
\begin{enumerate}
\item I en typisk iteration startes med en basis indeholdende basissøjlerne $A_{B(1)},\ldots,A_{B(m)}$, en associeret basal mulig løsning $\mathbf{v}$, samt en invers $B^{-1}$ af basismatricen. 
\item Udregn rækkevektoren $\mathbf{p}^T=\mathbf{c}_{\text{B}}^T B^{-1}$ og udregn dernæst den reducerede objektfunktion $c_j^* = c_j - \mathbf{p}^T \mathbf{A}_j$.
Hvis resultaterne alle er ikke-negative, vil det resultere i at den givne basal mulige løsning er optimal, og algoritmen vil slutte ved dette punkt. Hvis dette ikke er tilfældet, så vælg et $j$, hvor $c_j^* < 0$.
\item Udregn $\mathbf{u}=B^{-1}\mathbf{A}_j$.
Hvis intet komponent i $\mathbf{u}$ er positivt, så er den optimale pris $-\infty$, og algoritmen vil slutte ved dette trin. 
%
% Cost er vel ligmed pris her?? Eller hvad? %
%
\item Hvis mindst en komponent i $\mathbf{u}$ er positivt, lad 
\begin{align*}
\theta^*=\underset{\{i=1,\ldots,m \mid u_i>0\}}{\text{min}}\frac{x_{B(i)}}{u_i}.
\end{align*}
\item Bestem $l$, så $$\theta^*=\frac{x_{B(l)}}{u_l}$$ er gældende. Bestem dernæst en ny base ved at udskifte $\mathbf{A}_{B(l)}$ med $\mathbf{A}_j$. Hvis $\mathbf{w}$ er den nye basal mulige løsning, så er værdierne for de nye basale værdier $w_j=\theta^*$ samt $w_{B(i)}=v_{B(i)}-\theta^*u_i$, hvor $i \neq l$.
\item Endeligt bestem $m \times (m+1)$ matricen $\left [B^{-1} \mid \mathbf{u} \right ]$.
Tilføj enhver række et multiplum af den $l$'te række for at få den sidste søjle til at være lig med enhedsvektoren $\mathbf{e}_l$.
De første $m$ søjler er resultatet af matricen ${B^*}^{-1}$.
%Nogle der har en ide til at gøre dette pænere? Det er B-stjerne i -1% 
\end{enumerate}
\end{tcolorbox}
\noindent
%
%
\subsection{Fuld-tabel implementeringen}
Slutteligt er det interessant at beskrive fuld-tabel implementeringen. 
Det er gældende for fuld-tabel implementeringen, at den i stedet for at opretholde og opdatere matricen $B^{-1}$ ved hver iteration, som i den reviderede implementering; 
så opretholder og opdaterer den $m \times (n+1)$ matricen, givet ved $B^{-1} \left [ A \mid \mathbf{b} \right ]$, med søjlerne $B^{-1}\mathbf{b}$ og $B^{-1}\mathbf{A}_1,\ldots,B^{-1}\mathbf{A}_n$. 
Denne matrix betegnes som en simplextabel.
Søjlen $B^{-1}\mathbf{b}$ indeholder værdierne for de basale variable.
Søjlen $B^{-1}\mathbf{A}_i$ betegnes som den $i$'te søjle af tabellen.
Søjlen $\mathbf{u} = B^{-1}\mathbf{A}_j$ er svarende til den variabel, der er i basen, er betegnet som pivotsøjlen. 
Hvis den $l$'te basale variabel udgår fra basen, så er den $l$'te række, betegnet som pivotrækken. 
Endelig er elementet, der tilhører både pivotrækken og pivotsøjlen, betegnet ved pivoteringsindgangen.
Bemærk desuden, at dette element er $u_l$ og altid er positivt, da algoritmen ellers var stoppet i trinnet før. 
For at klarlægge hvordan fuld-tabel implementeringen fungerer, vil der i følgende blive gennemgået en iteration: 
%
\begin{tcolorbox}[
title=Fuld-tabel implementering,
colback		= myblue!15,
colframe	= myblue!15,
coltitle	= black,
before skip	= 20pt plus 2pt,
after skip	= 20pt plus 2pt,
fonttitle	= \bfseries]
\begin{enumerate}
\item I en typisk iteration, startes der med at opstille en simplextabel associeret med en basismatrix $\mathbf{B}$ og en tilsvarende basal mulig løsning $\mathbf{v}.$
\item Dernæst undersøges om den reducerede værdi i objektfunktionen i simplextabellen.
Hvis alle indgange er ikke-negative, så er den nuværende basal mulige løsning den optimale, og algoritmen vil slutte ved dette punkt.
Hvis dette ikke er tilfældet, så vælg ethvert $j$, hvor $c_j^* < 0$.
\item Betragt vektoren $\mathbf{u}=\mathbf{B}^{-1}\mathbf{A}_j$, hvor den $j$'te søjle, som er pivotsøjlen, i tabellen. Hvis intet komponent i $\mathbf{u}$ er positivt, så er den optimale pris $-\infty$, og algoritmen vil slutte ved dette trin. 
\item For hvert $i$, hvor $u_i$ er positivt, udregn forholdet $$\frac{x_{B(i)}}{u_i}.$$ Lad $l$ være det index  af en række, som tilsvarer det laveste forhold.
Søjlen $\mathbf{A}_{B(l)}$ udgår fra basen og søjlen $\mathbf{A}_j$ indgår i basen. 
\item Tilføj for hver række i tabellen en konstant multiplum af den $l$'te række, som er pivotrækken, så $u_l$, som er pivoteringsindgangen bliver lig med $1$ og alle andre indgange i pivotsøjlen bliver $0$. 
\end{enumerate}
\end{tcolorbox}
\noindent
%
Bemærk, at det er denne implementering, som er brugt i afsnit \ref{afsnittet}.