Som nævnt tidligere i rapporten, så findes der en optimal løsning til et lineært ligningssystem på standard form, så længe der findes mindst én basal mulig løsning. 
\textit{Simplex metoden} benytter sig af denne egenskab og forsøger at finde den optimale løsning for en lineært ligningssystem på standard form.
%
% -------------------------------------------------------------------------
% Noget introducerende
%
\section{Praktisk anvendelse af Simplex-metoden}
Simplex-metoden foretager som udgangspunkt følgende seks operationer, inden den afsluttes: 
%
\begin{col}{}{}
%
% Denne skal eventuelt beskrives mere i dybdegående - Tager kort tid.
\begin{enumerate}
\item Opskriv optimeringsproblemet på standardform med slack-variabler.  %1
\item Opstil simplex-tabellen for optimeringsproblemet.					 %2
\item Tjek optimering og identificér en pivotindgang.					 %3
\item Opstil en ny tabel ved hjælp af pivotering. 						 %4
\item Gentag operation 3 og 4, indtil den optimale løsning er fundet. 					 %5
\item Identificér den optimale løsning.									 %6
\end{enumerate}
%
\end{col}
\noindent
%
I følgende afsnit uddybes ovenstående punkter, samt tilhørende teori, med udgangspunkt i eksempel \ref{haribooooo}.
%
\\
%
\begin{eks}
\label{haribooooo}
Haribo har to typer slikblandinger $x_1$ og $x_2$, hvor profitten for disse er henholdvis $3$ og $4$ enheder(her kunne der også skrives tusind/millioner whatever ingen anelse om realistisk skala i haribos profitmargin)
Disse er begrænset kapiciteten af lakridsproduktion som har en begrænsning på $78$ enheder (igen ton whatever??) og vingummi som har en begrænsning på $36$ enheder, da disse indgår i begge slikblandinger.
%igen jeg er ikke sikker på om det også fungerer aok med din udgave Julie, den har jeg bare sværere ved at forstå.
%
Haribo ønsker at finde det mest optimale forhold mellem de to vingummismage i $"$Vingummi hindbær og brumbær$"$-blandingen, der ses på figur \ref{mums}, med udgangspunkt i en række betingelser. 
De ønsker at maksimere mængden af slik, begrænset af kulhydrat- og sukker-indholdet.
%
Optimeringsproblemet er dermed følgende:
%
\begin{align*}
\begin{array}{lrrlr} 
\text{Maksimer}		&	\multicolumn{2}{c}{z=5x_1+4x_2}  &\\
\text{begrænset af}	&3x_1& +5x_2			&\leq 	&78,\\
					&4x_1& + x_2				&\leq	& 36,\\
					&x_1& , x_2				&\geq	& 0.
\end{array}
\end{align*}
%
\end{eks}

\subsubsection{1. Opskriv optimeringsproblemet på standardform med slack-variabler.}
%
Generelt tager simplex-modellen udgangspunkt i, at alle variablerne er positive. Jævnfør afsnit \ref{sec:standard} tilføjes $x_i^+$ og $x_i^-$, hvis der i optimeringsproblemet ikke er ikke-negativitetsbetingelser for variablerne. 
I dette tilfælde er begge variabler positive, og det er derfor ikke nødvendigt at opdele variablerne. 
Med udgangspunkt i metoden fra afsnit \ref{sec:standard} opstilles optimeringsproblemet derfor på standardform som
%
\begin{align*}
\begin{array}{lrrrrrlr}
\text{Maksimér}		& -5x_1 &-4x_2 &&& + z & =0\\
\text{begrænset af}	&3x_1& +5x_2	& + \textcolor{blue}{s_1} 	&&&= 	&78,\\
					&4x_1& + x_2	& & + \textcolor{blue}{s_2}	&&=	&	 36.\\
\end{array}
\end{align*}
%Hvor kommer 78 og 36 fra? 
Dog er maksimeringsproblemet omskrevet med udgangspunkt i 
\begin{align*}
-\textbf{c}^T\textbf{x} + z	 =0,
\end{align*}
hvor $z$ er den optimale værdi. 
%
Slack-variablerne $s_1$ og $s_2$ er her markeret med blå, da de i sidste ende ikke tilføjes til løsningen. 
%
\subsubsection{2. Opstil simplex-tabellen for optimeringsproblemet}		
% 
Nu opstilles \textit{simplex-tabellen} for optimeringsproblemet. 
Med udgangspunkt i et generelt lineært optimeringsproblem på formen
%
% & =v, \text{ hvor } v=0
\begin{align*}
\begin{array}{lrl}
\text{Maksimér}		&-\textbf{c}^T\textbf{x} + z	& =0	\\
\text{begrænset af}	&A\textbf{x}	&=\mathbf{b},	\\
					&\mathbf{x}				&\geq \mathbf{0},
\end{array}
\end{align*}
er simplex-tabellen en matrix, som indeholder de lineære betingelser, slack-variablerne og objektfunktionen. 
Matricen $\mathbf{A}$ opskrives først, og derefter opskrives identitesmatricen $e_{m+1}$, hvor $m$ er mængden af slack-variabler, samt den optimale løsning $z$, og til sidst opskrives $\mathbf{b}$. 
Under $\mathbf{A}$ tilføjes matricen $- \mathbf{c}^T$. 
Med udgangspunkt i forrige optimeringsproblem ser en generel formel i simplex-tabellen dermed ud som følgende:
%
\begin{align*}
\begin{blockarray}{ccccccccccc}
x_1 & x_2 & \cdots & x_n & \textcolor{blue}{s_1} & \textcolor{blue}{s_2} &  \textcolor{blue}{\cdots} & \textcolor{blue}{s_m} & z & b \\
\begin{block}{[cccc|ccccc|c]c}
a_{1,1} & a_{1,2} & \cdots & a_{1,n} & 1 & 0 & \cdots & 0 & 0 & b_1 \\
a_{2,1} & a_{2,2} & \cdots & a_{2,n} & 0 & 1 & \cdots & 0 & 0 & b_2 \\
\vdots & \vdots & \ddots & \vdots & \vdots & \vdots & \ddots & \vdots & \vdots & \vdots \\
a_{m,1} & a_{m,2} & \cdots & a_{m,n} & 0 & 0 & \cdots  & 1  & 0 & b_{m}\\
\cline{1-10}
-c_1 & -c_2 & \cdots & -c_n & 0 & 0 & \cdots & 0 & 1 & v\\
\end{block}
\end{blockarray}.
\end{align*}
%
Optimeringsproblemet fra \ref{haribooooo} har dermed følgende simplex-tabel:
%
\begin{align*}
\begin{blockarray}{cccccc}
x_1 & x_2 & \textcolor{blue}{s_1} & \textcolor{blue}{s_2} & z & b \\
\begin{block}{[cc|ccc|c]}
3 & 5 & 1 & 0 & 0 & 78 \\
4 & 1 & 0 & 1 & 0 & 36 \\
\cline{1-6}
-5 & -4 & 0 & 0 & 1 & 0\\
\end{block}
\end{blockarray}.
\end{align*}
%
\subsubsection{3. Tjek optimering og identificér en pivotindgang}
%
Først tjekkes der efter optimering ved at finde det største negative tal i objektfunktionen, hvilket i dette tilfælde findes i nederste række i simplex-tabellen:
%
\begin{align*}
\begin{blockarray}{cccccc}
x_1 & x_2 & \textcolor{blue}{s_1} & \textcolor{blue}{s_2} & z & b \\
\begin{block}{[cc|ccc|c]}
3 & 5 & 1 & 0 & 0 & 78 \\
4 & 1 & 0 & 1 & 0 & 36 \\
\cline{1-6}
\hlight{-5} & -4 & 0 & 0 & 1 & 0\\
\end{block}
\end{blockarray}.
\end{align*}
%
Søjlen, hvori denne værdi er, kaldes \textit{pivot-søjlen}. 
Herefter findes pivotindgangen og \textit{pivot-rækken}, ved at finde den mindste $x_i$-værdi ud fra 
\begin{align*}
\frac{a_{i,j}}{b_j}=x_i
\end{align*}
%
i pivot-søjlen.
For eksemplet findes pivotindgangen
%
\begin{align*}
\frac{78}{3} =26 \text{  } \text{   } \frac{36}{4} =9.
\end{align*}
%
Eftersom $9$ er den laveste værdi, er $4$ pivotindgangen, mens den række, som den er i, er pivot-rækken. 
I nedenstående er pivotindgangen markeret.
%
\begin{align*}
\begin{blockarray}{cccccc}
x_1 & x_2 & \textcolor{blue}{s_1} & \textcolor{blue}{s_2} & z & b \\
\begin{block}{[cc|ccc|c]}
3 & 5 & 1 & 0 & 0 & 78 \\
\hlight{4} & 1 & 0 & 1 & 0 & 36 \\
\cline{1-6}
-5 & -4 & 0 & 0 & 1 & 0\\
\end{block}
\end{blockarray}.
\end{align*}	
%	
\subsubsection{4. Opstil en ny tabel ved hjælp af pivotering}
%
Med udgangspunkt i de elementære rækkeoperationer per definition \ref{defn:element}, samt Gauss-elimination jævnfør afsnit \ref{gauss}, skaleres pivotindgangen til $1$, og der skabes nuller under og over pivotindgangen ved hjælp af rækkeudskiftning.
Nedenfor ses dette gjort for eksemplet.
%
%& \begin{blockarray}{cccccc}
%x_1 & x_2 & \textcolor{blue}{s_1} & \textcolor{blue}{s_2} & z & b \\
%\begin{block}{[cc|ccc|c]}
%\hlight{3} & 5 & 1 & 0 & 0 & 78 \\
%4 & 1 & 0 & 1 & 0 & 36 \\
%\cline{1-6}
%\hlight{-5} & -4 & 0 & 0 & 1 & 0\\
%\end{block}
%\end{blockarray} \\
%
\begin{align*}
\xrightarrow[]{R_2 \rightarrow \frac{1}{4} R_2} &
\begin{blockarray}{cccccc}
x_1 & x_2 & \textcolor{blue}{s_1} & \textcolor{blue}{s_2} & z & b \\
\begin{block}{[cc|ccc|c]}
3 & 5 & 1 & 0 & 0 & 78 \\
1 & \frac{1}{4} & 0 & \frac{1}{4} & 0 & 9 \\
\cline{1-6}
-5 & -4 & 0 & 0 & 1 & 0\\
\end{block}
\end{blockarray} \\
\xrightarrow[R_3 \rightarrow R_3 + 5 R_2 ]{R_1 \rightarrow R_1 -3 R_2} &
\begin{blockarray}{cccccc}
x_1 & x_2 & \textcolor{blue}{s_1} & \textcolor{blue}{s_2} & z & b \\
\begin{block}{[cc|ccc|c]}
0 & \frac{17}{4} & 1 & \frac{-3}{4} & 0 & 51 \\
1 & \frac{1}{4} & 0 & \frac{1}{4} & 0 & 9 \\
\cline{1-6}
0 & \frac{-11}{4} & 0 & \frac{5}{4} & 1 & 45\\
\end{block}
\end{blockarray}.
\end{align*}	
%
\subsubsection{5. Kør 3 og 4 indtil den optimale løsning er fundet}
%
Denne proces forsættes nu, indtil der ikke er flere negative tal i objektfunktionen.
\textit{Pivotering} er processen fra valget af den største negative værdi i objektfunktionen til at en ny værdi i objektfunktionen kan vælges. 
Der tjekkes derfor efter optimering igen, og en ny pivotindgang er valgt ved den største negative værdi. 
Denne er markeret i nedenstående matrix.	
%Jeg tænker, at ovenstående kan præciseres, men jeg er blank pt.... Så hvad ved jeg :0) 
\begin{align*}
\begin{blockarray}{cccccc}
x_1 & x_2 & \textcolor{blue}{s_1} & \textcolor{blue}{s_2} & z & b \\
\begin{block}{[cc|ccc|c]}
0 & \frac{17}{4} & 1 & \frac{-3}{4} & 0 & 51 \\
1 & \frac{1}{4} & 0 & \frac{1}{4} & 0 & 9 \\
\cline{1-6}
0 & \hlight{\frac{-11}{4}} & 0 & \frac{5}{4} & 1 & 45\\
\end{block}
\end{blockarray}
\end{align*}
%
Pivotindgangen findes ved
%
\begin{align*}
\frac{51}{\frac{17}{4}} =12 \text{  } \text{   } \frac{9}{\frac{1}{4}} =36.
\end{align*}
%
Pivotindgangen er $\frac{17}{4}$, da denne giver den mindste værdi på $12$, hvilket er markeret i nedenstående matrix.
%
\begin{align*}
\begin{blockarray}{cccccc}
x_1 & x_2 & \textcolor{blue}{s_1} & \textcolor{blue}{s_2} & z & b \\
\begin{block}{[cc|ccc|c]}
0 & \hlight{\frac{17}{4}} & 1 & \frac{-3}{4} & 0 & 51 \\
1 & \frac{1}{4} & 0 & \frac{1}{4} & 0 & 9 \\
\cline{1-6}
0 & \frac{-11}{4} & 0 & \frac{5}{4} & 1 & 45\\
\end{block}
\end{blockarray}
\end{align*}
%
Pivotindgangen skaleres til $1$, og der skabes nuller under og over pivotindgangen, ved de elementære regneregler:
%& \begin{blockarray}{cccccc}
%x_1 & x_2 & \textcolor{blue}{s_1} & \textcolor{blue}{s_2} &z & b \\
%\begin{block}{[cc|ccc|c]}
%0 & \frac{17}{4} & 1 & \frac{-3}{4} & 0 & 51 \\
%1 & \hlight{\frac{1}{4}} & 0 & \frac{1}{4} & 0 & 9 \\
%\cline{1-6}
%0 & \hlight{\frac{-11}{4}} & 0 & \frac{5}{4} & 1 & 45\\
%\end{block}
%\end{blockarray}\\
%
\begin{align*}
\xrightarrow[]{R_1 \rightarrow \frac{4}{17} R_1} &
\begin{blockarray}{cccccc}
x_1 & x_2 & \textcolor{blue}{s_1} & \textcolor{blue}{s_2} & z & b \\
\begin{block}{[cc|ccc|c]}
0 & 1 & \frac{4}{17} & \frac{-3}{17} & 0 & 12 \\
1 & \frac{1}{4} & 0 & \frac{1}{4} & 0 & 9 \\
\cline{1-6}
0 & \frac{-11}{4} & 0 & \frac{5}{4} & 1 & 45\\
\end{block}
\end{blockarray} \\
\xrightarrow[R_3 \rightarrow R_3 + \frac{11}{4} R_2 ]{R_2 \rightarrow R_2 -\frac{11}{4} R_1} &
\begin{blockarray}{cccccc}
x_1 & x_2 & \textcolor{blue}{s_1} & \textcolor{blue}{s_2} & z & b \\
\begin{block}{[cc|ccc|c]}
0 & 1 & \frac{4}{17} & \frac{-3}{17} & 0 & 12 \\
1 & 0 & \frac{-1}{17} & \frac{20}{17} & 0 & 6 \\
\cline{1-6}
0 & 0 & \frac{11}{17} & \frac{52}{17} & 1 & 78\\
\end{block}
\end{blockarray}.
\end{align*}	
%
%
\subsubsection{6. Identificér den optimale løsning}
%
Eftersom der ikke er flere negative værdier i objektfunktionen, er den optimale løsning nu fundet. 
Denne løsning kan aflæses direkte af simplex-tabellen. 
Den optimale løsning for eksemplet er dermed
%
\begin{align*}
x_1 & = 6, \\
x_2 & = 12, \\
z   & = 78.
\end{align*}
%