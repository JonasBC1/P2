\section{Kompleksitet}
% he
Det er interessant at undersøge simplex-metodens tidskompleksitet.
Til beskrivelse af dette benyttes \textit{store-$O$} notation.
%
\begin{defn}{}{}
Lad $f(x)$ og $g(x)$ være vilkårlige funktioner. $f(x)$ er \textbf{store-$O$} af $g(x)$, hvis der eksisterer et $C$ og $k$, således at $|f(x)| \leq C|g(x)|, x \geq k$. $C$ og $k$ kaldes \textbf{vidner}.
\end{defn}\noindent
%
Der er flere måder at implementere simplex-metoden på, der hver har sin egen kompleksitet.
De to implementeringer, der vil blive betragtet i dette afsnit, er \textit{fuld tabel} og \textit{den reviderede} simplex-metode.
%
"Fuld tabel"-metoden kræver et konstant antal af operationer for at opdatere indgangene i simplex-matricen.
Derfor er antallet af operationer proportionelt med størrelsen på matricen og kompleksiteten $O(mn)$.
Den reviderede metode bruger lignende operationer, men opdaterer kun $O(m^2)$ indgange og har derfor kompleksiteten $O(m^2)$.
Det kan dog forekomme, at metoden skal opdatere alle variabler.
Hver udregning af en variabel kræver $O(m)$ operationer, hvilket medfører, at der i værste tilfælde bruges $O(mn)$ operationer.
Da $m \leq n$, er kompleksiteten i værste tilfælde $O(mn)$ for begge metoder.
Med dette konkluderes det, at den reviderede metode aldrig vil køre langsommere end "fuld tabel"-metoden.
Dog vil en iteration fra den reviderede metode være hurtigere i alle andre tilfælde end det værste.
Endnu et vigtigt element, som er tiltalende ved den reviderede metode, er rumkompleksiteten på $O(m^2)$, hvor "fuld tabel"-metodens er på $O(mn)$.
Her kan der igen være en ret betydelig forskel, afhængigt af størrelsen på $n$.