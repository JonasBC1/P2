\section{Kompleksitet}
\label{kompleksitet}
% hehi
Et vigtigt kriterium for valg af simplexmetoden er tidskompleksitet.
Til beskrivelse af dette benyttes \textit{store-$O$} notation.
%
\begin{defn}{}{}
Lad $f(x)$ og $g(x)$ være vilkårlige funktioner. $f(x)$ er \textbf{store-$O$} af $g(x)$, hvis der eksisterer et $C$ og $k$, således at $|f(x)| \leq C|g(x)|, x \geq k$. $C$ og $k$ kaldes \textbf{vidner}.
\end{defn}
\noindent
%
Som beskrevet i afsnit \ref{julieergudesmuk} findes der forskellige implementeringer af simplexmetoden.
Disse har hver deres tidskompleksitet og rumkompleksitet.\\\\
%
Fuld-tabel metoden kræver et konstant antal af operationer for at opdatere indgangene i simplextabellen.
Derfor er antallet af operationer proportionelt med størrelsen på tabellen, og dermed er tidskompleksiteten og rumkompleksiteten $O(mn)$.\\\\
%Aretmestiske operationer betyder bare udregninger
Den naive implementering behøver $O(m^3)$ aritmetiske operationer for at løse systemerne $\mathbf{p}^T B=\mathbf{c}_{\text{B}}^T$ og $B\mathbf{u}=\mathbf{A}_j$.
Ydermere kræver udregningen af den reducerede objektfunktion %gældende for alle variabler, 
$O(mn)$ aritmetiske operationer, da det er nødvendigt at udregne det indre produkt af vektoren $\mathbf{p}$ for alle ikke-basissøjler i $\mathbf{A}_j$.
Derfor er den samlede tidskompleksitet $O(m^3 + mn) = O(m^3)$ for den naive implementering.\\\\
%
Den reviderede metode bruger samme operationer som fuld-tabel implementeringen, men opdaterer kun $O(m^2)$ indgange og har derfor tidskompleksiteten $O(m^2)$.
Det kan dog forekomme, at metoden skal opdatere alle variable.
Hver udregning af en variabel kræver $O(m)$ operationer, hvilket medfører, at der i værste tilfælde bruges $O(mn)$ operationer.
Da $m \leq n$, er tidskompleksiteten i værste tilfælde $O(mn)$.
Med dette konkluderes, at den reviderede metode aldrig vil køre langsommere end fuld-tabel og den naive implementering.
Samtidigt vil en iteration fra den reviderede metode være hurtigere end fuld-tabel i alle andre tilfælde end det værste.
Endnu et vigtigt element, som er tiltalende ved den reviderede metode, er rumkompleksiteten på $O(m^2)$.
Her kan der igen være en ret betydelig forskel, afhængigt af størrelsen på $n$.\\\\
%
Selvom antallet af basale mulige løsninger kan stige eksponentielt med antallet af variable og betingelser, så har simplexmetoden typisk fundet en optimal løsning efter $O(m)$ iterationer.
Dog eksisterer der polyedre, hvor antallet af påkrævede pivoteringer  stiger eksponentielt.
For ikke-degenererede problemer bevæger simplexmetoden sig altid fra et ekstremumspunkt til en af dets naboer og forbedrer omkostningen.
Lad et polyeder have et eksponentielt antal af ekstremumspunkter, og lad en sti besøge alle disse.
Denne sti er konstrueret ved, fra et ekstremumspunkt, at besøge et ekstremumspunkt med mindre omkostning.
Med sådan et polyeder vil simplexmetoden, med en pivoteringsregel, som følger stien, kræve et eksponentielt antal af pivoteringer.
%%
\begin{thm}{}{}
Betragt det lineære optimeringsproblem, hvor objektfunktionen $-x_n$ minimeres med betingelserne $\varepsilon \leq x_1 \leq 1$ og $\varepsilon x_{i-1} \leq x_i \leq 1 - \varepsilon x_{i-1}$ for $i = 2, 3, \ldots, n$ og $0 < \varepsilon < \frac{1}{2}$.
Så gælder det, at
%
\begin{enumerate}[label=(\alph*)]
\item Løsningsmængden har $2^n$ ekstremumspunkter.
\item Ekstremumspunkterne kan arrangeres således, at hver er nabo til og har lavere værdi end det forrige ekstremumspunkt.
\item Der eksisterer en pivoteringsregel til simplexmetoden, således at den kræver $2^n-1$ pivoteringer, før algoritmen afsluttes.
\end{enumerate}
%
\end{thm}
%
\begin{proof}
%
Lad $\mathcal{P}_n$ være et polyeder i $\R^n$, begrænset af betingelserne $\varepsilon \leq x_1 \leq 1$ og $\varepsilon x_{i-1} \leq x_i \leq 1 - \varepsilon x_{i-1}$ for $i = 2, 3, \ldots, n$ og $0 < \varepsilon < \frac{1}{2}$.
%
\begin{enumerate}[label = (\alph*)]
\item Ved induktion vises, at $\mathcal{P}_n$ har $2^n$ ekstremumspunkter. 
Basistrinnet, hvor $n=1$, følger trivielt.
Antag nu, at resultatet gælder for $n=k$.
Polyederet $\mathcal{P}_{k+1}$ er begrænset af samme betingelser som $\mathcal{P}_{k}$ med tilføjelse af betingelsen $\varepsilon x_{k} \leq x_{k+1} \leq 1 - \epsilon x_{k}$.
Den øvre og nedre begrænsning i $\varepsilon x_{k} \leq x_{k+1} \leq 1 - \varepsilon x_{k}$ kan ikke skære hinanden, mens $ 0 \leq x_k \leq 1$ er opfyldt.
Dette medfører, at hvert ekstremumspunkt i $\mathcal{P}_{k}$ giver anledning til to entydige ekstremumspunkter i $\mathcal{P}_{k+1}$, hvilket viser (a).
\item Med induktion vises det, at ekstremumspunkterne i $\mathcal{P}_n$ kan arrangeres således, at hvert ekstremumspunkt er nabo til og har mindre værdi end det forrige ekstremumspunkt.
Basistrinnet for $n=1$ følger trivielt.
Antag nu, at resultatet gælder for  $n = k$.
$\mathcal{P}_{k+1}$ kan deles i to $\mathcal{P}_{k}$ polyedre.
Et polyeder, hvor $\varepsilon x_{k} \leq x_{k+1}$ er aktiv og et polyeder, hvor $x_{k+1} \leq 1 - \varepsilon x_{k}$ er aktiv.
Således er det muligt at finde den ønskede udspændende sti i $\mathcal{P}_{k+1}$ ved først at tage den udspændende sti i $\mathcal{P}_{k}$, hvor $\varepsilon x_{k} \leq x_{k+1}$ er aktiv, og dernæst at skifte til $\mathcal{P}_{k}$, hvor $x_{k+1} \leq 1 - \varepsilon x_{k}$ er aktiv og tage dennes udspændende sti i modsat rækkefølge.
Dette viser (b), da $x_k$ er strengt voksende i det første $\mathcal{P}_{k}$ og strengt faldende i det andet $\mathcal{P}_{k}$, hvilket medfører, at $x_{k+1}$ er strengt voksende gennem hele stien.
%
\item Jævnfør \textcolor{red}{afsnit \ref{julieerfantalastiskogvidunderlig}} bevæger simplexmetoden sig altid fra nabo til nabo, når problemet ikke er degenereret.
Med udgangspunkt i (b) vides det, at der eksisterer en udspændende sti, som ender i problemets optimale løsning og $\textbf{c}^T\textbf{x}_i < \textbf{c}^T\textbf{x}_{i-1}$ for $i = 2, 3, \ldots ,n$.
Dette medfører, at en pivoteringsregel, som mindsker omkostningen mindst muligt pr. pivotering, følger den udspændende sti og besøger alle $2^n$ ekstremumspunkter, hvilket viser (c).
\end{enumerate}
\end{proof}
%
Således er kompleksitet et vigtigt kriterium for valg af simplexmetode.
Ligeledes er valg af pivoteringsregel afgørende, eftersom det i praktiske applikationer er de afgørende faktorer for algoritmens brugbarhed.
%Køb en Yamaha.