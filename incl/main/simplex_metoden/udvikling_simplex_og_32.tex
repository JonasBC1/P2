Det skal nu vises, hvordan simplex-metoden kan udvikles ud fra geometriske repræsentationer.
Altså, hvordan der udvikles en metode til at gå fra en profitabel basal mulig løsning til en bedre basal mulig løsning.
Det antages indtil videre, at løsningen er ikke-degenereret.
Antag, at der stås i punktet for den basale mulige løsning $\mathbf{x}$ og at den reducerede objektfunktion $c_j$ er udregnet, hvis alle variable i denne er ikke-negative, haves det fra sætning (3.1 i bertsimas), at løsningen er den optimale, og der behøves derfor ikke yderligere skridt.
Såfremt dette ikke er tilfældet, altså når en ikke-basal variabel $x_j$ i den reducerede objektfunktion er negativ, ville en bedre løsning potentielt forefindes i retningen $\mathbf{d}$, givet ved  $d_j=1$, $d_i=0$ for $i \neq B(1),\ldots,B(m),j$ og $\mathbf{d}_b=-B^{-1}A_j$.
Hvis der foregår en bevægelse i retning $\mathbf{d}$, bliver $x_j$ positiv, imens de øvrige ikke-basale variable forbliver $0$.
Denne bevægelse kan beskrives med $\mathbf{x}+\theta \mathbf{d}$. 
Da værdien af objektfunktionen er aftagende i retning $\mathbf{d}$, er det ønskværdigt at følge $\mathbf{d}$, så langt som muligt med henblik på at ende i punktet $\mathbf{x}+\theta^{*} \mathbf{d}$, hvilket kan udtrykkes som 
$\theta^{*}=\text{max} \theta \geq 0 \mid \mathbf{x}+\theta \mathbf{d} \in \mathcal{P}$.
Forandringen i objektfunktionen er således $\theta^{*}\mathbf{c}^T \mathbf{d}=\theta^{*}c_j$.
Der skal herfra udledes en formel for $\theta^{*}$.
Givet at $A\mathbf{d}=\mathbf{0}$, følger det, at $A(\mathbf{x}+\theta \mathbf{d})=A\mathbf{x}=\mathbf{b}$ for alle $\theta$.
%and the equality constraints will never be violated. hvordan følger det af foregående??
Det gælder således, at $\mathbf{x}+\theta\mathbf{d}$ er en mulig løsning i alle tilfælde, hvor ingen af $\mathbf{d}$'s komponenter er negative.
%usikker på om det er ds komponenter i ovenstående
Der findes således et tilfælde, hvor $\mathbf{d} \geq \mathbf{0}$, hvorfor $\mathbf{x} + \theta \mathbf{d} \geq \mathbf{0}$ for $\forall\theta \geq 0$. 
Det gælder derfor, at $\mathbf{x} + \theta \mathbf{d} $ altid er en mulig løsning, hvormed $\theta^*=\infty$. \\
Såfremt et index i $\mathbf{d}$, $d_i<0$ kan begrænsningen $x_i + \theta d_i \geq 0$ omskrives til $\theta \leq \frac-{x_i}{d_i}$, som skal opfyldes for alle $i$, hvorom det gælder, at $d_i<0$.
Derfor følger det, at $\theta*=\text{min}_{i \mid d_i<0}(-\frac{x_i}{d_i}$.
Da $x_i$ ikke er en basal variabel 
%her kommer det jeg virkeligt ikke fatter umiddelbart
\begin{thm}{}{}
\begin{enumerate}[label = (\alph*)]
\item Søjlerne $A_{B(i)}$, $i \neq \mathcal{L}$ og $A_j$ er lineært uafhængige, og 
\item Vektoren $\mathbf{y}=x+ \theta \mathbf{d}$ er en basal mulig løsning til(associeret med??) basis matricen $B$.
\end{enumerate}
\end{thm}
%
\begin{proof}

(a) hvis vektorene $\mathbf{A}_{B(i)}$,$i=1,\ldots,m$ er linieært afhængige så eksisterer der koeficienter $\lambda_1,\ldots \lambda_m$ hvor ikke alle af disse er $0$, altså løsninger udover den trivielle løsning, hvorom det gælder:
$ \sum_{i=1}^m \lambda_i \mathbf{A}_{B(i)}=\mathbf{0} $ det følger derfor at $\sum_{i=1}^m \lambda_i B^{-1}\mathbf{A}_{B(i)}=\mathbf{0}$ og vektorene $B^{-1}\mathbf{A}_{B(i)}$ er ligeledes lineært uafhængige. Det skal nu vises at dette ikke er tilfældet, ved at vise at vektorene $B^{-1}\mathbf{A}_{B(i},i\neq \mathcal{L}$ og $B^{-1}\mathbf{A}_j$ er lineært uafhængige.
Jævnfør sætning (den om invers og identitetsmatricen) er $B^{-1}B=i$.
Da $A_{B(i)}$ er den $i$'te søjle af $B$ gælder derfor at vektorene $B^{-1}\mathbf{A}_{B(i},i\neq \mathcal{L}$  er alle enhedsvektorene med undtagelse af den $\mathcal{L}$'te enhedsvektor. 
Vektorene er således lineært uafhængige og deres  $\mathcal{L}$'te komponent er $0$.
Jævnfør sætning (igen en af dem med inverse tror) $B^{-1}\mathbf{A}_j=-\mathbf{d}_B$.(B er stadigt en funktion her men gives intet index, WTF?) 
Den $\mathcal{L}$'te indgang $-d_{B(\mathcal{L})}\neq 0$ ud fra definitionen af $\mathcal{L}$.
Derfor må det gælde at $B^{-1} \mathbf{A}_j$ er lineært uafhængige fra enhedsvektorene $B^{-1}\mathbf{A}_{B(i},i\neq \mathcal{L}$ \\
(b) lad $\mathbf{y}\geq \mathbf{0}$, $A\mathbf{y}=\mathbf{b}$ og $y_i=0$ for $i \neq B(1),\ldots,B(m)$. Søjerne $\mathbf{A}_{B(1)},\ldots,\mathbf{A}_{B(m)}$ er i (a) vist liniæert uafhængige. Det følger derfor at $y$ er en basal mulig løsning associeret med basismatricen $B$.
\end{proof}