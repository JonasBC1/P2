\section{Pivoterings-løkker}
 Der er en risiko for ved degenererede basale løsninger, at simplex metoden blot skifter variabler ind og ud af basis, sådan at den sidder fast i en uendelig løkke.  
For at garantere at simplex metoden stopper ved en optimal værdi, og ikke fortsætter med at pivotere uendeligt, så kan der indføres forskellige regler for hvilken variabel som skal komme ind i løsningen.

\subsection{Lexicografi}
%ingen ide om det hedder lexicografi på dansk men skiver det indtil videre

Før den \textit{lexicografiske} metode at vælge pivot søjler på introduceres, så er det nødvendigt at definere lexicografi
\begin{defn}{}{}
En vektor $\textbf{u}\in \R^n$ er \textbf{lexicografisk større}, eller mindre end en vektor $\textbf{v}\in \R^n$, hvis $\textbf{v} \neq \textbf{u}$ og den første ikke-nul indgang i $\textbf{u}-\textbf{v}$ er henholdsvis positiv eller negativ. Dette noters
\begin{align*}
\textbf{u} >^L \textbf{v} \text{ eller } \textbf{u} <^L \textbf{v}.
\end{align*} 
\end{defn}
\noindent
Hvis det første ikke-nul element af en vektor er positiv, så siges vektoren at være lexciografisk positiv, og negativ hvis den er negativ. Et eksempel på lexicografi ses i eksempel \ref{eks:lexi}.
\\
%
\begin{eks}\label{eks:lexi}
Givet vektorerne
$$\textbf{u}=
\begin{bmatrix}
2\\
3\\
1\\
\end{bmatrix}
,
\textbf{v}=
\begin{bmatrix}
2\\
1\\
2\\
\end{bmatrix}
\text{ og }
\textbf{z}=
\begin{bmatrix}
3\\
2\\
2\\
\end{bmatrix}
.$$
Så er $\textbf{u}$ lexicofrafisk større end $\textbf{v}$ da den første ikke-nul værdi af $\textbf{u}-\textbf{v}$ er 2.
$\textbf{u}$ er til gengæld lexicografisk mindre end $\textbf{z}$ da den første ikke-nul værdi er $-1$
\end{eks}
En lexicografisk metode at vælge hvilke rækker der skal pivoteres følger.
\begin{enumerate}
\item Vælg en arbitrær søjle $\textbf{A}_j$, til at indgå i basis, så længe dens reducererde pris $c_j$ er negativ.
Lad $\textbf{u}$ være den $j$'te søjle af tabulaen.
\item For hvert i med $u_i>0$ divideres den i'te række med $u_i$, og vælg den lexicografikske mindste række. Hvis rækken $L$ er den lexicografiske mindste række så forlader den $L$'te basiske variabel $x_B(L)$ basiset.
\end{enumerate}
Det følger at enhver valg af pivotering ved den lexicografiske metode er unik, da der ellers ville gælde at der var en anden række som var proportional med den valgte række. Dette er i strid med antagelsen om at rækkerne er lineært uafhængige, hvilket er bevist er tilfældet.

\begin{thm}{}{}
Antag at simplex metoden starter med at alle rækker af en simplex tabula, foruden den nulte række, er lexciografisk positiv. Hvis den lexicografiske metode at vælge pivotering er fulgt, så gælder:
\begin{enumerate}[label=(\alph*)]
\item Alle rækkerne, foruden den nulte række, forbliver lexciografiske positve gennem algoritmen.
\item Den nulte række vokser strengt lexicografisk for hver iteration.
\item Simplex metoden stopper efter et endeligt antal iterationer. 
\end{enumerate}
\end{thm}
%
\begin{proof}
Antag at alle rækkerne af en simplex tabula, udover den nulte række, er lexicografisk positiv i begyndelsen af en iteration af simplex. 
Antag dernæst at $x_j$ kommer ind i løsningen og pivot rækken er den $l$'te række, så gælder det per den lexciografiske metode at
$$\dfrac{l\text{'te række}}{u_l}<^L \dfrac{i\text{'te række}}{u_i}, \text{  hvis } i\neq l \text{ og }u_i>0.$$
For at bestemme den nye tabula, divideres den $l$'te række med den positive variabel $u_l$ og forbliver dermed lexciografisk positiv.
For rækker $i$ hvor $u_i<0$, skal der ligges et positiv antal af pivot rækken til for at få den $i,j$'te indgang til at give nul. 
Grundet begge rækkers lexciografiske positivitet, er den resulterende række også lexciografisk positiv.
For rækker $i$ hvor $u_i>0$ og $i\neq l$, haves
$$\text{(nye }i\text{'te række)}=\text{(gamle }i\text{'te række)}-\dfrac{u_i}{u_l}\text{(gamle }l\text{'te række)}.$$
Grundet den tidligere ulighed som opfyldes af de gamle rækker, vides det at den nye $i$te række også er lexciografisk positiv.
%
I begyndelsen af hver iteration, er pivot værdien i den nulte række negativ, så der skal ligges et positiv antal af pivot rækken til. 
Da pivot rækken er lexciografisk postiv, stiger den nulte række lexciografisk.
%
Eftersom den nulte rækker stige lexciografisk for hver iteration, kan den aldrig returnere til en tidligere værdi.
Eftersom den nulte række er bestemt unik ud fra nuværende basis, så kan intet basis gentage sig selv, og derfor må simplex metoden stoppe efter en endelig antal iterationer.
\end{proof}
%Jeg tror ikke Bland's regel er relevant for os medmindre at vi skriver om den revidererde simpelx metode.
%\subsection{Bland's regel}
%En anden metode
%
%
%
%