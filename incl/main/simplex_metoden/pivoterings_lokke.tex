\section{Pivoteringsløkker}
Ved degenererede basale løsninger er der en risiko for, at simplexmetoden blot skifter variable ind og ud af basen, således at den sidder fast i en uendelig \textit{pivoteringsløkke}.  
For at garantere at simplexmetoden stopper ved en optimal værdi og ikke fortsætter med at pivotere uendeligt, kan der indføres forskellige regler for, hvilken variabel, der bringes ind i løsningen.

\subsection{Lexicografi}
%ingen ide om det hedder lexicografi på dansk men skiver det indtil videre
%
Før den \textit{lexicografiske metode} til at vælge pivotsøjler introduceres, er det nødvendigt at definere \textit{lexicografi}.
\begin{defn}{}{}
En vektor $\textbf{u}\in \R^n$ er \textbf{lexicografisk} større, eller mindre end en vektor $\textbf{v}\in \R^n$, hvis $\textbf{v} \neq \textbf{u}$ og den første ikke-nul indgang i $\textbf{u}-\textbf{v}$ er henholdsvis positiv eller negativ. Dette noteres
\begin{align*}
\textbf{u} >^L \textbf{v} \phantom{..} \text{ eller }\phantom{..} \textbf{u} <^L \textbf{v}.
\end{align*} 
\end{defn}
\noindent
Hvis det første ikke-nul element af en vektor er positiv, siges vektoren at være lexicografisk positiv. Modsat siges den at være lexicografisk negativ, hvis det første ikke-nul element af en vektor er negativt. Et eksempel på lexicografi ses i \ref{eks:lexi}.
\\
%
\begin{eks}\label{eks:lexi}
Givet vektorerne
$$\textbf{u}=
\begin{bmatrix}
2\\
3\\
1\\
\end{bmatrix}
,\phantom{..}
\textbf{v}=
\begin{bmatrix}
2\\
1\\
2\\
\end{bmatrix}
\phantom{..}\text{ og }\phantom{..}
\textbf{z}=
\begin{bmatrix}
3\\
2\\
2\\
\end{bmatrix}
.$$
Så er $\textbf{u}$  lexicografisk større end $\textbf{v}$, $\textbf{u} >^L \textbf{v}$, da den første ikke-nul værdi af $\textbf{u}-\textbf{v}$ er $2$.
Til gengæld er $\textbf{u}$ lexicografisk mindre end $\textbf{z}$, $\textbf{u} <^L \textbf{z}$, da den første ikke-nul værdi er $-1$.
Bemærk, at alle $3$ vektorer er lexicografisk positive.
\end{eks}
%
Følgende er en lexicografisk metode til at udvælge de rækker, der skal pivoteres.
\begin{enumerate}
\item Vælg en arbitrær søjle $\textbf{A}_j$ til at indgå i basen, så længe dens reducerede omkostning $c_j^*$ er negativ.
Lad $\textbf{u}$ være den $j$'te søjle i simplextabellen.
\item For hvert $i$, hvor $u_i>0$, divideres den $i$'te række med $u_i$, og den lexicografisk mindste række vælges. 
Hvis rækken $L$ er den lexicografisk mindste række, så forlader den $L$'te basale variabel, $x_{B(L)}$, basen.
\end{enumerate}
%
Det følger, at ethvert valg af pivotering via den lexicografiske metode er entydig, da der ellers ville være en anden række, som var proportional med den valgte række. 
Dette er i strid med antagelsen om, at rækkerne er lineært uafhængige.
%
\begin{thm}{}{}
Antag, at simplexmetoden starter med, at alle rækker i en $m \times n$ simplextabel foruden rækken med omkostningsvektoren, den $m$'te række, er lexicografisk positive.
Hvis den lexicografiske metode til at vælge pivotering er fulgt, så gælder det, at
%
\begin{enumerate}[label=(\alph*)]
\item Alle rækkerne foruden den $m$'te række forbliver lexicografiske positive gennem algoritmen.
\item Den $m$'te række vokser strengt lexicografisk for hver iteration.
\item Simplexmetoden stopper efter et endeligt antal iterationer. 
\end{enumerate}
\end{thm}
%
\begin{proof}
\begin{enumerate}[label=(\alph*)]
\item Antag, at alle rækkerne af en simplextabel, udover den $m$'te række, er lexicografisk positive i begyndelsen af en iteration af simplex. 
Antag dernæst, at $x_j$ bringes ind i løsningen, og at pivotrækken er den $l$'te række.
Så gælder der jævnfør den lexicografiske metode, at
$$\dfrac{l\text{'te række}}{u_l}<^L \dfrac{i\text{'te række}}{u_i}, \text{  hvis } i\neq l \text{ og }u_i>0.$$
For at bestemme den nye tabel divideres den $l$'te række med den positive variabel $u_l$ og forbliver dermed lexicografisk positiv.
For rækker $i$, hvor $u_i<0$, skal der lægges et positivt multiplum af pivotrækken til, således at den $(i,j)$'te indgang bliver nul. 
Grundet begge rækkers lexicografiske positivitet er den resulterende række også lexicografisk positiv. 
For rækker $i$, hvor $u_i>0$ og $i\neq l$, haves, at
$$\text{(nye }i\text{'te række)}=\text{(gamle }i\text{'te række)}-\dfrac{u_i}{u_l}\text{(gamle }l\text{'te række)}.$$
Grundet den tidligere ulighed, som opfyldes af de gamle rækker, vides det, at den nye $i$'te række også er lexicografisk positiv.
% 
\item I begyndelsen af hver iteration, er pivotværdien i den $m$'te række negativ, så der skal ligges et positiv multiplum af pivotrækken til. 
Da pivotrækken er lexicografisk positiv, stiger den $m$'te række lexicografisk.
% 
\item Eftersom den $m$'te række stiger lexicografisk for hver iteration, kan den aldrig returnere til en tidligere værdi.
Da den $m$'te række er bestemt entydig ud fra nuværende basis, så kan intet basis gentage sig selv, og derfor må simplexmetoden stoppe efter et endeligt antal iterationer.
\end{enumerate}
\end{proof}
%
%Jeg tror ikke Bland's regel er relevant for os medmindre at vi skriver om den revidererde simpelx metode.
%\subsection{Bland's regel}
%En anden metode
%
%
%
%