\chapter{Simplex metoden}
% Eksempel
\begin{eks}\label{lineu}
Lad en mængde af vektorer være
\begin{align*}
\S &= \left\{
\begin{bmatrix}
           4 \\
           1 \\
\end{bmatrix}
,
\begin{bmatrix}
           -3 \\
           0 \\
\end{bmatrix}
,
\begin{bmatrix}
           1 \\
           -2 \\
\end{bmatrix}
\right\}.
\end{align*}
\noindent
% Skal skrives om fra et spørgsmål 
Afgør om vektorerne er lineært uafhængige, altså hvis $x_1$ og $x_2 = 0$, er den eneste løsning til linearkombination.
\begin{align*}
x_1+
\begin{bmatrix}
           4 \\
           1 \\
\end{bmatrix}
+ x_2
\begin{bmatrix}
           -3 \\
           0 \\
\end{bmatrix}
+ x_3
\begin{bmatrix}
           1 \\
           -2 \\
\end{bmatrix}
=0 
\end{align*}
%
Totalmatricen opkrives 
\noindent
\begin{align*}
A=
\begin{blockarray}{cccc}
x_1 & x_2 & x_3 & b \\
\begin{block}{[ccc|c]}
4 & -3 & 1 & 0\\
1 & 0 & -2 & 0 \\
\end{block}
\end{blockarray}
\xrightarrow[R_2 \rightarrow R_2+(-1R_1)]{R_1 \rightarrow \frac{1}{4}R_1} 
\begin{blockarray}{cccc}
x_1 & x_2 & x_3 & b \\ 
\begin{block}{[ccc|c]}
1 & \frac{-3}{4} & \frac{1}{4} & 0\\
0 & \frac{3}{4} & \frac{-9}{4} & 0 \\
\end{block}
\end{blockarray}
\xrightarrow[R_1 \rightarrow R_1+(\frac{3}{4} R_2)]{R_2 \rightarrow \frac{4}{3} \times R_2} 
\begin{blockarray}{cccc}
x_1 & x_2 & x_3 & b \\
\begin{block}{[ccc|c]}
\hlight{1} & 1 & -2 & 0\\
0 & \hlight{1} & -3 & 0\\
\end{block}
\end{blockarray}
\end{align*}
%
Da tolalmatricen ikke har nogle frie variable, har den kun en løsning og derfor lineært uafhængig.
\\
\\
\noindent
Lad i stedet en mængde af vektorer være 
%
\begin{align*}
\S &= \left\{
\begin{bmatrix}
           4 \\
           1 \\
           -2 \\
\end{bmatrix}
,
\begin{bmatrix}
           -3 \\
           0 \\
           1 \\
\end{bmatrix}
,
\begin{bmatrix}
           1 \\
           -2 \\
           1 \\
\end{bmatrix}
\right\}.
\end{align*}
\noindent
Afgør om vektorerne er lineært uafhængige, altså hvis $x_1$, $x_2$ og $x_3 = 0$, er den eneste løsning til linearkombinationen.
\begin{align*}
x_1+
\begin{bmatrix}
           4 \\
           1 \\
           -2 \\
\end{bmatrix}
+ x_2
\begin{bmatrix}
          -3 \\
           0 \\
           1 \\
\end{bmatrix}
+ x_3
\begin{bmatrix}
           1 \\
           -2 \\
           1 \\
\end{bmatrix}
=0.
\end{align*}
%
Totalmatricen opskrives 
%
\begin{align*}
A=
\begin{blockarray}{cccc}
x_1 & x_2 & x_3 & b \\
\begin{block}{[ccc|c]}
4 & -3 & 1 & 0\\
1 & 0 & -2 & 0 \\
-2 & 1 & 1 & 0 \\
\end{block}
\end{blockarray}
&\xrightarrow[R_2\rightarrow R_2+(-4R_1)]{R_1 \leftrightarrow R_2}
\begin{blockarray}{cccc}
x_1 & x_2 & x_3 & b \\
\begin{block}{[ccc|c]}
1 & 0 & -2 & 0\\
0 & -3 & 9 & 0\\
-2 & 1 & 1 & 0 \\
\end{block}
\end{blockarray}
\xrightarrow[R_3 \rightarrow R_3+\frac{1}{3}R_2]{R_3 \rightarrow R_3+2R_1}
\begin{blockarray}{cccc}
x_1 & x_2 & x_3 & b \\
\begin{block}{[ccc|c]}
\hlight{1} & 0 & -2 & 0\\
0 & -3 & 9 & 0 \\
0 & 0 & 0 & 0 \\
\end{block}
\end{blockarray}
\end{align*}
\noindent
Dette betyder, at der findes en ikke-triviel lineær kombination, som giver nulvektoren. Derfor har vi minimum fundet én linearkombination som er ikke-nul og derfor er det bevist at vektorsættet lineært afhængigt.
\end{eks}