\chapter{Simplex metoden}
Som nævnt tidligere i rapporten, så findes der en optimal løsning til et lineært ligningssystem på standard form, så længe der findes mindst én basal mulig løsning. 
Simplex metoden benytter sig af denne egenskab og forsøger at finde den optimale løsning. Før simplex metoden bliver præsenteret, skal nogle begreber introduceres.
%
\\\\
%
Når der undersøges om der findes en nabo punkt hvis løsning er bedre, så er det ikke nødvendigt at undersøge i retningen som leder ud af den basal mulige område.

\begin{defn}{}{}
Lad $\textbf{x}$ være et element af en polyede $P$.
En vektor $\textbf{d}\in \R^n$ siges at være en mulig retning fra $\textbf{x}$, hvis der eksister en positiv skalar $\theta$, sådan at $x+\theta \textbf{d}\in P$
\end{defn}
%Simplex-metoden er en metode til at løse lineære optimeringsproblemer.
%Metoden er bygget til at håndterer problemer på standardform.
%mOTIVERENDE Tkst\\\\
%%
%Simplex metoden udfører i grove træk følgende operationer:
%%
%\begin{enumerate}
%\item Et ekstremum $x$ vælges inde for løsningsmængden.
%\item
%\item
%\end{enumerate}
%


\subsection{implementering}
Der er tre forskellige implementeringer af simplex metoden, den naive, den reviderede, og den fulde matrix implementering. 
