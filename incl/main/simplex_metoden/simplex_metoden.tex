\chapter{Simplexmetoden}
\label{coronaaaaaaaaaaa}
%
Som vist i afsnit \ref{klogskab}, findes der en optimal løsning til et lineært ligningssystem på standardform, så længe der findes mindst én basal mulig løsning. 
\textit{Simplexmetoden} benytter sig af denne egenskab og søger at finde den optimale løsning.
Dette kapitel tager udgangspunkt i \citep[side 25-34]{lial} og \citep[side 82-87 og 94-100]{bert}, hvis ikke andet er angivet.
Bemærk, at afsnit \ref{afsnittet} tager afsæt i et maksimumsproblem, imens øvrige afsnit tager udgangspunkt i minimumsproblemer.
%
% Geometrisk tilgang til simplexmetoden
% ---------------------------------------------------------
	\section{Optimale betingelser}
\label{julieerfantalastiskogvidunderlig}
% --------------------------------------------------------------
%
For at finde den optimale løsning kan der udvælges en basal løsning og undersøges, hvorvidt der findes andre løsninger, som er bedre, end den valgte. 
Hvis der ikke findes bedre løsninger, er den valgte basale løsning den optimale løsning. 
Eftersom der jævnfør afsnit \ref{julieerlakker} optimeres for en konveks funktion over en konveks mængde, %Det er det med at et lokal minimum er et globalt minimum xD
vil den lokale optimale løsning være den globale optimale løsning. 
%
%Når der undersøges om der findes et nabo punkt, hvis løsning er bedre, så er det ikke nødvendigt at undersøge i retningen, som leder ud af den basal mulige område.
%
% Definition 3.1 i bert
\begin{defn}{}{}
Lad $\textbf{v}$ være et element af et polyeder $\mathcal{P}$.
En vektor $\textbf{d}\in \R^n$ siges at være en \textbf{mulig retning} fra $\textbf{v}$, hvis der eksisterer en positiv skalar $\theta$, således at $\textbf{v}+\theta \textbf{d}\in \mathcal{P}$.
\end{defn}
\noindent
%
På figur \ref{fig:julieersmuuuuuk}, ses vektorerne $\mathbf{u}, \mathbf{v}$ og $\mathbf{w}$ i polyederet $\mathcal{P}$, samt tilhørende forskellige mulige retninger, som stadig er indeholdt i $\mathcal{P}$.
%
\begin{center}
%
\begin{tikzpicture}[scale=8]
%
% Koordinater
% -------------------------------------------------------
\coordinate (a) at (0,0,0); 
\coordinate (a1) at (0.084,0,0);
\coordinate (a2) at (0,0.084,0);
\coordinate (a3) at (0.07,0.07,0);
%
\coordinate (b) at (0.7,0,0); 
\coordinate (c) at (0.393,0.55,0); 
%
\coordinate (e) at (0.066,0.434,0);
\coordinate (e1) at (0.15,0.46,0);
\coordinate (e2) at (0.025,0.35,0);
\coordinate (e3) at (0.094,0.36,0);
\coordinate (e4) at (0.14,0.4,0);
%
\coordinate (g) at (0,0.305,0);
%
\coordinate (m) at (0.35,0.15,0);
\coordinate (m1) at (0.31,0.066,0);
\coordinate (m2) at (0.31,0.234,0);
\coordinate (m3) at (0.266,0.15,0);
\coordinate (m4) at (0.434,0.15,0);
\coordinate (m5) at (0.39,0.066,0);
\coordinate (m6) at (0.39,0.234,0);
%
% Farvning
% -------------------------------------------------------
\filldraw[fill=myblue,opacity=0.3](a)--(b)--(c)--(e)--(g)--(a);
%
\draw[thick, color= myblue](a)--(b); 
\draw[thick, color= myblue](c)--(b);
\draw[thick, color= myblue](a)--(g);
\draw[thick, color= myblue](c)--(e);
\draw[thick, color= myblue](e)--(g);
%
% Punkt A
% -------------------------------------------------------
\draw[very thick,->](a)--(a1);
\draw[very thick,->](a)--(a2);
\draw[very thick,->](a)--(a3);
%
%
% Punkt E
% -------------------------------------------------------
\draw[very thick,->](e)--(e1);
\draw[very thick,->](e)--(e2);
\draw[very thick,->](e)--(e3);
\draw[very thick,->](e)--(e4);
%
% Punkt Mid
% -------------------------------------------------------
\draw[very thick,->](m)--(m1);
\draw[very thick,->](m)--(m2);
\draw[very thick,->](m)--(m3);
\draw[very thick,->](m)--(m4);
\draw[very thick,->](m)--(m5);
\draw[very thick,->](m)--(m6);
%
% Punkterne 
% -------------------------------------------------------
\filldraw [black] (0.35,0.35,0) circle (0pt) node[above] {$\mathcal{P}$};
\filldraw [black] (0.31,0.17,0) circle (0pt) node[anchor=south east]{$\mathbf{v}$};
\filldraw [black] (e) circle (0pt) node[anchor=
south east]{$\mathbf{u}$};
\filldraw [black] (a) circle (0pt) node[anchor= north east]{$\mathbf{w}$};

%
%
\end{tikzpicture}
  \captionof{figure}{Vektorerne $\mathbf{u}, \mathbf{v}$ og $\mathbf{w}$, samt tilhørende forskellige mulige retninger, som stadig er indeholdt i polyederet $\mathcal{P}$.}
  \label{fig:julieersmuuuuuk}
\end{center}
%
%
%Hjørnerne afsøges derved gennem en reduceret objektfunktion.
%Se \ref{defn:kogtskinke}.
%
\textcolor{red}{
For at finde en ny mulig retning $\mathbf{d}$ bevæges der i en retning, hvor værdien af en ikke-basal variable $x_j$ øges, og andre ikke-basale variable fastholdes. 
Dermed vil vektoren $\mathbf{d}_j = 1$, og $\mathbf{d}_i = 0$ for henholdsvis den ikke-basale variable $x_j$ og de andre ikke-basale variable $x_i$, hvor $i \neq j$.
Vektoren 
\begin{align*}
\mathbf{d}_B = \left( d_{B(1)}, d_{B(2)}, \ldots , d_{B(m)} \right),
\end{align*} 
bestående af de mulige retninger for de basale index, 
kontureres.
Da $\mathbf{x}$ er en basal mulig løsning, gælder det, at $A \mathbf{x} = \mathbf{b}$. 
Eftersom $\mathbf{x}+ \theta \mathbf{d}$ skal være en mulig løsning, gælder det, at $A ( \mathbf{x}+ \theta \mathbf{d}) = \mathbf{b}$.
Deraf haves, at 
\begin{align*}
A ( \mathbf{x}+ \theta \mathbf{d}) &=  A \mathbf{x} + \theta A \mathbf{d} = \mathbf{b} \\
& \Updownarrow \\
\mathbf{b} + \theta A \mathbf{d} & = \mathbf{b} \\
& \Updownarrow \\
\theta A \mathbf{d} & = \mathbf{0}.
\end{align*} 
Bemærk, at $\theta$ er en positiv skalar. 
Dermed skal $\mathbf{d}$ tilhøre $\text{nrum}(A)$.
Eftersom $\mathbf{d}_j = 1$ og $\mathbf{d}_i = 0$, så er 
%
\begin{align*}
\mathbf{0} = A \mathbf{d} & = \sum^n_{i = 1} \mathbf{A}_i d_i \\
& =  \mathbf{A}_j  + \sum^m_{i = 1} \mathbf{A}_{B(i)} d_{D(i)} \\
& =  \mathbf{A}_j  + B \mathbf{d}_B.
\end{align*}
%
Da basis matricen $B$ er invertibel, haves at
% 
\begin{align*}
\mathbf{A}_j  + B \mathbf{d}_B & = 0 \\
& \Updownarrow \\
B \mathbf{d}_B & = - \mathbf{A}_j \\
& \Updownarrow \\
\mathbf{d}_B & = - B^{-1} \mathbf{A}_j .
\end{align*}
% 
Denne mulige retning $d$, tilhørende $x_j$, kaldes den \textit{$j$'te basale retning}.
% 
\begin{defn}{}{}
Lad $\mathbf{x}$ være en basal mulig løsning. 
Den \textbf{$j$'te basale retning} $\mathbf{d}$, fra $\mathbf{x}$, for en ikke-basal variabel $x_j$, har komponenterne 
%
\begin{align*}
d_j & = 1, \\
d_i & = 0 \phantom{hej} \text{ for } i \neq b(1), \ldots , B(m) \wedge i \neq j, \\
\mathbf{d}_B & = - B^{-1} \mathbf{A}_j.
\end{align*}
% 
\end{defn}
\noindent
%
Bemærk, at overstående sikrer lighedsbegrænsningerne, men det sikrer ikke ikke-negativitetsbetingelserne. 
Eftersom $x_j$ øges, og de andre ikke-basale variable fastholdes, kan følgende scenarier opstå. 
\begin{itemize}
\item Antag, at $\mathbf{x}$ er en ikke-degenereret basal mulig løsning. 
Dermed er $\mathbf{x}_B > \mathbf{0}$. 
Hvis $\theta$ er småt nok, vil $\mathbf{x}_B + \theta \mathbf{d}_B \geq 0 $, og $\mathbf{d}$ er dermed en mulig retning. 
\item Antag, at $\mathbf{x}$ er en degenereret basal mulig løsning. 
Dermed er $\mathbf{d}$ ikke altid en mulig retning. 
Eftersom $\mathbf{d}$ er degenereret, er det muligt, at en basal variabel $x_{B(i)}=0$, og den tilsvarende komponent $d_{B(i)}$ er negativ. 
I dette tilfælde vil en bevægelse i den $j$'te mulige retning forlade polyederet, da en ikke-negativitetsbetingelse brydes af $\mathbf{x} + \theta \mathbf{d}$ for alle positive $\theta$.
\end{itemize}
%
På figur \ref{fig:julieerengud} ses de to scenarier. 
Bemærk, at $A$ har de to ikke-basale variable $x_1$ og $x_3$ og $x_2,$ $x_4$ og $x_5$ er basale variable.
I hjørnet $A$ fastholdes den ikke-basale variable $x_3$, og $x_1$ øges. 
Dermed bevæges der mod randen $AB$. 
Bemærk, at $B$ har de to ikke-basale variable $x_3$ og $x_5$ og $x_1,$ $x_1$ og $x_4$ er basale variable.
I hjørnet $B$ fastholdes den ikke-basale variabel $x_5$, og $x_3$ øges. 
Dermed bevæges der mod randen $BC$, hvilket bryder en ikke-negativitetsbetingelse, og dermed ikke en mulig retning. 
}
%
\begin{center}
%
\begin{tikzpicture}[scale=5]
%
% Koordinater
% -------------------------------------------------------
\coordinate (a) at (0,0,0); 
\coordinate (b) at (0.7,0,0); 
\coordinate (c) at (0.393,0.55,0); 
\coordinate (d) at (0.252,0.805,0);
\coordinate (e) at (0.066,0.434,0);
\coordinate (f) at (0,0.41,0);
\coordinate (g) at (0,0.305,0);
\coordinate (h) at (-0.15,0,0); 
%
% Farvning
% -------------------------------------------------------
\filldraw[fill=myblue,opacity=0.3](a)--(b)--(c)--(f)--(a);
  %\draw[thick](-0.2,-0.1,0)--(0.3,0.9,0); % n -> d  
  \draw[thick](0.2,0.9,0)-- node[anchor=north east] {$x_4=0$} (0.757,-0.1,0); % d -> b
  \draw[thick](-0.3,0.3,0)-- node[above] {$x_3=0$ \phantom{hej}} (0.8,0.7,0); % f -> c
  \draw[thick](0.12,0.9,0)-- node[anchor=south west] {$x_5=0$} (0.9,-0.1,0); % ekstra
% 
%
% Punkterne 
% -------------------------------------------------------
%\filldraw [black] (a) circle (0.2pt) node[anchor=north west] {$A$};
%\filldraw [black] (b) circle (0.2pt) node[anchor=north east] {$B$};
\filldraw [black] (c) circle (0.2pt) node[anchor=south west] {$B$};
\filldraw [black] (0.822,0,0) circle (0.2pt) node[anchor=south west] {$C$};
%\filldraw [black] (d) circle (0.2pt) node[anchor=west] {$D$};
%\filldraw [black] (e) circle (0.2pt) node[anchor=north west] {$E$};
\filldraw [black] (f) circle (0.2pt) node[anchor=south east] {$A$};
%\filldraw [black] (g) circle (0.2pt) node[anchor=east] {$G$};
%\filldraw [black] (h) circle (0.2pt) node[anchor=north west] {$H$};
%
% 
\filldraw [black] (0.27,0.17,0) circle (0pt) node[above] {$\mathcal{P}$};
% 
% Koordinatssystem 
% -------------------------------------------------------
\draw[thick] (0,0,0) -- node[anchor=north east] {$x_2=0$} (0.9,0,0);
\draw[thick] (0,0,0) -- (-0.2,0,0);
\draw[thick] (0,0,0) -- node[anchor=north east] {$x_1=0$} (0,0.5,0);
\draw[thick] (0,0.5,0) -- (0,0.9,0);
\draw[thick] (0,0,0) -- (0,-0.2,0);
%
\end{tikzpicture}
  \captionof{figure}{Illustration af en mulig retning, der svarer til randen $AB$, og en ikke-mulig retning, der svarer til randen $BC$, som bryder en ikke-negativitetsbetingelse.}
  \label{fig:julieerengud}
\end{center}
%
\textcolor{red}{
Heraf opstår et problem med degenererede løsninger, og af den grund forsøges disse undgået.
\\\\
%
Ved en bevægelse i en mulig retning medfører det, at 
\begin{align*}
\mathbf{c}^T \mathbf{d} = \mathbf{c}^T_B \mathbf{d}_B + c_j,
\end{align*}
hvor 
$$\mathbf{c}_B=[ c_{B(1)},c_{B(2)}, \cdots , c_{B(m)} ]^T.$$
Bemærk, at da $\mathbf{d}_B = - B^{-1} \mathbf{A}_j$, er 
\begin{align*}
\mathbf{c}^T \mathbf{d} &= \mathbf{c}^T_B ( - B^{-1} \mathbf{A}_j ) + c_j \\
& = c_j - \mathbf{c}^T_B (B^{-1} \mathbf{A}_j).
\end{align*}
%
For at omkostningen minimeres, skal $c_j - \mathbf{c}^T_B (B^{-1} \mathbf{A}_j ) < 0$, da 
% 
\begin{align*}
\mathbf{c}^T  ( \mathbf{x} + \theta \mathbf{d} )  = \mathbf{c}^T  \mathbf{x} + \theta \mathbf{c}^T \mathbf{d} = \mathbf{c}^T  \mathbf{x} + \theta ( c_j - \mathbf{c}^T_B (B^{-1} \mathbf{A}_j ) ) < \mathbf{c}^T \mathbf{x} 
\end{align*}
%
for alle positive $\theta$.
Størrelsen for reduktionen for hvert $\theta$ kaldes \textit{den reducerede omkostning}.
%
% Definition 3.2 
\begin{defn}{}{kogtskinke}
Lad $\mathbf{v}$ være en basal løsning til en basismatrix $B$ og tilhørende omkostningsvektoren $$\mathbf{c}_B=[ c_{B(1)},c_{B(2)}, \cdots , c_{B(m)} ]^T.$$
For hver $j$'te basale retning defineres den \textbf{reducerede omkostning} $c_j^*$ som
\begin{align*}
c_j^* = c_j - \mathbf{c}_B^T B^{-1} \mathbf{A}_j.
\end{align*} 
%
\end{defn}
\noindent
%
\textit{Den reducerede omkostningsvektor} $\mathbf{c}^*$ indeholder alle de reducerede omkostninger, hvor den $k$'te række betegner den reducerede omkostning for den $k$'te basale retning.
%
\begin{align*}
\mathbf{c}^* = (\mathbf{c}^T - \mathbf{c}_B^T B^{-1} A)^T = 
\begin{blockarray}{[c]}
c_1 - \mathbf{c}_B^T B^{-1} \mathbf{A}_1 \\
\vdots \\
c_k - \mathbf{c}_B^T B^{-1} \mathbf{A}_k \\
\vdots \\
c_n - \mathbf{c}_B^T B^{-1} \mathbf{A}_n
\end{blockarray}.
\end{align*}
%
Betragt de indgange, for de basale variable i den reducerede omkostningsvektor
%
\begin{align*}
c^*_{B(j)} & = c_{B(j)} - \mathbf{c}_B^T B^{-1} \mathbf{A}_{B(j)} \\
& = c_{B(j)} - \mathbf{c}_B^T \mathbf{e_j} \\
& = c_{B(j)} -  c_{B(j)}  \\
& = 0.
\end{align*}
%
Dermed er indgangene til de basale variable lig nul, da de mulige retninger er nulvektoren. 
Det kan derfor ikke betale sig at øge basale variable.
\\\\
Hvis den reducerede omkostningsvektor $\mathbf{c}^* \geq \mathbf{0}$, er den basale mulige løsning et lokalt minimum i objektfunktionen og dermed et globalt minimum. Heraf er den optimale løsning fundet. Dette tilfælde belyses i sætning \ref{thm:julieerguden}.
%
\begin{thm}{}{julieerguden}
Lad $\mathbf{x}$ være en basal mulig løsning til en basismatrix $B$, og lad $\mathbf{c}^*$ være den reducerede omkostningsvektor. 
\begin{enumerate}[label = (\alph*)]
\item Hvis $\mathbf{c}^* \geq 0$, så er $\mathbf{x}$ optimal.
\item Hvis $\mathbf{x}$ er optimal og ikke-degenereret, så er $\mathbf{c}^* \geq 0$.
\end{enumerate}
\end{thm}
%
\begin{proof}
Bevis udelades.
%\begin{enumerate}[label = (\alph*)]
%\item Antag, at $\bar{\textbf{c}} \geq \textbf{0}$. Lad $\textbf{y}$ være en arbitrær mulig løsning, og $\textbf{d} = \textbf{y} - \textbf{x}$. Grundet de mulige løsninger, er $\textbf{A} \textbf{x} = \textbf{A} \textbf{y} = \textbf{b}$, hvormed $\textbf{A} \textbf{d} = \textbf{0}$. Sidstnævnte kan skrives som 
%$$\textbf{B} \textbf{d}_B + \sum_{i \in N} \textbf{A}_i d_i = \textbf{0},$$
%hvor $N$ er mængden af indekser, der svarer til de ikke-basale variabler i basen. %wtf fatter slet ikke det her lol
%$\textbf{B}$ er invertibel, hvormed det haves, at 
%$$\textbf{d}_B = -\sum_{i \in N} \textbf{B}^{-1} \textbf{A}_i d_i ,$$
%og
%$$\textbf{c}' \textbf{d} = \textbf{c}'_B \textbf{d}_B + \sum_{i \in N} c_i d_i = \sum_{i \in N} (c_i - \textbf{c}'_B \textbf{B}^{-1} \textbf{A}_i) d_i = \sum_{i \in N} \bar{c}_i d_i .$$
%%nanananana fatter intet, hvorfor har vi dette med? 
%For ethvert ikke-basalt indeks $i \in N$ må der eksistere et $x_i = 0$, samt et $y_i \geq 0$, da $\textbf{y}$ er en mulig løsning. 
%Deraf haves, at $d_i \geq 0$ og $\bar{c}_i d_i \geq 0$ for alle $i \in N$. 
%Dermed er $\textbf{c}' (\textbf{y} - \textbf{x} = \textbf{c}' \textbf{d} \geq 0$, og $\textbf{x}$ er en optimal løsning, eftersom $\textbf{y}$ er en arbitrær mulig løsning. 
%%
%%
%\item Antag, at $x$ er en ikke-degenereret basal mulig løsning, og at $\bar{c}_j < 0$ for et $j$. Eftersom den reducerede objektfunktion for en basal variabel altid er $0$, må $x_j$ være en ikke-basal variabel, og $\bar{c}_j$ må være ændringshastigheden for objektfunktionen i den $j$'te basale retning. 
%%Wtf er ovenstående for noget lort
%Da $\textbf{x}$ ikke er degenereret, er den $j$'te basale retning en mulig retning for mindskelse af objektfunktionen. 
%Der findes dermed mulige løsninger, hvis objektfunktion har lavere værdi end $\textbf{x}$'s, ved at bevæge sig i denne retning, hvormed $\textbf{x}$ ikke er optimal. 
%\end{enumerate}
\end{proof}
\\
%
Dermed er alle basale retninger mulige retninger, hvis et problem er givet uden degenererede løsninger. 
Heraf defineres \textit{optimal basismatrix}.}
%
% Definition 3.3 
\begin{defn}{}{julieergudenoveralleguder}
En basismatrix $B$ kaldes \textbf{optimal}, hvis
%
\begin{enumerate}[label = (\alph*)]
\item $B^{-1} \mathbf{b} \geq \mathbf{0}$, og
\item $\mathbf{c}^* \geq \mathbf{0}$.
\end{enumerate}
%
\end{defn}
\noindent
%
%
% Eksempel til simplexmetoden
% ---------------------------------------------------------
	\section{Praktisk anvendelse af simplexmetoden}
\label{afsnittet}
Simplexmetoden foretager som udgangspunkt følgende seks trin, inden den afsluttes:
%
\begin{tcolorbox}[
title=Simplexmetoden,
colback		= myblue!15,
colframe	= myblue!15,
coltitle	= black,
before skip	= 20pt plus 2pt,
after skip	= 20pt plus 2pt,
fonttitle	= \bfseries]
% Denne skal eventuelt beskrives mere i dybdegående - Tager kort tid.
\begin{enumerate}
\item Omskriv optimeringsproblemet til standardform med slack-variable.  %1
\item Opstil \textit{simplextabellen} for optimeringsproblemet.					 %2
\item Kontroller for optimalitet ellers identificér en pivoteringsindgang.					 %3
\item Opstil en ny tabel ved hjælp af pivotering. 						 %4
\item Gentag trin 3 og 4, indtil den optimale løsning er fundet. 					 %5
\item Identificér den optimale løsning.									 %6
\end{enumerate}
%
\end{tcolorbox}
\noindent
%
I følgende afsnit uddybes ovenstående punkter, samt tilhørende teori, med udgangspunkt i \ref{haribooooo}.
%
\\
%
\begin{eks}
\label{haribooooo}
Haribo har to typer slikblandinger, $x_1$ og $x_2$, hvor profitten for disse henholdsvis er $5$ og $4$ valutaenheder.
%(her kunne der også skrives tusind/millioner whatever ingen anelse om realistisk skala i haribos profitmargin).
Disse er begrænset af produktionskapaciteten af lakrids og vingummi.
Lakridsproduktion har en begrænsning på $78$ enheder, hvor der skal $3$ enheder lakrids i $x_1$ blandingen og $5$ enheder lakrids i $x_2$.
Vingummiproduktionen har en begrænsning på $36$ enheder, hvor der skal $4$ enheder vingummi i $x_1$ blandingen og $1$ enhed vingummi i $x_2$. 
%
Disse betingelser danner optimeringsproblemet
%
\begin{align*}
\begin{array}{lrrlr} 
\text{Maksimer}		&	\multicolumn{2}{c}{z=5x_1+4x_2}  &\\
\text{begrænset af}	&3x_1& +5x_2			&\leq 	&78,\\
					&4x_1& + x_2				&\leq	& 36,\\
					&x_1,& x_2				&\geq	& 0.
\end{array}
\end{align*}
%
\end{eks}
%
\subsubsection{1. Opskriv optimeringsproblemet på standardform med slack-variable.}
%
Generelt tager simplexmetoden udgangspunkt i, at alle variable er positive. Jævnfør afsnit \ref{sec:standard} tilføjes $x_i^+$ og $x_i^-$, hvis der i optimeringsproblemet ikke eksisterer ikke-negativitetsbetingelser for variablene. 
I dette tilfælde er begge variable positive, og det er derfor ikke nødvendigt at opdele variablene. 
Med udgangspunkt i metoden fra afsnit \ref{sec:standard} opstilles optimeringsproblemet derfor på standardform som
%
\begin{align*}
\begin{array}{lrrrrrlr}
\text{Maksimér}		& -5x_1 &-4x_2 &&& + z & =&0\phantom{,}\\
\text{begrænset af}	&3x_1& +5x_2	& + \textcolor{blue}{s_1} 	&&&= 	&78,\\
					&4x_1& + x_2	& & + \textcolor{blue}{s_2}	&&=	&	 36.\\
\end{array}
\end{align*}
%
Slack-variablene $s_1$ og $s_2$ er her markeret med blå, da de ikke indgår i den endelige løsning.
%Slackvariablerne indgår i dualproblemets løsning. xd lit fam 420 
%
\subsubsection{2. Opstil simplextabellen for optimeringsproblemet}		
% 
Nu opstilles simplextabellen for optimeringsproblemet. 
Med udgangspunkt i et generelt lineært optimeringsproblem på formen
%
% & =v, \text{ hvor } v=0
\begin{align*}
\begin{array}{lrl}
\text{Maksimér}		&z -\textbf{c}^T\textbf{x}	& =0	\\
\text{begrænset af}	&A\textbf{x}	&=\mathbf{b},	\\
					&\mathbf{x}				&\geq \mathbf{0},
\end{array}
\end{align*}
er simplextabellen en matrix, som indeholder de lineære betingelser, slack-variablene og objektfunktionen. 
Matricen $\mathbf{A}$ opskrives først, og derefter opskrives identitesmatricen $e_{m+1}$, hvor $m$ er antallet af slack-variable, samt den optimale løsning $z$, og til sidst opskrives $\mathbf{b}$. 
Under $\mathbf{A}$ tilføjes objektfunktionen $- \mathbf{c}^T$. 
Identitetsmatricen $I_m$, som dannes med udgangspunkt i antallet af slack-variable danner basismatricen $B$.
En general form for simplextabbelen ser ud som følger:
%
\begin{align*}
\begin{blockarray}{ccccccccccc}
x_1 & x_2 & \cdots & x_n & \textcolor{blue}{s_1} & \textcolor{blue}{s_2} &  \textcolor{blue}{\cdots} & \textcolor{blue}{s_m} & z & b \\
\begin{block}{[cccc|ccccc|c]c}
a_{1,1} & a_{1,2} & \cdots & a_{1,n} & 1 & 0 & \cdots & 0 & 0 & b_1 \\
a_{2,1} & a_{2,2} & \cdots & a_{2,n} & 0 & 1 & \cdots & 0 & 0 & b_2 \\
\vdots & \vdots & \ddots & \vdots & \vdots & \vdots & \ddots & \vdots & \vdots & \vdots \\
a_{m,1} & a_{m,2} & \cdots & a_{m,n} & 0 & 0 & \cdots  & 1  & 0 & b_{m}\\
\cline{1-10}
-c_1 & -c_2 & \cdots & -c_n & 0 & 0 & \cdots & 0 & 1 & v\\
\end{block}
\end{blockarray}.
\end{align*}
%
Optimeringsproblemet fra \ref{haribooooo} har dermed følgende simplextabel:
%
\begin{align*}
\begin{blockarray}{cccccc}
x_1 & x_2 & \textcolor{blue}{s_1} & \textcolor{blue}{s_2} & z & b \\
\begin{block}{[cc|ccc|c]}
3 & 5 & 1 & 0 & 0 & 78 \\
4 & 1 & 0 & 1 & 0 & 36 \\
\cline{1-6}
-5 & -4 & 0 & 0 & 1 & 0\\
\end{block}
\end{blockarray}.
\end{align*}
%
\subsubsection{3. Kontroller for optimalitet ellers identificér en pivoteringsindgang}
%
Først kontrolleres for optimalitet ved at finde det mindste negative koefficient i objektfunktionen, hvilket findes i nederste række i simplextabellen:
%
\begin{align*}
\begin{blockarray}{cccccc}
x_1 & x_2 & \textcolor{blue}{s_1} & \textcolor{blue}{s_2} & z & b \\
\begin{block}{[cc|ccc|c]}
3 & 5 & 1 & 0 & 0 & 78 \\
4 & 1 & 0 & 1 & 0 & 36 \\
\cline{1-6}
\hlight{-5} & -4 & 0 & 0 & 1 & 0\\
\end{block}
\end{blockarray}.
\end{align*}
%
Søjlen, hvori denne værdi er, kaldes \textit{pivotsøjlen}. 
Herefter findes \textit{pivotrækken} og \textit{pivoteringsindgangen}, ved at finde den mindste $u_i$-værdi ud fra 
\begin{align*}
\frac{a_{i,j}}{b_j}=u_i
\end{align*}
%
i pivotsøjlen.
Pivoteringsindgangen findes ved at betragte 
%
\begin{align*}
u_1 = \frac{78}{3} = 26 \text{  } \text{ og } \text{   } u_2 = \frac{36}{4} = 9.
\end{align*}
%
Eftersom $9$ er den laveste værdi, er $a_{2,1}$ pivoteringsindgangen, mens den række, som den er i, er pivotrækken. 
I nedenstående er pivoteringsindgangen markeret.
%
\begin{align*}
\begin{blockarray}{cccccc}
x_1 & x_2 & \textcolor{blue}{s_1} & \textcolor{blue}{s_2} & z & b \\
\begin{block}{[cc|ccc|c]}
3 & 5 & 1 & 0 & 0 & 78 \\
\hlight{4} & 1 & 0 & 1 & 0 & 36 \\
\cline{1-6}
-5 & -4 & 0 & 0 & 1 & 0\\
\end{block}
\end{blockarray}.
\end{align*}	
%	
\subsubsection{4. Opstil en ny tabel ved hjælp af pivotering}
%
Ved brug af de elementære rækkeoperationer jævnfør \ref{defn:element}, skaleres pivoteringsindgangen til $1$, og der skabes nulindgange under og over pivoteringsindgangen ved hjælp af rækkeudskiftning.
Nedenfor ses dette gjort for eksemplet.
%
%& \begin{blockarray}{cccccc}
%x_1 & x_2 & \textcolor{blue}{s_1} & \textcolor{blue}{s_2} & z & b \\
%\begin{block}{[cc|ccc|c]}
%\hlight{3} & 5 & 1 & 0 & 0 & 78 \\
%4 & 1 & 0 & 1 & 0 & 36 \\
%\cline{1-6}
%\hlight{-5} & -4 & 0 & 0 & 1 & 0\\
%\end{block}
%\end{blockarray} \\
%
\begin{align*}
\xrightarrow[]{R_2 \rightarrow \frac{1}{4} R_2} &
\begin{blockarray}{cccccc}
x_1 & x_2 & \textcolor{blue}{s_1} & \textcolor{blue}{s_2} & z & b \\
\begin{block}{[cc|ccc|c]}
3 & 5 & 1 & 0 & 0 & 78 \\
1 & \frac{1}{4} & 0 & \frac{1}{4} & 0 & 9 \\
\cline{1-6}
-5 & -4 & 0 & 0 & 1 & 0\\
\end{block}
\end{blockarray} \\
\xrightarrow[R_3 \rightarrow R_3 + 5 R_2 ]{R_1 \rightarrow R_1 -3 R_2} &
\begin{blockarray}{cccccc}
x_1 & x_2 & \textcolor{blue}{s_1} & \textcolor{blue}{s_2} & z & b \\
\begin{block}{[cc|ccc|c]}
0 & \frac{17}{4} & 1 & \frac{-3}{4} & 0 & 51 \\
1 & \frac{1}{4} & 0 & \frac{1}{4} & 0 & 9 \\
\cline{1-6}
0 & \frac{-11}{4} & 0 & \frac{5}{4} & 1 & 45\\
\end{block}
\end{blockarray}.
\end{align*}	
%
\subsubsection{5. Gentag trin 3 og 4 indtil den optimale løsning er fundet}
%
Denne proces fortsættes nu, indtil der ikke er flere negative tal i objektfunktionen.
Bemærk, at \textit{pivotering} er processen fra valget af den mindste negative værdi i objektfunktionen indtil en ny værdi i objektfunktionen kan vælges.
Der kontrolleres for optimalitet igen, og en ny pivoteringsindgang er valgt ved den mindste negative værdi. 
Denne er markeret i nedenstående matrix.	
%Jeg tænker, at ovenstående kan præciseres, men jeg er blank pt.... Så hvad ved jeg :0) 
\begin{align*}
\begin{blockarray}{cccccc}
x_1 & x_2 & \textcolor{blue}{s_1} & \textcolor{blue}{s_2} & z & b \\
\begin{block}{[cc|ccc|c]}
0 & \frac{17}{4} & 1 & \frac{-3}{4} & 0 & 51 \\
1 & \frac{1}{4} & 0 & \frac{1}{4} & 0 & 9 \\
\cline{1-6}
0 & \hlight{\frac{-11}{4}} & 0 & \frac{5}{4} & 1 & 45\\
\end{block}
\end{blockarray}
\end{align*}
%
Pivoteringsindgangen findes ved at betragte
%
\begin{align*}
u_1 = \frac{51}{\frac{17}{4}} =12 \text{  }  \text{ og }  \text{   } u_2 = \frac{9}{\frac{1}{4}} =36.
\end{align*}
%
Pivoteringsindgangen er $a_{1,2}$, da denne giver den mindste værdi på $12$, hvilket er markeret i nedenstående matrix.
%
\begin{align*}
\begin{blockarray}{cccccc}
x_1 & x_2 & \textcolor{blue}{s_1} & \textcolor{blue}{s_2} & z & b \\
\begin{block}{[cc|ccc|c]}
0 & \hlight{\frac{17}{4}} & 1 & \frac{-3}{4} & 0 & 51 \\
1 & \frac{1}{4} & 0 & \frac{1}{4} & 0 & 9 \\
\cline{1-6}
0 & \frac{-11}{4} & 0 & \frac{5}{4} & 1 & 45\\
\end{block}
\end{blockarray}
\end{align*}
%
Pivotindgangen skaleres til $1$, og der skabes nulindgange under og over pivotindgangen:
%& \begin{blockarray}{cccccc}
%x_1 & x_2 & \textcolor{blue}{s_1} & \textcolor{blue}{s_2} &z & b \\
%\begin{block}{[cc|ccc|c]}
%0 & \frac{17}{4} & 1 & \frac{-3}{4} & 0 & 51 \\
%1 & \hlight{\frac{1}{4}} & 0 & \frac{1}{4} & 0 & 9 \\
%\cline{1-6}
%0 & \hlight{\frac{-11}{4}} & 0 & \frac{5}{4} & 1 & 45\\
%\end{block}
%\end{blockarray}\\
%
\begin{align*}
\xrightarrow[]{R_1 \rightarrow \frac{4}{17} R_1} &
\begin{blockarray}{cccccc}
x_1 & x_2 & \textcolor{blue}{s_1} & \textcolor{blue}{s_2} & z & b \\
\begin{block}{[cc|ccc|c]}
0 & 1 & \frac{4}{17} & \frac{-3}{17} & 0 & 12 \\
1 & \frac{1}{4} & 0 & \frac{1}{4} & 0 & 9 \\
\cline{1-6}
0 & \frac{-11}{4} & 0 & \frac{5}{4} & 1 & 45\\
\end{block}
\end{blockarray} \\
\xrightarrow[R_3 \rightarrow R_3 + \frac{11}{4} R_2 ]{R_2 \rightarrow R_2 -\frac{11}{4} R_1} &
\begin{blockarray}{cccccc}
x_1 & x_2 & \textcolor{blue}{s_1} & \textcolor{blue}{s_2} & z & b \\
\begin{block}{[cc|ccc|c]}
0 & 1 & \frac{4}{17} & \frac{-3}{17} & 0 & 12 \\
1 & 0 & \frac{-1}{17} & \frac{20}{17} & 0 & 6 \\
\cline{1-6}
0 & 0 & \frac{11}{17} & \frac{52}{17} & 1 & 78\\
\end{block}
\end{blockarray}.
\end{align*}	
%
%
\subsubsection{6. Identificér den optimale løsning}
%
Eftersom der ikke er flere negative værdier i objektfunktionen, er den optimale løsning nu fundet. 
Denne løsning kan aflæses direkte af simplextabellen, og den optimale løsning for eksemplet er dermed
%
\begin{align*}
x_1 & = 6, \\
x_2 & = 12, \\
z   & = 78.
\end{align*}
%
Haribos profit vil derfor være størst ved produktion af $6$ enheder af $x_1$ blandingen og $12$ enheder af $x_2$ blandingen. Dette giver en profit på $78$ enheder.
%
%
% Typer af simplexmetoden
% ---------------------------------------------------------
	%
\section{Implementering}
\label{julieergudesmuk}
Der findes tre forskellige implementeringer af simplexmetoden, henholdsvis den \textit{naive}, den \textit{reviderede}, og den \textit{fuld-tabel} implementering. Disse implementeringer vil blive gennemgået i følgende afsnit. 

\subsection{Den naive implementering}
Først introduceres den naive implementering af simplexmetoden, hvor ingen af de bærende elementer overgår fra en iteration til den næste iteration. 
For den indledende iteration, haves indekserne
$B(1),\ldots,B(m)$ for de givne basisvariable. 
Derudfra kan der dannes en basis matrix $B$ og beregne $\mathbf{p}^T=\mathbf{c}_{\text{B}}^T B^{-1}$ ved at løse det lineære ligningssystem $\mathbf{p}^T B=\mathbf{c}_{\text{B}}^T$ for den ukendte vektor $\mathbf{p}$. 
Hertil kan det ydermere nævnes, at $\mathbf{p}$ betegnes som den vektor bestående af de simplexmultiplikationer, som tilknyttet med basen $B$. 
Den reducerede objektfunktion $c_j^* = c_j - \mathbf{c}_B^T B^{-1}A_j$ for enhver variable $x_j$, er derfor givet ved følgende formel:
%
\begin{align*}
c_j^* = c_j - \mathbf{p}^T A_j.
\end{align*}
%
Afhængigt af om pivoteringsreglen er benyttet, er det nødvendigt at udregne alle de mulige reducerede objektfunktionerne eller i så fald udregne indtil en negativ objektfunktion opstår. Når en søjle $\mathbf{A}_j$ er valgt til at være en del af basen, løses det lineære ligningssystem $\mathbf{Bu}=\mathbf{A}_j$ for at bestemme vektoren $\mathbf{u}=\mathbf{B}^{-1}\mathbf{A}_j$. Ved hjælp af dette vil der kunne konstrueres den retning, hvor man bevæger sig væk fra de nuværende basale mulige løsninger. Endeligt kan $\theta^*$ bestemmes, samt den variabel der endeligt gør, at basen forlades og derved konstruere en ny basel mulig løsning. \\\\
%%%%
%
\subsection{Den reviderede implementering}
Problem ved den naive implementering er dens tunge beregningsmæssige ulempe, da den er nødsaget til at løse to sæt af lineære ligningssystemer. En alternativ implementering til dette, er den revidere implementering, som benytter matricen $B^{-1}$ ved hver iteration, samt er  vektorerne $\mathbf{c}_{\text{B}}^T B^{-1}$ og $B^{-1} \mathbf{A}_j$ udregnet ved hjælp af matrix-vektor multiplikation. En iteration af den reviderede simpleximplementering gennemløber følgende punkter: 
% 
\begin{col}{}{}
\begin{enumerate}
\item I en typisk iteration, startes med en basis indeholdende basissøjlerne $A_{B(1)},\ldots,A_{B(m)}$, en associeret basal mulig løsning $\mathbf{x}$ samt en invers $B^{-1}$ af basismatricen. 
\item Udregn rækkevektoren $\mathbf{p}^T=\mathbf{c}_{\text{B}}^T B^{-1}$ og udregn dernæst den reducerede objektfunktion $c_j^* = c_j - \mathbf{p}^T \mathbf{A}_j$. Hvis resultaterne alle er ikke-negative, vil det resultere i den givne basal mulige løsning være optimal, og algoritmen vil slutte ved dette punkt. Hvis dette ikke er tilfældet, så vælg et $j$, hvor $c_j^* < 0$.
\item Udregn $\mathbf{u}=B^{-1}\mathbf{A}_j$. Hvis intet komponent i $\mathbf{u}$ er positivt, så er den optimale pris $-\infty$, og algoritmen vil slutte ved dette trin. 
%
% Cost er vel ligmed pris her?? Eller hvad? %
%
\item Hvis mindst en komponent i $\mathbf{u}$ er positivt, lad 
\begin{align*}
\theta^*=\underset{\{i=1,\ldots,m \mid u_i>0\}}{\text{min}}\frac{x_{B(i)}}{u_i}.
\end{align*}
\item Bestem $l$, så $\theta^*=\frac{x_{B(l)}}{u_l}$ er gældende. Bestem dernæst en ny base ved at udskifte $\mathbf{A}_{B(l)}$ med $\mathbf{A}_j$. Hvis $\mathbf{y}$ er den nye basal mulige løsning, så er værdierne for de nye basale værdier $y_j=\theta^*$ samt $y_{B(i)}=x_{B(i)}-\theta^*u_i$, hvor $i \neq l$.
\item Bestem til slut den $m \times (m+1)$ matrix $\left [B^{-1} \mid \mathbf{u} \right ]$. Tilføj enhver række et multiplum af den $l$-te række for at få den sidste søjle til at være lig med enhedsvektoren $\mathbf{e}_l$. De første $m$ søjler er resultatet af matricen $\mathbf{B}^{*-1}$.
%Nogle der har en ide til at gøre dette pænere? Det er B-stjerne i -1% 
\end{enumerate}
\end{col}
\noindent
%
%
\subsection{Fuld-tabel implementeringen}
Slutteligt er det interessant at beskrive den sidste implementering af simplexmetoden, nemlig den fuld-tabel implementeringen. 
Det er gældende for den fulde tabel implementering, at den i stedet for at opretholde og opdatere matricen $\mathbf{B}^{-1}$ ved hver iteration, som i den reviderede implementering; så opretholder og opdaterer den $m \times (n+1)$ matricen, givet ved $\mathbf{B}^{-1} \left [ \mathbf{b} \mid \mathbf{A} \right ]$, med søjlerne $\mathbf{B}^{-1}\mathbf{b}$ og $\mathbf{B}^{-1}\mathbf{A}_1,\ldots,\mathbf{B}^{-1}\mathbf{A}_n$. 
Denne matrix betegnes som en simplextabel. Bemærk, at søjlen $\mathbf{B}^{-1}\mathbf{b}$, betegnes som \textit{nulsøjlen} og indeholder værdierne for de basale variable. 
Søjlen $\mathbf{B}^{-1}\mathbf{A}_i$ betegnes som den $i$-te søjle af tabellen. Søjlen $\mathbf{u} = \mathbf{B}^{-1}\mathbf{A}_j$ er svarende til den variabel, som er i basen, er betegnet som pivotsøjlen. 
Hvis den $l$-te basale variabel udgår fra basen, så er den $l$-te række, betegnet som pivotrækken. 
Endelig er elementet, der tilhører både pivotrække og pivotsøjlen betegnet ved pivoteringsindgangen og bemærk yderligere, at dette element er $u_l$ og altid er positivt. 
For at klarlægge hvordan den fulde tabel implementering fungerer, vil der i følgende bliver gennemgået en iteration af den fulde tabel implementering: 
%
\begin{col}{}{}
\begin{enumerate}
\item I en typisk iteration, startes der med at opstille en tabel associeret med en basis matrix $\mathbf{B}$ og en tilsvarende basal mulig løsning $\mathbf{x}.$
\item Dernæst undersøges om den reducerede pris i nulrækken af tabellen. Hvis alle indgange er ikke-negative, så er den nuværende basal mulige løsning den optimale, og algoritmen vil slutte ved dette punkt. Hvis dette ikke er tilfældet, så vælg ethvert $j$, hvor $c_j^* < 0$.
\item Overvej vektoren $\mathbf{u}=\mathbf{B}^{-1}\mathbf{A}_j$, hvor den $j$-te søjle, som er pivotsøjlen, i tabellen. Hvis intet komponent i $\mathbf{u}$ er positivt, så er den optimale pris $-\infty$, og algoritmen vil slutte ved dette trin. 
\item For hvert $i$, hvor $u_i$ er positivt, udregn forholdet $\frac{x_{B(i)}}{u_i}$. Lad $l$ være det index  af en række, som tilsvarer det laveste forhold. Søjlen $\mathbf{A}_{B(l)}$ udgår fra basen og søjlen $\mathbf{A}_j$ indgår i basen. 
\item Tilføj for hver række i tabellen en konstant multiplum af den $l$-te række, som er pivotrækken, så $u_l$, som er pivoteringsindgangen bliver lig med $1$ og alle andre indgange i pivotsøjlen bliver 0. 
\end{enumerate}
\end{col}
\noindent
%
%
%
% Pivoterings-løkker
% ---------------------------------------------------------
	\section{Pivoteringsløkker}
Ved degenererede basale løsninger er der en risiko for, at simplexmetoden blot skifter variable ind og ud af basis, således at den sidder fast i en uendelig \textit{pivoteringsløkke}.  
For at garantere at simplexmetoden stopper ved en optimal værdi og ikke fortsætter med at pivotere uendeligt, kan der indføres forskellige regler for, hvilken variabel, der bringes ind i løsningen.

\subsection{Lexicografi}
%ingen ide om det hedder lexicografi på dansk men skiver det indtil videre
%
Før den \textit{lexicografiske metode} til at vælge pivotsøjler introduceres, er det nødvendigt at definere \textit{lexicografi}.
\begin{defn}{}{}
En vektor $\textbf{u}\in \R^n$ er \textbf{lexicografisk} større, eller mindre end en vektor $\textbf{v}\in \R^n$, hvis $\textbf{v} \neq \textbf{u}$ og den første ikke-nul indgang i $\textbf{u}-\textbf{v}$ er henholdsvis positiv eller negativ. Dette noteres
\begin{align*}
\textbf{u} >^L \textbf{v} \phantom{..} \text{ eller }\phantom{..} \textbf{u} <^L \textbf{v}.
\end{align*} 
\end{defn}
\noindent
Hvis det første ikke-nul element af en vektor er positiv, siges vektoren at være lexicografisk positiv. Modsat siges den at være lexicografisk negativ, hvis det første ikke-nul element af en vektor er negativt. Et eksempel på lexicografi ses i \ref{eks:lexi}.
\\
%
\begin{eks}\label{eks:lexi}
Givet vektorerne
$$\textbf{u}=
\begin{bmatrix}
2\\
3\\
1\\
\end{bmatrix}
,\phantom{..}
\textbf{v}=
\begin{bmatrix}
2\\
1\\
2\\
\end{bmatrix}
\phantom{..}\text{ og }\phantom{..}
\textbf{z}=
\begin{bmatrix}
3\\
2\\
2\\
\end{bmatrix}
.$$
Så er $\textbf{u}$  lexicografisk større end $\textbf{v}$, $\textbf{u} >^L \textbf{v}$, da den første ikke-nul værdi af $\textbf{u}-\textbf{v}$ er $2$.
Til gengæld er $\textbf{u}$ lexicografisk mindre end $\textbf{z}$, $\textbf{u} <^L \textbf{z}$, da den første ikke-nul værdi er $-1$.
Bemærk, at alle $3$ vektorer er lexicografisk positive.
\end{eks}
%
Følgende er en lexicografisk metode til at udvælge de rækker, der skal pivoteres.
\begin{enumerate}
\item Vælg en arbitrær søjle $\textbf{A}_j$ til at indgå i basen, så længe dens reducerede omkostning $c_j^*$ er negativ.
Lad $\textbf{u}$ være den $j$'te søjle i simplextabellen.
\item For hvert $i$, hvor $u_i>0$, divideres den $i$'te række med $u_i$, og den lexicografisk mindste række vælges. 
Hvis rækken $L$ er den lexicografisk mindste række, så forlader den $L$'te basale variabel, $x_{B(L)}$, basen.
\end{enumerate}
%
Det følger, at ethvert valg af pivotering via den lexicografiske metode er entydig, da der ellers ville være en anden række, som var proportional med den valgte række. 
Dette er i strid med antagelsen om, at rækkerne er lineært uafhængige.
%
\begin{thm}{}{}
Antag, at simplexmetoden starter med, at alle rækker i en $m \times n$ simplextabel foruden rækken med objektfunktionen, den $m$'te række, er lexicografisk positive.
Hvis den lexicografiske metode til at vælge pivotering er fulgt, så gælder det, at
%
\begin{enumerate}[label=(\alph*)]
\item Alle rækkerne foruden den $m$'te række forbliver lexicografiske positive gennem algoritmen.
\item Den $m$'te række vokser strengt lexicografisk for hver iteration.
\item Simplexmetoden stopper efter et endeligt antal iterationer. 
\end{enumerate}
\end{thm}
%
\begin{proof}
\begin{enumerate}[label=(\alph*)]
\item Antag, at alle rækkerne af en simplextabel, udover den $m$'te række, er lexicografisk positive i begyndelsen af en iteration af simplex. 
Antag dernæst, at $x_j$ bringes ind i løsningen, og at pivotrækken er den $l$'te række.
Så gælder der jævnfør den lexicografiske metode, at
$$\dfrac{l\text{'te række}}{u_l}<^L \dfrac{i\text{'te række}}{u_i}, \text{  hvis } i\neq l \text{ og }u_i>0.$$
For at bestemme den nye tabel divideres den $l$'te række med den positive variabel $u_l$ og forbliver dermed lexicografisk positiv.
For rækker $i$, hvor $u_i<0$, skal der lægges et positivt multiplum af pivotrækken til, således at den $(i,j)$'te indgang bliver nul. 
Grundet begge rækkers lexicografiske positivitet er den resulterende række også lexicografisk positiv. 
For rækker $i$, hvor $u_i>0$ og $i\neq l$, haves, at
$$\text{(nye }i\text{'te række)}=\text{(gamle }i\text{'te række)}-\dfrac{u_i}{u_l}\text{(gamle }l\text{'te række)}.$$
Grundet den tidligere ulighed, som opfyldes af de gamle rækker, vides det, at den nye $i$'te række også er lexicografisk positiv.
% 
\item I begyndelsen af hver iteration, er pivotværdien i den $m$'te række negativ, så der skal ligges et positiv multiplum af pivotrækken til. 
Da pivotrækken er lexicografisk positiv, stiger den $m$'te række lexicografisk.
% 
\item Eftersom den $m$'te række stiger lexicografisk for hver iteration, kan den aldrig returnere til en tidligere værdi.
Da den $m$'te række er bestemt entydig ud fra nuværende basis, så kan intet basis gentage sig selv, og derfor må simplexmetoden stoppe efter et endeligt antal iterationer.
\end{enumerate}
\end{proof}
%
%Jeg tror ikke Bland's regel er relevant for os medmindre at vi skriver om den revidererde simpelx metode.
%\subsection{Bland's regel}
%En anden metode
%
%
%
%
%
%
% Kompleksitet
% ---------------------------------------------------------
	\section{Kompleksitet}
\label{kompleksitet}
% hehi
Et vigtigt kriterium for valg af simplexmetoden er tidskompleksitet.
Til beskrivelse af dette benyttes \textit{store-$O$} notation \citep[side 205]{dmat}.
% Side 226 i PDF'en (diskret matematik 7th)
%
\begin{defn}{}{}
Lad $f(x)$ og $g(x)$ være vilkårlige funktioner. $f(x)$ er \textbf{store-$O$} af $g(x)$, hvis der eksisterer et $C$ og $k$, således at $|f(x)| \leq C|g(x)|, x \geq k$. $C$ og $k$ kaldes \textbf{vidner}.
\end{defn}
\noindent
%
Som beskrevet i afsnit \ref{julieergudesmuk} findes der forskellige implementeringer af simplexmetoden.
Disse har hver deres tidskompleksitet og rumkompleksitet.\\\\
%
Fuld-tabel metoden kræver et konstant antal af operationer for at opdatere indgangene i simplextabellen.
Derfor er antallet af operationer proportionelt med størrelsen på tabellen, og dermed er tidskompleksiteten og rumkompleksiteten $O(mn)$.\\\\
%Aretmestiske operationer betyder bare udregninger
Den naive implementering behøver $O(m^3)$ aritmetiske operationer for at løse systemerne $\mathbf{p}^T B=\mathbf{c}_{\text{B}}^T$ og $B\mathbf{u}=\mathbf{A}_j$.
Ydermere kræver udregningen af den reducerede objektfunktion %gældende for alle variabler, 
$O(mn)$ aritmetiske operationer, da det er nødvendigt at udregne det indre produkt af vektoren $\mathbf{p}$ for alle ikke-basissøjler i $\mathbf{A}_j$.
Derfor er den samlede tidskompleksitet $O(m^3 + mn) = O(m^3)$ for den naive implementering.\\\\
%
Den reviderede metode bruger samme operationer som fuld-tabel implementeringen, men opdaterer kun $O(m^2)$ indgange og har derfor tidskompleksiteten $O(m^2)$.
Det kan dog forekomme, at metoden skal opdatere alle variable.
Hver udregning af en variabel kræver $O(m)$ operationer, hvilket medfører, at der i værste tilfælde bruges $O(mn)$ operationer.
Da $m \leq n$, er tidskompleksiteten i værste tilfælde $O(mn)$.
Med dette konkluderes, at den reviderede metode aldrig vil køre langsommere end fuld-tabel og den naive implementering.
Samtidigt vil en iteration fra den reviderede metode være hurtigere end fuld-tabel i alle andre tilfælde end det værste.
Endnu et vigtigt element, som er tiltalende ved den reviderede metode, er rumkompleksiteten på $O(m^2)$.
Her kan der igen være en ret betydelig forskel, afhængigt af størrelsen på $n$.\\\\
%
Selvom antallet af basale mulige løsninger kan stige eksponentielt med antallet af variable og betingelser, så har simplexmetoden typisk fundet en optimal løsning efter $O(m)$ iterationer.
Dog eksisterer der polyedre, hvor antallet af påkrævede pivoteringer stiger eksponentielt.
For ikke-degenererede problemer bevæger simplexmetoden sig altid fra et ekstremumspunkt til et af dets tilstødende ekstremumspunkter og forbedrer omkostningen.
Lad et polyeder have et eksponentielt antal af ekstremumspunkter, og lad en sti besøge alle disse.
Denne sti er konstrueret ved, fra et ekstremumspunkt, at besøge et ekstremumspunkt med mindre omkostning.
Med sådan et polyeder vil simplexmetoden, med en pivoteringsregel, som følger stien, kræve et eksponentielt antal af pivoteringer.
%%
\begin{thm}{}{}
Betragt det lineære optimeringsproblem, hvor objektfunktionen $\textbf{c}_n^T \textbf{x} = -x_n$ minimeres med betingelserne $\varepsilon \leq x_1 \leq 1$ og $\varepsilon x_{i-1} \leq x_i \leq 1 - \varepsilon x_{i-1}$ for $i = 2, 3, \ldots, n$ og $0 < \varepsilon < \frac{1}{2}$.
Så gælder det, at
%
\begin{enumerate}[label=(\alph*)]
\item Løsningsmængden har $2^n$ ekstremumspunkter.
\item Ekstremumspunkterne kan arrangeres således, at hvert tilstødende ekstremumspunkt har lavere omkostning end det forrige ekstremumspunkt.
\item Der eksisterer en pivoteringsregel til simplexmetoden, således at den kræver $2^n-1$ pivoteringer, før algoritmen afsluttes.
\end{enumerate}
%
\end{thm}
%
\begin{proof}
%
Lad $\mathcal{P}_n$ være et polyeder i $\R^n$, begrænset af betingelserne $\varepsilon \leq x_1 \leq 1$ og $$\varepsilon x_{i-1} \leq x_i \leq 1 - \varepsilon x_{i-1}$$ for $i = 2, 3, \ldots, n$ og $0 < \varepsilon < \frac{1}{2}$.
%
\begin{enumerate}[label = (\alph*)]
\item Ved induktion vises, at $\mathcal{P}_n$ har $2^n$ ekstremumspunkter. 
Basistrinnet, hvor $n=1$, følger trivielt.
Antag nu, at resultatet gælder for $n=k$.
Polyederet $\mathcal{P}_{k+1}$ er begrænset af samme betingelser som $\mathcal{P}_{k}$ med tilføjelse af betingelsen $\varepsilon x_{k} \leq x_{k+1} \leq 1 - \epsilon x_{k}$.
Den øvre og nedre begrænsning i $\varepsilon x_{k} \leq x_{k+1} \leq 1 - \varepsilon x_{k}$ skærer ikke hinanden, mens $ x_{k-1} \leq x_k \leq 1 - \varepsilon x_{k-1}$ er opfyldt.
Dette medfører, at hvert ekstremumspunkt i $\mathcal{P}_{k}$ giver anledning til to entydige ekstremumspunkter i $\mathcal{P}_{k+1}$, hvilket viser (a).
%
%
\item Med induktion vises det, at ekstremumspunkterne $\textbf{p}_1, \ldots \textbf{p}_{2^n}$ i $\mathcal{P}_n$ kan arrangeres således, at hvert ekstremumspunkt tilstøder det forrige og $\textbf{c}_n^T  \textbf{p}_1 > \cdots > \textbf{c}_n^T  \textbf{p}_{2^n}$, når $\textbf{c}_n^T \textbf{x} = -x_n$.
Basistrinnet for $n=1$ følger trivielt.
Antag nu, at resultatet gælder for  $n = k$.
Da et polyeder er afgrænset af polyedre i en dimension lavere, kan to $\mathcal{P}_k$ afgrænse polyederet $\mathcal{P}_{k+1}$.
Disse to afgrænsende polyedre vil henholdsvis have $\varepsilon x_{k} \leq x_{k+1}$ eller $x_{k+1} \leq 1 - \varepsilon x_{k}$ som aktiv betingelse i alle ekstremumspunkter.
Da ekstremumspunkterne i $\mathcal{P}_k$ er arrangeret således, at $\textbf{c}_n^T  \textbf{p}_1 > \cdots > \textbf{c}_n^T  \textbf{p}_{2^n}$, kan den ønskede sti findes.
Ekstremumspunkterne i $\mathcal{P}_{k+1}$ skal således arrangeres sådan, at $$1 - \varepsilon \textbf{c}_n^T \textbf{p}_{2^n} > \cdots > 1 - \varepsilon \textbf{c}_n^T  \textbf{p}_1 > \varepsilon \textbf{c}_n^T  \textbf{p}_1 > \cdots > \varepsilon \textbf{c}_n^T \textbf{p}_{2^n},$$
hvilket viser (b).

%$\mathcal{P}_{k+1}$ kan deles i to $\mathcal{P}_{k}$ polyedre.
%Et polyeder, hvor $\varepsilon x_{k} \leq x_{k+1}$ er aktiv og et polyeder, hvor $x_{k+1} \leq 1 - \varepsilon x_{k}$ er aktiv.
%Således er det muligt at finde den ønskede udspændende sti i $\mathcal{P}_{k+1}$ ved først at tage den udspændende sti i $\mathcal{P}_{k}$, hvor $\varepsilon x_{k} \leq x_{k+1}$ er aktiv, og dernæst at skifte til $\mathcal{P}_{k}$, hvor $x_{k+1} \leq 1 - \varepsilon x_{k}$ er aktiv og tage dennes udspændende sti i modsat rækkefølge.
%Dette viser (b), da $x_k$ er strengt voksende i det første $\mathcal{P}_{k}$ og strengt faldende i det andet $\mathcal{P}_{k}$, hvilket medfører, at $x_{k+1}$ er strengt voksende gennem hele stien.
%
\item Jævnfør afsnit \ref{julieerfantalastiskogvidunderlig} bevæger simplexmetoden sig altid til et tilstødende ekstremumspunkt, når problemet ikke er degenereret.
Med udgangspunkt i (b) vides det, at der eksisterer en udspændende sti, som ender i problemets optimale løsning og $\textbf{c}^T\textbf{x}_i < \textbf{c}^T\textbf{x}_{i-1}$ for $i = 2, 3, \ldots ,n$.
Dette medfører, at en pivoteringsregel, som mindsker omkostningen mindst muligt pr. pivotering, følger den udspændende sti og besøger alle $2^n$ ekstremumspunkter, hvilket viser (c).
\end{enumerate}
\end{proof}
%
Således er kompleksitet et vigtigt kriterium for valg af simplexmetode.
Ligeledes er valg af pivoteringsregel afgørende, eftersom det i praktiske applikationer er de afgørende faktorer for algoritmens brugbarhed.
%Køb en Yamaha.
%
%


