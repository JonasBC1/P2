\chapter{Simplexmetoden}
\label{coronaaaaaaaaaaa}
%
Som vist i afsnit \ref{klogskab}, findes der en optimal løsning til et lineært ligningssystem på standardform, så længe der findes mindst én basal mulig løsning. 
\textit{Simplexmetoden} benytter sig af denne egenskab og søger at finde den optimale løsning.
Dette kapitel tager udgangspunkt i \citep[side 25-34]{lay} og \citep[side 82-87 og 94-100]{bert}, hvis ikke andet er angivet.
Bemærk, at afsnit \ref{afsnittet} tager afsæt i et maksimumsproblem, imens øvrige afsnit tager udgangspunkt i minimumsproblemer.
%
% Geometrisk tilgang til simplexmetoden
% ---------------------------------------------------------
	\section{Optimale betingelser}
\label{julieerfantalastiskogvidunderlig}
% --------------------------------------------------------------
%
For at finde den optimale løsning kan der udvælges en basal mulig løsning og undersøges, hvorvidt der findes andre løsninger, som er bedre, end den valgte. 
Hvis der ikke findes bedre løsninger, er den valgte basale mulig løsning den optimale løsning. 
Eftersom der jævnfør afsnit \ref{julieerlakker} optimeres for en konveks funktion over en konveks mængde, %Det er det med at et lokal minimum er et globalt minimum xD
vil den lokale optimale løsning være den globale optimale løsning. 
%
%Når der undersøges om der findes et nabo punkt, hvis løsning er bedre, så er det ikke nødvendigt at undersøge i retningen, som leder ud af den basal mulige område.
%
% Definition 3.1 i bert
\begin{defn}{}{}
Lad $\textbf{v}$ være et element af et polyeder $\mathcal{P}$.
En vektor $\textbf{d}\in \R^n$ siges at være en \textbf{mulig retning} fra $\textbf{v}$, hvis der eksisterer en positiv skalar $\theta$, således at $\textbf{v}+\theta \textbf{d}\in \mathcal{P}$.
\end{defn}
\noindent
%
På figur \ref{fig:julieersmuuuuuk}, ses vektorerne $\mathbf{u}, \mathbf{v}$ og $\mathbf{w}$ i polyederet $\mathcal{P}$, samt tilhørende forskellige mulige retninger, som stadig er indeholdt i $\mathcal{P}$.
%
\input{fig/tikz/simplex/mulig_retning}
%
%Hjørnerne afsøges derved gennem en reduceret objektfunktion.
%Se \ref{defn:kogtskinke}.
%
%\textcolor{red}{
For at finde en ny mulig retning $\mathbf{d}$ bevæges der i en retning, hvor værdien af en ikke-basal variable $x_j$ øges, og andre ikke-basale variable fastholdes. 
Dermed vil indgangen $d_j = 1$, og $d_i = 0$ for henholdsvis den ikke-basale variable $x_j$ og de andre ikke-basale variable $x_i$, hvor $i \neq j$.
Vektoren 
\begin{align*}
\mathbf{d}_B = \left[ d_{B(1)}, d_{B(2)}, \ldots , d_{B(m)} \right]^T,
\end{align*} 
bestående af de mulige retninger for de basale index, 
kontureres.
Da $\mathbf{v}$ er en basal mulig løsning, gælder det, at $A \mathbf{v} = \mathbf{b}$. 
Eftersom $\mathbf{v}+ \theta \mathbf{d}$ skal være en mulig løsning, gælder det, at $A ( \mathbf{v}+ \theta \mathbf{d}) = \mathbf{b}$.
Deraf haves, at 
\begin{align*}
A ( \mathbf{v}+ \theta \mathbf{d}) &=  A \mathbf{v} + \theta A \mathbf{d} = \mathbf{b} \\
& \Updownarrow \\
\mathbf{b} + \theta A \mathbf{d} & = \mathbf{b} \\
& \Updownarrow \\
\theta A \mathbf{d} & = \mathbf{0}.
\end{align*} 
Bemærk, at $\theta$ er en positiv skalar. 
Dermed skal $\mathbf{d}$ tilhøre $\text{nrum}(A)$.
Eftersom $d_j = 1$ og $d_i = 0$, så er 
%
\begin{align*}
\mathbf{0} = A \mathbf{d} & = \sum^n_{i = 1} \mathbf{A}_i d_i \\
& =  \mathbf{A}_j  + \sum^m_{i = 1} \mathbf{A}_{B(i)} d_{B(i)} \\
& =  \mathbf{A}_j  + B \mathbf{d}_B.
\end{align*}
%
Da basismatricen $B$ er invertibel, haves at
% 
\begin{align*}
\mathbf{A}_j  + B \mathbf{d}_B & = 0 \\
& \Updownarrow \\
B \mathbf{d}_B & = - \mathbf{A}_j \\
& \Updownarrow \\
\mathbf{d}_B & = - B^{-1} \mathbf{A}_j .
\end{align*}
% 
Denne mulige retning $\textbf{d}$, tilhørende $v_j$, kaldes den \textit{$j$'te basale retning}.
% 
\begin{defn}{}{}
Lad $\mathbf{v}$ være en basal mulig løsning. 
Den \textbf{$\textbf{j}$'te basale retning} $\mathbf{d}$, fra $\mathbf{v}$, for en ikke-basal variabel $v_j$, har komponenterne 
%
\begin{align*}
d_j & = 1, \\
d_i & = 0 \phantom{hej} \text{ for } i \neq B(1), \ldots , B(m) \wedge i \neq j, \\
\mathbf{d}_B & = - B^{-1} \mathbf{A}_j.
\end{align*}
% 
\end{defn}
\noindent
%
Bemærk, at overstående sikrer lighedsbegrænsningerne, men det sikrer ikke ikke-negativitets-betingelserne.
Eftersom $v_j$ øges, og de andre ikke-basale variable fastholdes, kan følgende scenarier opstå.
%
\begin{itemize}
\item Antag, at $\mathbf{v}$ er en ikke-degenereret basal mulig løsning. 
Dermed er $\mathbf{v}_B > \mathbf{0}$. 
Hvis $\theta$ er småt nok, vil $\mathbf{v}_B + \theta \mathbf{d}_B \geq 0 $, og $\mathbf{d}$ er dermed en mulig retning. 
\item Antag, at $\mathbf{v}$ er en degenereret basal mulig løsning. 
Dermed er $\mathbf{d}$ ikke altid en mulig retning. 
Eftersom $\mathbf{d}$ er degenereret, er det muligt, at en basal variabel $v_{B(i)}=0$, og den tilsvarende komponent $d_{B(i)}$ er negativ. 
I dette tilfælde vil en bevægelse i den $j$'te mulige retning forlade polyederet, da en ikke-negativitetsbetingelse brydes af $\mathbf{v} + \theta \mathbf{d}$ for alle positive $\theta$.
\end{itemize}
%
På figur \ref{fig:julieerengud} ses de to scenarier. 
Bemærk, at $\textbf{u}$ har de to ikke-basale variable $x_1$ og $x_3$ imens $x_2,$ $x_4$ og $x_5$ er basale variable.
I hjørnet $\textbf{u}$ fastholdes den ikke-basale variable $x_3$, og $x_1$ øges. 
Dermed traverseres der mod $\textbf{v}$ langs linjen mellem $\textbf{u}$ og $\textbf{v}$. 
Bemærk, at $\textbf{v}$ har de to ikke-basale variable $x_3$ og $x_5$ og $x_1,$ $x_1$ og $x_4$ er basale variable.
I hjørnet $\textbf{v}$ fastholdes den ikke-basale variabel $x_5$, og $x_3$ øges. 
Dermed traverseres der langs linjen $\textbf{v}$ og $\textbf{w}$, hvilket bryder en ikke-negativitetsbetingelse, og dermed ikke en mulig retning. 

%
\begin{center}
%
\begin{tikzpicture}[scale=5]
%
% Koordinater
% -------------------------------------------------------
\coordinate (a) at (0,0,0); 
\coordinate (b) at (0.7,0,0); 
\coordinate (c) at (0.393,0.55,0); 
\coordinate (d) at (0.252,0.805,0);
\coordinate (e) at (0.066,0.434,0);
\coordinate (f) at (0,0.41,0);
\coordinate (g) at (0,0.305,0);
\coordinate (h) at (-0.15,0,0); 
%
% Farvning
% -------------------------------------------------------
\filldraw[fill=myblue,opacity=0.3](a)--(b)--(c)--(f)--(a);
  %\draw[thick](-0.2,-0.1,0)--(0.3,0.9,0); % n -> d  
  \draw[thick](0.2,0.9,0)-- node[anchor=north east] {$x_4=0$} (0.757,-0.1,0); % d -> b
  \draw[thick](-0.3,0.3,0)-- node[above] {$x_3=0$ \phantom{hej}} (0.8,0.7,0); % f -> c
  \draw[thick](0.12,0.9,0)-- node[anchor=south west] {$x_5=0$} (0.9,-0.1,0); % ekstra
% 
%
% Punkterne 
% -------------------------------------------------------
%\filldraw [black] (a) circle (0.2pt) node[anchor=north west] {$A$};
%\filldraw [black] (b) circle (0.2pt) node[anchor=north east] {$B$};
\filldraw [black] (c) circle (0.2pt) node[anchor=south west] {$\textbf{v}$};
\filldraw [black] (0.822,0,0) circle (0.2pt) node[anchor=south west] {$\textbf{w}$};
%\filldraw [black] (d) circle (0.2pt) node[anchor=west] {$D$};
%\filldraw [black] (e) circle (0.2pt) node[anchor=north west] {$E$};
\filldraw [black] (f) circle (0.2pt) node[anchor=south east] {$\textbf{u}$};
%\filldraw [black] (g) circle (0.2pt) node[anchor=east] {$G$};
%\filldraw [black] (h) circle (0.2pt) node[anchor=north west] {$H$};
%
% 
\filldraw [black] (0.27,0.17,0) circle (0pt) node[above] {$\mathcal{P}$};
% 
% Koordinatssystem 
% -------------------------------------------------------
\draw[thick] (0,0,0) -- node[anchor=north east] {$x_2=0$} (0.9,0,0);
\draw[thick] (0,0,0) -- (-0.2,0,0);
\draw[thick] (0,0,0) -- node[anchor=north east] {$x_1=0$} (0,0.5,0);
\draw[thick] (0,0.5,0) -- (0,0.9,0);
\draw[thick] (0,0,0) -- (0,-0.2,0);
%
\end{tikzpicture}
  \captionof{figure}{Illustration af en mulig retning, der svarer til linjen mellem $\textbf{u}$ og $\textbf{v}$, og en ikke-mulig retning, der svarer til linjen mellem $\textbf{v}$ og $\textbf{w}$, som bryder en ikke-negativitetsbetingelse.}
  \label{fig:julieerengud}
\end{center}
%
Heraf opstår et problem med degenererede løsninger, og af den grund forsøges disse undgået.
\\\\
%
Ved en bevægelse i en mulig retning medfører det, at 
\begin{align*}
\mathbf{c}^T \mathbf{d} = \mathbf{c}^T_B \mathbf{d}_B + c_j,
\end{align*}
hvor 
$$\mathbf{c}_B=[ c_{B(1)},c_{B(2)}, \cdots , c_{B(m)} ]^T.$$
Bemærk, at da $\mathbf{d}_B = - B^{-1} \mathbf{a}_j$, er 
\begin{align*}
\mathbf{c}^T \mathbf{d} &= \mathbf{c}^T_B ( - B^{-1} \mathbf{a}_j ) + c_j \\
& = c_j - \mathbf{c}^T_B (B^{-1} \mathbf{a}_j).
\end{align*}
%
For, at omkostningen minimeres, skal $c_j - \mathbf{c}^T_B (B^{-1} \mathbf{a}_j ) < 0$, da 
% 
\begin{align*}
\mathbf{c}^T  ( \mathbf{x} + \theta \mathbf{d} )  = \mathbf{c}^T  \mathbf{x} + \theta \mathbf{c}^T \mathbf{d} = \mathbf{c}^T  \mathbf{x} + \theta ( c_j - \mathbf{c}^T_B (B^{-1} \mathbf{a}_j ) ) < \mathbf{c}^T \mathbf{x},
\end{align*}
%
for alle positive $\theta$.
Størrelsen for reduktionen for hvert $\theta$ kaldes \textit{den reducerede omkostning}.
%
% Definition 3.2 
\begin{defn}{}{kogtskinke}
Lad $\mathbf{v}$ være en basal løsning til en basismatrix $B$ med den tilhørende omkostningsvektor $$\mathbf{c}_B=[ c_{B(1)},c_{B(2)}, \cdots , c_{B(m)} ]^T.$$
For hver $j$'te basale retning defineres den \textbf{reducerede omkostning} $c_j^*$ som
\begin{align*}
c_j^* = c_j - \mathbf{c}_B^T (B^{-1} \mathbf{a}_j).
\end{align*} 
%
\end{defn}
\noindent
%
\textit{Den reducerede omkostningsvektor} $\mathbf{c}^*$ indeholder alle de reducerede omkostninger, hvor den $k$'te række betegner den reducerede omkostning for den $k$'te basale retning.
%
\begin{align*}
\mathbf{c}^* = (\mathbf{c}^T - \mathbf{c}_B^T B^{-1} A)^T = 
\begin{blockarray}{[c]}
c_1 - \mathbf{c}_B^T B^{-1} \mathbf{a}_1 \\
\vdots \\
c_k - \mathbf{c}_B^T B^{-1} \mathbf{a}_k \\
\vdots \\
c_n - \mathbf{c}_B^T B^{-1} \mathbf{a}_n
\end{blockarray}.
\end{align*}
%
Betragt de indgange, for de basale variable i den reducerede omkostningsvektor
%
\begin{align*}
c^*_{B(j)} & = c_{B(j)} - \mathbf{c}_B^T B^{-1} \mathbf{a}_{B(j)} \\
& = c_{B(j)} - \mathbf{c}_B^T \mathbf{e_j} \\
& = c_{B(j)} -  c_{B(j)}  \\
& = 0.
\end{align*}
%
Dermed er indgangene til de basale variable lig nul, da de mulige retninger er nulvektoren. 
Det kan derfor ikke betale sig at øge basale variable.
\\\\
Hvis den reducerede omkostningsvektor $\mathbf{c}^* \geq \mathbf{0}$, er den basale mulige løsning et lokalt minimum i objektfunktionen og dermed et globalt minimum. Heraf er den optimale løsning fundet. Dette tilfælde belyses i sætning \ref{thm:julieerguden}.
%
\begin{thm}{}{julieerguden}
Lad $\mathbf{v}$ være en basal mulig løsning til en basismatrix $B$, og lad $\mathbf{c}^*$ være den reducerede omkostningsvektor. 
\begin{enumerate}[label = (\alph*)]
\item Hvis $\mathbf{c}^* \geq \textbf{0}$, så er $\mathbf{v}$ optimal.
\item Hvis $\mathbf{v}$ er optimal og ikke-degenereret, så er $\mathbf{c}^* \geq 0$.
\end{enumerate}
\end{thm}
\noindent
%
Beviset udelades.
Se \citep[Side 86]{bert}.\\\\
%\begin{enumerate}[label = (\alph*)]
%\item Antag, at $\bar{\textbf{c}} \geq \textbf{0}$. Lad $\textbf{y}$ være en arbitrær mulig løsning, og $\textbf{d} = \textbf{y} - \textbf{x}$. Grundet de mulige løsninger, er $\textbf{A} \textbf{x} = \textbf{A} \textbf{y} = \textbf{b}$, hvormed $\textbf{A} \textbf{d} = \textbf{0}$. Sidstnævnte kan skrives som 
%$$\textbf{B} \textbf{d}_B + \sum_{i \in N} \textbf{A}_i d_i = \textbf{0},$$
%hvor $N$ er mængden af indekser, der svarer til de ikke-basale variabler i basen. %wtf fatter slet ikke det her lol
%$\textbf{B}$ er invertibel, hvormed det haves, at 
%$$\textbf{d}_B = -\sum_{i \in N} \textbf{B}^{-1} \textbf{A}_i d_i ,$$
%og
%$$\textbf{c}' \textbf{d} = \textbf{c}'_B \textbf{d}_B + \sum_{i \in N} c_i d_i = \sum_{i \in N} (c_i - \textbf{c}'_B \textbf{B}^{-1} \textbf{A}_i) d_i = \sum_{i \in N} \bar{c}_i d_i .$$
%%nanananana fatter intet, hvorfor har vi dette med? 
%For ethvert ikke-basalt indeks $i \in N$ må der eksistere et $x_i = 0$, samt et $y_i \geq 0$, da $\textbf{y}$ er en mulig løsning. 
%Deraf haves, at $d_i \geq 0$ og $\bar{c}_i d_i \geq 0$ for alle $i \in N$. 
%Dermed er $\textbf{c}' (\textbf{y} - \textbf{x} = \textbf{c}' \textbf{d} \geq 0$, og $\textbf{x}$ er en optimal løsning, eftersom $\textbf{y}$ er en arbitrær mulig løsning. 
%%
%%
%\item Antag, at $x$ er en ikke-degenereret basal mulig løsning, og at $\bar{c}_j < 0$ for et $j$. Eftersom den reducerede objektfunktion for en basal variabel altid er $0$, må $x_j$ være en ikke-basal variabel, og $\bar{c}_j$ må være ændringshastigheden for objektfunktionen i den $j$'te basale retning. 
%%Wtf er ovenstående for noget lort
%Da $\textbf{x}$ ikke er degenereret, er den $j$'te basale retning en mulig retning for mindskelse af objektfunktionen. 
%Der findes dermed mulige løsninger, hvis objektfunktion har lavere værdi end $\textbf{x}$'s, ved at bevæge sig i denne retning, hvormed $\textbf{x}$ ikke er optimal. 
%\end{enumerate}
%
Dermed er alle basale retninger, for en basal mulig løsning, mulige retninger, hvis problemet er givet uden degenererede løsninger.
For ikke-degenererede problemer bevæger simplexmetoden sig dermed fra et ekstremumspunkt til et af dets tilstødende ekstremumspunkter og forbedrer omkostningen.
Heraf defineres en \textit{optimal basismatrix}, for at have optimale betingelser i forbindelse med simplexmetoden.
%
% Definition 3.3 
\begin{defn}{}{julieergudenoveralleguder}
En basismatrix $B$ kaldes \textbf{optimal}, hvis
%
\begin{enumerate}[label = (\alph*)]
\item $B^{-1} \mathbf{b} \geq \mathbf{0}$, og
\item $\mathbf{c}^* \geq \mathbf{0}$.
\end{enumerate}
%
\end{defn}
%
%
% Konstruktion af simplexmetoden
% ---------------------------------------------------------
	\noindent
Simplexmetoden finder basale retninger, som reducerer omkostningerne i objektfunktionen. 
Med optimale betingelser undersøges nu, hvilken basal retning, der skal vælges. 
Omkostningsvektoren $\mathbf{c}^*$ udregnes for en given basal mulig løsning. 
Hvis ingen basale retninger kan reducere omkostningen, er løsningen optimal. 
Hvis den reducerede omkostning $c^*_j$ er negativ for en basal retning $\mathbf{d}$, vælges denne retning. 
Hermed vil $x_j$ stige, og de resterende ikke-basale variable fastholdes.
Eftersom der søges at minimere omkostningen, bevæges der så langt i retning af $\textbf{d}$ som muligt, så omkostningen falder mest muligt.
Det ønskes dermed at finde den største skalar
\begin{align*}
\theta^* = \max \{ \theta \geq 0 \mid \textbf{x} + \theta\textbf{d} \in \mathcal{P} \},
\end{align*}
%
der opfylder, at $\mathbf{x} + \theta^* \mathbf{d}$ er en mulig løsning.
% Nu kan vi eventuelt udelede \theta^* xD 
Herefter beregnes omkostningsvektoren for den nye mulige løsning, og samme procedure gentages.
Dette forsættes, indtil ingen basale retninger reducerer omkostningen, og en optimal løsning er fundet.
%\textcolor{red}{
%Den løsning, der traverseres fra, forlader hermed basen og den nye basale mulige løsning indtræder i basen og de tidligere basale variable bliver dermed inaktive.}
% og de tidligere basale variable bliver til ikke-basale variable.
%\\\\
% Vi kan eventuelt her beskrive kort med ord den formel som bliver brugt i den naive implimentering 
%
%Den største skalar 
%\begin{align*}
%\theta^*
%\end{align*}
%
%
%	\begin{thm}{}{}
Lad $\mathbf{x}$ være en basal mulig løsning, til en basis matrix $B$ og lad $\mathbf{c}^*$ være den reducerede objektfunktion. 
\begin{enumerate}[label = (\alph*)]
\item Hvis $\mathbf{c}^* \geq 0$, så er $\mathbf{x}$ optimal.
\item Hvis $\mathbf{x}$ er optimal og ikke-degenereret, så er $\mathbf{c}^* \geq 0$.
\end{enumerate}
\end{thm}
%
\begin{proof}
\begin{enumerate}[label = (\alph*)]
\item Antag, at $\bar{\textbf{c}} \geq \textbf{0}$. Lad $\textbf{y}$ være en arbitrær mulig løsning, og $\textbf{d} = \textbf{y} - \textbf{x}$. Grundet de mulige løsninger, er $\textbf{A} \textbf{x} = \textbf{A} \textbf{y} = \textbf{b}$, hvormed $\textbf{A} \textbf{d} = \textbf{0}$. Sidstnævnte kan skrives som 
$$\textbf{B} \textbf{d}_B + \sum_{i \in N} \textbf{A}_i d_i = \textbf{0},$$
hvor $N$ er mængden af indekser, der svarer til de ikke-basale variabler i basen. %wtf fatter slet ikke det her lol
$\textbf{B}$ er invertibel, hvormed det haves, at 
$$\textbf{d}_B = -\sum_{i \in N} \textbf{B}^{-1} \textbf{A}_i d_i ,$$
og
$$\textbf{c}' \textbf{d} = \textbf{c}'_B \textbf{d}_B + \sum_{i \in N} c_i d_i = \sum_{i \in N} (c_i - \textbf{c}'_B \textbf{B}^{-1} \textbf{A}_i) d_i = \sum_{i \in N} \bar{c}_i d_i .$$
%nanananana fatter intet, hvorfor har vi dette med? 
For ethvert ikke-basalt indeks $i \in N$ må der eksistere et $x_i = 0$, samt et $y_i \geq 0$, da $\textbf{y}$ er en mulig løsning. 
Deraf haves, at $d_i \geq 0$ og $\bar{c}_i d_i \geq 0$ for alle $i \in N$. 
Dermed er $\textbf{c}' (\textbf{y} - \textbf{x} = \textbf{c}' \textbf{d} \geq 0$, og $\textbf{x}$ er en optimal løsning, eftersom $\textbf{y}$ er en arbitrær mulig løsning. 
%
%
\item Antag, at $x$ er en ikke-degenereret basal mulig løsning, og at $\bar{c}_j < 0$ for et $j$. Eftersom den reducerede objektfunktion for en basal variabel altid er $0$, må $x_j$ være en ikke-basal variabel, og $\bar{c}_j$ må være ændringshastigheden for objektfunktionen i den $j$'te basale retning. 
%Wtf er ovenstående for noget lort
Da $\textbf{x}$ ikke er degenereret, er den $j$'te basale retning en mulig retning for mindskelse af objektfunktionen. 
Der findes dermed mulige løsninger, hvis objektfunktion har lavere værdi end $\textbf{x}$'s, ved at bevæge sig i denne retning, hvormed $\textbf{x}$ ikke er optimal. 
\end{enumerate}
\end{proof}
%	
%	\input{incl/main/simplex_metoden/thm_3.2}
%	En iteration af simplex metoden
\begin{enumerate}
\item En normal iteration starter med et basis bestående af de basale rækker $\textbf{A}_{B(1)},\ldots,\textbf{A}_{B(m)}$ og den tilhørende basale mulige løsning $\textbf{x}$.
\item Beregn den reducerede objektfunktion $c_j^* = c_j - \mathbf{c}_B^T \textbf{B}^{-1}A_j$ for alle ikke-basale indekser $j$. Hvis de alle er ikke-negative, så er den nuværende basale mulige løsning optimal; ellers vælges et indeks $j$, hvorom det gælder, at $c^*_j<0$
\item Udregn $\textbf{u}=\textbf{B}^{-1}\textbf{A}_j$. Hvis der ikke er nogen komponent af $\textbf{u}$, som er positiv, så er $\theta ^*=\infty$, den optimale løsnings værdi er $-\infty$, og algorithmen stopper.
\item Hvis en komponent af $\textbf{u}$ er positiv, så lad 
$$\theta^*= min_{ \{i=1,\ldots,m|u_i>0 \} }        \dfrac{x_{B(i)}}{u_i}$$
\item Lad $l$ være indekset for $\theta^*=  \dfrac{x_{B(l)}}{u_l}$. Dan et nyt basis ved at udskifte $\textbf{A}_{B(l)}$ med $\textbf{A}_j$. Hvis $\textbf{y}$ er den nye basale mulige løsning, så er værdierne af de nye basale variable $y_j=\theta^*$ og $y_{B(i)}=x_{B(i)}-\theta^*u_i,i\neq l$
\end{enumerate}
%
%
I det ikke-degenererede tilfælde siger følgende sætning, at simplex-metoden fungerer og stopper efter et endeligt antal iterationer.
% Sætning 3.3
\begin{thm}{}{}
Antag, at løsningsmængden er ikke-tom og at enhver basal mulig løsning er ikke-degenereret.
Så stopper simplex-metoden efter et endeligt antal iterationer. 
Ved termineringen er der to muligheder:
\begin{enumerate}[label = (\alph*)]
\item Et optimalt basis $\textbf{B}$ er fundet, og den tilsvarende basale mulige løsning er optimal.
\item Der findes en vektor $\textbf{d}$, der opfylder $\textbf{Ad}=0,\textbf{d}\geq 0,$ og $\textbf{c}^T\textbf{d}<0$, og den optimale værdi er $-\infty$
\end{enumerate}
\end{thm}
%
\begin{proof}
%jeg ville ikke lige til at pille i det andet dokument hvis andre skrev i det pr sætning 6.1(3.1 i bertsima)
Hvis algoritmen stopper i trin 2, så følger det af Sætning 6.1(mangler en ref), at $\textbf{B}$ er et optimalt basis og den nuværende basale mulige løsninger er optimal.
\\
Hvis algoritmen stopper på grund af stopkriteriet i trin $3$, så er den nuværende basale mulige løsning $\textbf{x}$. 
Der findes desuden en ikke-basal variabel $x_j$, således at $c^*_j<0$, og den tilsvarende basale retning $\textbf{d}$ tilfredsstiller $\textbf{Ad}=\textbf{0}$ og $\textbf{d} \geq \textbf{0}$. 
Dertil, at $x+\theta \textbf{d}\in P$ for alle $\theta>0$. 
Eftersom $\textbf{c}^T\textbf{d}=c_j^*<0$, ved at tage et arbitrært stort $\theta$, fås en arbitrær negativ objektfunktion, og den optimale objektsværdi er $-\infty$.
\\\\
Ved hver iteration bevæger algoritmen sig en positiv værdi $\theta*$ langs en retning $\textbf{d}$, der ofylder $\textbf{c}^T\textbf{d}<0$. 
Som følge af dette forbedres objektfunktionens værdi for hver basal mulig løsning simplex algoritmen undersøger, samt at den samme løsning ikke besøges mere end en gang. 
Eftersom der er et endeligt antal basale mulige løsninger, terminerer algoritmen efter et endeligt antal iterationer.
\end{proof}
%

%
%	
%		
% Eksempel til simplexmetoden
% ---------------------------------------------------------
	\section{Praktisk anvendelse af simplexmetoden}
\label{afsnittet}
Simplexmetoden foretager som udgangspunkt følgende seks trin, inden den afsluttes:
%
\begin{tcolorbox}[
title=Simplexmetoden,
colback		= myblue!15,
colframe	= myblue!15,
coltitle	= black,
before skip	= 20pt plus 2pt,
after skip	= 20pt plus 2pt,
fonttitle	= \bfseries]
% Denne skal eventuelt beskrives mere i dybdegående - Tager kort tid.
\begin{enumerate}
\item Omskriv optimeringsproblemet til standardform med slack-variable.  %1
\item Opstil \textit{simplextabellen} for optimeringsproblemet.					 %2
\item Kontroller for optimalitet ellers identificér en pivoteringsindgang.					 %3
\item Opstil en ny tabel ved hjælp af pivotering. 						 %4
\item Gentag trin 3 og 4, indtil den optimale løsning er fundet. 					 %5
\item Identificér den optimale løsning.									 %6
\end{enumerate}
%
\end{tcolorbox}
\noindent
%
I følgende afsnit uddybes ovenstående punkter, samt tilhørende teori, med udgangspunkt i \ref{haribooooo}.
%
\\
%
\begin{eks}
\label{haribooooo}
Haribo har to typer slikblandinger, $x_1$ og $x_2$, hvor profitten for disse henholdsvis er $5$ og $4$ valutaenheder.
%(her kunne der også skrives tusind/millioner whatever ingen anelse om realistisk skala i haribos profitmargin).
Disse er begrænset af produktionskapaciteten af lakrids og vingummi.
Lakridsproduktion har en begrænsning på $78$ enheder, hvor der skal $3$ enheder lakrids i $x_1$ blandingen og $5$ enheder lakrids i $x_2$.
Vingummiproduktionen har en begrænsning på $36$ enheder, hvor der skal $4$ enheder vingummi i $x_1$ blandingen og $1$ enhed vingummi i $x_2$. 
%
Disse betingelser danner optimeringsproblemet
%
\begin{align*}
\begin{array}{lrrlr} 
\text{Maksimer}		&	\multicolumn{2}{c}{z=5x_1+4x_2}  &\\
\text{begrænset af}	&3x_1 \phantom{,} & +5x_2			&\leq 	&78,\\
					&4x_1 \phantom{,}& + x_2				&\leq	& 36,\\
					&x_1,& x_2				&\geq	& 0.
\end{array}
\end{align*}
%
\end{eks}
%
\subsubsection{1. Opskriv optimeringsproblemet på standardform med slack-variable.}
%
Generelt tager simplexmetoden udgangspunkt i, at alle variable er positive. Jævnfør afsnit \ref{sec:standard} tilføjes $x_i^+$ og $x_i^-$, hvis der i optimeringsproblemet ikke eksisterer ikke-negativitetsbetingelser for variablene. 
I dette tilfælde er begge variable positive, og det er derfor ikke nødvendigt at opdele variablene. 
Med udgangspunkt i metoden fra afsnit \ref{sec:standard} opstilles optimeringsproblemet derfor på standardform som
%
\begin{align*}
\begin{array}{lrrrrrlr}
\text{Maksimér}		& -5x_1 &-4x_2 &&& + z & =&0\phantom{,}\\
\text{begrænset af}	&3x_1& +5x_2	& + \textcolor{blue}{s_1} 	&&&= 	&78,\\
					&4x_1& + x_2	& & + \textcolor{blue}{s_2}	&&=	&	 36.\\
\end{array}
\end{align*}
%
Slack-variablene $s_1$ og $s_2$ er her markeret med blå, da de ikke indgår i den endelige løsning.
%Slackvariablerne indgår i dualproblemets løsning. xd lit fam 420 
%
\subsubsection{2. Opstil simplextabellen for optimeringsproblemet}		
% 
Nu opstilles simplextabellen for optimeringsproblemet. 
Med udgangspunkt i et generelt lineært optimeringsproblem på formen
%
% & =v, \text{ hvor } v=0
\begin{align*}
\begin{array}{lrl}
\text{Maksimér}		&z -\textbf{c}^T\textbf{x}	& =0	\\
\text{begrænset af}	&A\textbf{x}	&=\mathbf{b},	\\
					&\mathbf{x}				&\geq \mathbf{0},
\end{array}
\end{align*}
er simplextabellen en matrix, som indeholder de lineære betingelser, slack-variablene og objektfunktionen. 
Matricen $\mathbf{A}$ opskrives først, og derefter opskrives identitesmatricen $e_{m+1}$, hvor $m$ er antallet af slack-variable, samt den optimale løsning $z$, og til sidst opskrives $\mathbf{b}$. 
Under $\mathbf{A}$ tilføjes omkostningsvektoren $- \mathbf{c}^T$. 
Identitetsmatricen $I_m$, som dannes med udgangspunkt i antallet af slack-variable danner basismatricen $B$.
En general form for simplextabellen ser ud som følger:
%
\begin{align*}
\begin{blockarray}{ccccccccccc}
x_1 & x_2 & \cdots & x_n & \textcolor{blue}{s_1} & \textcolor{blue}{s_2} &  \textcolor{blue}{\cdots} & \textcolor{blue}{s_m} & z & b \\
\begin{block}{[cccc|ccccc|c]c}
a_{1,1} & a_{1,2} & \cdots & a_{1,n} & 1 & 0 & \cdots & 0 & 0 & b_1 \\
a_{2,1} & a_{2,2} & \cdots & a_{2,n} & 0 & 1 & \cdots & 0 & 0 & b_2 \\
\vdots & \vdots & \ddots & \vdots & \vdots & \vdots & \ddots & \vdots & \vdots & \vdots \\
a_{m,1} & a_{m,2} & \cdots & a_{m,n} & 0 & 0 & \cdots  & 1  & 0 & b_{m}\\
\cline{1-10}
-c_1 & -c_2 & \cdots & -c_n & 0 & 0 & \cdots & 0 & 1 & v\\
\end{block}
\end{blockarray}.
\end{align*}
%
Optimeringsproblemet fra \ref{haribooooo} har dermed følgende simplextabel:
%
\begin{align*}
\begin{blockarray}{cccccc}
x_1 & x_2 & \textcolor{blue}{s_1} & \textcolor{blue}{s_2} & z & b \\
\begin{block}{[cc|ccc|c]}
3 & 5 & 1 & 0 & 0 & 78 \\
4 & 1 & 0 & 1 & 0 & 36 \\
\cline{1-6}
-5 & -4 & 0 & 0 & 1 & 0\\
\end{block}
\end{blockarray}.
\end{align*}
%
\subsubsection{3. Kontroller for optimalitet ellers identificér en pivoteringsindgang}
%
Først kontrolleres for optimalitet ved at finde det mindste negative koefficient i omkostningsvektoren, hvilket findes i nederste række i simplextabellen:
%
\begin{align*}
\begin{blockarray}{cccccc}
x_1 & x_2 & \textcolor{blue}{s_1} & \textcolor{blue}{s_2} & z & b \\
\begin{block}{[cc|ccc|c]}
3 & 5 & 1 & 0 & 0 & 78 \\
4 & 1 & 0 & 1 & 0 & 36 \\
\cline{1-6}
\hlight{-5} & -4 & 0 & 0 & 1 & 0\\
\end{block}
\end{blockarray}.
\end{align*}
%
Søjlen, hvori denne værdi er, kaldes \textit{pivotsøjlen}. 
Herefter findes \textit{pivotrækken} og \textit{pivoteringsindgangen}, ved at finde den mindste $u_i$-værdi ud fra 
\begin{align*}
\frac{a_{i,j}}{b_j}=u_i
\end{align*}
%
i pivotsøjlen.
Pivoteringsindgangen findes ved at betragte 
%
\begin{align*}
u_1 = \frac{78}{3} = 26 \text{  } \text{ og } \text{   } u_2 = \frac{36}{4} = 9.
\end{align*}
%
Eftersom $9$ er den laveste værdi, er $a_{2,1}$ pivoteringsindgangen, mens den række, som den er i, er pivotrækken. 
I nedenstående er pivoteringsindgangen markeret.
%
\begin{align*}
\begin{blockarray}{cccccc}
x_1 & x_2 & \textcolor{blue}{s_1} & \textcolor{blue}{s_2} & z & b \\
\begin{block}{[cc|ccc|c]}
3 & 5 & 1 & 0 & 0 & 78 \\
\hlight{4} & 1 & 0 & 1 & 0 & 36 \\
\cline{1-6}
-5 & -4 & 0 & 0 & 1 & 0\\
\end{block}
\end{blockarray}.
\end{align*}	
%	
\subsubsection{4. Opstil en ny tabel ved hjælp af pivotering}
%
Ved brug af de elementære rækkeoperationer jævnfør \ref{defn:element}, skaleres pivoteringsindgangen til $1$, og der skabes nulindgange under og over pivoteringsindgangen ved hjælp af rækkeudskiftning.
Nedenfor ses dette gjort for eksemplet.
%
%& \begin{blockarray}{cccccc}
%x_1 & x_2 & \textcolor{blue}{s_1} & \textcolor{blue}{s_2} & z & b \\
%\begin{block}{[cc|ccc|c]}
%\hlight{3} & 5 & 1 & 0 & 0 & 78 \\
%4 & 1 & 0 & 1 & 0 & 36 \\
%\cline{1-6}
%\hlight{-5} & -4 & 0 & 0 & 1 & 0\\
%\end{block}
%\end{blockarray} \\
%
\begin{align*}
\xrightarrow[]{R_2 \rightarrow \frac{1}{4} R_2} &
\begin{blockarray}{cccccc}
x_1 & x_2 & \textcolor{blue}{s_1} & \textcolor{blue}{s_2} & z & b \\
\begin{block}{[cc|ccc|c]}
3 & 5 & 1 & 0 & 0 & 78 \\
1 & \frac{1}{4} & 0 & \frac{1}{4} & 0 & 9 \\
\cline{1-6}
-5 & -4 & 0 & 0 & 1 & 0\\
\end{block}
\end{blockarray} \\
\xrightarrow[R_3 \rightarrow R_3 + 5 R_2 ]{R_1 \rightarrow R_1 -3 R_2} &
\begin{blockarray}{cccccc}
x_1 & x_2 & \textcolor{blue}{s_1} & \textcolor{blue}{s_2} & z & b \\
\begin{block}{[cc|ccc|c]}
0 & \frac{17}{4} & 1 & \frac{-3}{4} & 0 & 51 \\
1 & \frac{1}{4} & 0 & \frac{1}{4} & 0 & 9 \\
\cline{1-6}
0 & \frac{-11}{4} & 0 & \frac{5}{4} & 1 & 45\\
\end{block}
\end{blockarray}.
\end{align*}	
%
\subsubsection{5. Gentag trin 3 og 4 indtil den optimale løsning er fundet}
%
Denne proces fortsættes nu, indtil der ikke er flere negative tal i omkostningsvektoren.
Bemærk, at \textit{pivotering} er processen fra valget af den mindste negative værdi i omkostningsvektoren indtil en ny værdi i omkostningsvektoren kan vælges.
Der kontrolleres for optimalitet igen, og en ny pivoteringsindgang er valgt ved den mindste negative værdi. 
Denne er markeret i nedenstående matrix.	
%Jeg tænker, at ovenstående kan præciseres, men jeg er blank pt.... Så hvad ved jeg :0) 
\begin{align*}
\begin{blockarray}{cccccc}
x_1 & x_2 & \textcolor{blue}{s_1} & \textcolor{blue}{s_2} & z & b \\
\begin{block}{[cc|ccc|c]}
0 & \frac{17}{4} & 1 & \frac{-3}{4} & 0 & 51 \\
1 & \frac{1}{4} & 0 & \frac{1}{4} & 0 & 9 \\
\cline{1-6}
0 & \hlight{\frac{-11}{4}} & 0 & \frac{5}{4} & 1 & 45\\
\end{block}
\end{blockarray}
\end{align*}
%
Pivoteringsindgangen findes ved at betragte
%
\begin{align*}
u_1 = \frac{51}{\frac{17}{4}} =12 \text{  }  \text{ og }  \text{   } u_2 = \frac{9}{\frac{1}{4}} =36.
\end{align*}
%
Pivoteringsindgangen er $a_{1,2}$, da denne giver den mindste værdi på $12$, hvilket er markeret i nedenstående matrix.
%
\begin{align*}
\begin{blockarray}{cccccc}
x_1 & x_2 & \textcolor{blue}{s_1} & \textcolor{blue}{s_2} & z & b \\
\begin{block}{[cc|ccc|c]}
0 & \hlight{\frac{17}{4}} & 1 & \frac{-3}{4} & 0 & 51 \\
1 & \frac{1}{4} & 0 & \frac{1}{4} & 0 & 9 \\
\cline{1-6}
0 & \frac{-11}{4} & 0 & \frac{5}{4} & 1 & 45\\
\end{block}
\end{blockarray}
\end{align*}
%
Pivotindgangen skaleres til $1$, og der skabes nulindgange under og over pivotindgangen:
%& \begin{blockarray}{cccccc}
%x_1 & x_2 & \textcolor{blue}{s_1} & \textcolor{blue}{s_2} &z & b \\
%\begin{block}{[cc|ccc|c]}
%0 & \frac{17}{4} & 1 & \frac{-3}{4} & 0 & 51 \\
%1 & \hlight{\frac{1}{4}} & 0 & \frac{1}{4} & 0 & 9 \\
%\cline{1-6}
%0 & \hlight{\frac{-11}{4}} & 0 & \frac{5}{4} & 1 & 45\\
%\end{block}
%\end{blockarray}\\
%
\begin{align*}
\xrightarrow[]{R_1 \rightarrow \frac{4}{17} R_1} &
\begin{blockarray}{cccccc}
x_1 & x_2 & \textcolor{blue}{s_1} & \textcolor{blue}{s_2} & z & b \\
\begin{block}{[cc|ccc|c]}
0 & 1 & \frac{4}{17} & \frac{-3}{17} & 0 & 12 \\
1 & \frac{1}{4} & 0 & \frac{1}{4} & 0 & 9 \\
\cline{1-6}
0 & \frac{-11}{4} & 0 & \frac{5}{4} & 1 & 45\\
\end{block}
\end{blockarray} \\
\xrightarrow[R_3 \rightarrow R_3 + \frac{11}{4} R_2 ]{R_2 \rightarrow R_2 -\frac{11}{4} R_1} &
\begin{blockarray}{cccccc}
x_1 & x_2 & \textcolor{blue}{s_1} & \textcolor{blue}{s_2} & z & b \\
\begin{block}{[cc|ccc|c]}
0 & 1 & \frac{4}{17} & \frac{-3}{17} & 0 & 12 \\
1 & 0 & \frac{-1}{17} & \frac{20}{17} & 0 & 6 \\
\cline{1-6}
0 & 0 & \frac{11}{17} & \frac{52}{17} & 1 & 78\\
\end{block}
\end{blockarray}.
\end{align*}	
%
%
\subsubsection{6. Identificér den optimale løsning}
%
Eftersom der ikke er flere negative værdier i omkostningsvektoren, er den optimale løsning nu fundet. 
Denne løsning kan aflæses direkte af simplextabellen, og den optimale løsning for eksemplet er dermed
%
\begin{align*}
x_1 & = 6, \\
x_2 & = 12, \\
z   & = 78.
\end{align*}
%
Haribos profit vil derfor være størst ved produktion af $6$ enheder af $x_1$ blandingen og $12$ enheder af $x_2$ blandingen. Dette giver en profit på $78$ enheder.
%
%
% Typer af simplexmetoden
% ---------------------------------------------------------
	%
\section{Typer af simplex-metoden}
%

%
%
%
% Pivoterings-løkker
% ---------------------------------------------------------
	\section{Pivoterings-løkker}
 Der er en risiko for ved degenererede basale løsninger, at simplex metoden blot skifter variabler ind og ud af basis, sådan at den sidder fast i en uendelig løkke.  
For at garantere at simplex metoden stopper ved en optimal værdi, og ikke fortsætter med at pivotere uendeligt, så kan der indføres forskellige regler for hvilken variabel som skal komme ind i løsningen.

\subsection{lexicografi}
%ingen ide om det hedder lexicografi på dansk men skiver det indtil videre

Før den \textit{lexicografiske} metode at vælge pivot søjler på introduceres, så er det nødvendigt at definere lexicografi
\begin{defn}{}{}
En vektor $\textbf{u}\in \R^n$ er \textbf{lexicografisk større}, eller mindre end en vektor $\textbf{v}\in \R^n$, hvis $\textbf{v} \neq \textbf{u}$ og den første ikke-nul indgang i $\textbf{u}-\textbf{v}$ er henholdsvis positiv eller negativ.
\end{defn}
\noindent
Et eksempel på lexicografi ses i eksempel \ref{eks:lexi}

\begin{eks}\label{eks:lexi}
Givet vektorerne
$$\textbf{u}=
\begin{bmatrix}
2\\
3\\
1\\
\end{bmatrix}
,
\textbf{v}=
\begin{bmatrix}
2\\
1\\
2\\
\end{bmatrix}
\text{ og }
\textbf{z}=
\begin{bmatrix}
3\\
2\\
2\\
\end{bmatrix}
.$$
Så er $\textbf{u}$ lexicofrafisk større end $\textbf{v}$ da den første ikke-nul værdi af $\textbf{u}-\textbf{v}$ er 2.
$\textbf{u}$ er til gengæld lexicografisk mindre end $\textbf{z}$ da den første ikke-nul værdi er $-1$
\end{eks}
En lexicografisk metode at vælge hvilke rækker der skal pivoteres følger.
\begin{enumerate}
\item Vælg en arbitrær søjle $\textbf{A}_j$, til at indgå i basis, så længe dens reducererde pris $c_j$ er negativ.
Lad $\textbf{u}$ være den $j$'te søjle af tabulaen.
\item For hvert i med $u_i>0$ divideres den i'te række med $u_i$, og vælg den lexicografikske mindste række. Hvis rækken $L$ er den lexicografiske mindste række så forlader den $L$'te basiske variabel $x_B(L)$ basiset.
\end{enumerate}
Det følger at enhver valg af pivotering ved den lexicografiske metode er unik, da der ellers ville gælde at der var en anden række som var proportional med den valgte række. Dette er i strid med antagelsen om at rækkerne er lineært uafhængige, hvilket er bevist er tilfældet.


%Der er en sætning som jeg nok skal lave, men jeg har bare brug for at diskutere med nogen hvad det vil sige at noget er lexigrafisk positiv

%Jeg tror ikke Bland's regel er relevant for os medmindre at vi skriver om den revidererde simpelx metode.
%\subsection{Bland's regel}
%En anden metode
%
%
%
%
%
%
% Kompleksitet
% ---------------------------------------------------------
	\section{Kompleksitet}
% he
Det er interessant at undersøge simplex-metodens tidskompleksitet.
Til beskrivelse af dette benyttes \textit{store-$O$} notation.
%
\begin{defn}{}{}
Lad $f(x)$ og $g(x)$ være vilkårlige funktioner. $f(x)$ er \textbf{store-$O$} af $g(x)$, hvis der eksisterer et $C$ og $k$, således at $|f(x)| \leq C|g(x)|, x \geq k$. $C$ og $k$ kaldes \textbf{vidner}.
\end{defn}\noindent
%
Der er flere måder at implementere simplex-metoden på, der hver har sin egen kompleksitet.
De to implementeringer, der vil blive betragtet i dette afsnit, er \textit{fuld tabel} og \textit{den reviderede} simplex-metode.
%
"Fuld tabel"-metoden kræver et konstant antal af operationer for at opdatere indgangene i simplex-matricen.
Derfor er antallet af operationer proportionelt med størrelsen på matricen og kompleksiteten $O(mn)$.
Den reviderede metode bruger lignende operationer, men opdaterer kun $O(m^2)$ indgange og har derfor kompleksiteten $O(m^2)$.
Det kan dog forekomme, at metoden skal opdatere alle variabler.
Hver udregning af en variabel kræver $O(m)$ operationer, hvilket medfører, at der i værste tilfælde bruges $O(mn)$ operationer.
Da $m \leq n$, er kompleksiteten i værste tilfælde $O(mn)$ for begge metoder.
Med dette konkluderes det, at den reviderede metode aldrig vil køre langsommere end "fuld tabel"-metoden.
Dog vil en iteration fra den reviderede metode være hurtigere i alle andre tilfælde end det værste.
Endnu et vigtigt element, som er tiltalende ved den reviderede metode, er rumkompleksiteten på $O(m^2)$, hvor "fuld tabel"-metodens er på $O(mn)$.
Her kan der igen være en ret betydelig forskel, afhængigt af størrelsen på $n$.
%
%


