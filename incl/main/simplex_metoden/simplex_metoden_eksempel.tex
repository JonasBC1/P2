\section{Praktisk anvendelse af simplexmetoden}
\label{afsnittet}
Simplexmetoden foretager som udgangspunkt følgende seks trin, inden den afsluttes:
%
\begin{tcolorbox}[
title=Simplexmetoden,
colback		= myblue!15,
colframe	= myblue!15,
coltitle	= black,
before skip	= 20pt plus 2pt,
after skip	= 20pt plus 2pt,
fonttitle	= \bfseries]
% Denne skal eventuelt beskrives mere i dybdegående - Tager kort tid.
\begin{enumerate}
\item Omskriv optimeringsproblemet til standardform med slack-variable.  %1
\item Opstil \textit{simplextabellen} for optimeringsproblemet.					 %2
\item Kontroller for optimalitet ellers identificér en pivoteringsindgang.					 %3
\item Opstil en ny tabel ved hjælp af pivotering. 						 %4
\item Gentag trin 3 og 4, indtil den optimale løsning er fundet. 					 %5
\item Identificér den optimale løsning.									 %6
\end{enumerate}
%
\end{tcolorbox}
\noindent
%
I følgende afsnit uddybes ovenstående punkter, samt tilhørende teori, med udgangspunkt i \ref{haribooooo}.
%
\\
%
\begin{eks}
\label{haribooooo}
Haribo har to typer slikblandinger, $x_1$ og $x_2$, hvor profitten for disse henholdsvis er $5$ og $4$ valutaenheder.
%(her kunne der også skrives tusind/millioner whatever ingen anelse om realistisk skala i haribos profitmargin).
Disse er begrænset af produktionskapaciteten af lakrids og vingummi.
Lakridsproduktion har en begrænsning på $78$ enheder, hvor der skal $3$ enheder lakrids i $x_1$ blandingen og $5$ enheder lakrids i $x_2$.
Vingummiproduktionen har en begrænsning på $36$ enheder, hvor der skal $4$ enheder vingummi i $x_1$ blandingen og $1$ enhed vingummi i $x_2$. 
%
Disse betingelser danner optimeringsproblemet
%
\begin{align*}
\begin{array}{lrrlr} 
\text{Maksimer}		&	\multicolumn{2}{c}{z=5x_1+4x_2}  &\\
\text{begrænset af}	&3x_1 \phantom{,} & +5x_2			&\leq 	&78,\\
					&4x_1 \phantom{,}& + x_2				&\leq	& 36,\\
					&x_1,& x_2				&\geq	& 0.
\end{array}
\end{align*}
%
\end{eks}
%
\subsubsection{1. Opskriv optimeringsproblemet på standardform med slack-variable.}
%
Generelt tager simplexmetoden udgangspunkt i, at alle variable er positive. Jævnfør afsnit \ref{sec:standard} tilføjes $x_i^+$ og $x_i^-$, hvis der i optimeringsproblemet ikke eksisterer ikke-negativitetsbetingelser for variablene. 
I dette tilfælde er begge variable positive, og det er derfor ikke nødvendigt at opdele variablene. 
Med udgangspunkt i metoden fra afsnit \ref{sec:standard} opstilles optimeringsproblemet derfor på standardform som
%
\begin{align*}
\begin{array}{lrrrrrlr}
\text{Maksimér}		& -5x_1 &-4x_2 &&& + z & =&0\phantom{,}\\
\text{begrænset af}	&3x_1& +5x_2	& + \textcolor{blue}{s_1} 	&&&= 	&78,\\
					&4x_1& + x_2	& & + \textcolor{blue}{s_2}	&&=	&	 36.\\
\end{array}
\end{align*}
%
Slack-variablene $s_1$ og $s_2$ er her markeret med blå, da de ikke indgår i den endelige løsning.
%Slackvariablerne indgår i dualproblemets løsning. xd lit fam 420 
%
\subsubsection{2. Opstil simplextabellen for optimeringsproblemet}		
% 
Nu opstilles simplextabellen for optimeringsproblemet. 
Med udgangspunkt i et generelt lineært optimeringsproblem på formen
%
% & =v, \text{ hvor } v=0
\begin{align*}
\begin{array}{lrl}
\text{Maksimér}		&z -\textbf{c}^T\textbf{x}	& =0	\\
\text{begrænset af}	&A\textbf{x}	&=\mathbf{b},	\\
					&\mathbf{x}				&\geq \mathbf{0},
\end{array}
\end{align*}
er simplextabellen en matrix, som indeholder de lineære betingelser, slack-variablene og objektfunktionen. 
Matricen $\mathbf{A}$ opskrives først, og derefter opskrives identitesmatricen $e_{m+1}$, hvor $m$ er antallet af slack-variable, samt den optimale løsning $z$, og til sidst opskrives $\mathbf{b}$. 
Under $\mathbf{A}$ tilføjes omkostningsvektoren $- \mathbf{c}^T$. 
Identitetsmatricen $I_m$, som dannes med udgangspunkt i antallet af slack-variable danner basismatricen $B$.
En general form for simplextabellen ser ud som følger:
%
\begin{align*}
\begin{blockarray}{ccccccccccc}
x_1 & x_2 & \cdots & x_n & \textcolor{blue}{s_1} & \textcolor{blue}{s_2} &  \textcolor{blue}{\cdots} & \textcolor{blue}{s_m} & z & b \\
\begin{block}{[cccc|ccccc|c]c}
a_{1,1} & a_{1,2} & \cdots & a_{1,n} & 1 & 0 & \cdots & 0 & 0 & b_1 \\
a_{2,1} & a_{2,2} & \cdots & a_{2,n} & 0 & 1 & \cdots & 0 & 0 & b_2 \\
\vdots & \vdots & \ddots & \vdots & \vdots & \vdots & \ddots & \vdots & \vdots & \vdots \\
a_{m,1} & a_{m,2} & \cdots & a_{m,n} & 0 & 0 & \cdots  & 1  & 0 & b_{m}\\
\cline{1-10}
-c_1 & -c_2 & \cdots & -c_n & 0 & 0 & \cdots & 0 & 1 & v\\
\end{block}
\end{blockarray}.
\end{align*}
%
Optimeringsproblemet fra \ref{haribooooo} har dermed følgende simplextabel:
%
\begin{align*}
\begin{blockarray}{cccccc}
x_1 & x_2 & \textcolor{blue}{s_1} & \textcolor{blue}{s_2} & z & b \\
\begin{block}{[cc|ccc|c]}
3 & 5 & 1 & 0 & 0 & 78 \\
4 & 1 & 0 & 1 & 0 & 36 \\
\cline{1-6}
-5 & -4 & 0 & 0 & 1 & 0\\
\end{block}
\end{blockarray}.
\end{align*}
%
\subsubsection{3. Kontroller for optimalitet ellers identificér en pivoteringsindgang}
%
Først kontrolleres for optimalitet ved at finde det mindste negative koefficient i omkostningsvektoren, hvilket findes i nederste række i simplextabellen:
%
\begin{align*}
\begin{blockarray}{cccccc}
x_1 & x_2 & \textcolor{blue}{s_1} & \textcolor{blue}{s_2} & z & b \\
\begin{block}{[cc|ccc|c]}
3 & 5 & 1 & 0 & 0 & 78 \\
4 & 1 & 0 & 1 & 0 & 36 \\
\cline{1-6}
\hlight{-5} & -4 & 0 & 0 & 1 & 0\\
\end{block}
\end{blockarray}.
\end{align*}
%
Søjlen, hvori denne værdi er, kaldes \textit{pivotsøjlen}. 
Herefter findes \textit{pivotrækken} og \textit{pivoteringsindgangen}, ved at finde den mindste $u_i$-værdi ud fra 
\begin{align*}
\frac{a_{i,j}}{b_j}=u_i
\end{align*}
%
i pivotsøjlen.
Pivoteringsindgangen findes ved at betragte 
%
\begin{align*}
u_1 = \frac{78}{3} = 26 \text{  } \text{ og } \text{   } u_2 = \frac{36}{4} = 9.
\end{align*}
%
Eftersom $9$ er den laveste værdi, er $a_{2,1}$ pivoteringsindgangen, mens den række, som den er i, er pivotrækken. 
I nedenstående er pivoteringsindgangen markeret.
%
\begin{align*}
\begin{blockarray}{cccccc}
x_1 & x_2 & \textcolor{blue}{s_1} & \textcolor{blue}{s_2} & z & b \\
\begin{block}{[cc|ccc|c]}
3 & 5 & 1 & 0 & 0 & 78 \\
\hlight{4} & 1 & 0 & 1 & 0 & 36 \\
\cline{1-6}
-5 & -4 & 0 & 0 & 1 & 0\\
\end{block}
\end{blockarray}.
\end{align*}	
%	
\subsubsection{4. Opstil en ny tabel ved hjælp af pivotering}
%
Ved brug af de elementære rækkeoperationer jævnfør \ref{defn:element}, skaleres pivoteringsindgangen til $1$, og der skabes nulindgange under og over pivoteringsindgangen ved hjælp af rækkeudskiftning.
Nedenfor ses dette gjort for eksemplet.
%
%& \begin{blockarray}{cccccc}
%x_1 & x_2 & \textcolor{blue}{s_1} & \textcolor{blue}{s_2} & z & b \\
%\begin{block}{[cc|ccc|c]}
%\hlight{3} & 5 & 1 & 0 & 0 & 78 \\
%4 & 1 & 0 & 1 & 0 & 36 \\
%\cline{1-6}
%\hlight{-5} & -4 & 0 & 0 & 1 & 0\\
%\end{block}
%\end{blockarray} \\
%
\begin{align*}
\xrightarrow[]{R_2 \rightarrow \frac{1}{4} R_2} &
\begin{blockarray}{cccccc}
x_1 & x_2 & \textcolor{blue}{s_1} & \textcolor{blue}{s_2} & z & b \\
\begin{block}{[cc|ccc|c]}
3 & 5 & 1 & 0 & 0 & 78 \\
1 & \frac{1}{4} & 0 & \frac{1}{4} & 0 & 9 \\
\cline{1-6}
-5 & -4 & 0 & 0 & 1 & 0\\
\end{block}
\end{blockarray} \\
\xrightarrow[R_3 \rightarrow R_3 + 5 R_2 ]{R_1 \rightarrow R_1 -3 R_2} &
\begin{blockarray}{cccccc}
x_1 & x_2 & \textcolor{blue}{s_1} & \textcolor{blue}{s_2} & z & b \\
\begin{block}{[cc|ccc|c]}
0 & \frac{17}{4} & 1 & \frac{-3}{4} & 0 & 51 \\
1 & \frac{1}{4} & 0 & \frac{1}{4} & 0 & 9 \\
\cline{1-6}
0 & \frac{-11}{4} & 0 & \frac{5}{4} & 1 & 45\\
\end{block}
\end{blockarray}.
\end{align*}	
%
\subsubsection{5. Gentag trin 3 og 4 indtil den optimale løsning er fundet}
%
Denne proces fortsættes nu, indtil der ikke er flere negative tal i omkostningsvektoren.
Bemærk, at \textit{pivotering} er processen fra valget af den mindste negative værdi i omkostningsvektoren indtil en ny værdi i omkostningsvektoren kan vælges.
Der kontrolleres for optimalitet igen, og en ny pivoteringsindgang er valgt ved den mindste negative værdi. 
Denne er markeret i nedenstående matrix.	
%Jeg tænker, at ovenstående kan præciseres, men jeg er blank pt.... Så hvad ved jeg :0) 
\begin{align*}
\begin{blockarray}{cccccc}
x_1 & x_2 & \textcolor{blue}{s_1} & \textcolor{blue}{s_2} & z & b \\
\begin{block}{[cc|ccc|c]}
0 & \frac{17}{4} & 1 & \frac{-3}{4} & 0 & 51 \\
1 & \frac{1}{4} & 0 & \frac{1}{4} & 0 & 9 \\
\cline{1-6}
0 & \hlight{\frac{-11}{4}} & 0 & \frac{5}{4} & 1 & 45\\
\end{block}
\end{blockarray}
\end{align*}
%
Pivoteringsindgangen findes ved at betragte
%
\begin{align*}
u_1 = \frac{51}{\frac{17}{4}} =12 \text{  }  \text{ og }  \text{   } u_2 = \frac{9}{\frac{1}{4}} =36.
\end{align*}
%
Pivoteringsindgangen er $a_{1,2}$, da denne giver den mindste værdi på $12$, hvilket er markeret i nedenstående matrix.
%
\begin{align*}
\begin{blockarray}{cccccc}
x_1 & x_2 & \textcolor{blue}{s_1} & \textcolor{blue}{s_2} & z & b \\
\begin{block}{[cc|ccc|c]}
0 & \hlight{\frac{17}{4}} & 1 & \frac{-3}{4} & 0 & 51 \\
1 & \frac{1}{4} & 0 & \frac{1}{4} & 0 & 9 \\
\cline{1-6}
0 & \frac{-11}{4} & 0 & \frac{5}{4} & 1 & 45\\
\end{block}
\end{blockarray}
\end{align*}
%
Pivotindgangen skaleres til $1$, og der skabes nulindgange under og over pivotindgangen:
%& \begin{blockarray}{cccccc}
%x_1 & x_2 & \textcolor{blue}{s_1} & \textcolor{blue}{s_2} &z & b \\
%\begin{block}{[cc|ccc|c]}
%0 & \frac{17}{4} & 1 & \frac{-3}{4} & 0 & 51 \\
%1 & \hlight{\frac{1}{4}} & 0 & \frac{1}{4} & 0 & 9 \\
%\cline{1-6}
%0 & \hlight{\frac{-11}{4}} & 0 & \frac{5}{4} & 1 & 45\\
%\end{block}
%\end{blockarray}\\
%
\begin{align*}
\xrightarrow[]{R_1 \rightarrow \frac{4}{17} R_1} &
\begin{blockarray}{cccccc}
x_1 & x_2 & \textcolor{blue}{s_1} & \textcolor{blue}{s_2} & z & b \\
\begin{block}{[cc|ccc|c]}
0 & 1 & \frac{4}{17} & \frac{-3}{17} & 0 & 12 \\
1 & \frac{1}{4} & 0 & \frac{1}{4} & 0 & 9 \\
\cline{1-6}
0 & \frac{-11}{4} & 0 & \frac{5}{4} & 1 & 45\\
\end{block}
\end{blockarray} \\
\xrightarrow[R_3 \rightarrow R_3 + \frac{11}{4} R_2 ]{R_2 \rightarrow R_2 -\frac{11}{4} R_1} &
\begin{blockarray}{cccccc}
x_1 & x_2 & \textcolor{blue}{s_1} & \textcolor{blue}{s_2} & z & b \\
\begin{block}{[cc|ccc|c]}
0 & 1 & \frac{4}{17} & \frac{-3}{17} & 0 & 12 \\
1 & 0 & \frac{-1}{17} & \frac{20}{17} & 0 & 6 \\
\cline{1-6}
0 & 0 & \frac{11}{17} & \frac{52}{17} & 1 & 78\\
\end{block}
\end{blockarray}.
\end{align*}	
%
%
\subsubsection{6. Identificér den optimale løsning}
%
Eftersom der ikke er flere negative værdier i omkostningsvektoren, er den optimale løsning nu fundet. 
Denne løsning kan aflæses direkte af simplextabellen, og den optimale løsning for eksemplet er dermed
%
\begin{align*}
x_1 & = 6, \\
x_2 & = 12, \\
z   & = 78.
\end{align*}
%
Haribos profit vil derfor være størst ved produktion af $6$ enheder af $x_1$ blandingen og $12$ enheder af $x_2$ blandingen. Dette giver en profit på $78$ enheder.