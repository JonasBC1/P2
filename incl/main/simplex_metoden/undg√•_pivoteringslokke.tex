%
\section{Konstruktion af simplex-metoden}
% --------------------------------------------------------------
%
For at finde den mest optimale løsning kan der udvælges en basal løsning og undersøges, hvorvidt der findes andre løsninger, som er bedre, end den valgte. 
Hvis der ikke findes bedre løsninger, er den valgte basale løsning den optimale løsning. 
Eftersom der jævnfør afsnit \ref{julieerlakker} optimeres for en konveks funktion over en konveks mængde, %Det er det med at et lokal minimum er et globalt minimum xD
vil den lokale optimale løsning være en global optimal løsning. 
%
%Når der undersøges om der findes et nabo punkt, hvis løsning er bedre, så er det ikke nødvendigt at undersøge i retningen, som leder ud af den basal mulige område.
%
% Definition 3.1 i bert
\begin{defn}{}{}
Lad $\textbf{x}$ være et element af et polyeder $\mathcal{P}$.
En vektor $\textbf{d}\in \R^n$ siges at være en  \textbf{mulig retning} fra $\textbf{x}$, hvis der eksisterer en positiv skalar $\theta$, således at $\textbf{x}+\theta \textbf{d}\in \mathcal{P}$.
\end{defn}
\noindent
%
På figur \ref{fig:julieersmuuuuuk}, ses vektorerne $\mathbf{x}_1, \mathbf{x}_2$ og $\mathbf{x}_3$ i polyederet $\mathcal{P}$, samt tilhørende forskellige mulige retninger, som stadig er indholdt i $\mathcal{P}$.
%
\begin{center}
%
\begin{tikzpicture}[scale=8]
%
% Koordinater
% -------------------------------------------------------
\coordinate (a) at (0,0,0); 
\coordinate (a1) at (0.084,0,0);
\coordinate (a2) at (0,0.084,0);
\coordinate (a3) at (0.07,0.07,0);
%
\coordinate (b) at (0.7,0,0); 
\coordinate (c) at (0.393,0.55,0); 
%
\coordinate (e) at (0.066,0.434,0);
\coordinate (e1) at (0.15,0.46,0);
\coordinate (e2) at (0.025,0.35,0);
\coordinate (e3) at (0.094,0.36,0);
\coordinate (e4) at (0.14,0.4,0);
%
\coordinate (g) at (0,0.305,0);
%
\coordinate (m) at (0.35,0.15,0);
\coordinate (m1) at (0.31,0.066,0);
\coordinate (m2) at (0.31,0.234,0);
\coordinate (m3) at (0.266,0.15,0);
\coordinate (m4) at (0.434,0.15,0);
\coordinate (m5) at (0.39,0.066,0);
\coordinate (m6) at (0.39,0.234,0);
%
% Farvning
% -------------------------------------------------------
\filldraw[fill=myblue,opacity=0.3](a)--(b)--(c)--(e)--(g)--(a);
%
\draw[thick, color= myblue](a)--(b); 
\draw[thick, color= myblue](c)--(b);
\draw[thick, color= myblue](a)--(g);
\draw[thick, color= myblue](c)--(e);
\draw[thick, color= myblue](e)--(g);
%
% Punkt A
% -------------------------------------------------------
\draw[very thick,->](a)--(a1);
\draw[very thick,->](a)--(a2);
\draw[very thick,->](a)--(a3);
%
%
% Punkt E
% -------------------------------------------------------
\draw[very thick,->](e)--(e1);
\draw[very thick,->](e)--(e2);
\draw[very thick,->](e)--(e3);
\draw[very thick,->](e)--(e4);
%
% Punkt Mid
% -------------------------------------------------------
\draw[very thick,->](m)--(m1);
\draw[very thick,->](m)--(m2);
\draw[very thick,->](m)--(m3);
\draw[very thick,->](m)--(m4);
\draw[very thick,->](m)--(m5);
\draw[very thick,->](m)--(m6);
%
% Punkterne 
% -------------------------------------------------------
\filldraw [black] (0.35,0.35,0) circle (0pt) node[above] {$\mathcal{P}$};
\filldraw [black] (0.31,0.17,0) circle (0pt) node[anchor=south east]{$\mathbf{v}$};
\filldraw [black] (e) circle (0pt) node[anchor=
south east]{$\mathbf{u}$};
\filldraw [black] (a) circle (0pt) node[anchor= north east]{$\mathbf{w}$};

%
%
\end{tikzpicture}
  \captionof{figure}{Vektorerne $\mathbf{u}, \mathbf{v}$ og $\mathbf{w}$, samt tilhørende forskellige mulige retninger, som stadig er indeholdt i polyederet $\mathcal{P}$.}
  \label{fig:julieersmuuuuuk}
\end{center}
%
%
%
%
% Definition 3.2 
\begin{defn}{}{}
Lad $\mathbf{x}$ være en basal løsning til en basis matrix $B$ og tilhørende objektfunktionen $c_B=[ c_{B(1)},c_{B(2)}, \cdots , c_{B(m)} ]^T.$
For hver $j$'te basale retning defineres den \textbf{reducerede objektfunktion} $c_j^*$ som
\begin{align*}
c_j^* = c_j - \mathbf{c}_B^T B^{-1}A_j.
\end{align*} 
% Nogle kalder den den reducerede omkostning.
%
\end{defn}
\noindent
%
Fisk
%
% Sætning 3.1 i Bertsimas 
% Ikke skrevet rigtig endnu 
\begin{thm}{}{}
Lad $\mathbf{x}$ være en basal mulig løsning, til en basis matrix $B$ og lad $\mathbf{c}^*$ være den reducerede objektfunktion. 
\begin{enumerate}[label = (\alph*)]
\item Hvis $\mathbf{c}^* \geq 0$, så er $\mathbf{x}$ optimal.
\item Hvis $\mathbf{x}$ er optimal og ikke-degenereret, så er $\mathbf{c}^* \geq 0$.
\end{enumerate}
\end{thm}
%
\begin{proof}
fisk
\end{proof}
\\
%
fisk
%
%
% Definition 3.3 i Bertsimas
% Ikke skrevet rigtig endnu 
\begin{defn}{}{}
En basis matrix $B$ er optimal, hvis
\begin{enumerate}[label = (\alph*)]
\item $B^-1 \mathbf{b} \geq \mathbf{0}$, and
\item ${\mathbf{c}^*}^T = \mathbf{c}^T - \mathbf{c}_B^T B^{-1} A \geq \mathbf{0}^T$.
\end{enumerate}
\end{defn}
\noindent
%
%
% 
Afsnit 3.2 i bertsimas
% Sætning 3.2 
% Ikke skrevet rigtig endnu 
\begin{thm}{}{}
\begin{enumerate}[label = (\alph*)]
\item Søjlerne $A_{B(i)}$, $i \neq \mathcal{L}$ og $A_j$ er lineær uafhængige, og 
\item Vektoren $\mathbf{y}=x+ \theta \mathbf{d}$ er en basal mulig løsning til basis matrix $B$.
\end{enumerate}
\end{thm}
%
\begin{proof}
fisk
\end{proof}
\\
%
fisk
% Sætning 3.3
% Ikke skrevet rigtig endnu 
%\begin{thm}{}{}
%\begin{enumerate}[label = (\alph*)]
%\item fisk
%\item fisk
%\end{enumerate}
%\end{thm}
%%
%\begin{proof}
%fisk
%\end{proof}
%\\
%%
%fisk
En iteration af simplex metoden
\begin{enumerate}
\item En normal iteration starter med et basis bestående af de basale rækker $\textbf{A}_{B(1)},\ldots,\textbf{A}_{B(m)}$, og den tilhørende basale mulige løsning $\textbf{x}$.
\item Beregn den reducerede objektfunktion $c_j^* = c_j - \mathbf{c}_B^T \textbf{B}^{-1}A_j$ for alle ikke-basale indekser $j$. Hvis de alle er ikke-negative så er den nuværende basale mulige løsning optimal, ellers vælges et indeks $j$ hvorom det gælder at $c^*_j<0$
\item Udregn $\textbf{u}=\textbf{B}^{-1}\textbf{A}_j$. Hvis der ikke er noget komponenet af $\textbf{u}$ som er positiv, så er $\theta ^*=\infty$, samt at den optimale løsnings værdi er $-\infty$, og algorithmen stopper.
\item Hvis et komponent af $\textbf{u}$ er positiv, så lad 
$$\theta^*= min_{ \{i=1,\ldots,m|u_i>0 \} }        \dfrac{x_{B(i)}}{u_i}$$
\item Lad $l$ være indekset for $\theta^*=  \dfrac{x_{B(l)}}{u_l}$. Dan er nyt basis ved at uskifte $\textbf{A}_{B(l)}$ med $\textbf{A}_j$. Hvis $\textbf{y}$ er den nye basale mulige løsning, så er værdierne af de nye basale vaiable $y_j=\theta^*$ og $y_{B(i)}=x_{B(i)}-\theta^*u_i,i\neq l$

\end{enumerate}


I det ikke-degenerede tilfælde siger følgende sætning at simplex metoden fungerer og stopper efter et endeligt antal iterationer.
% Sætning 3.3
\begin{thm}{}{}
Antag at løsningsmængden er ikke-tom og at enhver basal mulig løsning er ikke-degeneret.
Så teminerer simplex metoden efter et endeligt antal iterationer. 
Ved termineringen er der to muligheder:
\begin{enumerate}[label = (\alph*)]
\item Et optimalt basis $\textbf{B}$ er fundet og den tilsvarende basale mulige løsning er optimalt.
\item Der findes en vektor $\textbf{d}$ der opfylder $\textbf{Ad}=0,\textbf{d}\geq 0,$ og $\textbf{c}^T\textbf{d}<0$ og den optimale værdi er $-\infty$
\end{enumerate}
\end{thm}
%
\begin{proof}
%jeg ville ikke lige til at pille i det andet dokument hvis andre skrev i det pr sætning 6.1(3.1 i bertsima)
Hvis algoritmen stopper i trin 2, så følger det af Sætning 6.1(mangler en ref) at $\textbf{B}$ er et optimalt basis og den nuværende basale mulige løsninger er optimal.
\\
Hvis algoritmen stopper på grund af stop kriteriet i trin 3, så er den nuværende basale mulige løsning $\textbf{x}$ og der findes en ikke-basal variabel $x_j$ sådan at $c^*_j<0$ og den tilsvarende basale retning $\textbf{d}$ tilfredsstiller $\textbf{Ad}=\textbf{0}$ og $\textbf{d} \geq \textbf{0}$. Dertil, at $x+\theta \textbf{d}\in P$ for alle $\theta>0$. Eftersom $\textbf{c}^T\textbf{d}=c_j^*<0$, ved at tage et arbitært stort $\theta$, fåes et arbitært negativ objektfunktion, og den optimale objektsværdi er $-\infty$
\\
Ved hver iteration bevæger algoritmen sig en positiv værdi $\theta*$ langs en retning $\textbf{d}$ der ofylder $\textbf{c}^T\textbf{d}<0$. Som følge af dette forbedres objektfunktionens værdi for hver basal mulig løsning simplex algoritmen undersøger, samt at den samme løsning ikke besøges mere end en gang. Eftersom der er et endeligt antal basale mulige løsninger, terminerer algoritmen efter et endeligt antal iterationer.
\end{proof}

%
