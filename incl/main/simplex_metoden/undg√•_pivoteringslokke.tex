%
\section{Konstruktion af simplexmetoden}
\label{julieerfantalastiskogvidunderlig}
% --------------------------------------------------------------
%
For at finde den optimale løsning kan der udvælges en basal løsning og undersøges, hvorvidt der findes andre løsninger, som er bedre, end den valgte. 
Hvis der ikke findes bedre løsninger, er den valgte basale løsning den optimale løsning. 
Eftersom der jævnfør afsnit \ref{julieerlakker} optimeres for en konveks funktion over en konveks mængde, %Det er det med at et lokal minimum er et globalt minimum xD
vil den lokale optimale løsning være en global optimal løsning. 
%
%Når der undersøges om der findes et nabo punkt, hvis løsning er bedre, så er det ikke nødvendigt at undersøge i retningen, som leder ud af den basal mulige område.
%
% Definition 3.1 i bert
\begin{defn}{}{}
Lad $\textbf{v}$ være et element af et polyeder $\mathcal{P}$.
En vektor $\textbf{d}\in \R^n$ siges at være en  \textbf{mulig retning} fra $\textbf{v}$, hvis der eksisterer en positiv skalar $\theta$, således at $\textbf{v}+\theta \textbf{d}\in \mathcal{P}$.
\end{defn}
\noindent
%
På figur \ref{fig:julieersmuuuuuk}, ses vektorerne $\mathbf{u}, \mathbf{v}$ og $\mathbf{w}$ i polyederet $\mathcal{P}$, samt tilhørende forskellige mulige retninger, som stadig er indholdt i $\mathcal{P}$.
%
\begin{center}
%
\begin{tikzpicture}[scale=8]
%
% Koordinater
% -------------------------------------------------------
\coordinate (a) at (0,0,0); 
\coordinate (a1) at (0.084,0,0);
\coordinate (a2) at (0,0.084,0);
\coordinate (a3) at (0.07,0.07,0);
%
\coordinate (b) at (0.7,0,0); 
\coordinate (c) at (0.393,0.55,0); 
%
\coordinate (e) at (0.066,0.434,0);
\coordinate (e1) at (0.15,0.46,0);
\coordinate (e2) at (0.025,0.35,0);
\coordinate (e3) at (0.094,0.36,0);
\coordinate (e4) at (0.14,0.4,0);
%
\coordinate (g) at (0,0.305,0);
%
\coordinate (m) at (0.35,0.15,0);
\coordinate (m1) at (0.31,0.066,0);
\coordinate (m2) at (0.31,0.234,0);
\coordinate (m3) at (0.266,0.15,0);
\coordinate (m4) at (0.434,0.15,0);
\coordinate (m5) at (0.39,0.066,0);
\coordinate (m6) at (0.39,0.234,0);
%
% Farvning
% -------------------------------------------------------
\filldraw[fill=myblue,opacity=0.3](a)--(b)--(c)--(e)--(g)--(a);
%
\draw[thick, color= myblue](a)--(b); 
\draw[thick, color= myblue](c)--(b);
\draw[thick, color= myblue](a)--(g);
\draw[thick, color= myblue](c)--(e);
\draw[thick, color= myblue](e)--(g);
%
% Punkt A
% -------------------------------------------------------
\draw[very thick,->](a)--(a1);
\draw[very thick,->](a)--(a2);
\draw[very thick,->](a)--(a3);
%
%
% Punkt E
% -------------------------------------------------------
\draw[very thick,->](e)--(e1);
\draw[very thick,->](e)--(e2);
\draw[very thick,->](e)--(e3);
\draw[very thick,->](e)--(e4);
%
% Punkt Mid
% -------------------------------------------------------
\draw[very thick,->](m)--(m1);
\draw[very thick,->](m)--(m2);
\draw[very thick,->](m)--(m3);
\draw[very thick,->](m)--(m4);
\draw[very thick,->](m)--(m5);
\draw[very thick,->](m)--(m6);
%
% Punkterne 
% -------------------------------------------------------
\filldraw [black] (0.35,0.35,0) circle (0pt) node[above] {$\mathcal{P}$};
\filldraw [black] (0.31,0.17,0) circle (0pt) node[anchor=south east]{$\mathbf{v}$};
\filldraw [black] (e) circle (0pt) node[anchor=
south east]{$\mathbf{u}$};
\filldraw [black] (a) circle (0pt) node[anchor= north east]{$\mathbf{w}$};

%
%
\end{tikzpicture}
  \captionof{figure}{Vektorerne $\mathbf{u}, \mathbf{v}$ og $\mathbf{w}$, samt tilhørende forskellige mulige retninger, som stadig er indeholdt i polyederet $\mathcal{P}$.}
  \label{fig:julieersmuuuuuk}
\end{center}
%
%
%
%
% Definition 3.2 
\begin{defn}{}{}
Lad $\mathbf{v}$ være en basal løsning til en basis matrix $B$ og tilhørende objektfunktionen $\mathbf{c}_B=[ c_{B(1)},c_{B(2)}, \cdots , c_{B(m)} ]^T.$
For hver $j$'te basale retning defineres den \textbf{reducerede objektfunktion} $c_j^*$ som
\begin{align*}
c_j^* = c_j - \mathbf{c}_B^T B^{-1} \mathbf{A}_j.
\end{align*} 
% Nogle kalder den den reducerede omkostning.
%
\end{defn}
\noindent
%
% Noget mere tekst her 
% Det med simplextabellen 