\section{Kontruktionen af Simplex-metoden}
%
\textcolor{red}{
%Simplexmetoden afsøger dermed hjørnerne på denne vis, hvis de ikke er degenererede basale mulige løsninger.
%Dermed vil en optimal basismatrix sikre en løsning indenfor løsningsmængden. % er lidt i tvivl om denne sætning er helt rigtigt.
%bla bla 
Afsnit \ref{julieerfantalastiskogvidunderlig} belyser at simplexmetoden finder basale retninger, som reducere omkostningen i objektfunktionen. 
Med optimale betingelser undersøges nu, hvilken mulig retning, der skal vælges. 
Omkostningsvektoren $\mathbf{c}^*$ udregnes for en given basal mulig løsning. 
Hvis ingen basale retninger vil reducerer omkostningen  er løsningen optimal. 
Hvis den reducerede omkostning $c^*_j$ er negativ, for en mulige retning $\mathbf{d}$, følges denne retning. 
Hermed vil $x_j$ stige, og de resterende ikke-basale variable fastholdes.
Eftersom det ønskes at minimere omkostningen, bevæges der så langt væk som muligt fra $\mathbf{x}$, så omkostningen falder.
Det ønskes dermed at finde den største skalar
\begin{align*}
\theta^* = \max \{ \theta \geq 0 \mid \textbf{x} + \theta^*\textbf{d} \in \mathcal{P} \},
\end{align*}
%
der opfylder, at $\mathbf{x} + \theta^* \mathbf{d}$ er en mulig løsning.
% Nu kan vi eventuelt udelede \theta^* xD 
Herefter beregnes omkostningsvektoren for den nye mulige løsning, og samme procedure gentages. Sådan forsættes dette indtil, at ingen basale retninger reducere omkostningen, og løsningen dermed er optimal.
%\\\\
% Vi kan eventuelt her beskrive kort med ord den formel som bliver brugt i den naive implimentering 
%
%Den største skalar 
%\begin{align*}
%\theta^*
%\end{align*}
}
%
%
%	\begin{thm}{}{}
Lad $\mathbf{x}$ være en basal mulig løsning, til en basis matrix $B$ og lad $\mathbf{c}^*$ være den reducerede objektfunktion. 
\begin{enumerate}[label = (\alph*)]
\item Hvis $\mathbf{c}^* \geq 0$, så er $\mathbf{x}$ optimal.
\item Hvis $\mathbf{x}$ er optimal og ikke-degenereret, så er $\mathbf{c}^* \geq 0$.
\end{enumerate}
\end{thm}
%
\begin{proof}
\begin{enumerate}[label = (\alph*)]
\item Antag, at $\bar{\textbf{c}} \geq \textbf{0}$. Lad $\textbf{y}$ være en arbitrær mulig løsning, og $\textbf{d} = \textbf{y} - \textbf{x}$. Grundet de mulige løsninger, er $\textbf{A} \textbf{x} = \textbf{A} \textbf{y} = \textbf{b}$, hvormed $\textbf{A} \textbf{d} = \textbf{0}$. Sidstnævnte kan skrives som 
$$\textbf{B} \textbf{d}_B + \sum_{i \in N} \textbf{A}_i d_i = \textbf{0},$$
hvor $N$ er mængden af indekser, der svarer til de ikke-basale variabler i basen. %wtf fatter slet ikke det her lol
$\textbf{B}$ er invertibel, hvormed det haves, at 
$$\textbf{d}_B = -\sum_{i \in N} \textbf{B}^{-1} \textbf{A}_i d_i ,$$
og
$$\textbf{c}' \textbf{d} = \textbf{c}'_B \textbf{d}_B + \sum_{i \in N} c_i d_i = \sum_{i \in N} (c_i - \textbf{c}'_B \textbf{B}^{-1} \textbf{A}_i) d_i = \sum_{i \in N} \bar{c}_i d_i .$$
%nanananana fatter intet, hvorfor har vi dette med? 
For ethvert ikke-basalt indeks $i \in N$ må der eksistere et $x_i = 0$, samt et $y_i \geq 0$, da $\textbf{y}$ er en mulig løsning. 
Deraf haves, at $d_i \geq 0$ og $\bar{c}_i d_i \geq 0$ for alle $i \in N$. 
Dermed er $\textbf{c}' (\textbf{y} - \textbf{x} = \textbf{c}' \textbf{d} \geq 0$, og $\textbf{x}$ er en optimal løsning, eftersom $\textbf{y}$ er en arbitrær mulig løsning. 
%
%
\item Antag, at $x$ er en ikke-degenereret basal mulig løsning, og at $\bar{c}_j < 0$ for et $j$. Eftersom den reducerede objektfunktion for en basal variabel altid er $0$, må $x_j$ være en ikke-basal variabel, og $\bar{c}_j$ må være ændringshastigheden for objektfunktionen i den $j$'te basale retning. 
%Wtf er ovenstående for noget lort
Da $\textbf{x}$ ikke er degenereret, er den $j$'te basale retning en mulig retning for mindskelse af objektfunktionen. 
Der findes dermed mulige løsninger, hvis objektfunktion har lavere værdi end $\textbf{x}$'s, ved at bevæge sig i denne retning, hvormed $\textbf{x}$ ikke er optimal. 
\end{enumerate}
\end{proof}
%	
%	\input{incl/main/simplex_metoden/thm_3.2}
%	En iteration af simplex metoden
\begin{enumerate}
\item En normal iteration starter med et basis bestående af de basale rækker $\textbf{A}_{B(1)},\ldots,\textbf{A}_{B(m)}$ og den tilhørende basale mulige løsning $\textbf{x}$.
\item Beregn den reducerede objektfunktion $c_j^* = c_j - \mathbf{c}_B^T \textbf{B}^{-1}A_j$ for alle ikke-basale indekser $j$. Hvis de alle er ikke-negative, så er den nuværende basale mulige løsning optimal; ellers vælges et indeks $j$, hvorom det gælder, at $c^*_j<0$
\item Udregn $\textbf{u}=\textbf{B}^{-1}\textbf{A}_j$. Hvis der ikke er nogen komponent af $\textbf{u}$, som er positiv, så er $\theta ^*=\infty$, den optimale løsnings værdi er $-\infty$, og algorithmen stopper.
\item Hvis en komponent af $\textbf{u}$ er positiv, så lad 
$$\theta^*= min_{ \{i=1,\ldots,m|u_i>0 \} }        \dfrac{x_{B(i)}}{u_i}$$
\item Lad $l$ være indekset for $\theta^*=  \dfrac{x_{B(l)}}{u_l}$. Dan et nyt basis ved at udskifte $\textbf{A}_{B(l)}$ med $\textbf{A}_j$. Hvis $\textbf{y}$ er den nye basale mulige løsning, så er værdierne af de nye basale variable $y_j=\theta^*$ og $y_{B(i)}=x_{B(i)}-\theta^*u_i,i\neq l$
\end{enumerate}
%
%
I det ikke-degenererede tilfælde siger følgende sætning, at simplex-metoden fungerer og stopper efter et endeligt antal iterationer.
% Sætning 3.3
\begin{thm}{}{}
Antag, at løsningsmængden er ikke-tom og at enhver basal mulig løsning er ikke-degenereret.
Så stopper simplex-metoden efter et endeligt antal iterationer. 
Ved termineringen er der to muligheder:
\begin{enumerate}[label = (\alph*)]
\item Et optimalt basis $\textbf{B}$ er fundet, og den tilsvarende basale mulige løsning er optimal.
\item Der findes en vektor $\textbf{d}$, der opfylder $\textbf{Ad}=0,\textbf{d}\geq 0,$ og $\textbf{c}^T\textbf{d}<0$, og den optimale værdi er $-\infty$
\end{enumerate}
\end{thm}
%
\begin{proof}
%jeg ville ikke lige til at pille i det andet dokument hvis andre skrev i det pr sætning 6.1(3.1 i bertsima)
Hvis algoritmen stopper i trin 2, så følger det af Sætning 6.1(mangler en ref), at $\textbf{B}$ er et optimalt basis og den nuværende basale mulige løsninger er optimal.
\\
Hvis algoritmen stopper på grund af stopkriteriet i trin $3$, så er den nuværende basale mulige løsning $\textbf{x}$. 
Der findes desuden en ikke-basal variabel $x_j$, således at $c^*_j<0$, og den tilsvarende basale retning $\textbf{d}$ tilfredsstiller $\textbf{Ad}=\textbf{0}$ og $\textbf{d} \geq \textbf{0}$. 
Dertil, at $x+\theta \textbf{d}\in P$ for alle $\theta>0$. 
Eftersom $\textbf{c}^T\textbf{d}=c_j^*<0$, ved at tage et arbitrært stort $\theta$, fås en arbitrær negativ objektfunktion, og den optimale objektsværdi er $-\infty$.
\\\\
Ved hver iteration bevæger algoritmen sig en positiv værdi $\theta*$ langs en retning $\textbf{d}$, der ofylder $\textbf{c}^T\textbf{d}<0$. 
Som følge af dette forbedres objektfunktionens værdi for hver basal mulig løsning simplex algoritmen undersøger, samt at den samme løsning ikke besøges mere end en gang. 
Eftersom der er et endeligt antal basale mulige løsninger, terminerer algoritmen efter et endeligt antal iterationer.
\end{proof}
%

%