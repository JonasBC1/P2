\section{Optimale betingelser}
\label{julieerfantalastiskogvidunderlig}
% --------------------------------------------------------------
%
For at finde den optimale løsning kan der udvælges en basal løsning og undersøges, hvorvidt der findes andre løsninger, som er bedre, end den valgte. 
Hvis der ikke findes bedre løsninger, er den valgte basale løsning den optimale løsning. 
Eftersom der jævnfør afsnit \ref{julieerlakker} optimeres for en konveks funktion over en konveks mængde, %Det er det med at et lokal minimum er et globalt minimum xD
vil den lokale optimale løsning være en global optimal løsning. 
%
%Når der undersøges om der findes et nabo punkt, hvis løsning er bedre, så er det ikke nødvendigt at undersøge i retningen, som leder ud af den basal mulige område.
%
% Definition 3.1 i bert
\begin{defn}{}{}
Lad $\textbf{v}$ være et element af et polyeder $\mathcal{P}$.
En vektor $\textbf{d}\in \R^n$ siges at være en \textbf{mulig retning} fra $\textbf{v}$, hvis der eksisterer en positiv skalar $\theta$, således at $\textbf{v}+\theta \textbf{d}\in \mathcal{P}$.
\end{defn}
\noindent
%
På figur \ref{fig:julieersmuuuuuk}, ses vektorerne $\mathbf{u}, \mathbf{v}$ og $\mathbf{w}$ i polyederet $\mathcal{P}$, samt tilhørende forskellige mulige retninger, som stadig er indeholdt i $\mathcal{P}$.
%
\begin{center}
%
\begin{tikzpicture}[scale=8]
%
% Koordinater
% -------------------------------------------------------
\coordinate (a) at (0,0,0); 
\coordinate (a1) at (0.084,0,0);
\coordinate (a2) at (0,0.084,0);
\coordinate (a3) at (0.07,0.07,0);
%
\coordinate (b) at (0.7,0,0); 
\coordinate (c) at (0.393,0.55,0); 
%
\coordinate (e) at (0.066,0.434,0);
\coordinate (e1) at (0.15,0.46,0);
\coordinate (e2) at (0.025,0.35,0);
\coordinate (e3) at (0.094,0.36,0);
\coordinate (e4) at (0.14,0.4,0);
%
\coordinate (g) at (0,0.305,0);
%
\coordinate (m) at (0.35,0.15,0);
\coordinate (m1) at (0.31,0.066,0);
\coordinate (m2) at (0.31,0.234,0);
\coordinate (m3) at (0.266,0.15,0);
\coordinate (m4) at (0.434,0.15,0);
\coordinate (m5) at (0.39,0.066,0);
\coordinate (m6) at (0.39,0.234,0);
%
% Farvning
% -------------------------------------------------------
\filldraw[fill=myblue,opacity=0.3](a)--(b)--(c)--(e)--(g)--(a);
%
\draw[thick, color= myblue](a)--(b); 
\draw[thick, color= myblue](c)--(b);
\draw[thick, color= myblue](a)--(g);
\draw[thick, color= myblue](c)--(e);
\draw[thick, color= myblue](e)--(g);
%
% Punkt A
% -------------------------------------------------------
\draw[very thick,->](a)--(a1);
\draw[very thick,->](a)--(a2);
\draw[very thick,->](a)--(a3);
%
%
% Punkt E
% -------------------------------------------------------
\draw[very thick,->](e)--(e1);
\draw[very thick,->](e)--(e2);
\draw[very thick,->](e)--(e3);
\draw[very thick,->](e)--(e4);
%
% Punkt Mid
% -------------------------------------------------------
\draw[very thick,->](m)--(m1);
\draw[very thick,->](m)--(m2);
\draw[very thick,->](m)--(m3);
\draw[very thick,->](m)--(m4);
\draw[very thick,->](m)--(m5);
\draw[very thick,->](m)--(m6);
%
% Punkterne 
% -------------------------------------------------------
\filldraw [black] (0.35,0.35,0) circle (0pt) node[above] {$\mathcal{P}$};
\filldraw [black] (0.31,0.17,0) circle (0pt) node[anchor=south east]{$\mathbf{v}$};
\filldraw [black] (e) circle (0pt) node[anchor=
south east]{$\mathbf{u}$};
\filldraw [black] (a) circle (0pt) node[anchor= north east]{$\mathbf{w}$};

%
%
\end{tikzpicture}
  \captionof{figure}{Vektorerne $\mathbf{u}, \mathbf{v}$ og $\mathbf{w}$, samt tilhørende forskellige mulige retninger, som stadig er indeholdt i polyederet $\mathcal{P}$.}
  \label{fig:julieersmuuuuuk}
\end{center}
%
%
%Hjørnerne afsøges derved gennem en reduceret objektfunktion.
%Se \ref{defn:kogtskinke}.
%
\textcolor{red}{
For at finde en ny mulig retning $\mathbf{d}$, bevæges der i en retning, hvor værdien af en ikke-basal variable $x_j$ øges, og andre ikke-basale variable fastholdes. 
Dermed vil vektoren $\mathbf{d}_j = 1$, og $\mathbf{d}_i = 0$ for henholdsvis den ikke-basale variable $x_j$ og de andre ikke-basale variable $x_i$, hvor $i \neq j$.
Vektoren 
\begin{align*}
\mathbf{d}_B = \left( d_{B(1)}, d_{B(2)}, \ldots , d_{B(m)} \right),
\end{align*} 
bestående af de mulige retninger for basis variablerne, 
kontureres.
Da $\mathbf{x}$ er en basal mulig løsning gælder, at $A \mathbf{x} = \mathbf{b}$. 
Eftersom $\mathbf{x}+ \theta \mathbf{d}$ skal være en mulig løsning, gælder $A ( \mathbf{x}+ \theta \mathbf{d}) = \mathbf{b}$.
Deraf haves, at 
\begin{align*}
A ( \mathbf{x}+ \theta \mathbf{d}) &=  A \mathbf{x} + \theta A \mathbf{d} = \mathbf{b} \\
& \Updownarrow \\
\mathbf{b} + \theta A \mathbf{d} & = \mathbf{b} \\
& \Updownarrow \\
\theta A \mathbf{d} & = \mathbf{0}.
\end{align*} 
Bemærk, at $\theta$ er en positiv skalar. 
Dermed skal $\mathbf{d}$ tilhører $\text{nrum}(A)$.
Eftersom $\mathbf{d}_j = 1$ og $\mathbf{d}_i = 0$ så er 
%
\begin{align*}
\mathbf{0} = A \mathbf{d} & = \sum^n_{i = 1} \mathbf{A}_i d_i \\
& =  \mathbf{A}_j  + \sum^m_{i = 1} \mathbf{A}_{B(i)} d_{D(i)} \\
& =  \mathbf{A}_j  + B \mathbf{d}_B.
\end{align*}
%
Da basis matricen $B$ er invertibel haves, at 
% 
\begin{align*}
\mathbf{A}_j  + B \mathbf{d}_B & = 0 \\
& \Updownarrow \\
B \mathbf{d}_B & = - \mathbf{A}_j \\
& \Updownarrow \\
\mathbf{d}_B & = - B^{-1} \mathbf{A}_j .
\end{align*}
% 
Denne mulige retning $d$, tilhørende $x_j$, kaldes den \textit{$j$'te basale retning}.
% 
\begin{defn}{}{}
Lad $\mathbf{x}$ være en basal mulig løsning. 
Den \textbf{$j$'te basale retning} $\mathbf{d}$, fra $\mathbf{x}$, for et ikke-basalt variabel $x_j$, har komponenterne 
%
\begin{align*}
d_j & = 1, \\
d_i & = 0 \phantom{hej} for i \neq b(1), \ldots , B(m) \wedge i \neq j, \\
\mathbf{d}_B & = - B^{-1} \mathbf{A}_j.
\end{align*}
% 
\end{defn}
\noindent
%
Bemærk, at overstående kun sikre lighedsbegrænsningerne, og ikke ikke-negativitetsbetingelserne. 
Eftersom $x_j$ øges, og de andre ikke-basale variable fastholdes, kan følgende scenarier opstå. 
\begin{itemize}
\item Antag, at $\mathbf{x}$ er en ikke-degenereret basal mulig løsning. 
Dermed er $\mathbf{x}_B > \mathbf{0}$. 
Hvis $\theta$ er lille nok, vil $\mathbf{x}_B + \theta \mathbf{d}_B \geq 0 $, og $\mathbf{d}$ er dermed en mulig retning. 
\item Antag, at $\mathbf{x}$ er en degenereret basal mulig løsning. 
Dermed er $\mathbf{d}$ ikke altid en mulig retning. 
Eftersom $\mathbf{d}$ er degenereret, er det muligt at en basis variable $x_{B(i)}=0$, og den tilsvarende komponent $d_{B(i)}$ er negativ. 
I dette tilfælde, vil en bevægelse i den $j$'te mulige retning forlade polyedret, da en ikke-negativitetsbetingelse brydes af $\mathbf{x} + \theta \mathbf{d}$ for alle positive $\theta$. 
\end{itemize}
%
På figur \ref{fig:julieerengud} de to scenarier. 
Bemærk, at $A$ har de to ikke-basis variable $x_1$ og $x_3$ og $x_2,$ $x_4$ og $x_5$ er basis variable.
I hjørnet $A$ fastholdes ikke-basis variablen $x_3$, og $x_1$ øges. 
Dermed bevæges der i retningen ad randen $AB$. 
Bemærk, at $B$ har de to ikke-basis variable $x_3$ og $x_5$ og $x_1,$ $x_1$ og $x_4$ er basis variable.
I hjørnet $B$ fastholdes ikke-basis variablen $x_5$, og $x_3$ øges. 
Dermed bevæges der i retningen ad randen $BC$, hvilket bryder en ikke-negativitetsbetingelse, og dermed ikke en mulig retning. 
}
%
\begin{center}
%
\begin{tikzpicture}[scale=5]
%
% Koordinater
% -------------------------------------------------------
\coordinate (a) at (0,0,0); 
\coordinate (b) at (0.7,0,0); 
\coordinate (c) at (0.393,0.55,0); 
\coordinate (d) at (0.252,0.805,0);
\coordinate (e) at (0.066,0.434,0);
\coordinate (f) at (0,0.41,0);
\coordinate (g) at (0,0.305,0);
\coordinate (h) at (-0.15,0,0); 
%
% Farvning
% -------------------------------------------------------
\filldraw[fill=myblue,opacity=0.3](a)--(b)--(c)--(f)--(a);
  %\draw[thick](-0.2,-0.1,0)--(0.3,0.9,0); % n -> d  
  \draw[thick](0.2,0.9,0)-- node[anchor=north east] {$x_4=0$} (0.757,-0.1,0); % d -> b
  \draw[thick](-0.3,0.3,0)-- node[above] {$x_3=0$ \phantom{hej}} (0.8,0.7,0); % f -> c
  \draw[thick](0.12,0.9,0)-- node[anchor=south west] {$x_5=0$} (0.9,-0.1,0); % ekstra
% 
%
% Punkterne 
% -------------------------------------------------------
%\filldraw [black] (a) circle (0.2pt) node[anchor=north west] {$A$};
%\filldraw [black] (b) circle (0.2pt) node[anchor=north east] {$B$};
\filldraw [black] (c) circle (0.2pt) node[anchor=south west] {$B$};
\filldraw [black] (0.822,0,0) circle (0.2pt) node[anchor=south west] {$C$};
%\filldraw [black] (d) circle (0.2pt) node[anchor=west] {$D$};
%\filldraw [black] (e) circle (0.2pt) node[anchor=north west] {$E$};
\filldraw [black] (f) circle (0.2pt) node[anchor=south east] {$A$};
%\filldraw [black] (g) circle (0.2pt) node[anchor=east] {$G$};
%\filldraw [black] (h) circle (0.2pt) node[anchor=north west] {$H$};
%
% 
\filldraw [black] (0.27,0.17,0) circle (0pt) node[above] {$\mathcal{P}$};
% 
% Koordinatssystem 
% -------------------------------------------------------
\draw[thick] (0,0,0) -- node[anchor=north east] {$x_2=0$} (0.9,0,0);
\draw[thick] (0,0,0) -- (-0.2,0,0);
\draw[thick] (0,0,0) -- node[anchor=north east] {$x_1=0$} (0,0.5,0);
\draw[thick] (0,0.5,0) -- (0,0.9,0);
\draw[thick] (0,0,0) -- (0,-0.2,0);
%
\end{tikzpicture}
  \captionof{figure}{Illustration af en mulig retning, der svarer til randen $AB$, og en ikke-mulig retning, der svarer til randen $BC$, som bryder en ikke-negativitetsbetingelse.}
  \label{fig:julieerengud}
\end{center}
%
\textcolor{red}{
Heraf, opstår et problem med degenerede løsninger, og af den grund forsøges disse undgået.
\\\\
%
Ved en bevægelse i en mulig retning, medfører det at 
\begin{align*}
\mathbf{c}^T \mathbf{d} = \mathbf{c}^T_B \mathbf{d}_B + c_j,
\end{align*}
hvor 
$$\mathbf{c}_B=[ c_{B(1)},c_{B(2)}, \cdots , c_{B(m)} ]^T.$$
Bemærk, da $\mathbf{d}_B = - B^{-1} \mathbf{A}_j$, er 
\begin{align*}
\mathbf{c}^T \mathbf{d} &= \mathbf{c}^T_B ( - B^{-1} \mathbf{A}_j ) + c_j \\
& = c_j - \mathbf{c}^T_B (B^{-1} \mathbf{A}_j).
\end{align*}
%
For at omkostningen minimeres skal $c_j - \mathbf{c}^T_B (B^{-1} \mathbf{A}_j ) < 0$, da 
% 
\begin{align*}
\mathbf{c}^T  ( \mathbf{x} + \theta \mathbf{d} )  = \mathbf{c}^T  \mathbf{x} + \theta \mathbf{c}^T \mathbf{d} = \mathbf{c}^T  \mathbf{x} + \theta ( c_j - \mathbf{c}^T_B (B^{-1} \mathbf{A}_j ) ) < \mathbf{c}^T \mathbf{x} 
\end{align*}
%
for alle positive $\theta$.
Størrelsen for reduktionen for hvert $\theta$ kaldes \textit{den reducerede omkostning}.
%
% Definition 3.2 
\begin{defn}{}{kogtskinke}
Lad $\mathbf{v}$ være en basal løsning til en basis matrix $B$ og tilhørende omkostningsvektoren $$\mathbf{c}_B=[ c_{B(1)},c_{B(2)}, \cdots , c_{B(m)} ]^T.$$
For hver $j$'te basale retning defineres den \textbf{reducerede omkostning} $c_j^*$ som
\begin{align*}
c_j^* = c_j - \mathbf{c}_B^T B^{-1} \mathbf{A}_j.
\end{align*} 
%
\end{defn}
\noindent
%
\textit{Den reducerede omkostningsvektor} $\mathbf{c}^*$ indeholder alle de reducerede omkostninger, hvor den $k$'te række betegner den reducerede omkostning, for den $k$'te basale retning.
%
\begin{align*}
\mathbf{c}^* = (\mathbf{c}^T - \mathbf{c}_B^T B^{-1} A)^T = 
\begin{blockarray}{[c]}
c_1 - \mathbf{c}_B^T B^{-1} \mathbf{A}_1 \\
\vdots \\
c_k - \mathbf{c}_B^T B^{-1} \mathbf{A}_k \\
\vdots \\
c_n - \mathbf{c}_B^T B^{-1} \mathbf{A}_n
\end{blockarray}.
\end{align*}
%
Betragt de basale indgange i den reducerede omkostningsvektor,
%
\begin{align*}
c^*_{B(j)} & = c_{B(j)} - \mathbf{c}_B^T B^{-1} \mathbf{A}_{B(j)} \\
& = c_{B(j)} - \mathbf{c}_B^T \mathbf{e_j} \\
& = c_{B(j)} -  c_{B(j)}  \\
& = 0.
\end{align*}
%
Dermed er indgangende til de basale variable, er lig nul, da de mulige retninger er nulvektoren. 
Det kan derfor ikke betale sig at øge basale variable.
\\\\
Hvis den reducerede omkostningsvektor $\mathbf{c} \geq \mathbf{0}$, er den basale mulige løsning et lokalt minimum i objektfunktionen og dermed et globalt minimum. Heraf er den optimale løsning fundet. Dette tilfælde belyses i sætning \ref{thm:julieerguden}.
%
\begin{thm}{}{julieerguden}
Lad $\mathbf{x}$ være en basal mulig løsning, til en basis matrix $B$ og lad $\mathbf{c}^*$ være den reducerede omkostningsvektor. 
\begin{enumerate}[label = (\alph*)]
\item Hvis $\mathbf{c}^* \geq 0$, så er $\mathbf{x}$ optimal.
\item Hvis $\mathbf{x}$ er optimal og ikke-degenereret, så er $\mathbf{c}^* \geq 0$.
\end{enumerate}
\end{thm}
%
\begin{proof}
Bevis udelades.
%\begin{enumerate}[label = (\alph*)]
%\item Antag, at $\bar{\textbf{c}} \geq \textbf{0}$. Lad $\textbf{y}$ være en arbitrær mulig løsning, og $\textbf{d} = \textbf{y} - \textbf{x}$. Grundet de mulige løsninger, er $\textbf{A} \textbf{x} = \textbf{A} \textbf{y} = \textbf{b}$, hvormed $\textbf{A} \textbf{d} = \textbf{0}$. Sidstnævnte kan skrives som 
%$$\textbf{B} \textbf{d}_B + \sum_{i \in N} \textbf{A}_i d_i = \textbf{0},$$
%hvor $N$ er mængden af indekser, der svarer til de ikke-basale variabler i basen. %wtf fatter slet ikke det her lol
%$\textbf{B}$ er invertibel, hvormed det haves, at 
%$$\textbf{d}_B = -\sum_{i \in N} \textbf{B}^{-1} \textbf{A}_i d_i ,$$
%og
%$$\textbf{c}' \textbf{d} = \textbf{c}'_B \textbf{d}_B + \sum_{i \in N} c_i d_i = \sum_{i \in N} (c_i - \textbf{c}'_B \textbf{B}^{-1} \textbf{A}_i) d_i = \sum_{i \in N} \bar{c}_i d_i .$$
%%nanananana fatter intet, hvorfor har vi dette med? 
%For ethvert ikke-basalt indeks $i \in N$ må der eksistere et $x_i = 0$, samt et $y_i \geq 0$, da $\textbf{y}$ er en mulig løsning. 
%Deraf haves, at $d_i \geq 0$ og $\bar{c}_i d_i \geq 0$ for alle $i \in N$. 
%Dermed er $\textbf{c}' (\textbf{y} - \textbf{x} = \textbf{c}' \textbf{d} \geq 0$, og $\textbf{x}$ er en optimal løsning, eftersom $\textbf{y}$ er en arbitrær mulig løsning. 
%%
%%
%\item Antag, at $x$ er en ikke-degenereret basal mulig løsning, og at $\bar{c}_j < 0$ for et $j$. Eftersom den reducerede objektfunktion for en basal variabel altid er $0$, må $x_j$ være en ikke-basal variabel, og $\bar{c}_j$ må være ændringshastigheden for objektfunktionen i den $j$'te basale retning. 
%%Wtf er ovenstående for noget lort
%Da $\textbf{x}$ ikke er degenereret, er den $j$'te basale retning en mulig retning for mindskelse af objektfunktionen. 
%Der findes dermed mulige løsninger, hvis objektfunktion har lavere værdi end $\textbf{x}$'s, ved at bevæge sig i denne retning, hvormed $\textbf{x}$ ikke er optimal. 
%\end{enumerate}
\end{proof}
\\
%
Dermed er alle basale retninger, mulige retninger, hvis et problem er givet uden degenererede løsninger. 
Heraf defineres \textit{optimal basismatrix}.
%
% Definition 3.3 
\begin{defn}{}{julieergudenoveralleguder}
En basismatrix $B$, kaldes optimal, hvis:
%
\begin{enumerate}[label = (\alph*)]
\item $B^{-1} \mathbf{b} \geq \mathbf{0}$, og
\item $\mathbf{c}^* \geq \mathbf{0}$.
\end{enumerate}
%
\end{defn}
\noindent
%
\section{Kontruktionen af Simplex-metoden}
%
%Simplexmetoden afsøger dermed hjørnerne på denne vis, hvis de ikke er degenererede basale mulige løsninger.
%Dermed vil en optimal basismatrix sikre en løsning indenfor løsningsmængden. % er lidt i tvivl om denne sætning er helt rigtigt.
%bla bla 
Afsnit \ref{julieerfantalastiskogvidunderlig} belyser at simplexmetoden fidner basale retninger, som reducere omkostningen i objektfunktionen. Nu undersøges hvilken mulig retning, der skal vælges. 
Omkostningsvektoren $\mathbf{c}^*$ udregnes for en given basal mulig løsning. 
Hvis ingen basale retninger vil reducerer omkostningen  er løsningen optimal. 
Hvis den reducerede omkostning $c^*_j$ er negativ, følges den mulige retning $\mathbf{d}$. 
Hermed vil $x_j$ stige, og de resterende ikke-basale variable fastholdes.
Eftersom det ønskes at minimere omkostningen, bevæges der så langt som muligt væk fra $\mathbf{x}$, så omkostningsen falder.
Det ønskes dermed at finde den største skalar
\begin{align*}
\theta^* = \max \{ \theta \geq 0 \mid \textbf{x} + \theta^*\textbf{d} \in \mathcal{P} \},
\end{align*}
%
der opfylder, at $\mathbf{x} + \theta^* \mathbf{d}$ er en mulig løsning.
% Nu kan vi eventuelt udelede \theta^* xD 
}
%
%
%	\begin{thm}{}{}
Lad $\mathbf{x}$ være en basal mulig løsning, til en basis matrix $B$ og lad $\mathbf{c}^*$ være den reducerede objektfunktion. 
\begin{enumerate}[label = (\alph*)]
\item Hvis $\mathbf{c}^* \geq 0$, så er $\mathbf{x}$ optimal.
\item Hvis $\mathbf{x}$ er optimal og ikke-degenereret, så er $\mathbf{c}^* \geq 0$.
\end{enumerate}
\end{thm}
%
\begin{proof}
\begin{enumerate}[label = (\alph*)]
\item Antag, at $\bar{\textbf{c}} \geq \textbf{0}$. Lad $\textbf{y}$ være en arbitrær mulig løsning, og $\textbf{d} = \textbf{y} - \textbf{x}$. Grundet de mulige løsninger, er $\textbf{A} \textbf{x} = \textbf{A} \textbf{y} = \textbf{b}$, hvormed $\textbf{A} \textbf{d} = \textbf{0}$. Sidstnævnte kan skrives som 
$$\textbf{B} \textbf{d}_B + \sum_{i \in N} \textbf{A}_i d_i = \textbf{0},$$
hvor $N$ er mængden af indekser, der svarer til de ikke-basale variabler i basen. %wtf fatter slet ikke det her lol
$\textbf{B}$ er invertibel, hvormed det haves, at 
$$\textbf{d}_B = -\sum_{i \in N} \textbf{B}^{-1} \textbf{A}_i d_i ,$$
og
$$\textbf{c}' \textbf{d} = \textbf{c}'_B \textbf{d}_B + \sum_{i \in N} c_i d_i = \sum_{i \in N} (c_i - \textbf{c}'_B \textbf{B}^{-1} \textbf{A}_i) d_i = \sum_{i \in N} \bar{c}_i d_i .$$
%nanananana fatter intet, hvorfor har vi dette med? 
For ethvert ikke-basalt indeks $i \in N$ må der eksistere et $x_i = 0$, samt et $y_i \geq 0$, da $\textbf{y}$ er en mulig løsning. 
Deraf haves, at $d_i \geq 0$ og $\bar{c}_i d_i \geq 0$ for alle $i \in N$. 
Dermed er $\textbf{c}' (\textbf{y} - \textbf{x} = \textbf{c}' \textbf{d} \geq 0$, og $\textbf{x}$ er en optimal løsning, eftersom $\textbf{y}$ er en arbitrær mulig løsning. 
%
%
\item Antag, at $x$ er en ikke-degenereret basal mulig løsning, og at $\bar{c}_j < 0$ for et $j$. Eftersom den reducerede objektfunktion for en basal variabel altid er $0$, må $x_j$ være en ikke-basal variabel, og $\bar{c}_j$ må være ændringshastigheden for objektfunktionen i den $j$'te basale retning. 
%Wtf er ovenstående for noget lort
Da $\textbf{x}$ ikke er degenereret, er den $j$'te basale retning en mulig retning for mindskelse af objektfunktionen. 
Der findes dermed mulige løsninger, hvis objektfunktion har lavere værdi end $\textbf{x}$'s, ved at bevæge sig i denne retning, hvormed $\textbf{x}$ ikke er optimal. 
\end{enumerate}
\end{proof}
%	
%	Det skal nu vises, hvordan simplex-metoden kan udvikles ud fra geometriske repræsentationer.
Altså, hvordan der udvikles en metode til at gå fra en profitabel basal mulig løsning til en bedre basal mulig løsning.
Det antages indtil videre, at løsningen er ikke-degenereret.
Antag, at der stås i punktet for den basale mulige løsning $\mathbf{x}$ og at den reducerede objektfunktion $c_j$ er udregnet, hvis alle variable i denne er ikke-negative, haves det fra sætning (3.1 i bertsimas), at løsningen er den optimale, og der behøves derfor ikke yderligere skridt.
Såfremt dette ikke er tilfældet, altså når en ikke-basal variabel $x_j$ i den reducerede objektfunktion er negativ, ville en bedre løsning potentielt forefindes i retningen $\mathbf{d}$, givet ved  $d_j=1$, $d_i=0$ for $i \neq B(1),\ldots,B(m),j$ og $\mathbf{d}_b=-B^{-1}A_j$.
Hvis der foregår en bevægelse i retning $\mathbf{d}$, bliver $x_j$ positiv, imens de øvrige ikke-basale variable forbliver $0$.
Denne bevægelse kan beskrives med $\mathbf{x}+\theta \mathbf{d}$. 
Da værdien af objektfunktionen er aftagende i retning $\mathbf{d}$, er det ønskværdigt at følge $\mathbf{d}$, så langt som muligt med henblik på at ende i punktet $\mathbf{x}+\theta^{*} \mathbf{d}$, hvilket kan udtrykkes som 
$\theta^{*}=\text{max} \theta \geq 0 \mid \mathbf{x}+\theta \mathbf{d} \in \mathcal{P}$.
Forandringen i objektfunktionen er således $\theta^{*}\mathbf{c}^T \mathbf{d}=\theta^{*}c_j$.
Der skal herfra udledes en formel for $\theta^{*}$.
Givet at $A\mathbf{d}=\mathbf{0}$, følger det, at $A(\mathbf{x}+\theta \mathbf{d})=A\mathbf{x}=\mathbf{b}$ for alle $\theta$.
%and the equality constraints will never be violated. hvordan følger det af foregående??
Det gælder således, at $\mathbf{x}+\theta\mathbf{d}$ er en mulig løsning i alle tilfælde, hvor ingen af $\mathbf{d}$'s komponenter er negative.
%usikker på om det er ds komponenter i ovenstående
Der findes således et tilfælde, hvor $\mathbf{d} \geq \mathbf{0}$, hvorfor $\mathbf{x} + \theta \mathbf{d} \geq \mathbf{0}$ for $\forall\theta \geq 0$. 
Det gælder derfor, at $\mathbf{x} + \theta \mathbf{d} $ altid er en mulig løsning, hvormed $\theta^*=\infty$. \\
Såfremt et index i $\mathbf{d}$, $d_i<0$ kan begrænsningen $x_i + \theta d_i \geq 0$ omskrives til $\theta \leq \frac-{x_i}{d_i}$, som skal opfyldes for alle $i$, hvorom det gælder, at $d_i<0$.
Derfor følger det, at $\theta*=\text{min}_{i \mid d_i<0}(-\frac{x_i}{d_i}$.
Da $x_i$ ikke er en basal variabel 
%her kommer det jeg virkeligt ikke fatter umiddelbart
\begin{thm}{}{}
\begin{enumerate}[label = (\alph*)]
\item Søjlerne $A_{B(i)}$, $i \neq \mathcal{L}$ og $A_j$ er lineært uafhængige, og 
\item Vektoren $\mathbf{y}=x+ \theta \mathbf{d}$ er en basal mulig løsning til(associeret med??) basis matricen $B$.
\end{enumerate}
\end{thm}
%
\begin{proof}

(a) hvis vektorene $\mathbf{A}_{B(i)}$,$i=1,\ldots,m$ er linieært afhængige så eksisterer der koeficienter $\lambda_1,\ldots \lambda_m$ hvor ikke alle af disse er $0$, altså løsninger udover den trivielle løsning, hvorom det gælder:
$ \sum_{i=1}^m \lambda_i \mathbf{A}_{B(i)}=\mathbf{0} $ det følger derfor at $\sum_{i=1}^m \lambda_i B^{-1}\mathbf{A}_{B(i)}=\mathbf{0}$ og vektorene $B^{-1}\mathbf{A}_{B(i)}$ er ligeledes lineært uafhængige. Det skal nu vises at dette ikke er tilfældet, ved at vise at vektorene $B^{-1}\mathbf{A}_{B(i},i\neq \mathcal{L}$ og $B^{-1}\mathbf{A}_j$ er lineært uafhængige.
Jævnfør sætning (den om invers og identitetsmatricen) er $B^{-1}B=i$.
Da $A_{B(i)}$ er den $i$'te søjle af $B$ gælder derfor at vektorene $B^{-1}\mathbf{A}_{B(i},i\neq \mathcal{L}$  er alle enhedsvektorene med undtagelse af den $\mathcal{L}$'te enhedsvektor. 
Vektorene er således lineært uafhængige og deres  $\mathcal{L}$'te komponent er $0$.
Jævnfør sætning (igen en af dem med inverse tror) $B^{-1}\mathbf{A}_j=-\mathbf{d}_B$.(B er stadigt en funktion her men gives intet index, WTF?) 
Den $\mathcal{L}$'te indgang $-d_{B(\mathcal{L})}\neq 0$ ud fra definitionen af $\mathcal{L}$.
Derfor må det gælde at $B^{-1} \mathbf{A}_j$ er lineært uafhængige fra enhedsvektorene $B^{-1}\mathbf{A}_{B(i},i\neq \mathcal{L}$ \\
(b) lad $\mathbf{y}\geq \mathbf{0}$, $A\mathbf{y}=\mathbf{b}$ og $y_i=0$ for $i \neq B(1),\ldots,B(m)$. Søjerne $\mathbf{A}_{B(1)},\ldots,\mathbf{A}_{B(m)}$ er i (a) vist liniæert uafhængige. Det følger derfor at $y$ er en basal mulig løsning associeret med basismatricen $B$.
\end{proof}
%	En iteration af simplex metoden
\begin{enumerate}
\item En normal iteration starter med et basis bestående af de basale rækker $\textbf{A}_{B(1)},\ldots,\textbf{A}_{B(m)}$, og den tilhørende basale mulige løsning $\textbf{x}$.
\item Beregn den reducerede objektfunktion $c_j^* = c_j - \mathbf{c}_B^T \textbf{B}^{-1}A_j$ for alle ikke-basale indekser $j$. Hvis de alle er ikke-negative så er den nuværende basale mulige løsning optimal, ellers vælges et indeks $j$ hvorom det gælder at $c^*_j<0$
\item Udregn $\textbf{u}=\textbf{B}^{-1}\textbf{A}_j$. Hvis der ikke er noget komponenet af $\textbf{u}$ som er positiv, så er $\theta ^*=\infty$, samt at den optimale løsnings værdi er $-\infty$, og algorithmen stopper.
\item Hvis et komponent af $\textbf{u}$ er positiv, så lad 
$$\theta^*= min_{ \{i=1,\ldots,m|u_i>0 \} }        \dfrac{x_{B(i)}}{u_i}$$
\item Lad $l$ være indekset for $\theta^*=  \dfrac{x_{B(l)}}{u_l}$. Dan er nyt basis ved at uskifte $\textbf{A}_{B(l)}$ med $\textbf{A}_j$. Hvis $\textbf{y}$ er den nye basale mulige løsning, så er værdierne af de nye basale vaiable $y_j=\theta^*$ og $y_{B(i)}=x_{B(i)}-\theta^*u_i,i\neq l$

\end{enumerate}


I det ikke-degenerede tilfælde siger følgende sætning at simplex metoden fungerer og stopper efter et endeligt antal iterationer.
% Sætning 3.3
\begin{thm}{}{}
Antag at løsningsmængden er ikke-tom og at enhver basal mulig løsning er ikke-degeneret.
Så teminerer simplex metoden efter et endeligt antal iterationer. 
Ved termineringen er der to muligheder:
\begin{enumerate}[label = (\alph*)]
\item Et optimalt basis $\textbf{B}$ er fundet og den tilsvarende basale mulige løsning er optimalt.
\item Der findes en vektor $\textbf{d}$ der opfylder $\textbf{Ad}=0,\textbf{d}\geq 0,$ og $\textbf{c}^T\textbf{d}<0$ og den optimale værdi er $-\infty$
\end{enumerate}
\end{thm}
%
\begin{proof}
%jeg ville ikke lige til at pille i det andet dokument hvis andre skrev i det pr sætning 6.1(3.1 i bertsima)
Hvis algoritmen stopper i trin 2, så følger det af Sætning 6.1(mangler en ref) at $\textbf{B}$ er et optimalt basis og den nuværende basale mulige løsninger er optimal.
\\
Hvis algoritmen stopper på grund af stop kriteriet i trin 3, så er den nuværende basale mulige løsning $\textbf{x}$ og der findes en ikke-basal variabel $x_j$ sådan at $c^*_j<0$ og den tilsvarende basale retning $\textbf{d}$ tilfredsstiller $\textbf{Ad}=\textbf{0}$ og $\textbf{d} \geq \textbf{0}$. Dertil, at $x+\theta \textbf{d}\in P$ for alle $\theta>0$. Eftersom $\textbf{c}^T\textbf{d}=c_j^*<0$, ved at tage et arbitært stort $\theta$, fåes et arbitært negativ objektfunktion, og den optimale objektsværdi er $-\infty$
\\
Ved hver iteration bevæger algoritmen sig en positiv værdi $\theta*$ langs en retning $\textbf{d}$ der ofylder $\textbf{c}^T\textbf{d}<0$. Som følge af dette forbedres objektfunktionens værdi for hver basal mulig løsning simplex algoritmen undersøger, samt at den samme løsning ikke besøges mere end en gang. Eftersom der er et endeligt antal basale mulige løsninger, terminerer algoritmen efter et endeligt antal iterationer.
\end{proof}

%

%

