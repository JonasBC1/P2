\section{Optimale betingelser}
\label{julieerfantalastiskogvidunderlig}
% --------------------------------------------------------------
%
For at finde den optimale løsning kan der udvælges en basal mulig løsning og undersøges, hvorvidt der findes andre løsninger, som er bedre, end den valgte. 
Hvis der ikke findes bedre løsninger, er den valgte basale mulig løsning den optimale løsning. 
Eftersom der jævnfør afsnit \ref{julieerlakker} optimeres for en konveks funktion over en konveks mængde, %Det er det med at et lokal minimum er et globalt minimum xD
vil den lokale optimale løsning være den globale optimale løsning. 
%
%Når der undersøges om der findes et nabo punkt, hvis løsning er bedre, så er det ikke nødvendigt at undersøge i retningen, som leder ud af den basal mulige område.
%
% Definition 3.1 i bert
\begin{defn}{}{}
Lad $\textbf{v}$ være et element af et polyeder $\mathcal{P}$.
En vektor $\textbf{d}\in \R^n$ siges at være en \textbf{mulig retning} fra $\textbf{v}$, hvis der eksisterer en positiv skalar $\theta$, således at $\textbf{v}+\theta \textbf{d}\in \mathcal{P}$.
\end{defn}
\noindent
%
På figur \ref{fig:julieersmuuuuuk}, ses vektorerne $\mathbf{u}, \mathbf{v}$ og $\mathbf{w}$ i polyederet $\mathcal{P}$, samt tilhørende forskellige mulige retninger, som stadig er indeholdt i $\mathcal{P}$.
%
\input{fig/tikz/simplex/mulig_retning}
%
%Hjørnerne afsøges derved gennem en reduceret objektfunktion.
%Se \ref{defn:kogtskinke}.
%
%\textcolor{red}{
For at finde en ny mulig retning $\mathbf{d}$ bevæges der i en retning, hvor værdien af en ikke-basal variable $x_j$ øges, og andre ikke-basale variable fastholdes. 
Dermed vil indgangen $d_j = 1$, og $d_i = 0$ for henholdsvis den ikke-basale variable $x_j$ og de andre ikke-basale variable $x_i$, hvor $i \neq j$.
Vektoren 
\begin{align*}
\mathbf{d}_B = \left[ d_{B(1)}, d_{B(2)}, \ldots , d_{B(m)} \right]^T,
\end{align*} 
bestående af de mulige retninger for de basale index, 
kontureres.
Da $\mathbf{v}$ er en basal mulig løsning, gælder det, at $A \mathbf{v} = \mathbf{b}$. 
Eftersom $\mathbf{v}+ \theta \mathbf{d}$ skal være en mulig løsning, gælder det, at $A ( \mathbf{v}+ \theta \mathbf{d}) = \mathbf{b}$.
Deraf haves, at 
\begin{align*}
A ( \mathbf{v}+ \theta \mathbf{d}) &=  A \mathbf{v} + \theta A \mathbf{d} = \mathbf{b} \\
& \Updownarrow \\
\mathbf{b} + \theta A \mathbf{d} & = \mathbf{b} \\
& \Updownarrow \\
\theta A \mathbf{d} & = \mathbf{0}.
\end{align*} 
Bemærk, at $\theta$ er en positiv skalar. 
Dermed skal $\mathbf{d}$ tilhøre $\text{nrum}(A)$.
Eftersom $d_j = 1$ og $d_i = 0$, så er 
%
\begin{align*}
\mathbf{0} = A \mathbf{d} & = \sum^n_{i = 1} \mathbf{A}_i d_i \\
& =  \mathbf{A}_j  + \sum^m_{i = 1} \mathbf{A}_{B(i)} d_{B(i)} \\
& =  \mathbf{A}_j  + B \mathbf{d}_B.
\end{align*}
%
Da basismatricen $B$ er invertibel, haves at
% 
\begin{align*}
\mathbf{A}_j  + B \mathbf{d}_B & = 0 \\
& \Updownarrow \\
B \mathbf{d}_B & = - \mathbf{A}_j \\
& \Updownarrow \\
\mathbf{d}_B & = - B^{-1} \mathbf{A}_j .
\end{align*}
% 
Denne mulige retning $\textbf{d}$, tilhørende $v_j$, kaldes den \textit{$j$'te basale retning}.
% 
\begin{defn}{}{}
Lad $\mathbf{v}$ være en basal mulig løsning. 
Den \textbf{$\textbf{j}$'te basale retning} $\mathbf{d}$, fra $\mathbf{v}$, for en ikke-basal variabel $v_j$, har komponenterne 
%
\begin{align*}
d_j & = 1, \\
d_i & = 0 \phantom{hej} \text{ for } i \neq B(1), \ldots , B(m) \wedge i \neq j, \\
\mathbf{d}_B & = - B^{-1} \mathbf{A}_j.
\end{align*}
% 
\end{defn}
\noindent
%
Bemærk, at overstående sikrer lighedsbegrænsningerne, men det sikrer ikke ikke-negativitets-betingelserne.
Eftersom $v_j$ øges, og de andre ikke-basale variable fastholdes, kan følgende scenarier opstå.
%
\begin{itemize}
\item Antag, at $\mathbf{v}$ er en ikke-degenereret basal mulig løsning. 
Dermed er $\mathbf{v}_B > \mathbf{0}$. 
Hvis $\theta$ er småt nok, vil $\mathbf{v}_B + \theta \mathbf{d}_B \geq 0 $, og $\mathbf{d}$ er dermed en mulig retning. 
\item Antag, at $\mathbf{v}$ er en degenereret basal mulig løsning. 
Dermed er $\mathbf{d}$ ikke altid en mulig retning. 
Eftersom $\mathbf{d}$ er degenereret, er det muligt, at en basal variabel $v_{B(i)}=0$, og den tilsvarende komponent $d_{B(i)}$ er negativ. 
I dette tilfælde vil en bevægelse i den $j$'te mulige retning forlade polyederet, da en ikke-negativitetsbetingelse brydes af $\mathbf{v} + \theta \mathbf{d}$ for alle positive $\theta$.
\end{itemize}
%
På figur \ref{fig:julieerengud} ses de to scenarier. 
Bemærk, at $\textbf{u}$ har de to ikke-basale variable $x_1$ og $x_3$ imens $x_2,$ $x_4$ og $x_5$ er basale variable.
I hjørnet $\textbf{u}$ fastholdes den ikke-basale variable $x_3$, og $x_1$ øges. 
Dermed traverseres der mod $\textbf{v}$ langs linjen mellem $\textbf{u}$ og $\textbf{v}$. 
Bemærk, at $\textbf{v}$ har de to ikke-basale variable $x_3$ og $x_5$ og $x_1,$ $x_1$ og $x_4$ er basale variable.
I hjørnet $\textbf{v}$ fastholdes den ikke-basale variabel $x_5$, og $x_3$ øges. 
Dermed traverseres der langs linjen $\textbf{v}$ og $\textbf{w}$, hvilket bryder en ikke-negativitetsbetingelse, og dermed ikke en mulig retning. 

%
\begin{center}
%
\begin{tikzpicture}[scale=5]
%
% Koordinater
% -------------------------------------------------------
\coordinate (a) at (0,0,0); 
\coordinate (b) at (0.7,0,0); 
\coordinate (c) at (0.393,0.55,0); 
\coordinate (d) at (0.252,0.805,0);
\coordinate (e) at (0.066,0.434,0);
\coordinate (f) at (0,0.41,0);
\coordinate (g) at (0,0.305,0);
\coordinate (h) at (-0.15,0,0); 
%
% Farvning
% -------------------------------------------------------
\filldraw[fill=myblue,opacity=0.3](a)--(b)--(c)--(f)--(a);
  %\draw[thick](-0.2,-0.1,0)--(0.3,0.9,0); % n -> d  
  \draw[thick](0.2,0.9,0)-- node[anchor=north east] {$x_4=0$} (0.757,-0.1,0); % d -> b
  \draw[thick](-0.3,0.3,0)-- node[above] {$x_3=0$ \phantom{hej}} (0.8,0.7,0); % f -> c
  \draw[thick](0.12,0.9,0)-- node[anchor=south west] {$x_5=0$} (0.9,-0.1,0); % ekstra
% 
%
% Punkterne 
% -------------------------------------------------------
%\filldraw [black] (a) circle (0.2pt) node[anchor=north west] {$A$};
%\filldraw [black] (b) circle (0.2pt) node[anchor=north east] {$B$};
\filldraw [black] (c) circle (0.2pt) node[anchor=south west] {$\textbf{v}$};
\filldraw [black] (0.822,0,0) circle (0.2pt) node[anchor=south west] {$\textbf{w}$};
%\filldraw [black] (d) circle (0.2pt) node[anchor=west] {$D$};
%\filldraw [black] (e) circle (0.2pt) node[anchor=north west] {$E$};
\filldraw [black] (f) circle (0.2pt) node[anchor=south east] {$\textbf{u}$};
%\filldraw [black] (g) circle (0.2pt) node[anchor=east] {$G$};
%\filldraw [black] (h) circle (0.2pt) node[anchor=north west] {$H$};
%
% 
\filldraw [black] (0.27,0.17,0) circle (0pt) node[above] {$\mathcal{P}$};
% 
% Koordinatssystem 
% -------------------------------------------------------
\draw[thick] (0,0,0) -- node[anchor=north east] {$x_2=0$} (0.9,0,0);
\draw[thick] (0,0,0) -- (-0.2,0,0);
\draw[thick] (0,0,0) -- node[anchor=north east] {$x_1=0$} (0,0.5,0);
\draw[thick] (0,0.5,0) -- (0,0.9,0);
\draw[thick] (0,0,0) -- (0,-0.2,0);
%
\end{tikzpicture}
  \captionof{figure}{Illustration af en mulig retning, der svarer til linjen mellem $\textbf{u}$ og $\textbf{v}$, og en ikke-mulig retning, der svarer til linjen mellem $\textbf{v}$ og $\textbf{w}$, som bryder en ikke-negativitetsbetingelse.}
  \label{fig:julieerengud}
\end{center}
%
Heraf opstår et problem med degenererede løsninger, og af den grund forsøges disse undgået.
\\\\
%
Ved en bevægelse i en mulig retning medfører det, at 
\begin{align*}
\mathbf{c}^T \mathbf{d} = \mathbf{c}^T_B \mathbf{d}_B + c_j,
\end{align*}
hvor 
$$\mathbf{c}_B=[ c_{B(1)},c_{B(2)}, \cdots , c_{B(m)} ]^T.$$
Bemærk, at da $\mathbf{d}_B = - B^{-1} \mathbf{a}_j$, er 
\begin{align*}
\mathbf{c}^T \mathbf{d} &= \mathbf{c}^T_B ( - B^{-1} \mathbf{a}_j ) + c_j \\
& = c_j - \mathbf{c}^T_B (B^{-1} \mathbf{a}_j).
\end{align*}
%
For, at omkostningen minimeres, skal $c_j - \mathbf{c}^T_B (B^{-1} \mathbf{a}_j ) < 0$, da 
% 
\begin{align*}
\mathbf{c}^T  ( \mathbf{x} + \theta \mathbf{d} )  = \mathbf{c}^T  \mathbf{x} + \theta \mathbf{c}^T \mathbf{d} = \mathbf{c}^T  \mathbf{x} + \theta ( c_j - \mathbf{c}^T_B (B^{-1} \mathbf{a}_j ) ) < \mathbf{c}^T \mathbf{x},
\end{align*}
%
for alle positive $\theta$.
Størrelsen for reduktionen for hvert $\theta$ kaldes \textit{den reducerede omkostning}.
%
% Definition 3.2 
\begin{defn}{}{kogtskinke}
Lad $\mathbf{v}$ være en basal løsning til en basismatrix $B$ med den tilhørende omkostningsvektor $$\mathbf{c}_B=[ c_{B(1)},c_{B(2)}, \cdots , c_{B(m)} ]^T.$$
For hver $j$'te basale retning defineres den \textbf{reducerede omkostning} $c_j^*$ som
\begin{align*}
c_j^* = c_j - \mathbf{c}_B^T (B^{-1} \mathbf{a}_j).
\end{align*} 
%
\end{defn}
\noindent
%
\textit{Den reducerede omkostningsvektor} $\mathbf{c}^*$ indeholder alle de reducerede omkostninger, hvor den $k$'te række betegner den reducerede omkostning for den $k$'te basale retning.
%
\begin{align*}
\mathbf{c}^* = (\mathbf{c}^T - \mathbf{c}_B^T B^{-1} A)^T = 
\begin{blockarray}{[c]}
c_1 - \mathbf{c}_B^T B^{-1} \mathbf{a}_1 \\
\vdots \\
c_k - \mathbf{c}_B^T B^{-1} \mathbf{a}_k \\
\vdots \\
c_n - \mathbf{c}_B^T B^{-1} \mathbf{a}_n
\end{blockarray}.
\end{align*}
%
Betragt de indgange, for de basale variable i den reducerede omkostningsvektor
%
\begin{align*}
c^*_{B(j)} & = c_{B(j)} - \mathbf{c}_B^T B^{-1} \mathbf{a}_{B(j)} \\
& = c_{B(j)} - \mathbf{c}_B^T \mathbf{e_j} \\
& = c_{B(j)} -  c_{B(j)}  \\
& = 0.
\end{align*}
%
Dermed er indgangene til de basale variable lig nul, da de mulige retninger er nulvektoren. 
Det kan derfor ikke betale sig at øge basale variable.
\\\\
Hvis den reducerede omkostningsvektor $\mathbf{c}^* \geq \mathbf{0}$, er den basale mulige løsning et lokalt minimum i objektfunktionen og dermed et globalt minimum. Heraf er den optimale løsning fundet. Dette tilfælde belyses i sætning \ref{thm:julieerguden}.
%
\begin{thm}{}{julieerguden}
Lad $\mathbf{v}$ være en basal mulig løsning til en basismatrix $B$, og lad $\mathbf{c}^*$ være den reducerede omkostningsvektor. 
\begin{enumerate}[label = (\alph*)]
\item Hvis $\mathbf{c}^* \geq \textbf{0}$, så er $\mathbf{v}$ optimal.
\item Hvis $\mathbf{v}$ er optimal og ikke-degenereret, så er $\mathbf{c}^* \geq 0$.
\end{enumerate}
\end{thm}
\noindent
%
Beviset udelades.
Se \citep[Side 86]{bert}.\\\\
%\begin{enumerate}[label = (\alph*)]
%\item Antag, at $\bar{\textbf{c}} \geq \textbf{0}$. Lad $\textbf{y}$ være en arbitrær mulig løsning, og $\textbf{d} = \textbf{y} - \textbf{x}$. Grundet de mulige løsninger, er $\textbf{A} \textbf{x} = \textbf{A} \textbf{y} = \textbf{b}$, hvormed $\textbf{A} \textbf{d} = \textbf{0}$. Sidstnævnte kan skrives som 
%$$\textbf{B} \textbf{d}_B + \sum_{i \in N} \textbf{A}_i d_i = \textbf{0},$$
%hvor $N$ er mængden af indekser, der svarer til de ikke-basale variabler i basen. %wtf fatter slet ikke det her lol
%$\textbf{B}$ er invertibel, hvormed det haves, at 
%$$\textbf{d}_B = -\sum_{i \in N} \textbf{B}^{-1} \textbf{A}_i d_i ,$$
%og
%$$\textbf{c}' \textbf{d} = \textbf{c}'_B \textbf{d}_B + \sum_{i \in N} c_i d_i = \sum_{i \in N} (c_i - \textbf{c}'_B \textbf{B}^{-1} \textbf{A}_i) d_i = \sum_{i \in N} \bar{c}_i d_i .$$
%%nanananana fatter intet, hvorfor har vi dette med? 
%For ethvert ikke-basalt indeks $i \in N$ må der eksistere et $x_i = 0$, samt et $y_i \geq 0$, da $\textbf{y}$ er en mulig løsning. 
%Deraf haves, at $d_i \geq 0$ og $\bar{c}_i d_i \geq 0$ for alle $i \in N$. 
%Dermed er $\textbf{c}' (\textbf{y} - \textbf{x} = \textbf{c}' \textbf{d} \geq 0$, og $\textbf{x}$ er en optimal løsning, eftersom $\textbf{y}$ er en arbitrær mulig løsning. 
%%
%%
%\item Antag, at $x$ er en ikke-degenereret basal mulig løsning, og at $\bar{c}_j < 0$ for et $j$. Eftersom den reducerede objektfunktion for en basal variabel altid er $0$, må $x_j$ være en ikke-basal variabel, og $\bar{c}_j$ må være ændringshastigheden for objektfunktionen i den $j$'te basale retning. 
%%Wtf er ovenstående for noget lort
%Da $\textbf{x}$ ikke er degenereret, er den $j$'te basale retning en mulig retning for mindskelse af objektfunktionen. 
%Der findes dermed mulige løsninger, hvis objektfunktion har lavere værdi end $\textbf{x}$'s, ved at bevæge sig i denne retning, hvormed $\textbf{x}$ ikke er optimal. 
%\end{enumerate}
%
Dermed er alle basale retninger, for en basal mulig løsning, mulige retninger, hvis problemet er givet uden degenererede løsninger.
For ikke-degenererede problemer bevæger simplexmetoden sig dermed fra et ekstremumspunkt til et af dets tilstødende ekstremumspunkter og forbedrer omkostningen.
Heraf defineres en \textit{optimal basismatrix}, for at have optimale betingelser i forbindelse med simplexmetoden.
%
% Definition 3.3 
\begin{defn}{}{julieergudenoveralleguder}
En basismatrix $B$ kaldes \textbf{optimal}, hvis
%
\begin{enumerate}[label = (\alph*)]
\item $B^{-1} \mathbf{b} \geq \mathbf{0}$, og
\item $\mathbf{c}^* \geq \mathbf{0}$.
\end{enumerate}
%
\end{defn}