\subsection{Lineær uafhængighed}
% 1.7
\textbf{Raaaasmusss hjælp mig xD}
% 
\begin{defn}{Lineær afhængighed og lineær uafhængighed}{}
Et sæt af vektorerne $\mathbf{v}_1, \ldots , \mathbf{v}_k$ i $\R^n$ kaldes \textbf{lineært uafhængige}, hvis der eksistere skalare $x_1,x_2, \ldots , x_l$, sådan at ligningen 
\begin{align*}
x_1\mathbf{v}_1 + \ldots + x_k \mathbf{v}_k = \mathbf{0}, 
\end{align*}
kun har løsningen $c_1 = \ldots = c_k = 0$.
Ellers kaldes vektorerne \textbf{lineært afhængige}.
\end{defn}
%
%%%%%%%%%%%%%%%%%%%%%%%%%%%%%%%%%%%%%%%%%%%%%%%%%
% Eksempel
\begin{eks}

\end{eks}
% 
tekst 
%
\begin{thm}{}{}
%
Lad $A$ være en $m \times n$ matrix.
Følgende udsagn er ækvivalente:
%
\begin{enumerate}[label=(\alph*)]
\item $A$'s søjler er lineært uafhængige. 
\item Ligningssystemet $A\mathbf{x}=\mathbf{b}$ har højst en løsning $\forall \mathbf{b} \in \R^m$.
\item $\text{null}(A)=0$.
\item $\text{rang}(A)=n$.
\item Søjlerne i $A$ på reduceret trappeform er standardvektoren i $\R^m$.
\item $A\mathbf{x}=\mathbf{0}$ har kun løsningen $\mathbf{x}=\mathbf{0}$.
\item Der er pivot-indgange i hver søjle i $A$. 
\end{enumerate}
%
\end{thm}
%
\begin{proof}
%
Lad $\mathbf{b}=\mathbf{0}$. 
Ligningssystemet $A\mathbf{x}=\mathbf{b}$ har altid løsningen $\mathbf{x}=\mathbf{0}$, der derfor er den eneste løsning til $A\mathbf{x}=\mathbf{0}$, hvilket beviser (b) og (e).
\\
Lad $\mathbf{u}, \mathbf{v}$ være løsninger til $A\mathbf{x}=\mathbf{b}$.
% Sætning 1.3
Jævnfør sætning \ref{fisk}, følger det at $A(\mathbf{u}-\mathbf{v})=A\mathbf{u}-A\mathbf{v}=\mathbf{b}-\mathbf{b}=\mathbf{0}$, hvilket tilsvarer $\mathbf{u}-\mathbf{v}=\mathbf{0}
\rightarrow  
\mathbf{u} =\mathbf{v}.$
Mangler resten 
%
\end{proof}
\\
%
tekst
%
\begin{thm}{}{}
%
Vektorerne $\mathbf{u}_1,\mathbf{u}_2, \ldots ,\mathbf{u}_k$ i $\R^n$ er lineært afhængige, 
hvis og kun hvis, $\mathbf{u}_1=\mathbf{0}$, eller hvis der findes $i \geq 2$ således $\mathbf{u}_i \in \text{span} \{ \mathbf{u}_1,\mathbf{u}_2, \ldots ,\mathbf{u}_{i-1} \}$.
%
\end{thm}
%
\begin{proof}
%
Lad $\mathbf{b}=\mathbf{0}$. 
Ligningssystemet $A\mathbf{x}=\mathbf{b}$ har altid løsningen $\mathbf{x}=\mathbf{0}$, der derfor er den eneste løsning til $A\mathbf{x}=\mathbf{0}$, hvilket beviser (b) og (e).
\\
Lad $\mathbf{u}, \mathbf{v}$ være løsninger til $A\mathbf{x}=\mathbf{b}$.
% Sætning 1.3
Jævnfør sætning \ref{fisk}, følger det at $A(\mathbf{u}-\mathbf{v})=A\mathbf{u}-A\mathbf{v}=\mathbf{b}-\mathbf{b}=\mathbf{0}$, hvilket tilsvarer $\mathbf{u}-\mathbf{v}=\mathbf{0}
\rightarrow  
\mathbf{u} =\mathbf{v}.$
Mangler resten 
%
\end{proof}
\\
%
tekst