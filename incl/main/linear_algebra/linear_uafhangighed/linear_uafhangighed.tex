\subsection{Lineær uafhængighed}
% 1.7
I forbindelse med en mængde af vektorer kan de være \textbf{lineært afhængige} eller \textbf{lineært uafhængige}.
%
%En mængde af vektorer er \textbf{lineært afhængig}, hvis alle vektorerne kan skrives som en linearkombination af de andre vektorer. 
%Hvis ikke alle vektorerne kan skrives som en linearkombination af de andre vektorer, så er mængden \textbf{lineært uafhængige}. 
%Det kan derfor siges, at hvis mindst en af vektorerne afhænger af de andre vektorer, er samlingen lineært afhængig, og dette leder frem til følgende definition. 
%
%%%%%%%%%%%%%%%%%%%%%%%%%%%%%%%%%%%%%%%%%%%%%%%%%%%%%%%%%%%%%%%%%%%%%%%%%
% 
\begin{defn}{Lineær afhængighed og lineær uafhængighed}{}
Et sæt af vektorerne $\mathbf{v}_1, \ldots , \mathbf{v}_k$ i $\R^n$ kaldes lineært uafhængige, hvis der eksisterer skalarer $x_1,x_2, \ldots , x_l$, sådan at ligningen 
\begin{align*}
x_1\mathbf{v}_1 + \ldots + x_k \mathbf{v}_k = \mathbf{0}, 
\end{align*}
kun har den trivielle løsningen $c_1 = \ldots = c_k = 0$.
Ellers kaldes vektorerne lineært afhængige.
\end{defn}  
%
%%%%%%%%%%%%%%%%%%%%%%%%%%%%%%%%%%%%%%%%%%%%%%%%%
% Eksempel
\begin{eks}\label{fisk}
Givet
\begin{align*}
\text{span}\{\S_1 \} =
\left\{ 
\begin{bmatrix}
           2 \\
           1 \\
\end{bmatrix} 
,
\begin{bmatrix}
           1 \\
           3 \\
\end{bmatrix}
\right\},
\end{align*}
opstilles den trivielle løsning  
%
\begin{align*}
x_1 
\begin{bmatrix}
           2 \\
           1 \\
\end{bmatrix} 
+ x_2
\begin{bmatrix}
           1 \\
           3 \\
\end{bmatrix}
= \mathbf{0}.
\end{align*}
%
Denne er kun opfyldt, hvis $x_1=x_2=0$. 
Eftersom alle koefficienterne er nul er spannet af $\S_1$ lineært uafhængigt.
\\\\
Havde spannet været 
\begin{align*}
\text{span}\{\S_2 \} =
\left\{ 
\begin{bmatrix}
           2 \\
           1 \\
\end{bmatrix} 
,
\begin{bmatrix}
           4 \\
           2 \\
\end{bmatrix}
\right\},
\end{align*}
%
kunne den trivielle løsning 
%
\begin{align*}
x_1 
\begin{bmatrix}
           2 \\
           1 \\
\end{bmatrix} 
+ x_2
\begin{bmatrix}
           4 \\
           2 \\
\end{bmatrix}
= \mathbf{0},
\end{align*}
%
være opfyldt, hvis $x_1=x_2=0$ eller $x_1=-2$ og $x_2=1$, og spannet er derfor lineært afhængig, fordi der er flere løsninger end den trivielle.
%
\end{eks}
%
%%%%%%%%%%%%%%%%%%%%%%%%%%%%%%%%%%%%%%%%%%%%%%%%%%%%%%%%%%%%%%%%%%%%%%%%%
%
%I eksempel \ref{lineu} viste en af måderne hvorpå man kunne bestemme om et bestemt set $\S$ var lineært afhængigt eller uafhængigt. 
%Der findes flere måder at bestemme dette på. Denne følgende sætning viser en lignende teknik til bestemmelse heraf. 
%
Egenskaberne ved mængder, der er lineært uafhængighed kan udtrykkes på forskellige måder, hvilket leder til følgende sætning:
%
%%%%%%%%%%%%%%%%%%%%%%%%%%%%%%%%%%%%%%%%%%%%%%%%%%%%%%%%%%%%%%%%%%%%%%%%%
%
\begin{thm}{}{}
%
Lad $A$ være en $m \times n$ matrix.
Følgende udsagn er ækvivalente:
%
\begin{enumerate}[label=(\alph*)]
\item $A$'s søjler er lineært uafhængige. 
\item Ligningssystemet $A\mathbf{x}=\mathbf{b}$ har højst en løsning $\forall \mathbf{b} \in \R^m$.
\item $\text{null}(A)=0$.
\item $\text{rang}(A)=n$.
\item Søjlerne i $A$ på reduceret trappeform er standardvektoren i $\R^m$.
\item $A\mathbf{x}=\mathbf{0}$ har kun løsningen $\mathbf{x}=\mathbf{0}$.                                                                                                                                                                                                                                                                                                                         
\item Der er pivot-indgange i hver søjle i $A$. 
\end{enumerate}
%
\end{thm}
%
%%%%%%%%%%%%%%%%%%%%%%%%%%%%%%%%%%%%%%%%%%%%%%%%%%%%%%%%%%%%%%%%%%%%%%%%%
%
\begin{proof}
\noindent
Det er forudsagt, at (a) og (f), samt (f) og (g) er ækvivalente, så for at færdiggøre beviset for denne sætning, skal det vises, at (b) $\rightarrow$ (c), (c) $\rightarrow$ (d), (d) $\rightarrow$ (e), (e) $\rightarrow$ (f) og (f) $\rightarrow$ (b).
\\\\
%
% (b) $\rightarrow$ (c). 
Idet, at $\mathbf{0}$ er en løsning til $A\mathbf{x}=\mathbf{b}$, så forudsætter (b) at $A\mathbf{x}=\mathbf{b}$ har ingen ikke-nul løsninger, da den generelle løsning ingen frie variable, har. 
Da antallet af frie variable er nul, og dermed at nulliteten i A er nul, er (b) $\rightarrow$ (c) hermed bevist. 
\\\\
%
% (c) $\rightarrow$ (d).
Idet, at $\text{rang}(A)+\text{null}(A)=n$, så medfører det, at (c) $\rightarrow$ (d) er bevist.
\\\\
%
% (d) $\rightarrow$ (e).
Hvis $\text{rang}(A)=n$, så betyder det, at enhver søjle i A har en pivotindgang i søjlen, hvilket betyder, at den reducerede trappeform indeholder udelukkende standard vektorer. 
Disse standardvektorer er nødvendigvis forskellige, da hver søjle indeholder den første ikke-nul indgang i hver række. 
Dette beviser (d) $\rightarrow$ (e).
%
\\\\
% (e) $\rightarrow$ (f).
Lad $R$ være den reducerede trappeform af matricen $A$. Hvis søjlerne i $R$ er forskellige standardvektorer i $R^m$, så er $R= [ \mathbf{e}_1, \mathbf{e}_2, \ldots, \mathbf{e}_n]$. 
Det er givet, at den eneste løsning til $R\mathbf{x}=\mathbf{0}$, er nul, og idet at $A\mathbf{x}=\mathbf{0}$, er ækvivalent til $R\mathbf{x}=\mathbf{0}$, så er den $\mathbf{0}$ eneste løsning til A$\mathbf{x}=\mathbf{0}$. 
Dette beviser (e) $\rightarrow$ (f).
\\\\
%
% (f) $\rightarrow$ (b).
Lad $\mathbf{u}, \mathbf{v}$ være løsninger til $A\mathbf{x}=\mathbf{b}$.
% Sætning 1.3
Jævnfør sætning \ref{thm:mxvpro}, følger det at $A(\mathbf{u}-\mathbf{v})=A\mathbf{u}-A\mathbf{v}=\mathbf{b}-\mathbf{b}=\mathbf{0}$, hvilket tilsvarer $\mathbf{u}-\mathbf{v}=\mathbf{0}
\rightarrow  
\mathbf{u} =\mathbf{v}$, hvilket beviser (f) $\rightarrow$ (b).
\\\\
%
Lad $\mathbf{b}=\mathbf{0}$. 
Ligningssystemet $A\mathbf{x}=\mathbf{b}$ har altid løsningen $\mathbf{x}=\mathbf{0}$, der derfor er den eneste løsning til $A\mathbf{x}=\mathbf{0}$, hvilket beviser (b) $\rightarrow$ (e).
%
\end{proof}
\\
%
%%%%%%%%%%%%%%%%%%%%%%%%%%%%%%%%%%%%%%%%%%%%%%%%%%%%%%%%%%%%%%%%%%%%%%%%%
%
Ligeså findes der en række egenskaber vedrørende mængder der er lineært afhængige, som er belyst i sætning \ref{thm:inspektion}.
%
%%%%%%%%%%%%%%%%%%%%%%%%%%%%%%%%%%%%%%%%%%%%%%%%%%%%%%%%%%%%%%%%%%%%%%%%%
%
\begin{thm}{}{inspektion}
%
Vektorerne $\mathbf{u}_1,\mathbf{u}_2, \ldots ,\mathbf{u}_k$ i $\R^n$ er lineært afhængige, 
hvis og kun hvis, $\mathbf{u}_1=\mathbf{0}$, hvor $\mathbf{u}_1$ er en vilkårlig vektor, 
eller hvis der findes $i \geq 2$ således $\mathbf{u}_i \in \text{span} \{ \mathbf{u}_1,\mathbf{u}_2, \ldots ,\mathbf{u}_{i-1} \}$.
%
\end{thm}
%
%%%%%%%%%%%%%%%%%%%%%%%%%%%%%%%%%%%%%%%%%%%%%%%%%%%%%%%%%%%%%%%%%%%%%%%%%
%
\begin{proof}
%
Lad $\mathbf{u}_1=\mathbf{0}$, så er sætningen bevist. 
Antag nu, at $\mathbf{u}_1 \neq \mathbf{0}$. 
Der findes skalarer, hvor koefficienterne ikke alle er nul, således at
% 
\begin{align*}
c_1 \mathbf{u}_1 + c_2 \mathbf{u}_2 + \ldots + c_k \mathbf{u}_k = \mathbf{0}.
\end{align*}
%
Lad $i$ være det største index $i$, hvor $x_i \neq 0$, sådan at 
\begin{align*}
c_1 \mathbf{u}_1 + c_2 \mathbf{u}_2 + \ldots + c_i \mathbf{u}_i = \mathbf{0}.
\end{align*}
Løses denne i forhold til $\mathbf{u}_i$, ses det at
\begin{align*}
\mathbf{u}_i = - \frac{c_1}{c_i} \mathbf{u}_1 - \frac{c_2}{c_i} \mathbf{u}_2 - \ldots - \frac{c_{p-1}}{c_i} \mathbf{u}_{i-1}.
\end{align*}
Hvilket vil sige at 
\begin{align*}
\mathbf{u}_i \in \text{span}\{ \mathbf{u}_1,\mathbf{u}_2, \ldots ,\mathbf{u}_k \},
\end{align*}
%
og sætningen er hermed bevist.
%
\\\\
Antag nu at $\ldots$ MAAAAAAAAAAAAAAAAAAAAAAAAAAAAAAAAAAAAAAAAAAAAAAAADS
\textcolor{red}{Mangler beviset modsat (da det er en hvis og kun hvis)}
%
\end{proof}
\\
%
%%%%%%%%%%%%%%%%%%%%%%%%%%%%%%%%%%%%%%%%%%%%%%%%%%%%%%%%%%%%%%%%%%%%%%%%%
%
Man kan derfor i nogle tilfælde afgøre ved inspektion, hvorvidt en mængde vektorer $\mathbf{v},\mathbf{v}_2,\ldots ,\mathbf{v}_k$ er lineær afhængige. 
Jævnfør sætning \ref{thm:inspektion}, er mængden af vektorer afhængige hvis $\mathbf{0}$ er en af dem, hvis 2 af vektorerne er parallelle eller hvis der er flere vektorer end deres antal af koordinater. 
Med udgangspunkt i eksempel \ref{fisk}, ses det at de to vektorer er paralelle, og de er derfor lineært afhængige. 
\\\\
Hvis man ikke ved hjælp af inspektion kan afgøre hvorvidt en matrix er lineært afhængig eller lineært uafhængig kan det undersøges det ved at opstilles totalmatricen af mængden af vektorer, hvor det derefter med rækkeopperationer laves på trappeform. 
Hvis der er pivot-indgange i alle søjlerne er mængden af vektorer lineært uafhængige, hvorimod hvis der er frie variable er mængden af vektorer lineært afhængige.
\\\\
%
% Eksempel
\begin{eks}\label{lineu}
Lad en mængde af vektorer være
\begin{align*}
\S &= \left\{
\begin{bmatrix}
           4 \\
           1 \\
\end{bmatrix}
,
\begin{bmatrix}
           -3 \\
           0 \\
\end{bmatrix}
,
\begin{bmatrix}
           1 \\
           -2 \\
\end{bmatrix}
\right\}.
\end{align*}
\noindent
% Skal skrives om fra et spørgsmål 
Afgør om vektorerne er lineært uafhængige, altså hvis $x_1$ og $x_2 = 0$, er den eneste løsning til linearkombination.
\begin{align*}
x_1+
\begin{bmatrix}
           4 \\
           1 \\
\end{bmatrix}
+ x_2
\begin{bmatrix}
           -3 \\
           0 \\
\end{bmatrix}
+ x_3
\begin{bmatrix}
           1 \\
           -2 \\
\end{bmatrix}
=0 
\end{align*}
%
Totalmatricen opkrives 
\noindent
\begin{align*}
A=
\begin{blockarray}{cccc}
x_1 & x_2 & x_3 & b \\
\begin{block}{[ccc|c]}
4 & -3 & 1 & 0\\
1 & 0 & -2 & 0 \\
\end{block}
\end{blockarray}
\xrightarrow[R_2 \rightarrow R_2+(-1R_1)]{R_1 \rightarrow \frac{1}{4}R_1} 
\begin{blockarray}{cccc}
x_1 & x_2 & x_3 & b \\ 
\begin{block}{[ccc|c]}
1 & \frac{-3}{4} & \frac{1}{4} & 0\\
0 & \frac{3}{4} & \frac{-9}{4} & 0 \\
\end{block}
\end{blockarray}
\xrightarrow[R_1 \rightarrow R_1+(\frac{3}{4} R_2)]{R_2 \rightarrow \frac{4}{3} \times R_2} 
\begin{blockarray}{cccc}
x_1 & x_2 & x_3 & b \\
\begin{block}{[ccc|c]}
\hlight{1} & 1 & -2 & 0\\
0 & \hlight{1} & -3 & 0\\
\end{block}
\end{blockarray}
\end{align*}
%
Da tolalmatricen ikke har nogle frie variable, har den kun en løsning og derfor lineært uafhængig.
\\
\\
\noindent
Lad i stedet en mængde af vektorer være 
%
\begin{align*}
\S &= \left\{
\begin{bmatrix}
           4 \\
           1 \\
           -2 \\
\end{bmatrix}
,
\begin{bmatrix}
           -3 \\
           0 \\
           1 \\
\end{bmatrix}
,
\begin{bmatrix}
           1 \\
           -2 \\
           1 \\
\end{bmatrix}
\right\}.
\end{align*}
\noindent
Afgør om vektorerne er lineært uafhængige, altså hvis $x_1$, $x_2$ og $x_3 = 0$, er den eneste løsning til linearkombinationen.
\begin{align*}
x_1+
\begin{bmatrix}
           4 \\
           1 \\
           -2 \\
\end{bmatrix}
+ x_2
\begin{bmatrix}
          -3 \\
           0 \\
           1 \\
\end{bmatrix}
+ x_3
\begin{bmatrix}
           1 \\
           -2 \\
           1 \\
\end{bmatrix}
=0.
\end{align*}
%
Totalmatricen opskrives 
%
\begin{align*}
A=
\begin{blockarray}{cccc}
x_1 & x_2 & x_3 & b \\
\begin{block}{[ccc|c]}
4 & -3 & 1 & 0\\
1 & 0 & -2 & 0 \\
-2 & 1 & 1 & 0 \\
\end{block}
\end{blockarray}
&\xrightarrow[R_2\rightarrow R_2+(-4R_1)]{R_1 \leftrightarrow R_2}
\begin{blockarray}{cccc}
x_1 & x_2 & x_3 & b \\
\begin{block}{[ccc|c]}
1 & 0 & -2 & 0\\
0 & -3 & 9 & 0\\
-2 & 1 & 1 & 0 \\
\end{block}
\end{blockarray}
\xrightarrow[R_3 \rightarrow R_3+\frac{1}{3}R_2]{R_3 \rightarrow R_3+2R_1}
\begin{blockarray}{cccc}
x_1 & x_2 & x_3 & b \\
\begin{block}{[ccc|c]}
\hlight{1} & 0 & -2 & 0\\
0 & -3 & 9 & 0 \\
0 & 0 & 0 & 0 \\
\end{block}
\end{blockarray}
\end{align*}
\noindent
Dette betyder, at der findes en ikke-triviel lineær kombination, som giver nulvektoren. Derfor har vi minimum fundet én linearkombination som er ikke-nul og derfor er det bevist at vektorsættet lineært afhængigt.
\end{eks}
%
%%%%%%%%%%%%%%%%%%%%%%%%%%%%%%%%%%%%%%%%%%%%%%%%%%%%%%%%%%%%%%%%%%%%%%%%%
%
% SKAL MÅSKE FLYTTES
I forbindelse med et givet mængde vektorer kan færre vektorer udvælges for at få en mere effektiv beskrivelse af spannet. Dette gøres ved at tage spannet af pivot-søjlerne i koefficientmatricen.
\\\\
%
%%%%%%%%%%%%%%%%%%%%%%%%%%%%%%%%%%%%%%%%%%%%%%%%%
% Eksempel
\begin{eks}
Givet 
\begin{align*}
\S &= \left\{
\begin{bmatrix}
           1 \\
           2 \\
           0 \\
\end{bmatrix}
,
\begin{bmatrix}
           -1 \\
           2 \\
           4 \\
\end{bmatrix}
,
\begin{bmatrix}
           3 \\
           0 \\
           1 \\
\end{bmatrix}
,
\begin{bmatrix}
           1 \\
           0 \\
           5 \\
\end{bmatrix}
\right\},
\end{align*}
%
opstilles koefficientmatricen. 
%
\begin{align*}
\begin{blockarray}{ccccc}
\begin{block}{[cccc]c}
  1 & -1 & 3 & 1 \\
  2 & 2 & 0 & 0 \\
  0 & 4 & 1 & 5 \\
\end{block}
\end{blockarray} 
%
\xrightarrow{R_2 \rightarrow R_2-2R_1}
\begin{blockarray}{ccccc}
\begin{block}{[cccc]c}
  1 & -1 & 3 & 1 \\
  0 & 4 & -6 & -2 \\
  0 & 4 & 1 & 5 \\
\end{block}
\end{blockarray} 
%
\xrightarrow{R_3 \rightarrow R_3-R_2}
\begin{blockarray}{ccccc}
\begin{block}{[cccc]c}
  1 & -1 & 3 & 1 \\
  0 & 4 & -6 & -2 \\
  0 & 0 & 7 & 7 \\
\end{block}
\end{blockarray} \\
%
\xrightarrow{R_3 \rightarrow 1/7 R_3}
\begin{blockarray}{ccccc}
\begin{block}{[cccc]c}
  1 & -1 & 3 & 1 \\
  0 & 4 & -6 & -2 \\
  0 & 0 & 1 & 1 \\
\end{block}
\end{blockarray} 
%
\xrightarrow{R_2 \rightarrow R_2+6R_3}
\begin{blockarray}{ccccc}
\begin{block}{[cccc]c}
  1 & -1 & 3 & 1 \\
  0 & 4 & 0 & 4 \\
  0 & 0 & 1 & 1 \\
\end{block}
\end{blockarray} 
%
\xrightarrow{R_1 \rightarrow R_1-3R_3}
\begin{blockarray}{ccccc}
\begin{block}{[cccc]c}
  1 & -1 & 0 & -2 \\
  0 & 4 & 0 & 4 \\
  0 & 0 & 1 & 1 \\
\end{block}
\end{blockarray} \\
%
\xrightarrow{R_2 \rightarrow 1/4R_2}
\begin{blockarray}{ccccc}
\begin{block}{[cccc]c}
  1 & -1 & 0 & -2 \\
  0 & 1 & 0 & 1 \\
  0 & 0 & 1 & 1 \\
\end{block}
\end{blockarray} 
%
\xrightarrow{R_1 \rightarrow R_1+R_2}
\begin{blockarray}{ccccc}
\begin{block}{[cccc]c}
  1 & 0 & 0 & -1 \\
  0 & 1 & 0 & 1 \\
  0 & 0 & 1 & 1 \\
\end{block}
\end{blockarray}.
%
\end{align*}
%
Hermed er den sidste vektor i spannet en lineær kombination af de andre vektorer:
%
  \begin{align*}
    \begin{bmatrix}
           -1 \\
           1 \\
           1 \\
         \end{bmatrix}
         = 
         -1 \begin{bmatrix}
           1 \\
           2 \\
           0 \\
         \end{bmatrix}
         +1
         \begin{bmatrix}
           -1 \\
           2 \\
           4 \\
         \end{bmatrix}
          +1
         \begin{bmatrix}
           3 \\
           0 \\
           1 \\
         \end{bmatrix}
  \end{align*} 
%
Dermed kan spannet af $\S$ vælges til
%
\begin{align*}
\S &= \left\{
\begin{bmatrix}
           1 \\
           2 \\
           0 \\
\end{bmatrix}
,
\begin{bmatrix}
           -1 \\
           2 \\
           4 \\
\end{bmatrix}
,
\begin{bmatrix}
           3 \\
           0 \\
           1 \\
\end{bmatrix}
\right\}.
\end{align*}
\end{eks}
%
%%%%%%%%%%%%%%%%%%%%%%%%%%%%%%%%%%%%%%%%%%%%%%%%%%%%%%%%%%%%%%%%%%%%%%%%%
%
%\begin{eks}\label{lineu}
%Lad et sæt af vektorer være
%\begin{align*}
%\S &= \left\{
%\begin{bmatrix}
%           2 \\
%           1 \\
%\end{bmatrix}
%,
%\begin{bmatrix}
%           3 \\
%           2 \\
%\end{bmatrix}
%\right\}.
%\end{align*}
%\noindent
%Bestem om vektorerne er lineært uafhængige ved ligningen, altså hvis $x_1$ og $x_2 = 0$
%\begin{align*}
%x_1+
%\begin{bmatrix}
%           2 \\
%           1 \\
%\end{bmatrix}
%+ x_2
%\begin{bmatrix}
%           3 \\
%           2 \\
%\end{bmatrix}
%=0 
%\end{align*}
%%
%Ligningerne udskrives
%\noindent
%\begin{align*}
%2x_1+3x_2=0\\
%x_1+2x_2=0\\
%\\
%x_1+3/2x_2=0\\
%x_1+2x_2=0\\
%\\
%x_1+2x_2=0 - x_1+3/2x_2=0\\
%1/2c_2=0\\
%c_2=0\\
%c_1=0
%\end{align*}
%%
%Da $c_1$ og $c_2 = 0$, så er sættet $\S$ lineært uafhængigt.
%\\
%\\
%\noindent
%Lad et sæt af vektorer være 
%%
%\begin{align*}
%\S &= \left\{
%\begin{bmatrix}
%           2 \\
%           1 \\
%\end{bmatrix}
%,
%\begin{bmatrix}
%           3 \\
%           2 \\
%\end{bmatrix}
%,
%\begin{bmatrix}
%           1 \\
%           2 \\
%\end{bmatrix}
%\right\}.
%\end{align*}
%\noindent
%Bestem om vektorerne er lineært afhængige ved ligningen, altså hvis $x_1$, $x_2$ og $x_3 \neq 0$
%\begin{align*}
%x_1+
%\begin{bmatrix}
%           2 \\
%           1 \\
%\end{bmatrix}
%+ x_2
%\begin{bmatrix}
%           3 \\
%           2 \\
%\end{bmatrix}
%+ x_3
%\begin{bmatrix}
%           1 \\
%           2 \\
%\end{bmatrix}
%=0.
%\end{align*}
%%
%Ligningerne udskrives 
%%
%\begin{align*}
%2x_1+3x_2+c_3=0\\
%x_1+2x_2+2c_3=0 
%\end{align*}
%\noindent
%Nu findes en linearkombination, som ikke er nul. 
%Vi sætter tilfældigt $x_3 = -1$ og forsøger om ligningen kan løses
%%
%\begin{align*}
%2x_1+3x_2-1=0\\
%x_1+2x_2-2=0\\
%\\
%2x_1+3x_2-1=0\\
%2x_1+4x_2-4=0\\
%\\	 
%2x_1+3x_2-1=0 - 2x_1+4x_2-4=0 \\
%-x_2+3=0\\
%-x_2=-3\\
%x_2=3\\
%\\
%x_1+6+2\\
%x_1+4=0\\
%x_1=-4
%\end{align*}
%%
%Linearkombinationen er derfor
%%
%\begin{align*}
%-4+
%\begin{bmatrix}
%           2 \\
%           1 \\
%\end{bmatrix}
%+ 3
%\begin{bmatrix}
%           3 \\
%           2 \\
%\end{bmatrix}
%+ -1
%\begin{bmatrix}
%           1 \\
%           2 \\
%\end{bmatrix}
%=0
%\end{align*}
%%
%Derfor har vi minimum fundet én linearkombination som er ikke-nul, som giver os nulvektoren, og derfor er det bevist at vektorsættet lineært afhængig.
%\end{eks}
%