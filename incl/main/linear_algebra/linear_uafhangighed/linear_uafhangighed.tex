\subsection{Lineær uafhængighed}
% 1.7
Et sæt af vektorer er \textbf{lineært afhængig}, hvis en af vektorerne kan skrives som en linearkombination af de andre vektorer. Hvis ingen af vektorerne kan skrives som en linearkombination af de andre vektorer, så er sættet \textbf{lineært uafhængige}. Det kan derfor siges, at hvis mindst en af vektorerne \textit{afhænger} af de andre vektorer, er samlingen \textit{lineært afhængig}, og dette leder frem til følgende definition. 
% 
\begin{defn}{Lineær afhængighed og lineær uafhængighed}{}
Et sæt af vektorerne $\mathbf{v}_1, \ldots , \mathbf{v}_k$ i $\R^n$ kaldes \textbf{lineært uafhængige}, hvis der eksisterer skalarer $x_1,x_2, \ldots , x_l$, sådan at ligningen 
\begin{align*}
x_1\mathbf{v}_1 + \ldots + x_k \mathbf{v}_k = \mathbf{0}, 
\end{align*}
kun har løsningen $c_1 = \ldots = c_k = 0$.
Ellers kaldes vektorerne \textbf{lineært afhængige}.
\end{defn}
%
%%%%%%%%%%%%%%%%%%%%%%%%%%%%%%%%%%%%%%%%%%%%%%%%%
% Eksempel
\begin{eks}\label{lineu}
Lad et sæt af vektorer være
\begin{align*}
\S &= \left\{
\begin{bmatrix}
           2 \\
           1 \\
\end{bmatrix}
,
\begin{bmatrix}
           3 \\
           2 \\
\end{bmatrix}
\right\}
\end{align*}
\noindent
Bestem om vektorerne er lineært uafhængige ved ligningen, altså hvis $x_1$ og $x_2 = 0$
\begin{align*}
x_1+
\begin{bmatrix}
           2 \\
           1 \\
\end{bmatrix}
+ x_2
\begin{bmatrix}
           3 \\
           2 \\
\end{bmatrix}
=0 \\
\end{align*}

Ligningerne udskrives
\noindent
\begin{align*}
2x_1+3x_2=0\\
x_1+2x_2=0\\
\\
x_1+3/2x_2=0\\
x_1+2x_2=0\\
\\
x_1+2x_2=0 - x_1+3/2x_2=0\\
1/2c_2=0\\
c_2=0\\
c_1=0\\
\end{align*}
Da $c_1$ og $c_2 = 0$, så er sættet $\S$ lineært uafhængigt.
\\
\\
\noindent
Lad et sæt af vektorer være 
\begin{align*}
\S &= \left\{
\begin{bmatrix}
           2 \\
           1 \\
\end{bmatrix}
,
\begin{bmatrix}
           3 \\
           2 \\
\end{bmatrix}
,
\begin{bmatrix}
           1 \\
           2 \\
\end{bmatrix}
\right\}
\end{align*}
\noindent
Bestem om vektorerne er lineært afhængige ved ligningen, altså hvis $x_1$, $x_2$ og $x_3 \neq 0$
\begin{align*}
x_1+
\begin{bmatrix}
           2 \\
           1 \\
\end{bmatrix}
+ x_2
\begin{bmatrix}
           3 \\
           2 \\
\end{bmatrix}
+ x_3
\begin{bmatrix}
           1 \\
           2 \\
\end{bmatrix}
=0 \\
\end{align*}
Ligningerne udskrives 

\begin{align*}
2x_1+3x_2+c_3=0\\
x_1+2x_2+2c_3=0 \\
\end{align*}
\noindent
Nu findes en linearkombination, som ikke er nul. Vi sætter tilfældigt $x_3 = -1$ og forsøger om ligningen kan løses
\begin{align*}
2x_1+3x_2-1=0\\
x_1+2x_2-2=0\\
\\
2x_1+3x_2-1=0\\
2x_1+4x_2-4=0\\
\\	 
2x_1+3x_2-1=0 - 2x_1+4x_2-4=0 \\
-x_2+3=0\\
-x_2=-3\\
x_2=3\\
\\
x_1+6+2\\
x_1+4=0\\
x_1=-4\\
\end{align*}
Linearkombinationen er derfor ¨
\begin{align*}
-4+
\begin{bmatrix}
           2 \\
           1 \\
\end{bmatrix}
+ 3
\begin{bmatrix}
           3 \\
           2 \\
\end{bmatrix}
+ -1
\begin{bmatrix}
           1 \\
           2 \\
\end{bmatrix}
=0 \\
\end{align*}
Derfor har vi minimum fundet én linearkombination som er ikke-nul, som giver os nulvektoren, og derfor er det bevist at vektorsættet lineært afhængig.
\end{eks}
I eksempel \ref{lineu} viste en af måderne hvorpå man kunne bestemme om et bestemt set $\S$ var lineært afhængigt eller uafhængigt. Der findes flere måder at bestemme dette på. Denne følgende sætning viser en lignende teknik til bestemmelse heraf. 

\begin{thm}{}{}
%
Lad $A$ være en $m \times n$ matrix.
Følgende udsagn er ækvivalente:
%
\begin{enumerate}[label=(\alph*)]
\item $A$'s søjler er lineært uafhængige. 
\item Ligningssystemet $A\mathbf{x}=\mathbf{b}$ har højst en løsning $\forall \mathbf{b} \in \R^m$.
\item $\text{null}(A)=0$.
\item $\text{rang}(A)=n$.
\item Søjlerne i $A$ på reduceret trappeform er standardvektoren i $\R^m$.
\item $A\mathbf{x}=\mathbf{0}$ har kun løsningen $\mathbf{x}=\mathbf{0}$.                                                                                                                                                                                                                                                                                                                         
\item Der er pivot-indgange i hver søjle i $A$. 
\end{enumerate}
%
\end{thm}
%
\begin{proof}
\noindent
Det er forudsagt, at (a) og (f), samt (f) og (g) er ækvivalente, så for at færdiggøre beviset for denne sætning, skal det vises, at(b) $\rightarrow$ (c), (c) $\rightarrow$ (d),(d) $\rightarrow$ (e),(e) $\rightarrow$ (f), (f) $\rightarrow$ (b).
%

\itemize 
\item (b) $\rightarrow$ (c) 
Idet, at $\mathbf{0}$ er en løsning til $A\mathbf{x}=\mathbf{b}$, så forudsætter (b) at $A\mathbf{x}=\mathbf{b}$ har ingen ikke-nul løsninger, da den generelle løsning ingen frie variable, har. Da antallet af frie variable er nulliteten i A og dette er 0, så beviser det (b) $\rightarrow$ (c). 

\item (c) $\rightarrow$ (d)
Idet, at $\text{rang}(A)+\text{null}(A)=n$, så medfører det øjeblikkeligt, at (c) $\rightarrow$ (d).


\item (d) $\rightarrow$ (e)
Hvis $\text{rang}(A)=n$, så betyder det, at enhver søjle i A har en pivotindgang i søjlen, som derfor betyder, at den reducerede trappeform indeholder udelukkende standard vektorer. Disse standardvektorer er nødvendigvis forskellige, da hver søjle indeholder den første ikke-nul indgang i hver række. Dette beviser (d) $\rightarrow$ (e)


\item (e) $\rightarrow$ (f)
Lad $R$ være den reducerede trappeform af matricen $A$. Hvis søjlerne i $R$ er forskellige standardvektorer i $R^m$, så er $R=\mathbf{e_1}, mathbf{e_2}, \ldots, mathbf{e_n}]$. Det er givet, at den eneste løsning til $R\mathbf{x}=\mathbf{0}$, er 0, og idet at $A\mathbf{x}=\mathbf{0}$, er ækvivalent til $R\mathbf{x}=\mathbf{0}$, så er den $\mathbf{0}$ eneste løsning til A$\mathbf{x}=\mathbf{0}$. Dette beviser (e) $\rightarrow$ (f).
 
\item (f) $\rightarrow$ (b)
Lad $\mathbf{u}, \mathbf{v}$ være løsninger til $A\mathbf{x}=\mathbf{b}$.
% Sætning 1.3
Jævnfør sætning \ref{fisk}, følger det at $A(\mathbf{u}-\mathbf{v})=A\mathbf{u}-A\mathbf{v}=\mathbf{b}-\mathbf{b}=\mathbf{0}$, hvilket tilsvarer $\mathbf{u}-\mathbf{v}=\mathbf{0}
\rightarrow  
\mathbf{u} =\mathbf{v}$, hvilket beviser (f) $\rightarrow$ (b)

\end{proof}
%
%Lad $\mathbf{b}=\mathbf{0}$. 
%Ligningssystemet $A\mathbf{x}=\mathbf{b}$ har altid løsningen $\mathbf{x}=\mathbf{0}$, der derfor er den eneste løsning til $A\mathbf{x}=\mathbf{0}$, hvilket beviser (b) $\rightarrow$(e).
% Julie, det her var noget du havde skrevet. Jeg ved ikke helt hvilken den hørte til /MG???
%
\\
%
\textcolor{green}{Tekst om det her med at den er afhængig hvis nul-vektoren er en af dem, 2 af dem er parallele eller hvis det er flere vektorer end deres koordinater.}
\\
%
%%%%%%%%%%%%%%%%%%%%%%%%%%%%%%%%%%%%%%%%%%%%%%%%%
% Eksempel
\begin{eks}
\textcolor{green}{En eksemepel XD}
\end{eks}
%
\textcolor{red}{Motiverende tekst til næste sætning}
%
\begin{thm}{}{}
%
Vektorerne $\mathbf{u}_1,\mathbf{u}_2, \ldots ,\mathbf{u}_k$ i $\R^n$ er lineært afhængige, 
hvis og kun hvis, $\mathbf{u}_1=\mathbf{0}$, eller hvis der findes $i \geq 2$ således $\mathbf{u}_i \in \text{span} \{ \mathbf{u}_1,\mathbf{u}_2, \ldots ,\mathbf{u}_{i-1} \}$.
%
\end{thm}
%
\begin{proof}
%
Lad $\mathbf{u}_1=\mathbf{0}$, så er sætningen bevist. 
Antag nu, at $\mathbf{u}_1 \neq \mathbf{0}$. 
Der findes skalarer, hvor koefficienterne ikke alle er nul, således at
% 
\begin{align*}
c_1 \mathbf{u}_1 + c_2 \mathbf{u}_2 + \ldots + c_k \mathbf{u}_k = \mathbf{0}.
\end{align*}
%
Lad $i$ være det største index $i$, hvor $x_i \neq 0$, sådan at 
\begin{align*}
c_1 \mathbf{u}_1 + c_2 \mathbf{u}_2 + \ldots + c_i \mathbf{u}_i = \mathbf{0}.
\end{align*}
Løses denne i forhold til $\mathbf{u}_i$, ses det at
\begin{align*}
\mathbf{u}_i = - \frac{c_1}{c_i} \mathbf{u}_1 - \frac{c_2}{c_i} \mathbf{u}_2 - \ldots - \frac{c_{p-1}}{c_i} \mathbf{u}_{i-1}.
\end{align*}
Hvilket vil sige at 
\begin{align*}
\mathbf{u}_i \in \text{span}\{ \mathbf{u}_1,\mathbf{u}_2, \ldots ,\mathbf{u}_k \},
\end{align*}
%
og sætningen er hermed bevist.
%
\\\\
\textcolor{red}{Mangler beviset modsat (da det er en hvis og kun hvis)}
%
\end{proof}
\\
%
%
%%%%%%%%%%%%%%%%%%%%%%%%%%%%%%%%%%%%%%%%%%%%%%%%%
% Eksempel
\begin{eks}
\textcolor{green}{En godt eksemepel XD} 
\\
Noget med udvælg færre vektrorer for at få en effektiv beskrivelse af $\text{span}\{\S\}$.
\end{eks}
%
\textcolor{green}{Mere motiverende tekst}