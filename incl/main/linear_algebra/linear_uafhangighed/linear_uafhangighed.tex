\subsection{Lineær uafhængighed}
% 1.7
Et sæt af vektorer er \textbf{lineært afhængig}, hvis en af vektorerne kan skrives som en linearkombination af de andre vektorer. Hvis ingen af vektorerne kan skrives som en linearkombination af de andre vektorer, så er sættet \textbf{lineært uafhængige}. Det kan derfor siges, at hvis mindst en af vektorerne \textit{afhænger} af de andre vektorer, er samlingen \textit{lineært afhængig}, og dette leder frem til følgende definition. 
% 
\begin{defn}{Lineær afhængighed og lineær uafhængighed}{}
Et sæt af vektorerne $\mathbf{v}_1, \ldots , \mathbf{v}_k$ i $\R^n$ kaldes \textbf{lineært uafhængige}, hvis der eksisterer skalarer $x_1,x_2, \ldots , x_l$, sådan at ligningen 
\begin{align*}
x_1\mathbf{v}_1 + \ldots + x_k \mathbf{v}_k = \mathbf{0}, 
\end{align*}
kun har løsningen $c_1 = \ldots = c_k = 0$.
Ellers kaldes vektorerne \textbf{lineært afhængige}.
\end{defn}
%
%%%%%%%%%%%%%%%%%%%%%%%%%%%%%%%%%%%%%%%%%%%%%%%%%
% Eksempel
\begin{eks}
\textcolor{red}{En godt eksemepel på noget der er lineært uafhængig og lineært afhængig XD}
\end{eks}
% 
\textcolor{red}{Motiverende tekst til næste sætning}
%
\begin{thm}{}{}
%
Lad $A$ være en $m \times n$ matrix.
Følgende udsagn er ækvivalente:
%
\begin{enumerate}[label=(\alph*)]
\item $A$'s søjler er lineært uafhængige. 
\item Ligningssystemet $A\mathbf{x}=\mathbf{b}$ har højst en løsning $\forall \mathbf{b} \in \R^m$.
\item $\text{null}(A)=0$.
\item $\text{rang}(A)=n$.
\item Søjlerne i $A$ på reduceret trappeform er standardvektoren i $\R^m$.
\item $A\mathbf{x}=\mathbf{0}$ har kun løsningen $\mathbf{x}=\mathbf{0}$.
\item Der er pivot-indgange i hver søjle i $A$. 
\end{enumerate}
%
\end{thm}
%
\begin{proof}
%
\textcolor{red}{Mangler resten af beviset - Hele beviset side 79 i bogen
%
(a) $\rightarrow$ (f)
(f) $\rightarrow$ (g)
(b) $\rightarrow$ (c)
(c) $\rightarrow$ (d)
(e) $\rightarrow$ (f)
(f) $\rightarrow$ (b)
%
}
Lad $\mathbf{b}=\mathbf{0}$. 
Ligningssystemet $A\mathbf{x}=\mathbf{b}$ har altid løsningen $\mathbf{x}=\mathbf{0}$, der derfor er den eneste løsning til $A\mathbf{x}=\mathbf{0}$, hvilket beviser (b) $\rightarrow$ (e).
\\
Lad $\mathbf{u}, \mathbf{v}$ være løsninger til $A\mathbf{x}=\mathbf{b}$.
% Sætning 1.3
Jævnfør sætning \ref{fisk}, følger det at $A(\mathbf{u}-\mathbf{v})=A\mathbf{u}-A\mathbf{v}=\mathbf{b}-\mathbf{b}=\mathbf{0}$, hvilket tilsvarer $\mathbf{u}-\mathbf{v}=\mathbf{0}
\rightarrow  
\mathbf{u} =\mathbf{v}$, hvilket beviser (e) $\rightarrow$ (b)
%
%
\end{proof}
\\
%
\textcolor{green}{Tekst om det her med at den er afhængig hvis nul-vektoren er en af dem, 2 af dem er parallele eller hvis det er flere vektorer end deres koordinater.}
\\
%
%%%%%%%%%%%%%%%%%%%%%%%%%%%%%%%%%%%%%%%%%%%%%%%%%
% Eksempel
\begin{eks}
\textcolor{green}{En eksemepel XD}
\end{eks}
%
\textcolor{red}{Motiverende tekst til næste sætning}
%
\begin{thm}{}{}
%
Vektorerne $\mathbf{u}_1,\mathbf{u}_2, \ldots ,\mathbf{u}_k$ i $\R^n$ er lineært afhængige, 
hvis og kun hvis, $\mathbf{u}_1=\mathbf{0}$, eller hvis der findes $i \geq 2$ således $\mathbf{u}_i \in \text{span} \{ \mathbf{u}_1,\mathbf{u}_2, \ldots ,\mathbf{u}_{i-1} \}$.
%
\end{thm}
%
\begin{proof}
%
Lad $\mathbf{u}_1=\mathbf{0}$, så er sætningen bevist. 
Antag nu, at $\mathbf{u}_1 \neq \mathbf{0}$. 
Der findes skalarer, hvor koefficienterne ikke alle er nul, således at
% 
\begin{align*}
c_1 \mathbf{u}_1 + c_2 \mathbf{u}_2 + \ldots + c_k \mathbf{u}_k = \mathbf{0}.
\end{align*}
%
Lad $i$ være det største index $i$, hvor $x_i \neq 0$, sådan at 
\begin{align*}
c_1 \mathbf{u}_1 + c_2 \mathbf{u}_2 + \ldots + c_i \mathbf{u}_i = \mathbf{0}.
\end{align*}
Løses denne i forhold til $\mathbf{u}_i$, ses det at
\begin{align*}
\mathbf{u}_i = - \frac{c_1}{c_i} \mathbf{u}_1 - \frac{c_2}{c_i} \mathbf{u}_2 - \ldots - \frac{c_{p-1}}{c_i} \mathbf{u}_{i-1}.
\end{align*}
Hvilket vil sige at 
\begin{align*}
\mathbf{u}_i \in \text{span}\{ \mathbf{u}_1,\mathbf{u}_2, \ldots ,\mathbf{u}_k \},
\end{align*}
%
og sætningen er hermed bevist.
%
\\\\
\textcolor{red}{Mangler beviset modsat (da det er en hvis og kun hvis)}
%
\end{proof}
\\
%
%
%%%%%%%%%%%%%%%%%%%%%%%%%%%%%%%%%%%%%%%%%%%%%%%%%
% Eksempel
\begin{eks}
\textcolor{green}{En godt eksemepel XD} 
\\
Noget med udvælg færre vektrorer for at få en effektiv beskrivelse af $\text{span}\{\S\}$.
\end{eks}
%
\textcolor{green}{Mere motiverende tekst}