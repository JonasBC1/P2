\subsection{Linearkombination}
%
% Et eller andet metatekst her.
%
\textbf{Linearkombination} er en essentiel del af lineær algebra. 
%
\begin{defn}{}{mxlinkom}
Givet matricerne $A_1, A_2, \ldots, A_k$ med størrelsen $m \times n$, så er linearkombinationen af $A_1, A_2, \ldots, A_k$ en matrix $B$ på formen 
%
$$c_1A_1+c_2A_2+\ldots+c_kA_k=\sum\limits_{i=1}^k c_iA_i=B,$$
%
med størrelsen $m \times n$. 
Linearkombinationen er mulig, hvis og kun, hvis skalarerne $c_1, c_2, \ldots, c_k$ eksisterer. 
Skalarerne kaldes linearkombinationens koefficienter.
\end{defn}
%
% KILDE primært: https://www.statlect.com/matrix-algebra/linear-combinations
%
\noindent 
%
I lineær algebra er det oftest række- eller søjlevektorer, der indgår i linearkombinationer. 
En linearkombination af én vektor er blot en skalering af vektoren.
\\
%
%%%%%%%%%%%%%%%%%%%%%%%%%%%%%%%%%%%%%%%%%%%
%
\begin{eks}
Lad 
$$
\textbf{a}=
\begin{bmatrix}
5 \\
9 \\
4
\end{bmatrix}
\text{, }
\textbf{b}=
\begin{bmatrix}
8  \\
12 \\ 
3
\end{bmatrix}
\text{og }
\textbf{c}=
\begin{bmatrix}
0 \\
0 \\
7
\end{bmatrix},
$$
samt $c_a=-4$, $c_b=100$ og $c_c=2$. 
Jævnfør definition \ref{defn:mxlinkom} er linearkombinationen af $\textbf{a}$, $\textbf{b}$ og $\textbf{c}$
$$
(-4)
\begin{bmatrix}
5 \\ 
9 \\ 
4
\end{bmatrix}
+
100
\begin{bmatrix}
8 \\
12 \\ 
3
\end{bmatrix}
+
2
\begin{bmatrix}
0 \\
0 \\ 
7
\end{bmatrix}
=
\begin{bmatrix}
780 \\
1164 \\ 
298
\end{bmatrix}
\text{.}
$$
\end{eks}
%
%%%%%%%%%%%%%%%%%%%%%%%%%%%%%%%%%%%%%%%%%%%%%%
%
Enhver vektor 
$
\textbf{a}=
\begin{bmatrix}
a_1 & a_2 & \ldots & a_n
\end{bmatrix}^T
$
i $\R^n$ kan skrives som linearkombination af $I_n$ i $R^n$. 
Givet to ikke-parallelle vektorer $u$ og $v$ i $\R^2$, så er enhver anden vektor i $\R^2$ en linearkombination af $u$ og $v$. 
Hvis koefficienterne i en linearkombination er vektorer, så er der tale om et \textbf{matrix-vektorprodukt}.
%
%
%
%
\subsection{Matrix-vektorprodukt}
%Altså, jeg vil måske mene, at nedenstående definition minder for meget om den i bogen (s. 19). Men igen... "Det er heller ikke ham, der har fundet på det." - Mads Vejleder, 2019. 
\begin{defn}{}{mvp}
Lad $A$ være en $m \times n$ matrix og \textbf{v} være en $n \times 1$ vektor. 
Matrix-vektorproduktet af $A$ og $\textbf{v}$ defineres som linearkombinationen af søjlerne i $A$, hvis koefficienter er de tilsvarende komponenter af $\textbf{v}$, og noteres $A\textbf{v}$. 
Dermed er
\begin{align*}
A\textbf{v} =v_1\textbf{a}_1 + v_2\textbf{a}_2 + \cdots + v_n\textbf{a}_n.
\end{align*}
\end{defn}
\noindent
For at $A\textbf{v}$ eksisterer, skal der være lige mange komponenter i \textbf{v}, som der er søjler i $A$. 
Et eksempel på et matrix-vektorprodukt mellem en matrix $B$ og en vektor \textbf{u} ses i eksempel \ref{Matrix-vektor}.
\\
%
\begin{eks}
\label{Matrix-vektor}
%
Lad
$$B=
\begin{blockarray}{c c c c}
\begin{block}{[c c c c]}
2 & 5 & 6 \\
4 & 7 & 3\\
1 & 4 & 1\\
2 & 3 & 8\\
\end{block}
\end{blockarray}
%
\text{ og }
%
\textbf{u}=
\begin{bmatrix}
4 \\
2 \\
3 \\ 
\end{bmatrix}.
$$
%
Eftersom $B$ har 3 søjler og \textbf{u} har 3 komponenter, så eksisterer $B\textbf{u}$ og findes ved
$$
B\textbf{u}=
\begin{bmatrix}
2 & 5 & 6 \\
4 & 7 & 3\\
1 & 4 & 1\\
2 & 3 & 8\\
\end{bmatrix}
\begin{bmatrix}
4 \\
2 \\
3 \\ 
\end{bmatrix}
=4
\begin{bmatrix}
2\\
4\\
1\\
2\\
\end{bmatrix}
+2
\begin{bmatrix}
5\\
7\\
4\\
3\\
\end{bmatrix}
+3
\begin{bmatrix}
6\\
3\\
1\\
8\\
\end{bmatrix}
=
\begin{bmatrix}
8\\
16\\
4\\
8\\
\end{bmatrix}
+
\begin{bmatrix}
10\\
14\\
8\\
6\\
\end{bmatrix}
+
\begin{bmatrix}
18\\
9\\
3\\
24\\
\end{bmatrix}
=
\begin{bmatrix}
36\\
39\\
15\\
38\\
\end{bmatrix}.
$$
\end{eks}