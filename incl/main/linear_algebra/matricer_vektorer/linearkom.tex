\subsection{Linearkombination}
% Det jo kinda en vigtig del for os 
\textit{Linearkombination} er en central del af lineær algebra, da det benyttes i forbindelse med løsning af ligningssystemer. 
%
\begin{defn}{}{mxlinkom}
Givet vektorer $\mathbf{u}_1, \mathbf{u}_2, \ldots, \mathbf{u}_k$ med størrelsen $m \times n$, så er \textbf{linearkombinationen} af $\mathbf{u}_1, \mathbf{u}_2, \ldots, \mathbf{u}_k$ en vektor $\mathbf{v}$ på formen 
%
\begin{align*}
c_1\mathbf{u}_1+c_2\mathbf{u}_2+\ldots+c_k\mathbf{u}_k=\sum\limits_{i=1}^k c_i\mathbf{u}_i=\mathbf{v},
\end{align*}
%
med størrelsen $m \times n$, hvor $c_1, c_2, \ldots, c_k$ er skalarerne.
Skalarerne kaldes linearkombinationens koefficienter.
\end{defn}
%
% KILDE primært: https://www.statlect.com/matrix-algebra/linear-combinations
%
\noindent 
%
En linearkombination af én vektor er blot en skalering af vektoren.
\\
%
%%%%%%%%%%%%%%%%%%%%%%%%%%%%%%%%%%%%%%%%%%%
%
\begin{eks}
Lad 
$$
\textbf{u}=
\begin{bmatrix}
5 \\
9 \\
4
\end{bmatrix}
\text{, }
\textbf{v}=
\begin{bmatrix}
8  \\
12 \\ 
3
\end{bmatrix}
\text{og }
\textbf{w}=
\begin{bmatrix}
0 \\
0 \\
7
\end{bmatrix},
$$
samt $c_a=-4$, $c_b=100$ og $c_c=2$. 
Jævnfør \ref{defn:mxlinkom} er linearkombinationen af $\textbf{u}$, $\textbf{v}$ og $\textbf{w}$:
$$
(-4)
\begin{bmatrix}
5 \\ 
9 \\ 
4
\end{bmatrix}
+
100
\begin{bmatrix}
8 \\
12 \\ 
3
\end{bmatrix}
+
2
\begin{bmatrix}
0 \\
0 \\ 
7
\end{bmatrix}
=
\begin{bmatrix}
780 \\
1164 \\ 
298
\end{bmatrix}
\text{.}
$$
\end{eks}
%
%%%%%%%%%%%%%%%%%%%%%%%%%%%%%%%%%%%%%%%%%%%%%%
%
Enhver vektor 
\begin{align*}
\textbf{u}=
\begin{bmatrix}
a_1 \\
a_2 \\
\vdots \\
a_n
\end{bmatrix}
\end{align*}
%
i $\R^n$ kan skrives som linearkombination af $I_n$ i $\R^n$. 
Givet to ikke-parallelle vektorer $\mathbf{u}$ og $\mathbf{v}$ i $\R^2$, så er enhver anden vektor i $\R^2$ en linearkombination af $\mathbf{u}$ og $\mathbf{v}$. 
Hvis koefficienterne i en linearkombination er vektorer, så er der tale om et matrix-vektorprodukt.
%
%
%
%
\subsection{Matrix-vektorprodukt}
%
Som udvidelse til linearkombinationen kan der tales om et \textit{matrix-vektorprodukt}.
%
%Altså, jeg vil måske mene, at nedenstående definition minder for meget om den i bogen (s. 19). Men igen... "Det er heller ikke ham, der har fundet på det." - Mads Vejleder, 2019. 
%
\begin{defn}{}{mvp}
Lad $A$ være en $m \times n$ matrix og \textbf{u} være en $n \times 1$ vektor. 
\textbf{Matrix-vektorproduktet} af $A$ og $\textbf{u}$ defineres som linearkombinationen af søjlerne i $A$, hvis koefficienter er de tilsvarende komponenter af $\textbf{u}$, hvilket noteres $A\textbf{u}$. 
Dermed er
\begin{align*}
A\textbf{u} =u_1\textbf{a}_1 + u_2\textbf{a}_2 + \cdots + u_n\textbf{a}_n.
\end{align*}
\end{defn}
\noindent
For at $A\textbf{u}$ eksisterer, skal der være lige mange komponenter i \textbf{v}, som der er søjler i $A$. 
Et eksempel på et matrix-vektorprodukt mellem en matrix $A$ og en vektor \textbf{u} ses i \ref{Matrix-vektor}.
\\
%
\begin{eks}
\label{Matrix-vektor}
%
Lad
$$A=
\begin{blockarray}{c c c c}
\begin{block}{[c c c c]}
2 & 5 & 6 \\
4 & 7 & 3\\
1 & 4 & 1\\
2 & 3 & 8\\
\end{block}
\end{blockarray}
%
\text{ og }
%
\textbf{u}=
\begin{bmatrix}
4 \\
2 \\
3 \\ 
\end{bmatrix}.
$$
%
Eftersom $A$ har 3 søjler og \textbf{u} har 3 komponenter, så eksisterer $A\textbf{u}$ og findes ved
$$
A\textbf{u}=
\begin{bmatrix}
2 & 5 & 6 \\
4 & 7 & 3\\
1 & 4 & 1\\
2 & 3 & 8\\
\end{bmatrix}
\begin{bmatrix}
4 \\
2 \\
3 \\ 
\end{bmatrix}
=4
\begin{bmatrix}
2\\
4\\
1\\
2\\
\end{bmatrix}
+2
\begin{bmatrix}
5\\
7\\
4\\
3\\
\end{bmatrix}
+3
\begin{bmatrix}
6\\
3\\
1\\
8\\
\end{bmatrix}
=
\begin{bmatrix}
8\\
16\\
4\\
8\\
\end{bmatrix}
+
\begin{bmatrix}
10\\
14\\
8\\
6\\
\end{bmatrix}
+
\begin{bmatrix}
18\\
9\\
3\\
24\\
\end{bmatrix}
=
\begin{bmatrix}
36\\
39\\
15\\
38\\
\end{bmatrix}.
$$
\end{eks}
% 
En række egenskaber ved matrix-vektorprodukt beskrives i \ref{thm:mxvpro}.
\\\\
\phantom{  }
\\\\
\phantom{  }
\\\\
\phantom{  }
\\\\
\phantom{  }
\\\\
\phantom{  }
\\\\
\phantom{  }
\\\\
%
\begin{thm}{}{mxvpro}
Lad $A$ og $B$ være $m \times n$ matricer, og lad $\mathbf{u}, \mathbf{v}$ og $\mathbf{w}$ være vektorer i $R^n$. Så gælder det at
\begin{enumerate}[label=(\alph*)]
\item $A(\mathbf{u}+\mathbf{v})=A\mathbf{u}+A\mathbf{v}$.
\item $A(c\mathbf{u})=c(A\mathbf{u})=(cA)\mathbf{u}$ for enhver skalar c.
\item $(A+B)\mathbf{u}=A\mathbf{u}+B\mathbf{u}$.
\item $A\mathbf{e}_j=\mathbf{a}_j$ for $j=1,2,\ldots,n$, hvor $\mathbf{e}_j$ er den j'te standardvektor i $R^n$.
\item Hvis $B$ er en $m \times n$ matrix, der opfylder, at $B\mathbf{w}=A\mathbf{w}$ for alle $\mathbf{w}$ i $R^n$, så gælder, at $B=A$.
\item $A\mathbf{0}$ er $m \times 1$ nulvektoren.
\item Hvis $O$ er $m \times n$ nulmatricen, så gælder, at $O\mathbf{v}$ er $m \times 1$ nulvektoren.
\item $I_n\mathbf{v}=\mathbf{v}$.
\end{enumerate}
\end{thm}
\begin{proof}
(a)
Da de i'te komponenter af $\textbf{u}$ og $\textbf{v}$ er henholdsvis $u_i$ og $v_i$, kan det opskrives som
\begin{align*}
A(\mathbf{u}+\mathbf{v})&=(u_1+v_1)\mathbf{a}_1+(u_2+v_2)\mathbf{a}_2+\cdots + (u_n+v_n)\mathbf{a}_n \\
&=(u_1\mathbf{a}_1+u_2\mathbf{a}_2+\cdots+u_n\mathbf{a}_n)+(v_1\mathbf{a}_1+v_2\mathbf{a}_2+\cdots+v_n\mathbf{a}_n) \\
&=A\mathbf{u}+A\mathbf{v}.
\end{align*}
(b) $A(c\mathbf{u})$ kan opskrives som
\begin{align*}
A(c \textbf{u})&=\textbf{A}_1(cu_1)+\textbf{A}_2(cu_2) + \cdots + \textbf{A}_n(cu_n) \\
&=\textbf{A}_1cu_1+\textbf{A}_2cu_2 + \cdots + \textbf{A}_ncu_n \\
&=c(\textbf{A}_1u_1)+c(\textbf{A}_2u_2)+\cdots+c(\textbf{A}_n u_n)
=c(A \textbf{u}) \\
&=(c\textbf{A}_1)u_1+(c\textbf{A}_2)u_2+\cdots+(c\textbf{A}_n)u_n=(cA)\textbf{u}.
\end{align*}
%
(c) $(A+B)\mathbf{u}$ opskrives som
%
\begin{align*}
(A+B)\textbf{u}&=(\textbf{A}_1+\textbf{b}_1)u_1 +(\textbf{A}_2+\textbf{B}_2)u_2 \cdots (\textbf{A}_n+\textbf{B}_n)u_n \\
&=(\textbf{A}_1u_1+\textbf{A}_2u_2+ \cdots + \textbf{A}_nu_n)+(\textbf{B}_1u_1+\textbf{B}_2u_2+ \cdots + \textbf{B}_nu_n) \\
&=A\textbf{u}+B\textbf{u}.
\end{align*}
(d) 
Søjlen $\mathbf{a}_j$ multipliceres med $1$, mens de resterende multipliceres med $0$. 
Derfor er resultatet af $A\mathbf{e}_j=\mathbf{a}_j.$
\\\\
% Denne align skal eventuelt skrives om xD
(e) Lad  $B\mathbf{w}=A\mathbf{w}$ for alle $\mathbf{w} \in \R^n$.
Så gælder det, at $B=A$, da
%
\begin{align*}
\textbf{b}_1w_1+\textbf{b}_2 w_2+\cdots+\textbf{b}_n w_n &=\textbf{a}_1 w_1+\textbf{a}_2 w_2+\cdots+\textbf{a}_n w_n \rightarrow\\   
B&=A.
\end{align*}
% \frac{\textbf{b}_1w_1+\textbf{b}_2 w_2+\cdots+\textbf{b}_n w_n}{w_1+ w_2+\cdots+ w_n} &=\frac{\textbf{a}_1 w_1+\textbf{a}_2 w_2+\cdots+\textbf{a}_n w_n}{w_1+ w_2+\cdots+ w_n} \rightarrow\\  
(f)
Dette kan opskrives som
\begin{align*}
0\textbf{A}_1+0\textbf{A}_2+\cdots+0\textbf{A}_n.
\end{align*}
Resultatet er $\textbf{0}$ med $m$ rækker. \\\\
(g)
Produktet af en vektor og en $m \times n$ matrix er en vektor på størrelsen $1 \times m$. Da alle indgange i matricen er $0$, bliver produktet netop en $1 \times m$ nulvektor. 
\\\\
(h)
Værdien af komponenterne i produktet af $I_n\mathbf{v}$ har følgende form 
$$
\begin{bmatrix}
1v_1 \\
1v_2 \\
\vdots \\
1v_n  \\
\end{bmatrix}.
$$
Det følger derfor, at $I_n\mathbf{v}=v$.
Således er alle punkter i \ref{thm:mxvpro} bevist. 
\end{proof}