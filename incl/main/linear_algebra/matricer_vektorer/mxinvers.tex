\subsection{Inverse matricer}

\begin{defn}{}{mxinvers}
En $n \times n$ matrix $A$ er \textbf{invertibel}, hvis en $n \times n$ matrix $B$ eksisterer, så $AB=BA=I_n$. $B$ er i så fald $A$'s invers. 
\end{defn}
%%%%%%%%%%%%%%%%%%%%%%%%%%%%%
Det er dog ikke altid, der findes en matrix $B$, så kriterierne i definition \ref{defn:mxinvers} opfyldes. 
Såfremt $A$ er en invertibel $n \times n$ matrix, gælder det for ethvert $B \in \R^n$, at $A\textbf{x}=B$ har løsningen $A^{-1}B$.
%
\begin{thm}{}{mxinvertibel}
Lad $A$ være en $n \times n$ matrix. 
$A$ er invertibel, hvis og kun hvis den reducerede trappeform af $A$ er $I_n$.
\end{thm}
%
\begin{proof}
$\rightarrow$ Antag, at $n\times n$ matricen $A$ er invertibel, og at der findes en vektor \textbf{v} i $R^n$, så $A\textbf{v}=\textbf{0}$. 
Deraf følger, at $\textbf{v}=A^{-1}\textbf{0}=\textbf{0}$, da $A$ er en invertibel matrix gælder det for ethvert $B \in \R^n$, at $A\textbf{x}=B$ har løsningen $A^{-1}B$ og derfor er $\textbf{v}=A^{-1}\textbf{0}=\textbf{0}$. 
Den eneste løsning til problemet er derfor $0$-vektoren $\textbf{0}$. 
Såfremt $A$ er invertibel gælder det derfor $rank(A)=n$(tjeck lige notationen).
Det følger derfor fra sætning (den er i liniær uafhængihedsafsnittet forstod ikke helt lablet) at $A$ på reducerettrappeform er $I_n$.
$leftarrow$
Antag nu at den reducerede trappeform af $A$=$I_n$. 
Så gælder det fra sætning, at der eksister en invertibel $n \times n$ matrix $P$ hvorom det gælder at $PA=I_n$.
$$A=I_nA=(P^{-1}P)A=^{-1}(PA)=P^{-1}$$
Fra sætning fremgår det at $P^{-1}$ er en invertibel matrix, og det er hermed bevist at $A$ er invertibel da $A=P^{-1}$.

\end{proof}
%%%%%%%%%%%%%%%%%%%%%%%%%%%%%%
Inverse matricer har en række egenskaber, som formuleret i sætning \ref{thm:mxinvers} og \ref{thm:mxinvers2}. 
%
\begin{thm}{}{mxinvers}
Givet to $m \times n$ matricer $A$ og $B$, så gælder, at
\begin{enumerate}[label=(\alph*)]
\item Hvis $A$ er invertibel, så er $A^{-1}$ invertibel, og $(A^{-1})^{-1}=A$.
\item Hvis $A$ og $B$ er invertible, så er $AB$ invertibel, og $(AB)^{-1}=A^{-1}B^{-1}$.
\item Hvis $A$ er invertibel, så er $A^T$ invertibel, og $(A^T)^{-1}=(A^{-1})^T$.
\end{enumerate}
\end{thm}
%
%
\begin{proof}
Punkt (a) følger af definition \ref{thm:mxinvers}. 
(b) Antag, at $A$ og $B$ er invertibel matricer, så $(AB)(B^{-1}A^{-1})=A(BB^{-1})A^{-1}=AI_nA^{-1}=AA^{-1}=I_n$. 
Det kan gennem lignende operationer vises at $(B^{-1}A^{-1})(AB)=I_n$.
Dermed er $AB$ en invertibel matrix, hvis invers er $A^{-1}B^{-1}$ altså er $AB^{-1}=A^{-1}B^{-1}$.
(c) Antag at A er invertibel. 
Det følger fra definition \ref{defn:mxinvers}, at $A^{-1}A=I_n$. 
Fra punkt (g) i sætning \ref{thm:mxmulti} følger det hermed, at $A^T(A^{-1})^T=(A^{-1}A)^T=I_n^T=I_n$. 
Ligeledes kan det vises, at $(A^{-1})^TA^T=I_n$. 
Derfor er $A^T$ en invertibel matrix med inversen $(A{-1})^T$ og $(A^T)^{-1}A=(A^{-1})^T$.
\end{proof}
%
%
\begin{thm}{}{mxinvers2}
%Givet to $m \times n$ matricer $A$ og $B$, så gælder, at (det er bare leftover aye?)
Lad $A$ være en $n \times n$ matrix. 
Da er følgende udsagn ækvivalente.
\begin{enumerate}[label=(\alph*)]
\item $A$ er invertibel.
\item Den reducerede trappeform af $A$ er $I_n$.
\item rang$(A)=n$.
\item Spannet af søjlerne i $A$ er $R^n$.
\item ligningssystemet $A\textbf{x}=\textbf{b}$ er konsistent for alle $\textbf{b}$ i $R^n$.
\item null$(A)$=0.
\item Søjlerne i $A$ er liniært uafhængige.
\item den eneste løsning for $A\textbf{x}=\textbf{O}$ er $\textbf{O}$.
\item Der eksisterer en $n \times n$ matrix $B$ så $BA=I_n$.
\item Der eksister en $n \times n$ matrix $C$ så $AC=I_n$
\item $A$ er et produkt af elementærmatricer.
%hvis vi tager det sidste punkt med er vi også nødt til at redegøre for det. 

\end{enumerate}
\end{thm}
\begin{proof}
(a) og (b) er ækvivalente som konsekvens af sætning \ref{thm:mxinvertibel}. Da $A$ er en $n \times n$ matrix er dette ligeledes ækvivalent med $c$, $d$ og $e$ som følger af sætning ??
\end{proof}