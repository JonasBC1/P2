\subsection{Elementære og inverse matricer}
At multiplicere en matrix $A$ med en $n \times n$ \textbf{elementær matrix} $E$ svarer til at udføre én rækkeoperation på $A$. 
Dermed kan enhver rækkeoperation udføres gennem matrixmultiplikation. 
En $n \times n$ matrix $E$ er elementær, hvis én rækkeoperation udført på $I_n$ resulterer i $E$. 
%Eksempel? 

Lad nu $A$ være en $m \times n$ matrix og $E$ en elementær $m \times m$ matrix, der er resultatet af én rækkeoperation udført på $I_m$. 
Matrixproduktet $EA$ kan opnås ved at udføre én rækkeoperation på $A$. 
Enhver elementær matrix er desuden \textbf{invertibel}, og dens \textbf{invers} er også en elementær matrix. 
%
\begin{defn}{}{mxinvers}
En $n \times n$ matrix $A$ er invertibel, hvis en $n \times n$ matrix $B$ eksisterer, så $AB=BA=I_n$. $B$ er i så fald $A$'s invers. 
\end{defn}
Såfremt $A$ er en invertibel $n \times n$ matrix, gælder det for ethvert $B \in \R^n$, at $A\textbf{x}=B$ har den entydige løsning $A^{-1}B$. 
%
Inverse matricer har en række egenskaber, hvoraf nogle centrale er formuleret i sætning \ref{thm:mxinvers}. 
%
\begin{thm}{}{mxinvers}
Givet to $m \times n$ matricer $A$ og $B$, så gælder, at
\begin{enumerate}[label=(\alph*)]
\item Hvis $A$ er invertibel, så er $A^{-1}$ invertibel, og $(A^{-1})^{-1}=A$.
\item Hvis $A$ og $B$ er invertible, så er $AB$ invertibel, og $(AB)^{-1}=A^{-1}B^{-1}$.
\item Hvis $A$ er invertibel, så er $A^T$ invertibel, og $(A^T)^{-1}=(A^{-1})^T$.
\end{enumerate}
\end{thm}
%
%
\begin{proof}
Punkt (a) følger af definition \ref{thm:mxinvers}. 
(b) Antag, at $A$ og $B$ er invertibel matricer, så $(AB)(B^{-1}A^{-1})=A(BB^{-1})A^{-1}=AI_nA^{-1}=AA^{-1}=I_n$. 
Det kan gennem lignende operationer vises at $(B^{-1}A^{-1})(AB)=I_n$.
Dermed er $AB$ en invertibel matrix, hvis invers er $A^{-1}B^{-1}$ altså er $AB^{-1}=A^{-1}B^{-1}$.
(c) Antag at A er invertibel. 
Det følger fra definition \ref{defn:mxinvers}, at $A^{-1}A=I_n$. 
Fra punkt (g) i sætning \ref{thm:mxmulti} følger det hermed, at $A^T(A^{-1})^T=(A^{-1}A)^T=I_n^T=I_n$. 
Ligeledes kan det vises, at $(A^{-1})^TA^T=I_n$. 
Derfor er $A^T$ en invertibel matrix med inversen $(A{-1})^T$ og $(A^T)^{-1}A=(A^{-1})^T$.
\end{proof}
%
%NOGET OM ELEMENTÆRE MATRICER OG INVERTIBILITET
%
\begin{thm}{}{eleinv}
Lad $A$ være en $m \times n$ matrix, og $R$ den reducerede trappeform af $A$. 
Så eksisterer der en invertibel $m \times n$ matrix $P$, så $PA=R$.
\end{thm}
%
\begin{proof}
Lad $P$ være en invertibel matrix, så 
$
\begin{bmatrix}
A & \textbf{b}
\end{bmatrix}
=
\begin{bmatrix}
R & \textbf{c}
\end{bmatrix}
$.
Så gælder der, at 
$$
\begin{bmatrix}
PA & P\textbf{b}
\end{bmatrix}
=P
\begin{bmatrix}
A & \textbf{b}
\end{bmatrix}
=
\begin{bmatrix}
R & \textbf{c}
\end{bmatrix}
\text{, }
$$
hvormed $PA=R$ og $P\textbf{b}=\textbf{c}$.
Heraf følger, at $A=P^{-1}R$ og $\textbf{b}=P^{-1}\textbf{c}$, da $P$ er invertibel. 

Antag, at $\textbf{v}$ er en løsning til $A\textbf{x}=\textbf{b}$. Så er
$$
R\textbf{v}=(PA)\textbf{v}=P(A\textbf{v})=P\textbf{b}=\textbf{c},
$$
hvormed $\textbf{v}$ er en løsning til $R\textbf{x}=\textbf{c}$. 
Antag nu i stedet, at $\textbf{v}$ er en løsning til $R\textbf{x}=\textbf{c}$. Så er
$$
A\textbf{v}=(P^{-1}R)\textbf{v}=P^{-1}(R\textbf{v})=P^{-1}\textbf{c}=\textbf{b},
$$
hvormed $\textbf{v}$ er en løsning til $A\textbf{x}=b$. Dermed er det bevist, at $A\textbf{x}=b$ og $R\textbf{x}=c$ har samme løsning.
\end{proof}
%
\noindent
Sætning \ref{thm:eleinv} medfører desuden korollar \ref{kor:eleinv}.
%måske noget mere her (at det er vigtigt for rækkeoperationer?).  
\begin{kor}{}{eleinv}
Matrixligningerne $A\textbf{x}=\textbf{b}$ og $R\textbf{x}=\textbf{c}$ har samme løsninger, hvor 
$
\begin{bmatrix}
R & \textbf{c}
\end{bmatrix}
$
er den reducerede trappeform af den udvidede matrix 
$
\begin{bmatrix}
A & \textbf{b}
\end{bmatrix}
$.
\end{kor}
Beviset for korollar \ref{kor:eleinv} følger af beviset for sætning \ref{thm:eleinv} med tilføjelsen, at hvis $\textbf{b}=\textbf{0}$, så er $\textbf{c}=\textbf{0}$, hvilket resulterer i nulvektoren i begge udtryk. 
%
%
%HER SAD DEF 3.8 FØR
%%%%%%%%%%%%%%%%%%%%%%%%%%%%%
Det er dog ikke altid, der findes en matrix $B$, så kriterierne i definition \ref{defn:mxinvers} opfyldes. 
%
\begin{thm}{}{mxinvertibel}
Lad $A$ være en $n \times n$ matrix. 
$A$ er invertibel, hvis og kun hvis den reducerede trappeform af $A$ er $I_n$.
\end{thm}
%
\begin{proof}
$\rightarrow$ Antag, at $n\times n$ matricen $A$ er invertibel, og at der findes en vektor \textbf{v} i $R^n$, så $A\textbf{v}=\textbf{0}$. 
Deraf følger, at $\textbf{v}=A^{-1}\textbf{0}=\textbf{0}$, da $A$ er en invertibel matrix gælder det for ethvert $B \in \R^n$, at $A\textbf{x}=B$ har løsningen $A^{-1}B$ og derfor er $\textbf{v}=A^{-1}\textbf{0}=\textbf{0}$. 
Den eneste løsning til problemet er derfor $0$-vektoren $\textbf{0}$. 
Såfremt $A$ er invertibel gælder det derfor $rank(A)=n$(tjeck lige notationen).
Det følger derfor fra sætning (den er i liniær uafhængihedsafsnittet forstod ikke helt lablet) at $A$ på reducerettrappeform er $I_n$.
$\leftarrow$
Antag nu at den reducerede trappeform af $A$=$I_n$. 
Så gælder det fra sætning, at der eksister en invertibel $n \times n$ matrix $P$ hvorom det gælder at $PA=I_n$. 
%
\begin{align*}
A=I_nA=(P^{-1}P)A=^{-1}(PA)=P^{-1}.
\end{align*}
%
Fra sætning fremgår det at $P^{-1}$ er en invertibel matrix, og det er hermed bevist at $A$ er invertibel da $A=P^{-1}$.
\end{proof}
%%%%%%%%%%%%%%%%%%%%%%%%%%%%%%
Sætning \ref{thm:mxinvers2} redegør for nogle yderligere egenskaber for invertible matricer. 
%
\begin{thm}{}{mxinvers2}
%Givet to $m \times n$ matricer $A$ og $B$, så gælder, at (det er bare leftover aye?)
Lad $A$ være en $n \times n$ matrix. 
Da er følgende udsagn ækvivalente.
\begin{enumerate}[label=(\alph*)]
\item $A$ er invertibel.
\item Den reducerede trappeform af $A$ er $I_n$.
\item rang$(A)=n$.
\item Spannet af søjlerne i $A$ er $R^n$.
\item ligningssystemet $A\textbf{x}=\textbf{b}$ er konsistent for alle $\textbf{b}$ i $R^n$.
\item null$(A)$=0.
\item Søjlerne i $A$ er lineært uafhængige.
\item Den eneste løsning for $A\textbf{x}=\textbf{O}$ er $\textbf{O}$.
\item Der eksisterer en $n \times n$ matrix $B$, så $BA=I_n$.
\item Der eksister en $n \times n$ matrix $C$, så $AC=I_n$.
\item $A$ er et produkt af elementære matricer.
%hvis vi tager det sidste punkt med er vi også nødt til at redegøre for det. 
\end{enumerate}
\end{thm}
\begin{proof}
Lad $A$ være en $n \times n$ matrix.
Jævnfør sætning \ref{thm:mxinvertibel} er (a) og (b) ækvivalente. 
Da $A$ er kvadratisk, er (b) ligeledes ækvivalent med (c), (d) og (e) jævnfør sætning \ref{thm:spaneqv}. 
Jævnfør sætning \ref{thm:mxlinuaf} er (b) desuden ækvivalent med (f), (g) og (h). 
Eftersom (a) er ækvivalent med (b), er (a) også ækvivalent med (c), (d), (e), (f), (g) og (h). 

At (a) og (k) er ækvivalente, bevises i to skridt. 
(1) Bemærk, at hvis $A$ er invertibel, så er $I_n$ er den reducerede trappeform af $A$. 
Jævnfør sætning \ref{thm:eleinv} og \ref{thm:mxinvertibel} eksisterer der en invertibel matrix $P$, så at $PA=I_n$, hvormed $A=P^{-1}$. 
Da $P=E_kE_{k-1} \cdots E_1$, så er $P^{-1}=E_1^{-1} \cdots E_{k-1}^{-1}E_k^{-1}$.
$A$ er dermed et produkt af elementære matricer, hvilket beviser, at (a) medfører (k). \\
(2) Antag nu, at $A$ er en produkt af elementære matricer. 
Elementære matricer er invertible, hvormed $A$ er et produkt af invertible matricer. 
$A$ er derfor invertibel, hvilket bekræfter (a), hvilket gør (a) og (k) ækvivalente. 

At (a) medfører (i) er tydeligt, da $B=A^{1}$. 
Antag nu, at der eksisterer en $n \times n$ matrix $B$, så $BA=I_n$. Lad $\textbf{v} \in \R^n$, så at $A\textbf{v}=\textbf{b}$. Så gælder, at 
$$
\textbf{v} = I_n\textbf{v} = (BA)\textbf{v} = B(A\textbf{v} = B\textbf{0} = \textbf{0}.
$$
Bemærk, at (i) dermed medfører (h).
Eftersom (h) er ækvivalent med (a), medfører (i) derfor også (a).
Således er (i) og (a) ækvivalente. 

At (a) medfører (j) er igen tydeligt, da $C=A^{1}$.
Antag nu, at der eksister en $n \times n$ matrix $C$, så $AC=I_n$. 
Lad $\textbf{b} \in \R^n$ og $\textbf{v}=C\textbf{b}$. 
Så gælder, at 
$$
A\textbf{v} = A(C\textbf{b}) = (AC)\textbf{b} = I_n\textbf{b} = \textbf{b}.
$$
Bemærk, at (j) dermed medfører (e).
Eftersom (e) er ækvivalent med (a), medfører (j) derfor også (a).
Således er (j) og (a) ækvivalente, og alle punkter i sætning \ref{thm:mxinvers2} er dermed bevist ækvivalente.
\end{proof}


