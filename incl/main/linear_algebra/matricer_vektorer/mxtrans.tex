\subsection{Matrixtransponering}
%
\textit{Matrixtransponering} er i nogen henseender brugbart, herunder i forbindelse med Simplexmetoden. 
%
% Henseender = Sammenhænge 
%
% \textbf{Matrixtransponering} er faktisk rimelig unødvendigt xD 
%
\begin{defn}{}{mxtrans}
Lad $A$ være en $m \times n$ matrix. \textbf{Transponeringen} af $A$, er en $n \times m$ matrix, noteret $A^T$, hvor $A$'s $(i,j)$'te indgang er den $(j,i)$'te indgang i $A^T$.
\end{defn}
\noindent
%
En transponering af en matrix resulterer i en ny matrix, hvor rækkerne i den oprindelige matrix er søjlerne i den transponerede matrix.
\\
%
\begin{eks}
\label{eks:trans}
%
Lad 
%
\begin{align*}
A= 
\begin{bmatrix}
3	&	-3	&	1 \\
2	&	0	&	4
\end{bmatrix}
\text{\phantom{---}og\phantom{---}}
B= 
\begin{bmatrix}
-1	&	4	&	1\\
5	&	3	&	2
\end{bmatrix}.
\end{align*}
%
Så er
%
\begin{align*}
A^T =
\begin{bmatrix}
3	&	2\\
-3	&	0\\
1	&	4
\end{bmatrix}
\text{\phantom{---}og\phantom{---}}
(A+B)^T=
\begin{bmatrix}
2	&	1	&	2\\
7	&	3	&	6
\end{bmatrix}^T
=
\begin{bmatrix}
2	&	7\\
1	&	3\\
2	&	6
\end{bmatrix}.
\end{align*}
%
\end{eks}
%
%%%%%%%%%%%%%%%%%%%%%%%%%%%%%%%%%%%%%%%%%%%%%%%%%%%%%%%%%%%%%
%
\ref{thm:mxtranspo} viser, at transponering bevarer summering og skalering af matricer.
%
\begin{thm}{}{mxtranspo}
Lad $A$ og $B$ være $m \times n$ matricer, og lad $c$ være en skalar.
Så har matricerne følgende egenskaber:
\begin{enumerate}[label=(\alph*)]
\item $(A + B)^T = A^T + B^T$.
\item $(cA)^T = cA^T$.
\item $(A^T)^T = A$.
\end{enumerate}
\end{thm}
%
\begin{proof}
Lad $A$ og $B$ være $m \times n$ matricer og lad $c$ være en skalar.\\
(a) Det vides, at indgangene i $A+B$ er $a_{i,j} + b_{i,j}$, og transponeringen af dette resulterer i en $n \times m$ matrix med indgangende $a_{j,i} + b_{j,i}$.
Det ses, at  $A^T + B^T$ også er en $n \times m$ matrix med indgangene $a_{i,j} + b_{i,j}$.\\\\
(b)
Ved skalering og transponering af en matrix ses, at opperationernes rækkefølge er uden betydning for slutresultatet.
Dette tydeliggøres ved at betragte $cA^T$ og $(cA)^T$.
For $cA^T$ er indgangene $ca_{j,i}$, og indgangene i $cA$ er $ca_{i,j}$, som efter transponering er $ca_{j,i}$.\\\\
(c)
Transponeres $A$ to gange vil resultatet være $A$.
Første transponering resulterer i en $n \times m$ matrix med indgangene $a_{j,i}$ og anden transponering resulterer i en $m \times n$ matrix med indgangene $a_{i,j}$, hvilket beviser (c).
\end{proof}