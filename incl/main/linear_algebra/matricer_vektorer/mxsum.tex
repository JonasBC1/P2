\subsection{Matrixsum og skalering}
% 
Der er behov for at lave opperationer på en matrix i forbindelse med løsning af lineære programmeringsproblemer. 
To af disse er \textit{matrixsum} og \textit{skalering}.
%
\begin{defn}{}{mxsum} % Jeg synes substration skal være fed
Lad $A$ og $B$ være $m \times n$ matricer.
\textbf{Summen} af $A$ og $B$ er $m \times n$ matricen $A + B$, hvor den $(i,j)$'te indgang er $a_{i,j} + b_{i,j}$.
Ligeledes er subtraktion mellem matricerne $A$ og $B$ muligt, og indgangene i den resulterende matrix er $a_{i,j} - b_{i,j}$.
\\
Lad nu $c$ være en skalar.
\textbf{Skalering}, også kaldet skalarmultiplikation, af $A$, noteret $cA$, er en $m \times n$ matrix, hvor den $(i,j)$'te indgang er $ca_{i,j}$.
\end{defn}
\noindent
%
Bemærk, at $1A = A$, $-1A = -A$ og $0A = O$.
\\
%
\begin{eks}
Lad 
\begin{align*}
A= 
\begin{bmatrix}
3	&	-3	&	1\\
2	&	0	&	4
\end{bmatrix}
\text{\phantom{---}og\phantom{---}}
B= 
\begin{bmatrix}
-1	&	4	&	1\\
5	&	3	&	2
\end{bmatrix}.
\end{align*}
Jævnfør \ref{defn:mxsum} er
\begin{align*}
3A= 
\begin{bmatrix}
9	&	-9	&	3\\
6	&	0	&	12
\end{bmatrix}
\text{,\phantom{---}}
-B= 
\begin{bmatrix}
1	&	-4	&	-1\\
-5	&	-3	&	-2
\end{bmatrix}
\end{align*}
og
\begin{align*}
3A+B= 
\begin{bmatrix}
9	&	-9	&	3\\
6	&	0	&	12
\end{bmatrix}
+ 
\begin{bmatrix}
-1	&	4	&	1\\
5	&	3	&	2
\end{bmatrix}
=
\begin{bmatrix}
8	&	-5	&	4\\
11	&	3	&	14
\end{bmatrix}.
\end{align*}
\end{eks}
%
Ud fra disse operationer har matricer en række egenskaber, som ses i \ref{thm:mxprop}.
%Nedenstående definition er ordret den samme som i bogen (s. 6), men vi kan jo nærmest ikke skrive den og lignende sætninger anderledes; skal vi ikke bare lige vende det med Horia, tror I? 
\begin{thm}{}{mxprop}
Lad $A$, $B$ og $C$ være $m \times n$ matricer, og lad $c$ og $k$ være skalarer.
Så har matricer følgende egenskaber:
\begin{enumerate}[label=(\alph*)]
\item $A + B = B + A$.
\item $(A + B) + C = A + (B + C)$.
\item $A + O = A$.
\item $A + (-A) = O$.
\item $(ck)A = k(cA)$.
\item $c(A + B) = cA + cB$.
\item $(c + k)A = cA + kA$.
\end{enumerate}
\end{thm}
%
\begin{proof} % Skal dette bevis være i punktform? det skal det jo vel egenlig 
Lad $A$, $B$ og $C$ være $m \times n$ matricer, og lad $c$ og $k$ være skalarer.
Betragt $A + B$ og $B + A$ og bemærk jævnfør \ref{defn:mxsum}, at de $(i,j)$'te indgange bliver henholdsvis $a_{i,j} + b_{i,j}$ og $b_{i,j} + a_{i,j}$, hvilket beviser (a).
Ligeledes ses det, at $(a_{i,j} + b_{i,j}) + c_{i,j} = a_{i,j} + (b_{i,j} + c_{i,j})$, hvilket beviser (b).
Endvidere kan $0$ isoleres i $a_{i,j} + 0 = a_{i,j}$, således $a_{i,j} + (-a_{i,j}) = 0$, hvilket beviser (c) og (d), da alle indgange i en nulmatrix er $0$.
\\\\
%
Betragt skaleringen af $A$ med skalaren $ck$, som er et produkt af skalarene $c$ og $k$, og skaleringen af $A$ med $c$ og herefter $k$.
For indgangene i disse skaleringer findes ligheden $(ck)a_{i,j} = k(ca_{i,j})$, hvilket beviser (e).
Betragt ligeledes (f) og bemærk, at $c(a_{i,j} + b_{i,j}) = ca_{i,j} + cb_{i,j}$, hvilket beviser (f).
Det samme gøres for ligheden i (g), hvor ligheden $(c + k)a_{i,j} = ca_{i,j} + ka_{i,j}$ opstilles, hvilket beviser (g).
\end{proof}
\noindent
\\
%
Da vektorer er matricer med enten én række eller én søjle er samme operationer mulige og vektorer besidder derfor de samme egenskaber.
%
% Mere sssssssssssssssssssssssssssssssssssss
\\\\
\phantom{Hejmeddig}
\\\\