\subsection{Matrixmultiplikation}

Det er ofte nødvendigt at multiplicere matricer med andre matricer. 
\begin{defn}{}{}\label{def:mxmulti}
Matrixproduktet $AB$ af en $m \times n$ matrix $A$ og en $n \times p$ matrix $B$ er en $m \times p$ matrix $C$, 
$$
C=
\begin{bmatrix}
A\textbf{b}_1 & A\textbf{b}_2 & \ldots & A\textbf{b}_p
\end{bmatrix}\text{,}
$$
hvor $A\textbf{b}_j$ er den $j$'te række i $C$.
\end{defn}
\noindent
%
Som det fremgår af definition \ref{def:mxmulti}, skal antallet af søjler i $A$ tilsvare antallet af rækker i $B$, for at $AB$ er defineret og matrixmultiplikation dermed er muligt. 
Bemærk, at selvom $AB$ er defineret, er $BA$ ikke nødvendigvis defineret. 
Hvis både $AB$ og $BA$ er defineret, er de desuden ikke nødvendigvis lig hinanden.\\\\
%
\begin{eks}
Lad 
$$
A=
\begin{bmatrix}
7 & 9 & 1 \\
4 & 0 & 0 \\
8 & 5 & 5
\end{bmatrix}
\text{og }
B=
\begin{bmatrix}
2 & 3 \\
5 & 1 \\
0 & 1 
\end{bmatrix}
\text{.}
$$
Ved at betragte $A$ og $B$ vides det, $AB$ er defineret og er en $3 \times 2$ matrix. 
Først findes søjlerne $A\textbf{b}_1$ og $A\textbf{b}_2$:
$$
A\textbf{b}_1=
\begin{bmatrix}
7 & 9 & 1 \\
4 & 0 & 0 \\
8 & 5 & 5
\end{bmatrix}
\begin{bmatrix}
2 \\
5 \\
0
\end{bmatrix}
=
\begin{bmatrix}
14 + 45 + 0 \\
8 + 0 + 0 \\
16 + 25 + 0
\end{bmatrix}
=
\begin{bmatrix}
59 \\
8 \\
41
\end{bmatrix}
$$
og
$$
A\textbf{b}_1=
\begin{bmatrix}
7 & 9 & 1 \\
4 & 0 & 0 \\
8 & 5 & 5
\end{bmatrix}
\begin{bmatrix}
3 \\
1 \\
1
\end{bmatrix}
=
\begin{bmatrix}
21 + 9 + 1 \\
12 + 0 + 0 \\
24 + 5 + 5
\end{bmatrix}
=
\begin{bmatrix}
31 \\
12 \\
34
\end{bmatrix}
\text{.}
$$
Dermed er 
$$
AB=
\begin{bmatrix}
A\textbf{b}_1 & A\textbf{b}_2
\end{bmatrix}
=
\begin{bmatrix}
59 & 31 \\
8 & 12 \\
41 & 34
\end{bmatrix}
$$
\end{eks}
Da matricen $B$ inddeles i sine søjlevektorer, er grundprincippet i matrixmultiplikation at finde matrix-vektorprodukter.\\\\
En række egenskaber gælder for matrixmultiplikation, som summeret i særning \ref{thm:mxmulti}.

\begin{thm}{}{}\label{thm:mxmulti}
Lad $A$ og $B$ være $k \times m$ matricer, $C$ være en $m \times n$ matrix, og $P$ og $Q$ være $n \times p$ matricer. Følgende udsagn er dermed sande:
\begin{enumerate}[label=(\alph*)]
\item $s(AC)=(sA)C=A(sC)$ for en vilkårlig skalar $s$.
\item $A(CP)=(AC)P$.
\item $(A+B)C=AC+BC$.
\item $C(P+Q)=CP+CQ$.
\item $I_kA=A=AI_m$.
\item Produktet af en nulmatrice og an vilkårlig matrice er altid $0$.
\item $(AC)^T=C^TA^T$.
\end{enumerate}
\end{thm}
\begin{proof}
asdf
\end{proof}
