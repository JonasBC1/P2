\subsection{Matrixmultiplikation}

Det er ofte nødvendigt at multiplicere matricer med andre matricer. 
\begin{defn}{}{mxmulti}
Lad $A$ være en $m \times n$ matrix og $B$ være en $n \times p$ matrix.
\textbf{Matrixproduktet} $AB$  er en $m \times p$ matrix 
$$
AB=
\begin{bmatrix}
A\textbf{b}_1 & A\textbf{b}_2 & \ldots & A\textbf{b}_p
\end{bmatrix}\text{,}
$$
hvor $A\textbf{b}_j$ er den $j$'te søjle i $C$.
\end{defn}
\noindent
%
Som det fremgår af \ref{defn:mxmulti}, skal antallet af søjler i $A$ tilsvare antallet af rækker i $B$, for at $AB$ er defineret og matrixmultiplikation dermed er muligt. 
Bemærk, at selvom $AB$ er defineret, er $BA$ ikke nødvendigvis defineret. 
Hvis både $AB$ og $BA$ er defineret, er de desuden ikke nødvendigvis lig hinanden. 
For at beregne værdien af komponenterne i matrixproduktet $AB$ kan følgende sammenhæng mellem den $i$'te række i $A$ og den $j$'te søjle i $B$ betragtes: 
\begin{align*}
a_{i1}b_{1j} + a_{i2}b_{2j} + \cdots + a_{in}b_{nj}
\text{.}
\end{align*}
%\\
%
\begin{eks}
Lad 
$$
A=
\begin{bmatrix}
7 & 9 & 1 \\
4 & 0 & 0 \\
8 & 5 & 5
\end{bmatrix}
\text{og }
B=
\begin{bmatrix}
2 & 3 \\
5 & 1 \\
0 & 1 
\end{bmatrix}
\text{.}
$$
Ved at betragte $A$ og $B$ vides det, $AB$ er defineret og er en $3 \times 2$ matrix. 
Først findes søjlerne $A\textbf{b}_1$ og $A\textbf{b}_2$:
$$
A\textbf{b}_1=
\begin{bmatrix}
7 & 9 & 1 \\
4 & 0 & 0 \\
8 & 5 & 5
\end{bmatrix}
\begin{bmatrix}
2 \\
5 \\
0
\end{bmatrix}
=
\begin{bmatrix}
14 + 45 + 0 \\
8 + 0 + 0 \\
16 + 25 + 0
\end{bmatrix}
=
\begin{bmatrix}
59 \\
8 \\
41
\end{bmatrix}
$$
og
$$
A\textbf{b}_1=
\begin{bmatrix}
7 & 9 & 1 \\
4 & 0 & 0 \\
8 & 5 & 5
\end{bmatrix}
\begin{bmatrix}
3 \\
1 \\
1
\end{bmatrix}
=
\begin{bmatrix}
21 + 9 + 1 \\
12 + 0 + 0 \\
24 + 5 + 5
\end{bmatrix}
=
\begin{bmatrix}
31 \\
12 \\
34
\end{bmatrix}
\text{.}
$$
Dermed er 
$$
AB=
\begin{bmatrix}
A\textbf{b}_1 & A\textbf{b}_2
\end{bmatrix}
=
\begin{bmatrix}
59 & 31 \\
8 & 12 \\
41 & 34
\end{bmatrix}.
$$
% Eventuelt en kommentar omkring om BA er eksisterende 
\end{eks}
Da matricen $B$ inddeles i sine søjlevektorer, er grundprincippet i matrixmultiplikation at finde matrix-vektorprodukter.\\\\
En række egenskaber gælder for matrixmultiplikation, hvilket er summeret i \ref{thm:mxmulti}.
%
\begin{thm}{}{mxmulti}
Lad $A$ og $B$ være $k \times m$ matricer, $C$ være en $m \times n$ matrix, og $P$ og $Q$ være $n \times p$ matricer. Følgende udsagn er dermed sande:
\begin{enumerate}[label=(\alph*)]
\item $s(AC)=(sA)C=A(sC)$ for en vilkårlig skalar $s$.
\item $A(CP)=(AC)P$.
\item $(A+B)C=AC+BC$.
\item $C(P+Q)=CP+CQ$.
\item $I_kA=A=AI_m$.
\item Produktet af en nulmatrix og en vilkårlig matrix er altid $O$.
\item $(AC)^T=C^TA^T$.
\end{enumerate}
\end{thm}
\begin{proof}
Lad $A$ og $B$ være $k \times m$ matricer, $C$ være en $m \times n$ matrix, og $P$ og $Q$ være $n \times p$ matricer. 
\\\\
(a) Den $(i,j)$'te indgang i matrixprodukterne $s(AC)$ og $(sA)C$ har formen $sa_{i,1}c_{1,j} + sa_{i,2}c_{2,j} + \cdots + sa_{i,m}c_{m,j}$, hvilket er det samme som $a_{i,1}sc_{1,j} + a_{i,2}sc_{2,j} + \cdots + a_{i,m}sc_{m,j}$, der er en indgang i $A(sC)$. 
Da indgangene er ens, er matrixproduktet det samme i de tre tilfælde.
\\\\
(b) Bemærk, at $A$ ikke kan multipliceres med $P$, mens $C$ kan multipliceres med begge. 
Derfor kan matrixprodukterne $AC$ og $CP$ betragtes som to matricer, således at både $A(CP)$ og $(AC)P$ er $k \times p$ matricer. 
At komponenten i den $(i,j)$'te indgang i $A(CP)$ er ækvivalent med komponenten i den $(i,j)$'te indgang $(AC)P$ fås ved samme fremgangsmåde som vist i (a). 
% Måske er (b) ikke helt nok? 
\\\\
(c) Lad $\textbf{c}_j$ være en vilkårlig søjle i $C$. 
Det følger deraf, at $(A+B)\textbf{c}_j=A\textbf{c}_j+B\textbf{c}_j$ jævnfør \ref{thm:mxvpro}. 
\\\\
(d) Lad $\textbf{p}_j$ og $\textbf{q}_j$ være søjler i henholdvis $P$ og $Q$. Det følger deraf, at $C(\textbf{p}_j+\textbf{q}_j)=C\textbf{p}_j+C\textbf{q}_j$ jævnfør \ref{thm:mxvpro}. 
\\\\
(e) Da $I_k$ og $I_m$ henholdsvis korresponderer til antallet af rækker og søjler i $A$, er matrixprodukterne $I_kA$ og $AI_m$ defineret. 
Da $I_k\textbf{a}_j=\textbf{a}_j$ og $\textbf{a}_iI_m=\textbf{a}_i$, er $I_kA=A$ og $AI_m=A$, hvormed $I_kA=AI_m$.
\\\\
(f) Eftersom enhver komponent i en vilkårlig matrix multipliceret med $0$ er $0$, følger udsagnet trivielt. 
\\\\
(g) Den $(i,j)$'te indgang i $(AC)^T$ er den $(j,i)$'te indgang i $AC$. De to udtryks ækvivalens vises ved sammen fremgangsmåde som i (a).
\end{proof}