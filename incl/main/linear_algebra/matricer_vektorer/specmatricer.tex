\subsection{Udvalgte matricer og vektorer} 
% 
Hvis alle indgange i en matrix er nul, kaldes dette for en \textit{nulmatrix}, noteret $O$. 
En $m \times n$ nulmatrix noteres $O_{m,n}$.
%
Et andet særtilfælde er \textit{standardvektorerne} i $\R^n$, som er defineret ved  
% Jeg er lidt i tvivl om om der skal være et ekstra nul i længen så der ikke står 1 ... ? - Julie 
$$
\textbf{e}_1=
\begin{bmatrix}
1 \\ 
0 \\
0 \\
\vdots \\
0
\end{bmatrix}
\text{, }
\textbf{e}_2=
\begin{bmatrix}
0 \\ 
1 \\
0 \\
\vdots \\
0
\end{bmatrix}
\text{, }
\ldots
\text{, }
\textbf{e}_n=
\begin{bmatrix}
0 \\ 
0 \\
0 \\
\vdots \\
1
\end{bmatrix}
\text{. }
$$
%
Sammensættes standardvektorer i en matrix opnås en \textit{identitetsmatrix}.
%
\begin{defn}{}{}
%
En $n \times n$ \textbf{identitetsmatrix} $I_n$, hvor $n \in \Z^+$, består af alle standardvektorer $\textbf{e}_1, \textbf{e}_2, \ldots, \textbf{e}_n$ i $\R^n$, således at
$$
I_n=
\begin{bmatrix}
\textbf{e}_1 & \textbf{e}_2 & \ldots & \textbf{e}_n
\end{bmatrix}.
$$ 
\end{defn}
\noindent
%
