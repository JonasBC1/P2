\subsection{Dimension}
I forlængelse af underrum og basis kan der introduceres 
$\mathbf{dimension}$. 
%
\begin{defn}{}{dimension}
Mængden af vektorer, som udgør en basis for et givent ikke-nul undderrum $V$ i $\R^n$ betegnes som dimensionen af V, noteret dim$V$. 
%Det er dertil fordelagtigt at definere dimensionen af nulrummet i $\R^n$ til at være 0. Nødvendigt?? 
\end{defn}
\noindent
Det betyder derfor generelt, at standard basen for $\R^n$ indeholder $n$ vektorer og derfor er dim$\R^n = n$. Dertil kan man betragte bestemte underrum associeret med matricer samt deres dimensioner. Det gælder følgende for dimensionerne af underrummet for en $m \times n$ matrix, at \\ 
\begin{center}
 \begin{tabular}{||c c c||} 
 \hline
 Underrum & Indeholder rum & Dimension\\
 \hline\hline
 Søj $A$ & $\R^m$ & rang($A$)\\ 
 \hline
 Nul $A$ & $\R^n$ & null(A)= $n-\text{rang}(A)$\\
 \hline
 Ræk $A$ & $\R^n$ & rang($A$)\\
 \hline
\end{tabular}
\end{center}
\noindent
Pivotsøjlerne i enhver matrix danner en basis for dens søjlerum. Derfor er dim$\text{Søj}\text{  } A$ lig med mængden af pivotsøjler i en matrix $A$ og derfor lig med $\text{rang}(A)$. Nulrummets dimension for en matrix er givet ved nulliteten af matricen $A$, altså $n-\text{rang} (A)$. Til sidst er det gældende for rækkerummet for en matrix $A$, at dens dimension ligeledes er lig med matricens rang. Da rækkerummet og søjlerummet har samme dimension gælder følgende \\
\begin{equation}
\text{dim}(\text{Ræk}\text{  } A)=\text{dim}(\text{Søj}\text{  } A)=\text{dim}(\text{Ræk}\text{  } A^T)
\end{equation}
Det medfølger derfor, at rangen af en enhver matrix er lig med rangen af dens transponerede. 

