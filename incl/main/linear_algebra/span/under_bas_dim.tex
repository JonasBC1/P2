\subsection{Underrum}
Det kan ofte være belejligt at studere vektorer i en mindre del af $\R^n$, dertil kan der gøres brug af \textbf{underum}.

\begin{defn}{}{}
En mængde $W$ bestående af vektorer i $\R^n$ er et underrum til $\R^n$ hvis den opfylder følgende betingelser:	
\begin{enumerate}[label=(\alph*)]
	\item  $\textbf{0}\in W$.
	\item $\textbf{u}$,$\textbf{v} \in W \rightarrow \textbf{u}+\textbf{v} \in W $.
	\item $\textbf{u} \in W$ og $c$ er en skalar så er $\textbf{u}c \in W$.
\end{enumerate}
\end{defn}
\noindent
Mængden $\R^n$ er et underrum til sig selv, da summen af enhver 2 vektorer i $\R^n$ tilhører $\R^n$, enhver vektor ganget med en skalar er også i $\R^n$ og $\textbf{0}$ er i $\R^n$. Mængden $\{\textbf{0}\}$ er også et underrum til $\R^n$, og kaldes \textbf{nul-underrummet}.


\subsection{Basis}




\subsection{Dimension}