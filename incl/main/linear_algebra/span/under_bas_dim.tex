\subsection{Underrum}
Det kan ofte være belejligt at studere vektorer i en mindre del af $\R^n$, dertil kan der gøres brug af \textbf{underum}.
\begin{defn}{}{underrum}
En mængde $W$ bestående af vektorer i $\R^n$ er et underrum til $\R^n$ hvis den opfylder følgende betingelser:	
\begin{enumerate}[label=(\alph*)]
	\item  $\textbf{0}\in W$.
	\item $\textbf{u}$,$\textbf{v} \in W \rightarrow \textbf{u}+\textbf{v} \in W $.
	\item $\textbf{u} \in W$ og $c$ er en skalar så er $\textbf{u}c \in W$.
\end{enumerate}
\end{defn}
\noindent
Mængden $\R^n$ er et underrum til sig selv, da summen af enhver 2 vektorer i $\R^n$ tilhører $\R^n$, enhver vektor multipliciteret med en skalar er også i $\R^n$ og $\textbf{0}$ er i $\R^n$. Mængden $\{\textbf{0}\}$ er også et underrum til $\R^n$, og kaldes \textbf{nul-underrummet}.


\begin{thm}{}{}
Spanet af et tælleligt ikke tomt delmængde af $\R^n$ er et underrum af $\R^n$
\end{thm}
\begin{proof}
Lad spannet af en tællelig ikke tom mængde være 
$\S=\{\textbf{w}_1 , \textbf{w}_2 , \ldots , \textbf{w}_j\}$.
Eftersom at der kan opskrives en linearkombination $$0\textbf{w}_1 + 0\textbf{w}_2 + \cdots + 0\textbf{w}_j=\textbf{0},$$ så er $\textbf{0}$ i  $\S$, hvilket opfylder betingelse (a) i definition \ref{defn:underrum}.\\
Lad $\textbf{u}$ og $\textbf{v}$ tilhøre $\S$. Så er\\
$\textbf{u}=c_1\textbf{w}_1+c_2\textbf{w}_2+\cdots+c_j\textbf{w}_j$ og $\textbf{v}=k_1\textbf{w}_1+k_2\textbf{w}_2+\cdots+k_j\textbf{w}_j$, for nogle skalarer $c_1,c_2,\ldots,c_j$ og $k_1,k_2,\ldots,k_j$. Så kan det omskrives til en linearkombination
\begin{align*}
\textbf{u}+\textbf{v}&=(c_1\textbf{w}_1+c_2\textbf{w}_2+\cdots+c_j\textbf{w}_j)+(k_1\textbf{w}_1+k_2\textbf{w}_2+\cdots+k_j\textbf{w}_j)\\
&=(c_1+k_1)\textbf{w}_1+(c_2+k_2)\textbf{w}_2+\cdots+(c_j+k_j)\textbf{w}_j
\end{align*}
hvilket viser at $\textbf{u}+\textbf{v}\in \S$, hvilket opfylder (b) i definition \ref{defn:underrum}.\\
For enhver skalar $h$, gælder
\begin{align*}
h\textbf{u}&=h(c_1\textbf{w}_1+c_2\textbf{w}_2+\cdots+c_j\textbf{w}_j)\\
&=(h_1c_1)\textbf{w}_1+(h_2c_2)\textbf{w}_2+\cdots+(h_jc_j)\textbf{w}_j,
\end{align*}
og eftersom $h_1c_1,h_2c_2,\ldots,h_jc_j$ bare er skalarer så er $h\textbf{u}\in \S$ hvilket opfylder (c) i definition \ref{defn:underrum}.
Hermed er det bevist at spannet af delmængde af $\R^n$ er et underrum til $\R^n$.
\end{proof}

\subsubsection{Underrum for matricer}
I forbindelse med underrum er det nogle udvalgte typer af underrum som er relevante i forbindelse med matricer. Den første der vil blive nævnt er \textbf{nulrummet}.
\begin{defn}{}{}
Nulrummet til en matrix $A$ er løsningsmængden til matrixligningen $A\textbf{x}=\textbf{0}$. 
\end{defn}
Et eksempel nulrummet til en matrix ses i eksempel \ref{nulrum}

\begin{eks}\label{nulrum}
Nulrummet til matrixen
$$A=\begin{bmatrix}
2 & 3 & 2\\
1 & -1 & 5
\end{bmatrix}$$
er løsningssættet til matrixligningen $A\textbf{x}=\textbf{0}$
som kan opskrives som ligningssystemet
\begin{align*}
2x_1+3x_2+2x_3&=0\\
1x_1-x_2+5x_3&=0
\end{align*}
\end{eks} 
\begin{thm}{}{}
Hvis A er en $m\times n$ matrice, så er nulrummet af A et underrum af $\R^n$.
\end{thm}
\begin{proof}
Eftersom A er en $m\times n$ matrice så er vektorerne i nulrummet i $\R^n$, da det er løsningerne til $A\textbf{x}=\textbf{0}$. $\textbf{0}$ er en del af nulrummet da $A\textbf{0}=\textbf{0}$. Hvis $\textbf{u}$ og $\textbf{v}$ er i nulrummet så er $A\textbf{u}=\textbf{0}$ og $A\textbf{v}=\textbf{0}$. Deraf har vi at
$$A(\textbf{u}+\textbf{v})=A\textbf{u}+A\textbf{v}=\textbf{0}+\textbf{0}=\textbf{0}.$$
Deraf haves at $\textbf{u}+\textbf{v}$ er i nulrummet. For alle skalarer $c$ haves at
$$A(c\textbf{u})=c(A\textbf{u})=c\textbf{0}=\textbf{0}.$$
Dette viser at $c\textbf{u}$ er i nulrummet.
Dette vises at nulrummet af A er et underrum af $\R^n$
\end{proof}
\noindent
En anden type underrum er \textbf{rækkerum} og \textbf{søjlerum}.
\begin{defn}{}{}
Rækkerum er spannet som kommer af rækkerne i en matrice.\\
Søjlerum er spannet af søjlerne i en matrice.
\end{defn}
\noindent
Speciele underrum er også relevant i forbindelse med lineær afbildning.
Billedmængden af en lineær afbildning er det samme som søjlerummet af dens standard matrix.

\subsection{Basis}
