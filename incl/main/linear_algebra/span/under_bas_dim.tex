\section{Underrum}
Det kan ofte være belejligt at undersøge vektorer i en begrænset del af $\R^n$. Den begrænsede del kaldes et \textit{underum}.
%
\begin{defn}{}{underrum}
En mængde $W$, bestående af vektorer i $\R^n$, er et \textbf{underrum} af $\R^n$, hvis den opfylder følgende betingelser:	
\begin{enumerate}[label=(\alph*)]
	\item $\textbf{0}\in W$.
	\item $\textbf{u}$,$\textbf{v} \in W \rightarrow \textbf{u}+\textbf{v} \in W $.
	\item $\textbf{u} \in W$ og $c$ er en skalar, så er $\textbf{u}c \in W$.
\end{enumerate}
\end{defn}
\noindent
Mængden $\R^n$ er et underrum til sig selv, da summen af to vilkårlige vektorer i $\R^n$ tilhører $\R^n$.
Enhver vektor multipilciteret med en skalar er i $\R^n$, og $\textbf{0}$ er ligeledes i $\R^n$. 
Mængden $\{\textbf{0}\}$ er ligeså et underrum til $\R^n$, kaldet \textit{nul-underrummet}.
%
%
\begin{thm}{}{}
Spannet af en endelig ikke-tom delmængde af $\R^n$ er et underrum af $\R^n$.
\end{thm}
\begin{proof}
Lad $\S$ være spannet af en endelig ikke-tom mængde 
$\{\textbf{w}_1 , \textbf{w}_2 , \ldots , \textbf{w}_j\}$.
Eftersom der kan opskrives en linearkombination 
$$0\textbf{w}_1 + 0\textbf{w}_2 + \cdots + 0\textbf{w}_j=\textbf{0},$$ 
så er $\textbf{0}$ i  $\S$, hvilket opfylder betingelse (a) i \ref{defn:underrum}.
\\\\
%
Lad $\textbf{u}$ og $\textbf{v}$ tilhøre $\S$. Så er
%
\begin{align*}
\textbf{u}&=c_1\textbf{w}_1+c_2\textbf{w}_2+\cdots+c_j\textbf{w}_j, \\ 
\textbf{v}&=k_1\textbf{w}_1+k_2\textbf{w}_2+\cdots+k_j\textbf{w}_j,
\end{align*}
%
for skalarerne $c_1,c_2,\ldots,c_j$ og $k_1,k_2,\ldots,k_j$. 
Dette kan omskrives til linearkombinationen
%
\begin{align*}
\textbf{u}+\textbf{v}&=(c_1\textbf{w}_1+c_2\textbf{w}_2+\cdots+c_j\textbf{w}_j)+(k_1\textbf{w}_1+k_2\textbf{w}_2+\cdots+k_j\textbf{w}_j)\\
&=(c_1+k_1)\textbf{w}_1+(c_2+k_2)\textbf{w}_2+\cdots+(c_j+k_j)\textbf{w}_j,
\end{align*}
hvilket viser, at $\textbf{u}+\textbf{v}\in \S$, hvormed (b) i \ref{defn:underrum} er opfyldt.
\\\\
For enhver skalar $h$ gælder, at
\begin{align*}
h\textbf{u}&=h(c_1\textbf{w}_1+c_2\textbf{w}_2+\cdots+c_j\textbf{w}_j)\\
&=(h_1c_1)\textbf{w}_1+(h_2c_2)\textbf{w}_2+\cdots+(h_jc_j)\textbf{w}_j,
\end{align*}
og eftersom $h_1c_1,h_2c_2,\ldots,h_jc_j$ er skalarer, så er $h\textbf{u}\in \S$, hvilket opfylder (c) i \ref{defn:underrum}.
Hermed er det bevist, at spannet af en delmængde af $\R^n$ er et underrum til $\R^n$.
\end{proof}

\subsection{Underrum for matricer}
Der findes bestemte typer af underrum, som er relevante i forbindelse med matricer. Den første type, der beskrives, er \textit{nulrummet}.
%
\begin{defn}{}{}
\textbf{Nulrummet} til en matrix $A$, noteret $\text{nrum}(A)$, er løsningsmængden til matrixligningen $A\textbf{x}=\textbf{0}$. 
\end{defn}
%
\noindent
Et eksempel på nulrummet til en matrix ses i \ref{nulrum}.

\begin{eks}
\label{nulrum}
Lad 
$$A=\begin{bmatrix}
2 & 3 & 2\\
1 & -1 & 5
\end{bmatrix}.$$
Matrixligningen $A\textbf{x}=\textbf{0}$,
kan opskrives som ligningssystemet
\begin{align*}
2x_1+3x_2+2x_3&=0\\
1x_1-x_2+5x_3&=0.
\end{align*}
Nulrummet, $\text{nrum}(A)$, til matricen er dermed de vektorer $\textbf{x}$, som opfylder ligningssystemet.
\end{eks} 
%
\begin{thm}{}{}
Hvis $A$ er en $m\times n$ matrix, så er nulrummet af $A$ et underrum af $\R^n$.
\end{thm}
%
\begin{proof}
Eftersom $A$ er en $m\times n$ matrix, så er vektorerne i nulrummet i $\R^n$, da det er løsningerne til $A\textbf{x}=\textbf{0}$. $\textbf{0}$ er en del af nulrummet, da $A\textbf{0}=\textbf{0}$. 
Hvis $\textbf{u}$ og $\textbf{v}$ er i nulrummet, så er $A\textbf{u}=\textbf{0}$ og $A\textbf{v}=\textbf{0}$. 
Deraf har vi, at
%
$$A(\textbf{u}+\textbf{v})=A\textbf{u}+A\textbf{v}=\textbf{0}+\textbf{0}=\textbf{0}.$$
%
Deraf haves, at $\textbf{u}+\textbf{v}$ er i nulrummet. 
For alle skalarer $c$ haves, at
$$A(c\textbf{u})=c(A\textbf{u})=c\textbf{0}=\textbf{0}.$$
Dette viser, at $c\textbf{u}$ er i nulrummet, hvilket viser, at nulrummet af $A$ er et underrum af $\R^n$
\end{proof}
\\
\noindent
Andre relevante typer underrum er \textit{rækkerum} og \textit{søjlerum}.
%
\begin{defn}{}{}
%Til nedenstående "Rækkerum er spannet af rækkerne i en matrix" - right? 
\textbf{Rækkerum} er spannet, som kommer af rækkerne i en matrix, og noteres $\text{ræk}(A)$.
\\
\textbf{Søjlerum} er spannet af søjlerne i en matrix, og noteres $\text{søj}(A)$.
\end{defn}
\noindent
Underrum med særegne karakteristika er også relevante i forbindelse med lineær afbildning.
Værdimængden af en lineær afbildning er det samme som søjlerummet af dens standardmatrix.
%
\subsection{Basis}
Mens et underrum kan beskrives med en given mængde vektorer, så er det fordelagtigt at beskrive det med så få vektorer som muligt. 
Denne mængde kaldes \textit{basis} for underrummet.
%
\begin{defn}{}{basisunderrum}
Lad $\mathcal{P}$ være et ikke-nul-underrum af $\R^n$. 
En \textbf{basis} for $\mathcal{P}$ er en lineært uafhængig mængde af generatorer for $\mathcal{P}$.
\end{defn}
\noindent
Eksempelvis kan en mængde af standardvektorer $\{\textbf{e}_1,\textbf{e}_2,\ldots,\textbf{e}_n\}$ bruges som basis for $\R^n$.
Denne type basis kaldes for \textit{standard basis}.
Der findes dog flere mulige baser for det samme underrum.
\\\\
%
Pivotsøjlerne i en matrix udgør basis for dens søjlerum. Dette følger af bevis \ref{thm:pivotu}.
\\\\
\begin{thm}{}{}
Lad $V$ være et ikke nul underrum af $\R^n$. Enhver to baser for $V$ indeholder det samme antal vektorer.
\end{thm}
\begin{proof}
Antag at $\{\textbf{u}_1,\textbf{u}_2,\ldots,\textbf{u}_k\} $ og $\{\textbf{v}_1,\textbf{v}_2,\ldots,\textbf{v}_p\} $ er baser for $V$ og lad $A=[\textbf{u}_1 \text{ } \textbf{u}_2 \text{ }  \ldots \text{ } \textbf{u}_k] $ og $B=[\textbf{v}_1 \text{ } \textbf{v}_2 \text{ }  \ldots \text{ } \textbf{v}_k] $.
Efter som $\{\textbf{u}_1,\textbf{u}_2,\ldots,\textbf{u}_k\} $ er en mængde af generatorer for $V$, så eksisterer der vektorer $\textbf{c}_i$ i $\R^k$, for $I=1,2,\ldots,p$ sådan at det opfylder $A\textbf{c}_i=\textbf{v}_i$.
Lad $C=[\textbf{c}_1 \text{ } \textbf{c}_2 \text{ } \ldots \text{ } \textbf{c}_p]$.
$C$ er dermed en $ k \text{ } \times \text{ } p$ matrix, hvorom det gælder at $AC=B$.
Antag nu er der for nogle vektorer $\textbf{x}$ i $\R^p$ gælder $C\textbf{x}=\textbf{0}$. 
Så er $B\textbf{x}=AC\textbf{x}=\textbf{0}$. 
Men da søjlerne af $B$ er lineært uafhængige følger det af sætning \ref{thm:mxlinuaf} at $\textbf{x}=\textbf{0}$.
Af samme sætning haves det så at søjlerne af $C$ er lineært uafhængige vektorer i $\R^k$.
Da en mængde af vektorer i $\R^k$ med mere en k vektorer er lineært afhængig, haves det at $p\leq k$. 
Vendes rollerne af baserne om, haves $k \leq p$. 
Deraf konkluderes der at $k=p$, og dermed at to baser indeholder samme antal vektorer. 
% egenskab (2) for lineært afhængige og uafhængige mængder, se side \pageref{egenskab_lin}.
\end{proof}