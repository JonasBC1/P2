\section{Span}
% 1.6
\textit{Spannet} er den del af $\R^n$, som linearkombinationen af en given mængde vektorer dækker over. 
%
\begin{defn}{}{}
%
For en ikke-tom mængde af vektorer i $\R^n$ $\S = \{\mathbf{u}_1, \mathbf{u}_2 , \ldots , \mathbf{u}_k \}$ er \textbf{spannet} af $\S$ mængden af alle linearkombinationer af $\mathbf{u}_1, \mathbf{u}_2 , \ldots , \mathbf{u}_k$. 
Denne mængde noteres $\text{span} \{ \S \}$ eller $\text{span}\{ \mathbf{u}_1, \mathbf{u}_2 , \ldots , \mathbf{u}_k \}$.
%
\end{defn}
%
\noindent
Mængden, som indeholder vektorerne, der udgør spannet, kaldes for \textit{mængden af generatorer}.
\\
%
%%%%%%%%%%%%%%%%%%%%%%%%%%%%%%%%%%%%%%%%%%%%%%%%%%%%%%%%%%%%%%%%%%%%%%%%
%
\begin{eks}
%
Givet mængderne af vektorer $\S_1$ og $\S_2$, kan $\text{span}\{\S_1 \}$ og $\text{span}\{\S_2 \}$ findes.
%
\begin{align*}
\S_1 &= \left\{
\begin{bmatrix}
           -1 \\
           1 \\
\end{bmatrix}
,
\begin{bmatrix}
           2 \\
           -2 \\
\end{bmatrix}
\right\}
= \left\{ \mathbf{u}_1, \mathbf{u}_2 \right\}
\\
\S_2 &= \left\{
\begin{bmatrix}
           -1 \\
           1 \\
\end{bmatrix}
,
\begin{bmatrix}
           2 \\
           -2 \\
\end{bmatrix}
,
\begin{bmatrix}
           1 \\
           2 \\
\end{bmatrix}
\right\}
= \left\{ \mathbf{u}_1, \mathbf{u}_2,  \mathbf{u}_3 \right\}
\end{align*}
%
Spannet for $\S_1$ bliver derfor:
%
\begin{align*}
\text{span}\{\S_1 \} =
\left\{ x_1 
\begin{bmatrix}
           -1 \\
           1 \\
\end{bmatrix} 
+ x_2
\begin{bmatrix}
           2 \\
           -2 \\
\end{bmatrix}
\right\}.
\end{align*}
%
Som det fremgår geometrisk på figur \ref{span_eks}, er $\text{span}\{\S_1 \}$ en ret linje. 
Eftersom $\mathbf{u}_2$ er en linearkombination af $\mathbf{u}_1$, kan dette reduceres til  
%
\begin{align*}
\text{span}\{\S_1 \} =
\left\{ x_1 
\begin{bmatrix}
           -1 \\
           1 \\
\end{bmatrix} 
\right\}.
\end{align*}
%
Mængden $\S_2$ kan derfor reduceres til $\S_2=\{ \mathbf{u}_1, \mathbf{u}_3 \}$. 
Spannet for $\S_2$ er derfor
%
\begin{align*}
\text{span}\{\S_2 \} = 
\left\{ x_1 
\begin{bmatrix}
           -1 \\
           1 \\
\end{bmatrix} 
+ x_2
\begin{bmatrix}
           1 \\
           2 \\
\end{bmatrix}
\right\}.
\end{align*}
%
Eftersom spannet dækker alle punkter i planen er $\text{span}\{\S_2\}=\R^2.$
% Eventuelt planet xD
Dette gør den, da det er muligt at besøge alle punkter ved at indføre passende værdier for $x_1$ og $x_2$.
%
% Skal sættes rigtigt inden aflevering!
\begin{figure}[h!]
%
\centering
\begin{tikzpicture} [scale=1.4]
  \draw[thin,gray!40] (-2.5,-2.5) grid (2.5,2.5);
  \draw[<->] (-2.5,0)--(2.5,0) node[right]{$x$};
  \draw[<->] (0,-2.5)--(0,2.5) node[above]{$y$};
  \draw[line width=2pt,AAUblue1,-stealth](0,0)--(-1,1) node[anchor=south west]{$\boldsymbol{\mathbf{u}_1}$};
  \draw[line width=2pt,AAUred,-stealth](0,0)--(2,-2) node[anchor=north east]{$\boldsymbol{\mathbf{u}_2}$};
  \draw[line width=2pt,AAUgreen,-stealth](0,0)--(1,2) node[anchor=north east]{$\boldsymbol{\mathbf{u}_3}$};
\end{tikzpicture}
%
\caption{Grafisk repræsentation af vektorerne $\mathbf{u}_1, \mathbf{u}_2,  \mathbf{u}_3$.}
\label{span_eks}
\end{figure}
%
\end{eks}
%
%%%%%%%%%%%%%%%%%%%%%%%%%%%%%%%%%%%%%%%%%%%%%%%%%%%%%%%%%%%%%%%%%%%%%%%%
%
Rækkereduktionsalgoritmen kan bruges til at afgøre, hvorvidt en given vektor er en del af et givet span.
Dette betyder, at rækkereduktionsalgoritmen kan bruges til at undersøge om en vektor er en linearkombination af en vektormængde.
\\
%
%%%%%%%%%%%%%%%%%%%%%%%%%%%%%%%%%%%%%%%%%%%%%%%%%%%%%%%%%%%%%%%%%%%%%%%%
%
\begin{eks}
%
Det skal undersøges, hvorvidt vektoren $\mathbf{v}$ er i spannet for $\S_1$. 
Givet
\begin{align*}
\mathbf{v}= \begin{bmatrix}
           -1 \\
           2 \\
           3 \\
\end{bmatrix} 
\text{ og }
\text{span}\{\S_1 \} =
\left\{ 
\begin{bmatrix}
           1 \\
           2 \\
           1 \\
\end{bmatrix} 
,
\begin{bmatrix}
           1 \\
           1 \\
           0 \\
\end{bmatrix}
,
\begin{bmatrix}
           1 \\
           -1 \\
           -2 \\
\end{bmatrix}
\right\},
\end{align*}
%
opskrives totalmatricen
%
\begin{align*}
[A \mid \textbf{v}]=&
\begin{blockarray}{ccccc}
x_1 & x_2 & x_3 & v \\
\begin{block}{[ccc|c]c}
  1 & 1 & 1 & -1 \\
  2 & 1 & -1 & 2 \\
  1 & 0 & -2 & 3 \\
\end{block}
\end{blockarray}.
\end{align*}
\newpage
\noindent
%
Totalmatricen omskrives nu til trappeform.
%
\begin{align*}
\xrightarrow[R_3 \rightarrow R_3-R_1]{R_2 \rightarrow R_2-2R_1}&
\begin{blockarray}{ccccc}
x_1 & x_2 & x_3 & v \\
\begin{block}{[ccc|c]c}
  1 & 1 & 1 & -1 \\
  0 & -1 & -3 & 4 \\
  0 & -1 & -3 & 4 \\
\end{block}
\end{blockarray} \\
%
\xrightarrow{R_3 \rightarrow R_3-R_2}&
\begin{blockarray}{ccccc}
x_1 & x_2 & x_3 & v \\
\begin{block}{[ccc|c]c}
  \hlight{1} & 1 & 1 & -1 \\
  0 & \hlight{-1} & -3 & 4 \\
  0 & 0 & 0 & 0 \\
\end{block}
\end{blockarray} 
\end{align*}
%
Da ligningssystemet er konsistent haves, at $\mathbf{v}$ er i $\text{span}\{\S \}$.
%
\end{eks}
%
%%%%%%%%%%%%%%%%%%%%%%%%%%%%%%%%%%%%%%%%%%%%%%%%%%%%%%%%%%%%%%%%%%%%%%%%
%
\ref{thm:spaneqv} beskriver ækvivalente udsagn om span.
% 
\begin{thm}{}{spaneqv}
%
De følgende udsagn om en $m \times n$ matrix $A$ er ækvivalente:
%
\begin{enumerate}[label=(\alph*)]
\item Spannet af søjlerne i $A$ er $\R^m$.
\item Matricen $A\mathbf{x}=\mathbf{b}$ har mindst én løsning for alle $b$ i $\R^m$.
\item Rangen af $A$ er antallet af rækker $m$.
\item Den reducerede trappeform har ingen nulrækker.
\item Der er pivotindgang i hver række i $A$. 
\end{enumerate}
%
\end{thm}
%
%
\begin{proof}
%
Udsagn (a) og (b) er ækvivalente, eftersom det er en forudsætning, at der kan skabes en linearkombination af søjlerne i $A$ for ethvert $\mathbf{b}$ i $\R^m$, da det netop er definitionen af spannet, der dækker $\R^m$. 
%
Hvis der ikke er nogle nulrækker i den reducerede trappeform, følger det, at $\text{rang}\{A\} = m$, hvorfor (c) og (d) er ækvivalente. 
Eftersom matricen er på reduceret trappeform, er denne derfor ækvivalent med (e).
\\\\
Det skal nu bevises at (b) og (c) er ækvivalente. 
Lad $A_R$ være den reducerede trappeform af $A$ og
%
\begin{align*}
\mathbf{\mathbf{e}}_m = \begin{bmatrix}
		0 \\
        \vdots \\
        0 \\
        1 
\end{bmatrix}
\end{align*}
%
i $\R^m$. 
Gennem rækkereduktionsalgoritmen kan $A$ transformeres til $A_R$. 
Da disse rækkeoperationer er reversible, følger det, at der findes en sekvens af rækkeoperationer, som kan transformere $A_R$ til $A$.
Hvis disse operationer udføres på totalmatricen 
$[A_R \mid \mathbf{e}_m]$ 
for at konstruere matricen 
$[A \mid \mathbf{d}]$, 
$\mathbf{d} \in \R^n$, 
så følger det, at ligningssystemet $A\mathbf{x}=\mathbf{
d}$ er ækvivalent med $A_R\mathbf{x}=\mathbf{e}_m$.
\\\\
Hvis (b) er sand, følger det, at de to ligningssystemer er konsistente. 
Det følger derfor af \ref{thm:konsistens}, at den sidste række i $A$ og $A_R$ ikke kan være en nulrække, da dette ellers medører, at $\textbf{b} = \mathbf{e}_m$ giver pivotindgang i sidste søjle, som medfører, at ligningssystemet er inkonsistent.
Dermed er $\text{rang}(A)=m$, hvilket leder til (c).
% Dette kan eventuelt forklares dybere xD
\\\\
Antag, at (c) er sand. 
Lad $[A_R \mid \mathbf{c}]$ 
være den reducerede trappeform af 
$[A \mid \mathbf{b}]$.
Eftersom $A$ har rangen $m$, følger det, at der ikke findes en nulrække i $A_R$.
Derfor kan $[A_R \mid \mathbf{c}]$ ikke indeholde en nulrække, og der er derfor ikke pivot i sidste søjle.
Det følger derfor jævnfør \ref{thm:konsistens}, at $A\mathbf{x}=\mathbf{b}$ er konsistent for alle $\mathbf{b}$, hvilket beviser, at (b) og (c) er ækvivalente.
%
\end{proof}
\\
%
%%%%%%%%%%%%%%%%%%%%%%%%%%%%%%%%%%%%%%%%%%%%%%%%%%%%%%%%%%%%%%%%%%%%%%%%
%
\ref{thm:span} beskriver, hvornår en vektor tilhører et span.
%
\begin{thm}{}{span}
%
Lad $\S =  \{ \mathbf{u}_1, \mathbf{u}_2 , \ldots , \mathbf{u}_k \}$ være en mængde vektorer i $\R^n$, og lad $\mathbf{v}$ være en vektor i $\R^n$.
Så er $\text{span}\{ \mathbf{u}_1, \mathbf{u}_2 , \ldots , \mathbf{u}_k , \mathbf{v}\} =\text{span}\{\S\}$, hvis og kun hvis $\mathbf{v}$ tilhører $\text{span}\{\S\}$.
%
\end{thm}
%
%
\begin{proof}
%
Antag, at $\mathbf{v}$ er i $\text{span}\{\S\}$. Så er $\mathbf{v}$ en linearkombination af $\S$, hvilket kan opskrives som $\mathbf{v}=c_1\mathbf{u}_1+c_2\mathbf{u}_2+ \ldots + c_k\mathbf{u}_k$, hvor $c_1,c_2,\ldots, c_k$ er skalarer. 
Hvis en vektor $\mathbf{w}$ er i $\text{span}\{ \mathbf{u}_1, \mathbf{u}_2 , \ldots , \mathbf{u}_k , \mathbf{v}\}$, 
kan denne opskrives $\mathbf{w}=c_1\mathbf{u}_1+c_2\mathbf{u}_2+ \ldots + c_k\mathbf{u}_k+b\mathbf{v}$, hvor $c_1,c_2,\ldots, c_k, b$ er skalarer.
Ved substitution af $\mathbf{v}$ med $\{ \mathbf{u}_1, \mathbf{u}_2 , \ldots , \mathbf{u}_k , \mathbf{v}\}$ vides det, at $\mathbf{w}$ kan opskrives som en linearkombination af $\S$.
Det gælder endvidere, at alle linearkombinationer i $\S$ også kan dannes fra vektorerne $\mathbf{u}_1,\mathbf{u}_1, \mathbf{u}_2 , \ldots , \mathbf{u}_k , \mathbf{v}$, hvis $\mathbf{v}$ multipliceres med skalaren $0$.
Det følger derfor, at de to mængder har samme span. 
\\\\
%
Antag nu, at $\textbf{v}$ ikke er i $\text{span}\{\S\}$. 
Det vil stadig gælde, at $\mathbf{v}$ er i spannet af 
$$\{ \mathbf{u}_1,\mathbf{u}_1, \mathbf{u}_2 , \ldots , \mathbf{u}_k , \mathbf{v}\},$$ da 
$\mathbf{v}=0\mathbf{u}_1+0\mathbf{u}_2+ \ldots + 0\mathbf{u}_k+1\mathbf{v}$.
Derfor er
$$\text{span}\{ \mathbf{u}_1, \mathbf{u}_2 , \ldots , \mathbf{u}_k \}
\neq
\text{span}\{ \mathbf{u}_1, \mathbf{u}_2 , \ldots , \mathbf{u}_k , \mathbf{v}\},$$ 
da mængderne ikke er ækvivalente, idet kun én af dem indeholder $\mathbf{v}$. 
%
\end{proof}
\\
%
%%%%%%%%%%%%%%%%%%%%%%%%%%%%%%%%%%%%%%%%%%%%%%%%%%%%%%%%%%%%%%%%%%%%%%%%
%
Som det fremgår af ovenstående, gælder det derfor, at enhver løsning i et lineært optimeringsproblem skal findes i spannet for vektorerne i ligningssystemet. 
%