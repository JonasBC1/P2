\section{Løsning til lineær ligningssystemer}


\subsection{Elementære rækkeopperationer}
Der eksisterer tre opperationer som kan udføres på rækker i et lineært ligningssystem.
Dette gøres mest når systemet er skrevet som en matrice.
De er ombytning, udskiftning og skalaring.

\begin{defn}{}{element}
\textbf{Ombytning} af rækker, bytter rundt på pladserne af to rækker og noteres
\begin{align*}
A \xrightarrow{R_i \leftrightarrow R_h} B, 
\end{align*}
hvor $A$ er matricen før opperationen, $B$ er matricen efter opperationen og $R_h$ og $R_i$ er rækker i $A$.\\
\textbf{Skalaring} betyder at en bestemt række skaleres med en ikke nul værdi, og noteres
\begin{align*}
A \xrightarrow{R_i \rightarrow cR_i} B, 
\end{align*}
\textbf{Udskiftning} af rækker betyder at en række udskiftes med rækken selv plus en skalaring af en anden række og noteres
\begin{align*}
A \xrightarrow{R_i \rightarrow R_i + cR_h} B
\end{align*}

\end{defn}

Som eksempel på elementære rækkeoperationer ses eksempler på dem i eksempel \thechapter .\ref{eks1}

\begin{eks}\label{eks1}
\begin{align*}
\begin{bmatrix}
5 & 5 & 3 \\
2 & 2 & 1\\
3 & 4 & 8
\end{bmatrix}
&\xrightarrow{R_1 \leftrightarrow R_3}
\begin{bmatrix}
3 & 4 & 8\\
2 & 2 & 1\\
5 & 5 & 3
\end{bmatrix}\\
%
%
\begin{bmatrix}
3 & 4 & 8\\
2 & 2 & 1\\
5 & 5 & 3
\end{bmatrix}
&\xrightarrow{R_2 \rightarrow -2R_2}
\begin{bmatrix}
3 & 4 & 8\\
-4 & -4 & -2\\
5 & 5 & 3
\end{bmatrix}\\
\begin{bmatrix}
3 & 4 & 8\\
-4 & -4 & -2\\
5 & 5 & 3
\end{bmatrix}
&\xrightarrow{R_3 \rightarrow 3R_1+R_3}
\begin{bmatrix}
3 & 4 & 8\\
-4 & -4 & -2\\
14 & 17 & 27
\end{bmatrix}
\end{align*}
\end{eks}


\subsection{Trappeform og reduceret trappeform}
En række kaldes for en nulrække hvis alle indgange er nul, og en ikke nul række hvis der bare er en indgang der er forskellig fra nul.
Dette bruges til definitionen af trappeform.
\begin{defn}{}{}
En matrix er på \textbf{trappeform}, hvis den opfylder følgende krav.
\itemize
\item Enhver ikke nul række ligger over alle nul rækker
\item Den første ikke nul indgang i en ikke nul række ligger i en søjle til højre for første ikke nul indgange i forrige række.
\item Hvis en søjle har den første ikke nul indgang i en række, så er alle indgangene i søjlen under den nul.
\end{defn}

\begin{eks}
Det kan ses på de to følgende matricer at $A$ er på trappeform, mens $B$ ikke er på trappeform, da den første indgang i række fire ikke er en nul indgang. Hvis der laves en række ombytning på række to og fire, så ville matrix $B$ også være på trappeform
\begin{align*}
A=
\begin{bmatrix}
2 & 4 & 8 & 2\\
0 & 5 & -2 & 5\\
0 & 0 & 2 & 7\\
0 & 0 & 0 & 0
\end{bmatrix}
\text{ }
B=
\begin{bmatrix}
2 & 4 & 8 & 2\\
0 & 5 & -2 & 5\\
0 & 4 & 2 & 7\\
2 & 0 & 0 & 0
\end{bmatrix}
\end{align*}

\end{eks}

\begin{defn}{}{}
En matrix er på \textbf{reduceret trappeform}, hvis den opfylder følgende krav.
\itemize
\item Hvis en søjle har den første ikke nul indgang i en række, så er alle indgange i søjlen nul.
\item Den første indgang i hver ikke nul række er 1. 
\end{defn}

\begin{eks}
Det kan ses på følgende matrix, at $A$ er på reduceret trappeform.
\begin{align*}
A=
\begin{bmatrix}
1 & 0 & 5 & 0 & 0\\
0 & 1 & 3 & 0 & 0\\
0 & 0 & 0 & 1 & 0\\
0 & 0 & 0 & 0 & 1
\end{bmatrix}
\end{align*}
\end{eks}

\subsection{Pivotindgange}
\begin{defn}{}{}
Pivotindgange betegnes som den første ikke nul indgang i enhver række i en matrix på trappeform. 
\end{defn}

\begin{align*}
A=
\begin{blockarray}{ccccc}
x_1 & x_2 & x_3 & x_4 & b \\
\begin{block}{[cccc|c]}
1 & 0 & 5 & 0 & 0\\
0 & 1 & 3 & 0 & 0\\
0 & 0 & 0 & 1 & 0\\
0 & 0 & 0 & 0 & 1\\
\end{block}
\end{blockarray}
\end{align*}

\begin{align*}
x_1+5x_3&=0\\
x_2+3x_3&=0\\
x_4&=0\\
0&=1
\end{align*}