\begin{defn}{}{}
En funktion $f:\R^n \rightarrow \R^m$ siges at være \textbf{surjektiv}, hvis dens værdimængde er hele $\R^m$, tilsvarende at alle vektorer i $\R^m$ er en afbildning af mindst én vektor i $\R^n$.\\
Funktionen siges at være \textbf{injektiv}, hvis hver entydig vektor i $\R^n$ har en entydig vektor i $\R^m$, som afbildning.\\
Hvis en funktion er både injektiv og surjektiv, så er funktionen \textbf{bijektiv}.
\end{defn}
%
% EKSEMPEL! tilføjes 
%
\subsection{Invertibilitet af funktioner}
%
Bemærk, at notationen for inverse funktioner $f^{-1}$ ikke skal forstås som $\dfrac{1}{f}$, men blot notation, der betegner den inverse funktion af $f$.
Givet en funktion $T:\R^n \rightarrow \R^n$, som er en lineær afbildning med standardmatricen $A$, så er $T$ invertibel, hvis og kun hvis $A$ er invertibel. 
Lad A være en $n \times n$ invertibel matrix. 
Så haves for alle $\textbf{v}$, at
$$T_AT_{A^-1}(\textbf{v})=T_A(T_{A^-1}(\textbf{v}))=T_A(A^{-1}\textbf{v})=A(A^{-1}\textbf{v})=AA^{-1}\textbf{v}=I_n\textbf{v}=\textbf{v}.$$ Dertil er
$$T_{A^-1}T_A(\textbf{v})=\textbf{v}.$$
%
Dette medfører $T^{-1}=T_{A^{-1}}$. 
Som følge deraf er $T^{-1}$ linear og dens standard matrix er $A^{-1}$.
%
\begin{eks}
Givet matricen 
\begin{equation*}
A=
\begin{bmatrix}
1 & 1 & 1\\
1 & 2 & 2\\
1 & 3 & 2
\end{bmatrix}
\end{equation*}  
fra eksempel \ref{eks:fisk5}, sådan at:
\begin{equation*}
T_A
\left(
\begin{bmatrix}
x_1\\
x_2\\
x_3
\end{bmatrix}
\right)
=
\begin{blockarray}{c}
\begin{block}{[c]}
x_1+x_2+x_3\\
x_1+2x_2+2x_3\\
x_1+3x_2+2x_3\\
\end{block}
\end{blockarray}.
\end{equation*}
Er den inverse funktion:
\begin{equation*}
T_{A}^{-1}\left(
\begin{bmatrix}
x_1\\
x_2\\
x_3
\end{bmatrix}
\right)=
\begin{blockarray}{c}
\begin{block}{[c]}
2x_1+-1x_2\\
-x_2+x_3\\
-x_1+2x_2-x_3\\
\end{block}
\end{blockarray}.
\end{equation*}
\end{eks}