
\begin{defn}{}{}
En funktion $f:\R^n \rightarrow \R^m$ siges at være \textbf{surjektiv} hvis dens billedmængde er hele $\R^m$, altså at alle vektorer i $\R^m$ er en afbildning af mindst en vektor i $\R^n$.\\
Funktionen siges at være \textbf{injektiv} hvis hver entydig vektor i $\R^n$ har en entydig vektor i $\R^m$ som afbildning.\\
Hvis en funktion er både injektiv og surjektiv så er funktionen \textbf{bijektiv}.
\end{defn}

\subsection{Invertibilitet af funktioner}



\begin{eks}
Givet matricen 
\begin{equation*}
A=
\begin{bmatrix}
1 & 1 & 1\\
1 & 2 & 2\\
1 & 3 & 2
\end{bmatrix}
\end{equation*}  fra eksempel ''mangler ref'', sådan at:
\begin{equation*}
T_A
\left(
\begin{bmatrix}
x_1\\
x_2\\
x_3
\end{bmatrix}
\right)
=
\begin{blockarray}{c}
\begin{block}{[c]}
x_1+x_2+x_3\\
x_1+2x_2+2x_3\\
x_1+3x_2+2x_3\\
\end{block}
\end{blockarray}
\end{equation*}
Er den inverse funktion:
\begin{equation*}
T_{A}^{-1}\left(
\begin{bmatrix}
x_1\\
x_2\\
x_3
\end{bmatrix}
\right)=
\begin{blockarray}{c}
\begin{block}{[c]}
2x_1+-1x_2\\
-x_2+x_3\\
-x_1+2x_2-x_3\\
\end{block}
\end{blockarray}
\end{equation*}
\end{eks}