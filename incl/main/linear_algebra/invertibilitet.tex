%
% EKSEMPEL! tilføjes 
%
\subsection{Invertibilitet af funktioner}
%
En to tresko - indlednende tekst.
%
\begin{defn}{}{}
En funktion $f:\R^n \rightarrow \R^m$ siges at være \textbf{surjektiv}, hvis dens værdimængde er hele $\R^m$, tilsvarende at alle vektorer i $\R^m$ er en afbildning af mindst én vektor i $\R^n$.
Funktionen siges at være \textbf{injektiv}, hvis hver entydig vektor i $\R^n$ har en entydig vektor i $\R^m$, som afbildning.
Hvis en funktion er både injektiv og surjektiv, så er funktionen \textbf{bijektiv}.
\end{defn}
\noindent
%
Bemærk, at
$$f^{-1} \neq \frac{1}{f},$$
men blot notation for den \textit{inverse funktion} til $f$.
Givet en funktion $T:\R^n \rightarrow \R^n$, som er en lineær afbildning med standardmatricen $A$, så er $T$ invertibel, hvis og kun hvis $A$ er invertibel. 
Lad A være en $n \times n$ invertibel matrix. 
Så haves for alle $\textbf{v}$, at
$$T_AT_{A^{-1}}(\textbf{v})=T_A(T_{A^{-1}}(\textbf{v}))=T_A(A^{-1}\textbf{v})=A(A^{-1}\textbf{v})=AA^{-1}\textbf{v}=I_n\textbf{v}=\textbf{v}.$$
Det kan gennem lignende operationer vises, at
$$T_{A^{-1}}T_A(\textbf{v})=\textbf{v}.$$
%
Dette medfører $T^{-1}=T_{A^{-1}}$. 
Som følge deraf er $T^{-1}$ lineær og dens standardmatrix er $A^{-1}$.
\\
%
\begin{eks}
Fra \ref{eks:fisk5} haves, at
\begin{equation*}
A=
\begin{bmatrix}
1 & 1 & 1\\
1 & 2 & 2\\
1 & 3 & 2
\end{bmatrix} \text{\phantom{jd}og\phantom{jd}}
A^{-1}=
\begin{bmatrix}
2 & -1 & 0 \\
0 & -1 & 1 \\
-1 & 2 & -1 \\
\end{bmatrix}
\end{equation*}  
så er
\begin{equation*}
T_A
\left(
\begin{bmatrix}
x_1\\
x_2\\
x_3
\end{bmatrix}
\right)
=
\begin{blockarray}{c}
\begin{block}{[c]}
x_1+x_2+x_3\\
x_1+2x_2+2x_3\\
x_1+3x_2+2x_3\\
\end{block}
\end{blockarray}.
\end{equation*}
Og den inverse funktion er
\begin{equation*}
T_{A^{-1}}\left(
\begin{bmatrix}
x_1\\
x_2\\
x_3
\end{bmatrix}
\right)=
\begin{blockarray}{c}
\begin{block}{[c]}
2x_1-1x_2\\
-x_2+x_3\\
-x_1+2x_2-x_3\\
\end{block}
\end{blockarray}.
\end{equation*}
\end{eks}