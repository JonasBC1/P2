\section{Gauss elimination}
%sætning 1.4 kan vi bevise den
En algoritme til løsningen af linære ligningssystemer er  rækkereduktionsalgoritmen, som omdanner en matrix fra en totalmatrice til reduceret trappeform.
Når den bruges til at løse linære ligningssystemer, kaldes denne process for Gauss elimination.
I algoritmen bruges de elemæntere rækkeoperationer for at omdanne en matrix $A$ til trappeform. 
Herefter benyttes rækkeoperationerne igen til at omdanne matricen på trappeform til reduceret trappeform.
Algortimen gennemgår således følgende skridt:
%
\begin{enumerate}
\item Find første ikke-nul søjlse i $A$ fra venstre.
\item Ved rækkeombytning placeres en ikke-nul indgang øverst i pivot-søjlen.
\item Skab nuller under pivot-indgangen øverst i pivot-søjlen ved hjælp af rækkeudskiftning.
\item Øverste række markeres som afsluttet og trin $1-3$ gennemføres nu på den næste række.
\item Alle rækker med pivot-indgange skaleres så alle pivot-indgange er lig $1$.
\item Ved rækkeudskiftning sikres nu $0$er over og under pivot-indgangene.
\end{enumerate}
% Eksempel
Givet ligningssystemet:
\begin{align*}
x_1-x_3-2x_4-8x_5&=-3 \\
-2x_1+x_3+2x_4+9x_5&=5 \\
3x_1-2x_3-3x_4-15x_5&=-9 \\
\end{align*}
\[
\begin{blockarray}{cccccc}
a & b & c & d & e \\
\begin{block}{[ccccc]c}
  1 & 1 & 1 & 1 & 1 & f \\
  0 & 1 & 0 & 0 & 1 & g \\
  0 & 0 & 1 & 0 & 1 & h \\
  0 & 0 & 0 & 1 & 1 & i \\
  0 & 0 & 0 & 0 & 1 & j \\
\end{block}
\end{blockarray}
 \]

\end{document}

\begin{thm}{}{konsitens}
\end{thm}