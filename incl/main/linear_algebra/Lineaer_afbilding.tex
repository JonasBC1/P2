\section{Lineær afbildning}
%
Vi definerer en funktion ved følgende:
%
\begin{defn}{}{}
Lad $\S _1 \subset \R^n$ og $\S _2 \subset \R^m$.
En funktion $f$ fra $\S_1$ til $\S_2$, noteret $f:\S_1\rightarrow\S_2$, tildeler enhver vektor $\textbf{v}$ i $\S_1$, en unik vektor $f(\textbf{v})$ i $\S_2$.
Vektoren $f(\textbf{v})$ kaldes en \textbf{afbildning} af $\textbf{v}$.
$\S_1$ er domænet for funktionen $f$ mens $\S_2$ kaldes codomænet til $f$.
\textbf{Billedmængden} for $f$ er defineret som mængden af afbildninger, $f(\textbf{v})$, for alle $\textbf{v}\in \S_1$.
\end{defn}
%
I figur \ref{fig:afbild} ses at både $\textbf{u}$ og $\textbf{v}$ har $\textbf{w}$ som afbildning, da $f(\textbf{u})=f(\textbf{v})=\textbf{w}$.
%Lav flot figur som 2.9 på side 167.
\begin{figure}[h!]
  \centering
  \begin{tikzpicture}
    \tikzset{punkt/.style={point, draw=black}}\draw[color=black](1,-2) circle (2.5);
	\draw[color=black](7,-2) circle (2.5);  
	\draw[color=black, fill=myblue!15](7,-2) circle (1.5);  
%Punkter
	\draw  node[fill,circle,inner sep=0pt,minimum size=3pt] at (1,-2.65)  (v1) {};
	\draw  node[fill,circle,inner sep=0pt,minimum size=3pt] at (1,-1.35)  (v2) {};
	\draw  node[fill,circle,inner sep=0pt,minimum size=3pt] at (7,-2)  (v3) {};

%Labels
	\draw  node at (0.75,-2.55)  {$\textbf{u}$};
	\draw  node at (0.75,-1.25)  {$\textbf{v}$};
	\draw  node at (7.25,-1.85)  {$\textbf{w}$};
	
	
	
	%Navn på cirklerne 
	\draw[black, fill=black] (1.25,1) circle (0pt) node[anchor=west] {$\S_1$};    
	\draw[black, fill=black] (6.75,1) circle (0pt) node[anchor=west] {$\S_2$};
	%Navn på punkter 

	\draw[black, fill=black] (4,1.5) circle (0pt) node[anchor=west] {$f$};
	
	\draw [->,thick, draw=black] (2,1) -- (6.5,1);

	 \draw [->, thick, draw=red] ([xshift=5pt, yshift=-1pt]v1.north) -- ([xshift=-5pt, yshift=1pt]v3.south);
     \draw [->, thick, draw=black] ([xshift=5pt, yshift=1pt]v2.south) -- ([xshift=-5pt, yshift=-1pt]v3.north);


	\draw[black, fill=black] (0,-5) circle (0pt) node[anchor=west] {Domæne};
	\draw[black, fill=black] (6,-5) circle (0pt) node[anchor=west] {Kodomæne};
	\draw[black, fill=black] (5.5,-2.5) circle (0pt) node[anchor=west] {Billedmængde};    
	%Streger mellem punkterne
  \end{tikzpicture}
  \caption{En funktion fra $\S_1$ til $\S_2$, hvor $f(\textbf{v})=\textbf{w}$ og f$(\textbf{u})=\textbf{w}$. }
  \label{fig:afbild}
\end{figure}
%
%
%    \node[punkt] at (-4,0.5)      (v1){$v_1$};
%    \node[punkt] at (-2,0.5)      (v2){$v_2$};
%    \node[punkt] at (-4,-1.5)     (v3){$v_3$};
%    \node[punkt] at (-2,-1.5)     (v4){$v_4$};
%    \node at (-3,2)     (v){$K_{4}$};
%
%
%    \node[punkt] at (4.6,-0.2)      (k1){$v_3$};
%    \node[punkt] at (1.4,-0.2)      (k2){$v_2$};
%    \node[punkt] at (2,-2)     (k3){$v_4$};
%    \node[punkt] at (4,-2)     (k4){$v_5$};
%    \node[punkt] at (3,1)      (k5){$v_1$};
%    \node at (3,2)      (k){$K_{5}$};
%
%
%
%
%
%    \draw [-, thick, draw=black] (v1) -- (v2);
%    \draw [-, thick, draw=black] (v1) -- (v3);
%    \draw [-, thick, draw=black] (v1) -- (v4);
%    \draw [-, thick, draw=black] (v2) -- (v3);
%    \draw [-, thick, draw=black] (v2) -- (v4);
%    \draw [-, thick, draw=black] (v3) -- (v4);
%
