\section{Lineær afbildning}
%
En matrix kan betragtes som en funktion, der afbilder en vektor i et rum.
%
\begin{defn}{}{}
Lad $\S _1 \subset \R^n$ og $\S _2 \subset \R^m$.
En funktion $f$ fra $\S_1$ til $\S_2$, noteret $f:\S_1\rightarrow\S_2$, tildeler enhver vektor $\textbf{u}$ i $\S_1$ en entydig vektor $f(\textbf{u})$ i $\S_2$.
Vektoren $f(\textbf{u})$ kaldes en \textbf{afbildning} af $\textbf{v}$.
%Hedder det codomænet på dansk? Lyder underligt. 
$\S_1$ er \textbf{domænet} for funktionen $f$, mens $\S_2$ kaldes \textbf{codomænet} til $f$.
\textbf{Værdimængden} for $f$ er defineret som mængden af afbildninger $f(\textbf{u})$ for alle $\textbf{v}\in \S_1$.
\end{defn}\noindent
%
På figur \ref{fig:afbild} ses, at både $\textbf{u}$ og $\textbf{v}$ har $\textbf{w}$ som afbildning, da $f(\textbf{u})=f(\textbf{v})=\textbf{w}$.
%
% Lav flot figur som 2.9 på side 167.
%
\begin{figure}[h!]
  \centering
  \begin{tikzpicture}
    \tikzset{punkt/.style={point, draw=black}}\draw[color=black](1,-2) circle (2.5);
	\draw[color=black](7,-2) circle (2.5);  
	\draw[color=black, fill=myblue!15](7,-2) circle (1.5);  
%Punkter
	\draw  node[fill,circle,inner sep=0pt,minimum size=3pt] at (1,-2.65)  (v1) {};
	\draw  node[fill,circle,inner sep=0pt,minimum size=3pt] at (1,-1.35)  (v2) {};
	\draw  node[fill,circle,inner sep=0pt,minimum size=3pt] at (7,-2)  (v3) {};

%Labels
	\draw  node at (0.75,-2.55)  {$\textbf{u}$};
	\draw  node at (0.75,-1.25)  {$\textbf{v}$};
	\draw  node at (7.25,-1.85)  {$\textbf{w}$};
	
	
	
	%Navn på cirklerne 
	\draw[black, fill=black] (1.25,1) circle (0pt) node[anchor=west] {$\S_1$};    
	\draw[black, fill=black] (6.75,1) circle (0pt) node[anchor=west] {$\S_2$};
	%Navn på punkter 

	\draw[black, fill=black] (4,1.5) circle (0pt) node[anchor=west] {$f$};
	
	\draw [->,thick, draw=black] (2,1) -- (6.5,1);

	 \draw [->, thick, draw=red] ([xshift=5pt, yshift=-1pt]v1.north) -- ([xshift=-5pt, yshift=1pt]v3.south);
     \draw [->, thick, draw=black] ([xshift=5pt, yshift=1pt]v2.south) -- ([xshift=-5pt, yshift=-1pt]v3.north);


	\draw[black, fill=black] (0,-5) circle (0pt) node[anchor=west] {Domæne};
	\draw[black, fill=black] (6,-5) circle (0pt) node[anchor=west] {Kodomæne};
	\draw[black, fill=black] (5.5,-2.5) circle (0pt) node[anchor=west] {Billedmængde};    
	%Streger mellem punkterne
  \end{tikzpicture}
  \caption{En funktion fra $\S_1$ til $\S_2$, hvor $f(\textbf{v})=\textbf{w}$ og f$(\textbf{u})=\textbf{w}$. }
  \label{fig:afbild}
\end{figure}
%
%
%    \node[punkt] at (-4,0.5)      (v1){$v_1$};
%    \node[punkt] at (-2,0.5)      (v2){$v_2$};
%    \node[punkt] at (-4,-1.5)     (v3){$v_3$};
%    \node[punkt] at (-2,-1.5)     (v4){$v_4$};
%    \node at (-3,2)     (v){$K_{4}$};
%
%
%    \node[punkt] at (4.6,-0.2)      (k1){$v_3$};
%    \node[punkt] at (1.4,-0.2)      (k2){$v_2$};
%    \node[punkt] at (2,-2)     (k3){$v_4$};
%    \node[punkt] at (4,-2)     (k4){$v_5$};
%    \node[punkt] at (3,1)      (k5){$v_1$};
%    \node at (3,2)      (k){$K_{5}$};
%
%
%
%
%
%    \draw [-, thick, draw=black] (v1) -- (v2);
%    \draw [-, thick, draw=black] (v1) -- (v3);
%    \draw [-, thick, draw=black] (v1) -- (v4);
%    \draw [-, thick, draw=black] (v2) -- (v3);
%    \draw [-, thick, draw=black] (v2) -- (v4);
%    \draw [-, thick, draw=black] (v3) -- (v4);
%
For funktioner, som er matrix-vektorprodukter, introduceres notationen $T_A$.
%
\begin{defn}{}{fisk3}
Lad $A$ være en $m \times n$ matrix.
Funktionen $T_A:\R^n \rightarrow \R^m$, defineret ved $T_A(\textbf{x}) = A\textbf{x}$ for alle $\textbf{x} \in \R^n$, kaldes \textbf{matrixafbildning fremkaldt af $\mathbf{A}$.}
\end{defn}\noindent
%
Direkte fra sætning \ref{thm:mxvpro} ses det, at $T_A$ har egenskaberne $T_A(\textbf{u}+\textbf{v})=T_A(\textbf{u}) + T_A(\textbf{v})$ og $cT_A(\textbf{u}) = T_A(c\textbf{u})$.
Funktioner, der opfylder disse egenskaber, får deres egen definition.
%
\begin{defn}{}{}
Funktionen $T: \R^n \rightarrow \R^m$ kaldes en \textbf{lineær afbildning}, hvis følgende vilkår gør sig gældende for alle $\textbf{u},\textbf{v} \in \R^n$ og alle skalarer $c$:
\begin{enumerate}[label=(\alph*)]
\item $T_A(\textbf{u}+\textbf{v})=T_A(\textbf{u}) + T_A(\textbf{v})$.
\item $cT_A(\textbf{u}) = T_A(c\textbf{u})$.
\end{enumerate}
\end{defn}\noindent
%
Da matrixafbildning har disse egenskaber, er alle matrixafbildninger lineære.
Følgende sætning viser nogle af egenskaberne ved lineær afbildning.
%Matrixafbildninger er ét ord. 
\begin{thm}{}{afbegen}
Lad $\textbf{u}, \textbf{v} \in \R^n$ og lad $c$ og $k$ være skalarer.
For enhver lineær afbildning $T: \R^n \rightarrow \R^m$ er følgende udsagn sande:
\begin{enumerate}[label = (\alph*)]
\item $T(\textbf{0}) = \textbf{0}$.
\item $T(-\textbf{u}) = -T(\textbf{u})$.
\item $T(\textbf{u}-\textbf{v}) = T(\textbf{u})-T(\textbf{v})$.
\item $T(c\textbf{u} + k\textbf{v}) = cT(\textbf{u}) + kT(\textbf{u})$.
\end{enumerate}
\end{thm}
%
\begin{proof}
(a) 
Lad $\textbf{u}, \textbf{v}$ være vektorer i $\R^n$, $c$ og $k$ være skalarer og lad $T: \R^n \rightarrow \R^m$ være en lineær afbildning.
Da $T$ bevarer vektoraddition haves, at
%
\begin{align*}
T(\textbf{0}) = T(\textbf{0} + \textbf{0}) = T(\textbf{0}) + T(\textbf{0}).
\end{align*}
%
(b) 
Trækkes $T(\textbf{0})$ fra begge sider haves, at $\textbf{0} = T(\textbf{0})$, hvilket viser (b).
%
Da $T$ bevarer skalering haves
% 
\begin{align*}
T(-\textbf{u}) = T((-1)\textbf{u}) = (-1)T(\textbf{u}) = -T(\textbf{u}).
\end{align*}
%
(c) 
Ved kombination af, at $T$ bevarer vektor addition og (b), haves, at
%
\begin{align*}
T(\textbf{u}-\textbf{v}) = T(\textbf{u}+(-\textbf{v})) = T(\textbf{u})+T(-\textbf{v}) = T(\textbf{u}) - T(\textbf{v}).
\end{align*}
%
(d) 
Da $T$ både bevarer både vektor addition og skalering haves, at 
%
\begin{align*}
T(c\textbf{u} + k\textbf{v}) = T(c\textbf{u}) + T(k\textbf{v}) = cT(\textbf{u}) + kT(\textbf{u}).
\end{align*}
\end{proof}
%
\\
Sætning \ref{thm:afbegen}(d) kan generaliseres til at vise, at lineær afbildning bevarer linearkombinationer.
%Linearkombinationer
\begin{lem}{}{}
Lad $T:\R^n \rightarrow \R^m$ være en lineær afbildning, $\textbf{u}_1,\textbf{u}_2,\ldots,\textbf{u}_k$ være vektorer i $\R^n$ og lad $a_1,a_2,\ldots,a_k$ være skalarer. 
Så gælder, at
%
\begin{align*}
T(a_1\textbf{u}_1 + a_2\textbf{u}_2 + \cdots + a_k\textbf{u}_k) = T(a_1\textbf{u}_1) + T(a_2\textbf{u}_2) + \cdots + T(a_k\textbf{u}_k).
\end{align*}
%
\end{lem}
%
\begin{proof}
Lemmaet følger direkte af sætning \ref{thm:afbegen}.
\end{proof}
\\
%
Bemærk, at sætning \ref{thm:afbegen}(a) i nogle tilfælde kan bruges til at vise, hvorvidt en funktion er ikke-lineær, hvis den ikke opfylder betingelsen. 
Det kan dog forekomme, at en funktion opfylder betingelsen og stadig er ikke-lineær.
% 
\\\\
% 
Hvis afbildningen $T$ er lineær, kan en tilsvarende matrix $A$ opstilles, sådan at $T=T_A$, hvilket er belyst i sætning \ref{thm:fisk2}.
%
%%%%%%%%%%%%%%%%%%%%%%%%%%%%%%%%%%%%%%%%%%%%%%%%%%%%%%%%%%%%%%%%%%%%%%%%%%
% 
\begin{thm}{}{fisk2}
Lad $T: \R^n \rightarrow \R^m$ være lineær. 
Så findes der en entydig $m \times n$ matrix
\begin{align*}
A= [T(\mathbf{e}_1)\text{    } T(\mathbf{e}_2) \text{    } \ldots \text{    } T(\mathbf{e}_n)],
\end{align*}
hvor søjlerne er afbildninger af $T$ ud fra standardvektorene i $\R^n$, så det gælder at $T(\mathbf{v})=A \mathbf{v}$  $\forall \mathbf{v} \in \R^n$.
\end{thm}
%
%%%%%%%%%%%%%%%%%%%%%%%%%%%%%%%%%%%%%%%%%%%%%%%%%%%%%%%%%%%%%%%%%%%%%%%%%%
%
\begin{proof}
Lad $A= [T(\mathbf{e}_1)\text{    } T(\mathbf{e}_2) \text{    } \ldots \text{    } T(\mathbf{e}_n)]$. 
Den lineære afbildning af $\mathbf{v}$ er
%
\begin{align*}
T(\mathbf{v})= T(v_1 \mathbf{e}_1+v_2 \mathbf{e}_2+ \ldots + v_n \mathbf{e}_n).
\end{align*}
%
Jævnfør sætning \ref{thm:afbegen}(d) er
%
\begin{align*}
T(\mathbf{v})= v_1 T( \mathbf{e}_1)+ v_2 T( \mathbf{e}_2) + \ldots + v_n T( \mathbf{e}_n),
\end{align*}
%
hvilket kan erstattes med elementerne i matrix A
%
\begin{align*}
T(\mathbf{v})&= v_1 \mathbf{a}_1+ v_2 \mathbf{a}_2 + \ldots + v_n \mathbf{a}_n \\
&= A \mathbf{v},
\end{align*}
%Er det meningen, at ovenstående skal være på to linjer? 
hvilket jævnfør \ref{defn:fisk3} beviser, at
%
\begin{align*}
T(\mathbf{v})= T_A (\mathbf{v}).
\end{align*}
%
Lad $B$ være en $m \times n$ matrix. 
For at bevise entydigheden, lad da $T_A=T_B$. 
Jævnfør sætning \ref{thm:mxvpro}(e) gælder det, at $A=B$, eftersom $A \mathbf{v}=B \mathbf{v}$ for alle $\mathbf{v}$ i $\R^n$.
\end{proof}
%
%%%%%%%%%%%%%%%%%%%%%%%%%%%%%%%%%%%%%%%%%%%%%%%%%%%%%%%%%%%%%
\begin{eks}\label{entydigeks}
%
Lad $T: R^3 \rightarrow R^2$ være defineret ved 
$$T\left(
\begin{bmatrix}
x_1\\
x_2\\
x_3
\end{bmatrix}
\right)
=
%\begin{blockarray}{c}
\begin{bmatrix}%{[c]}
5x_1+x_2\\
3x_2+2x_3
\end{bmatrix}
%\end{blockarray}.
$$ \\
For at danne standardmatricen for $T$, dannes søjlerne 
\begin{align*}
T(\mathbf{e}_1)=
\begin{bmatrix}
5\\
0
\end{bmatrix}\text{, }
T(\mathbf{e}_2)=
\begin{bmatrix}
1\\
3
\end{bmatrix}\text{ og }
T(\mathbf{e}_3)=
\begin{bmatrix}
0\\
2
\end{bmatrix}.
\end{align*}
Derfor er standardmatricen for
\begin{align*}
T=
\begin{bmatrix}
5 & 1 & 0\\
0 & 3 & 2
\end{bmatrix}.
\end{align*}
\end{eks}