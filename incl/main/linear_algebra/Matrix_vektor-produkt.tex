%Matrixvektorproduct
\begin{defn}{}{mvp}
\phantom{gdfs}\\Lad $A$ være en $m\times n$ matrix og \textbf{v} være en $n\times 1$ vektor. \textbf{Matrix-vektor-produktet} af A og \textbf{v} defineres som den lineære kombination af søjlerne i A hvis koefficienter er de tilsvarende komponenter af \textbf{v}, og noteres A\textbf{v}. Det vil sige
\begin{align*}
A\textbf{v} =v_1\textbf{a}_1 + v_2\textbf{a}_2 + \cdots + v_n\textbf{a}_n.
\end{align*}
\end{defn}
\noindent
For at $A\textbf{v}$ eksisterer, skal der være lige mange komponenter i \textbf{v} som søjler i $A$. Et eksempel på et matrix-vektor-produkt mellem en matrix $B$ og en vektor \textbf{u} ses i eksempel \ref{Matrix-vektor}.
\begin{eks}\label{Matrix-vektor}
\begin{center} 
$$B=
\begin{blockarray}{c c c c}
\begin{block}{[c c c c]}
2 & 5 & 6 \\
4 & 7 & 3\\
1 & 4 & 1\\
2 & 3 & 8\\
\end{block}
\end{blockarray}
%
\text{ og }
%
\textbf{u}=
\begin{bmatrix}
4 \\
2 \\
3 \\ 
\end{bmatrix}.
$$
Eftersom $B$ har 3 søjler og \textbf{u} har 3 komponenter, så eksisterer $B\textbf{u}$ og findes ved
$$
B\textbf{u}=
\begin{bmatrix}
2 & 5 & 6 \\
4 & 7 & 3\\
1 & 4 & 1\\
2 & 3 & 8\\
\end{bmatrix}
\begin{bmatrix}
4 \\
2 \\
3 \\ 
\end{bmatrix}
=4
\begin{bmatrix}
2\\
4\\
1\\
2\\
\end{bmatrix}
+2
\begin{bmatrix}
5\\
7\\
4\\
3\\
\end{bmatrix}
+3
\begin{bmatrix}
6\\
3\\
1\\
8\\
\end{bmatrix}
=
\begin{bmatrix}
8\\
16\\
4\\
8\\
\end{bmatrix}
+
\begin{bmatrix}
10\\
14\\
8\\
6\\
\end{bmatrix}
+
\begin{bmatrix}
18\\
9\\
3\\
24\\
\end{bmatrix}
=
\begin{bmatrix}
36\\
39\\
15\\
38\\
\end{bmatrix}.
$$
\end{center}
\end{eks}
 
