\section{Lineære ligningssystemer }
%
En \textbf{lineær ligning} med variablerne $x_1,x_2,\ldots ,x_n$ er en ligning, som har formen
$$ c_1x_1+c_2x_2+\cdots+c_nx_n=b \text{,}$$ 
hvor $\{c_1,c_2,\ldots,c_n\} \in \R$. 
%
\begin{defn}{}{linlignsys}
Et \textbf{lineært ligningssystem} er en mængde af $m$ lineære ligninger med de samme $n$ variable, hvor $\{m, n\} \in \Z^+$.
Dette kan skrives på formen
\begin{align*}
c_{11}x_1+c_{12}x_2+&\cdots+c_{1n}x_n=b_1\\
c_{21}x_1+c_{22}x_2+&\cdots+c_{2n}x_n=b_2\\
&\vdots\\
c_{m1}x_1+c_{m2}x_2+&\cdots +c_{mn}x_n=b_m
\text{,}
\end{align*}
hvor $c_{ij}$ betegner koefficienten af $x_j$ i ligning $i$.
\end{defn}
\noindent
Løsningen på et sådant system er en vektor
\begin{center}
$
\mathbf{s} = 
\begin{bmatrix}
s_1\\
s_2\\
\vdots\\
s_n
\end{bmatrix},
$
\end{center}
i $\R ^n$, således at alle ligninger er opfyldt, når $x_i$ erstattes med $s_i$. Et eksempel på et lineært ligningssystem ses i eksempel \ref{ekslignsys}.
%
\\
\begin{eks}
\label{ekslignsys}
Understående er et lineært ligningssystem, bestående af tre ligninger.
%
\begin{align*}
x_1-x_3-2x_4-8x_5&=-3 \\
-2x_1+x_3+2x_4+9x_5&=5 \\
3x_1-2x_3-3x_4-15x_5&=-9.
\end{align*}
%
Løsningen på dette system ses i eksempel \ref{eks_gauss}.
%
%%%%%%%%%%%%%%%%%%%%%%%%%%%%%%%%%%%%%%%
%
\end{eks}
Alle lineære ligningssystemer har enten ingen, én eller uendeligt mange løsninger.
Ligninger i $\R^2$ kan repræsenteres ved linjer og i $\R^3$ kan de repræsenteres ved planer.
% Er det en eller et plan? Hess siger altid "en"... Hmmmmm.. 
Hvis disse ligninger er parallelle og forskellige, har systemet ingen løsning.
Hvis ligningerne er forskellige, så har systemet præcis én løsning, hvor ligningerne krydser hinanden. 
Hvis ligningerne er ens, har systemet uendeligt mange løsninger.\\
Et system der har én eller flere løsninger kaldes \textbf{konsistent}; ellers kaldes det \textbf{inkonsistent}, da der ingen løsninger eksisterer.
%
%
%
\subsection{Ligningssystemer og matricer}
Et lineært ligningssystem kan opstilles som en matrixligning $A\textbf{x}=\textbf{b}$, hvor
$$A=
\begin{bmatrix}
a_{1,1} & a_{1,2} & \cdots & a_{1,n}\\
a_{2,1} & a_{2,2} & \cdots & a_{2,n}\\
\vdots & \vdots & \ddots & \vdots\\
a_{m,1} & a_{m,2} & \cdots & a_{m,n}
\end{bmatrix}
\text{, } 
\textbf{x}=
\begin{bmatrix}
x_1\\
x_2\\
\vdots\\
x_n\\
\end{bmatrix}
\text{og }
\textbf{b}=\begin{bmatrix}
b_1\\
b_2\\
\vdots\\
b_m
\end{bmatrix}.
$$
%
A kaldes for \textbf{koefficientmatricen}. 
%efter hårdt sammenleje bliver A matricen og b til totalmatricen
\textbf{Totalmatrixen} $[A \text{    } | \mathbf{b}]$ dannes ved at kombinere koefficientmatricen med vektoren \textbf{b}.
Totalmatrixen indeholder alle informationer, der skal bruges til at løse et lineært ligningssystem.
Ligningssystemet fra eksempel \ref{ekslignsys} skrives som totalmatricen
\begin{equation*}
  A=
\begin{blockarray}{cccccc}
x_1 & x_2 & \cdots & x_n & b \\
\begin{block}{[cccc|c]c}
a_{1,1} & a_{1,2} & \cdots & a_{1,n} & b_1 \\
a_{2,1} & a_{2,2} & \cdots & a_{2,n} & b_2 \\
\vdots & \vdots & \ddots & \vdots & \vdots \\
a_{m,1} & a_{m,2} & \cdots & a_{m,n} & b_{m}\\
\end{block}
\end{blockarray}.
\end{equation*}
%
\begin{eks}
Med udgangspunkt i ligningssystemet fra eksempel \ref{ekslignsys} følger det at
$$A=
\begin{bmatrix}
1 & 0 & -1 & -2 & -8\\
-2 & 0 & 1 & 2 & 9\\
3 & 0 & -2 & -3 & -15
\end{bmatrix}
\text{, } 
\textbf{x}=
\begin{bmatrix}
x_1\\
x_2\\
x_3\\
x_4\\
x_5
\end{bmatrix}
\text{, }
\textbf{b}=\begin{bmatrix}
-3\\
5\\
-9
\end{bmatrix}
$$
%
og
\begin{equation*}
  [A \text{    } | \mathbf{b}] =
\begin{blockarray}{ccccccc}
x_1 & x_2 & x_3 & x_4 & x_5 & b \\
\begin{block}{[ccccc|c]c}
  1 & 0 & -1 & -2 & -8 & -3 \\
  -2 & 0 & 1 & 2 & 9 & 5 \\
  3 & 0 & -2 & -3 & -15 & -9 \\
\end{block}
\end{blockarray}.
\end{equation*}

\end{eks}
