\section{Lineære ligningssystemer }
%
En \textit{lineær ligning} med variablene $x_1,x_2,\ldots ,x_n$ er en ligning, som har formen
$$ c_1x_1+c_2x_2+\cdots+c_nx_n=b \text{,}$$ 
hvor $c_1,c_2,\ldots,c_n \in \R$. 
%
\begin{defn}{}{linlignsys}
Et \textbf{lineært ligningssystem} er en mængde af $m$ lineære ligninger med de samme $n$ variable, hvor $m, n \in \Z^+$.
Dette kan skrives på formen
\begin{align*}
c_{1,1}x_1+c_{1,2}x_2+&\cdots+c_{1,n}x_n=b_1\\
c_{2,1}x_1+c_{2,2}x_2+&\cdots+c_{2,n}x_n=b_2\\
&\vdots\\
c_{m,1}x_1+c_{m,2}x_2+&\cdots +c_{m,n}x_n=b_m
\text{,}
\end{align*}
hvor $c_{i,j}$ betegner koefficienten af $x_j$ i ligning $i$.
\end{defn}
\noindent
Løsningen på et sådant system er en vektor
\begin{center}
$
\mathbf{v} = 
\begin{bmatrix}
v_1\\
v_2\\
\vdots\\
v_n
\end{bmatrix},
$
\end{center}
i $\R ^n$, således at alle ligninger er opfyldt, når $x_i = v_i$.
%
\\
%
\begin{eks}
\label{eks:lignsys}
Understående er et lineært ligningssystem, bestående af tre lineære ligninger.
%
\begin{align*}
x_1-x_3-2x_4-8x_5&=-3 \\
-2x_1+x_3+2x_4+9x_5&=5 \\
3x_1-2x_3-3x_4-15x_5&=-9.
\end{align*}
%
% Løsningen på dette system ses i \ref{eks_gauss}.
%
%%%%%%%%%%%%%%%%%%%%%%%%%%%%%%%%%%%%%%%
%
\end{eks}
%
Alle lineære ligningssystemer har enten ingen, én eller uendeligt mange løsninger.
Ligninger i $\R^2$ kan repræsenteres ved linjer, og ligninger i $\R^3$ kan repræsenteres ved planer.
% Er det en eller et plan? Hess siger altid "en"... Hmmmmm.. 
Hvis disse ligninger er parallelle og forskellige, har systemet ingen løsning.
Hvis ligningerne er forskellige og ikke parallelle, så har systemet præcis én løsning, hvor ligningerne er lig hinanden.
Hvis ligningerne er ækvivalente, har systemet uendeligt mange løsninger.
\\\\
%
Et system der har én eller flere løsninger kaldes \textit{konsistent}; ellers kaldes det \textit{inkonsistent}.
%
%%%%%%%%%%%%%%%%%%%%%%%%%%%%%%
%
\subsection{Ligningssystemer og matricer}
Et lineært ligningssystem kan opstilles som en \textit{matrixligning} $A\textbf{x}=\textbf{b}$, hvor
$$A=
\begin{bmatrix}
a_{1,1} & a_{1,2} & \cdots & a_{1,n}\\
a_{2,1} & a_{2,2} & \cdots & a_{2,n}\\
\vdots & \vdots & \ddots & \vdots\\
a_{m,1} & a_{m,2} & \cdots & a_{m,n}
\end{bmatrix}
\text{, } 
\textbf{x}=
\begin{bmatrix}
x_1\\
x_2\\
\vdots\\
x_n\\
\end{bmatrix}
\text{og }
\textbf{b}=\begin{bmatrix}
b_1\\
b_2\\
\vdots\\
b_m
\end{bmatrix}.
$$
%
A kaldes for \textit{koefficientmatricen}. 
%efter hårdt sammenleje bliver A matricen og b til totalmatricen
\textit{Totalmatrixen} $[A \mid \mathbf{b}]$ dannes ved at kombinere koefficientmatricen med vektoren \textbf{b}.
Totalmatrixen indeholder alle informationer, der skal bruges til at løse det lineære ligningssystem.
%
Totalmatricen opskrives på formen:
%
\begin{equation*}
[A \mid \mathbf{b}]=
\begin{blockarray}{cccccc}
x_1 & x_2 & \cdots & x_n & b \\
\begin{block}{[cccc|c]c}
a_{1,1} & a_{1,2} & \cdots & a_{1,n} & b_1 \\
a_{2,1} & a_{2,2} & \cdots & a_{2,n} & b_2 \\
\vdots & \vdots & \ddots & \vdots & \vdots \\
a_{m,1} & a_{m,2} & \cdots & a_{m,n} & b_{m}\\
\end{block}
\end{blockarray}.
\end{equation*}
%
En matrixligning kan også opskrives ved
%
\begin{align*}
\sum^n_{i=1}{A_ix_i}=\textbf{b}.
\end{align*}
\begin{eks}\label{eks:lig_mat}
Med udgangspunkt i ligningssystemet fra \ref{eks:lignsys} følger det, at
$$A=
\begin{bmatrix}
1 & 0 & -1 & -2 & -8\\
-2 & 0 & 1 & 2 & 9\\
3 & 0 & -2 & -3 & -15
\end{bmatrix}
\text{, } 
\textbf{x}=
\begin{bmatrix}
x_1\\
x_2\\
x_3\\
x_4\\
x_5
\end{bmatrix}
\text{, }
\textbf{b}=\begin{bmatrix}
-3\\
5\\
-9
\end{bmatrix}
$$
%
og
%
\begin{equation*}
  [A \mid \mathbf{b}] =
\begin{blockarray}{ccccccc}
x_1 & x_2 & x_3 & x_4 & x_5 & b \\
\begin{block}{[ccccc|c]c}
  1 & 0 & -1 & -2 & -8 & -3 \\
  -2 & 0 & 1 & 2 & 9 & 5 \\
  3 & 0 & -2 & -3 & -15 & -9 \\
\end{block}
\end{blockarray}.
\end{equation*}
%
\end{eks}
\\\\
\phantom{hej}
\\\\
\phantom{hej}
\\\\
\phantom{hej}
\\\\
