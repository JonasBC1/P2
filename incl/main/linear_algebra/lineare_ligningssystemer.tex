\section{Lineære ligningssystemer }
%

\textbf{En lineær ligning} med variablerne, $x_1,x_2,\ldots ,x_n$, er en ligning som har formen
$$ a_1x_1+a_2x_2+\cdots+a_nx_n=b$$ 
hvor $a_1,a_2,\ldots,a_n$ er reelle tal. 

\begin{defn}{}{}
Et lineært ligningssystem er en mængde af $m $lineære ligninger  med de samme $n$ variabler, hvor $m$ og $n$ er positive heltal. Dette kan skrives på formen
\begin{align*}
a_{11}x_1+a_{12}x_2+&\cdots+a_{1n}x_n=b_1\\
a_{21}x_1+a_{22}x_2+&\cdots+a_{2n}x_n=b_2\\
&\vdots\\
a_{m1}x_1+a_{m2}x_2+&\cdots +a_{mn}x_n=b_m
\end{align*}
hvor er $a_{ij}$ betegner koefficienten af $x_j$ i ligning $i$
\end{defn}

\begin{eks}
\begin{align*}
2x_1+3x_2+4x_4&=6\\
-2x_2+4x_3&=2\\
4x_1+7x_3+3x_4&=8\\
\end{align*}

\end{eks}

