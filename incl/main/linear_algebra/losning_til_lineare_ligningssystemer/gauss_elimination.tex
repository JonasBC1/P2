\subsection{Gauss-elimination}
\label{gauss}
% 1.4
% Sætning 1.4 kan vi bevise den
En algoritme til løsning af lineære ligningssystemer er  \textit{rækkereduktionsalgoritmen}, som reducerer en totalmatrix til reduceret trappeform.
Når den bruges til at løse lineære ligningssystemer, kaldes denne proces for \textit{Gauss-elimination}.
I algoritmen bruges de elementære rækkeoperationer til at omdanne en matrix til trappeform. 
Herefter benyttes rækkeoperationerne til at reducere matricen til reduceret trappeform.
Algoritmen gennemgår således følgende skridt:
%
\begin{enumerate}
\item Find første ikke-nulsøjle fra venstre i matricen.
\item Ved rækkeombytning placeres en ikke-nulindgang øverst i pivotsøjlen.
\item Skab nulindgange under pivotindgangen ved hjælp af rækkeudskiftning.
\item Øverste række markeres som afsluttet og trin $1-4$, hvor afsluttede rækker ignoreres.
Dette gentages, indtil alle rækker er markeret som afsluttet.
\item Alle rækker med pivotindgange skaleres, så alle pivotindgange er lig $1$.
\item Ved rækkeudskiftning sikres nu nulindgange over og under pivotindgangene.
\end{enumerate}
%
%%%%%%%%%%%%%%%%%%%%%%%%%%%%%%%%%%%%%%%%%%%%%%%%%
% Eksempel
\begin{eks}
Betragt ligningssystemet fra \ref{eks:lignsys}
%
\begin{align*}
\begin{array}{rlr}
x_1-x_3-2x_4-8x_5    &=  &-3 \\
-2x_1+x_3+2x_4+9x_5  &=  &5 \\
3x_1-2x_3-3x_4-15x_5 &=  &-9.
\end{array}
\end{align*}
%
Ligningssystemet opskrevet som totalmatrix
%
\begin{equation*}
[A \mid \mathbf{b}] =
\begin{blockarray}{ccccccc}
x_1 & x_2 & x_3 & x_4 & x_5 & b \\
\begin{block}{[ccccc|c]c}
1 & 0 & -1 & -2 & -8 & -3 \\
-2 & 0 & 1 & 2 & 9 & 5 \\
3 & 0 & -2 & -3 & -15 & -9 \\
\end{block}
\end{blockarray}.
\end{equation*}
%
De elementære rækkeoperationer benyttes nu med henblik på at reducere totalmatricen til trappeform.
%
\begin{equation*}
\xrightarrow[R_3 \rightarrow R_3-3R_1]{R_2 \rightarrow R_2+2R_1} 
\begin{blockarray}{ccccccc}
x_1 & x_2 & x_3 & x_4 & x_5 & b \\
\begin{block}{[ccccc|c]c}
  1 & 0 & -1 & -2 & -8 & -3 \\
  0 & 0 & -1 & -2 & -7 & -1 \\
  0 & 0 & 1 & 3 & 9 & 0 \\
\end{block}
\end{blockarray}
\end{equation*}
%
\begin{equation*}
\xrightarrow{R_3 \rightarrow R_3+R_2}
\begin{blockarray}{ccccccc}
x_1 & x_2 & x_3 & x_4 & x_5 & b \\
\begin{block}{[ccccc|c]c}
  \hlight{1} & 0 & -1 & -2 & -8 & -3 \\
  0 & 0 & \hlight{-1} & -2 & -7 & -1 \\
  0 & 0 & 0 & \hlight{1} & 2 & -1 \\
\end{block}
\end{blockarray}
\end{equation*}
%
Algoritmen fortsættes nu med henblik på at reducere totalmatricen fra trappeform til reduceret trappeform.
%
\begin{equation*}
\xrightarrow[R_2 \rightarrow R_2+2R_3]{R_1 \rightarrow R_1+2R_3}
\begin{blockarray}{ccccccc}
x_1 & x_2 & x_3 & x_4 & x_5 & b \\
\begin{block}{[ccccc|c]c}
  1 & 0 & -1 & 0 & -4 & -5 \\
  0 & 0 & -1 & 0 & -3 & -3 \\
  0 & 0 & 0 & 1 & 2 & -1 \\
\end{block}
\end{blockarray}
\end{equation*}
%
\begin{equation*}
\xrightarrow{R_2 \rightarrow -1R_2}
\begin{blockarray}{ccccccc}
x_1 & x_2 & x_3 & x_4 & x_5 & b \\
\begin{block}{[ccccc|c]c}
  1 & 0 & -1 & 0 & -4 & -5 \\
  0 & 0 & 1 & 0 & 3 & 3 \\
  0 & 0 & 0 & 1 & 2 & -1 \\
\end{block}
\end{blockarray}
\end{equation*}
%
\begin{equation*}
\xrightarrow{R_1 \rightarrow R_1+R_2}
\begin{blockarray}{ccccccc}
x_1 & x_2 & x_3 & x_4 & x_5 & b \\
\begin{block}{[ccccc|c]c}
  \hlight{1} & 0 & 0 & 0 & -1 & -2 \\
  0 & 0 & \hlight{1} & 0 & 3 & 3 \\
  0 & 0 & 0 & \hlight{1} & 2 & -1 \\
\end{block}
\end{blockarray}
\end{equation*}
%
Herefter kan følgende løsninger opskrives:
%
\begin{align*}
\begin{array}{rrcll}
x_1-x_5     =&-2   &\iff &x_1   =&x_5-2 \\
x_3 + 3x_5    =&3    &\iff &x_3   =&-3x_5+3 \\
x_4+2x_5    =&-1   &\iff &x_4   =&-2x_5-1.
\end{array}
\end{align*}
%
Løsningen, opskrevet som en parameterfremstilling, er dermed givet ved
%
  \begin{align*}
    \mathbf{x} &= \begin{bmatrix}
           x_{1} \\
           x_{2} \\
           x_{3} \\
           x_{4} \\
           x_{5} \\
         \end{bmatrix} 
         = \begin{bmatrix}
           -2 \\
           0 \\
           3 \\
           -1 \\
           0 \\
         \end{bmatrix}
         +x_2 \begin{bmatrix}
           0 \\
           1 \\
           0 \\
           0 \\
           0 \\
         \end{bmatrix}
         +x_5 \begin{bmatrix}
           1 \\
           0 \\
           -3 \\
           -2 \\
           1 \\
         \end{bmatrix}.
  \end{align*} 
%
\label{eks_gauss}
%
% Frie variable
%
\end{eks}
%
%%%%%%%%%%%%%%%%%%%%%%%%%%%%%%%
%%                           %%
%%    Rang og nullitet       %%
%%                           %%
%%%%%%%%%%%%%%%%%%%%%%%%%%%%%%
%
\subsection{Rang og nullitet}
% 
For at beskrive matricers egenskaber i forhold til pivotindgange er der brug for at definere \textit{rang} og \textit{nullitet}.
%
\begin{defn}{}{rangnull}
\textbf{Rangen} af en matrix $A$ er antallet af pivotindgange, hvilket noteres $\text{rang}(A)$.
\textbf{Nulliteten} af $A$ er antallet af søjler uden pivotindgang, hvilket noteres $\text{null}(A)$.
\end{defn}
%
\noindent
Bemærk, at der for en $m \times n$ matrix gælder, at $\text{rang}(A)+\text{null}(A)=n$.\\
%
\begin{eks}
Med udgangspunkt i matrix $[A \mid \textbf{b}]$ på reduceret trappeform fra \ref{eks_gauss} bliver rang og nullitet i således $\text{rang}(A)=3$ og $\text{null}(A)=2$.
\end{eks}
%
\begin{thm}{}{konsistens}
%
Følgende udsagn er ækvivalente:
%
\begin{enumerate}[label=(\alph*)]
\item Ligningssystemet $A\mathbf{x}=\mathbf{b}$ er konsistent.
\item Vektoren $\mathbf{b}$ er en linearkombination af søjlerne i $A$.
\item Den reducerede trappeform af totalmatricen $[A \mid \mathbf{b}]$ har ingen rækker på formen $[ 0 \ldots 0 \mid d ]$, hvor $d \neq 0$.
\end{enumerate}
%
\end{thm}
%
%
\begin{proof}
%
Lad $A$ være en $m \times n$ matrix og $\mathbf{b} \in \R^m$. Jævnfør \ref{defn:mvp} følger det, at der eksisterer en vektor 
%
\begin{align*}    
       \mathbf{v} = \begin{bmatrix}
		v_{1} \\
        v_{2} \\
        \vdots \\
        v_{n} 
        \end{bmatrix},  
\end{align*}
%
i $\R^n$, der opfylder $A\mathbf{v}=\mathbf{b}$,
hvilket er linearkombinationen
%
$$v_1 \mathbf{a}_1+v_2 \mathbf{a}_2 + \ldots + v_n \mathbf{a}_n = \textbf{b}$$
%
af $A$ og $\textbf{v}$. 
Det følger derfor, at (a) og (b) er ækvivalente.\\\\
%
Herefter bevises, at (a) og (c) er ækvivalente. 
Hvis (c) er falsk, følger det, at
%
$$0 x_1+0 x_2 + \ldots + 0x_n =d,$$
%
hvor $d\neq0$. Da dette ikke har nogen løsning, følger det, at ligningssystemet således bliver inkonsistent. 
Såfremt (c) er sandt, gælder det, at alle rækker i den reducerede trappeform har en løsning, som også er løsning til $A\mathbf{x}=\mathbf{b}$. 
Således er (a) og (c) ækvivalente.
%
\end{proof}
\\
%
\begin{eks}
Med udgangspunkt i \ref{eks_gauss} ses derfor, at ligningssystemet for $A\textbf{x}=\textbf{b}$ er konsistent, da der ikke er pivotindgang i sidste søjle. Såfremt ligningssystemet skulle være inkonsistent, kunne den have formen
%
\begin{align*}
[B \mid \textbf{b} ]=\begin{blockarray}{ccccccc}
x_1 & x_2 & x_3 & x_4 & x_5 & b \\
\begin{block}{[ccccc|c]c}
  \hlight{1} & 0 & 0 & 0 & -1 & -2 \\
  0 & 0 & \hlight{1} & 0 & 3 & 3 \\
  0 & 0 & 0 & 0 & 0 & \hlight{-1} \\
\end{block}
\end{blockarray},
\end{align*}
%
da der eksisterer en pivotindgang i sidste søjle. 
%
\end{eks}