\subsection{Gauss elimination}
% 1.4
% Sætning 1.4 kan vi bevise den
En algoritme til løsningen af linære ligningssystemer er  \textbf{rækkereduktionsalgoritmen}, som omdanner en matrix fra en totalmatrice til reduceret trappeform.
Når den bruges til at løse linære ligningssystemer, kaldes denne process for \textbf{Gauss elimination}.
I algoritmen bruges de elemæntere rækkeoperationer for at omdanne en matrix $A$ til trappeform. 
Herefter benyttes rækkeoperationerne igen til at reducere matricen til trappeform og videre til reduceret trappeform.
Algortimen gennemgår således følgende skridt:
%
\begin{enumerate}
\item Find første ikke-nul søjlse i $A$ fra venstre.
\item Ved rækkeombytning placeres en ikke-nul indgang øverst i pivot-søjlen.
\item Skab nuller under pivot-indgangen øverst i pivot-søjlen ved hjælp af rækkeudskiftning.
\item Øverste række markeres som afsluttet og trin $1-3$ gennemføres nu på den næste række.
\item Alle rækker med pivot-indgange skaleres så alle pivot-indgange er lig $1$.
\item Ved rækkeudskiftning sikres nu $0$'er over og under pivot-indgangene.
\end{enumerate}
% Eksempel
\begin{eks}
Givet ligningssystemet:
\begin{align*}
x_1-x_3-2x_4-8x_5&=-3 \\
-2x_1+x_3+2x_4+9x_5&=5 \\
3x_1-2x_3-3x_4-15x_5&=-9 \\
\end{align*}
Ligningssystemet opskrevet som totalmatrix $[A|\mathbf{b}]$
%
\begin{equation*}
  A=
\begin{blockarray}{ccccccc}
x_1 & x_2 & x_3 & x_4 & x_5 & b \\
\begin{block}{[ccccc|c]c}
  1 & 0 & -1 & -2 & -8 & -3 \\
  -2 & 0 & 1 & 2 & 9 & 5 \\
  3 & 0 & -2 & -3 & -15 & -9 \\
\end{block}
\end{blockarray}
\end{equation*}
De elementære rækkeopperationer benyttes nu med henblik på at reducere totalmatricen til trappeform.
\begin{equation*}
\xrightarrow[R_3 \rightarrow R_3-3R_1]{R_2 \rightarrow R_2+2R_1} 
\begin{blockarray}{ccccccc}
x_1 & x_2 & x_3 & x_4 & x_5 & b \\
\begin{block}{[ccccc|c]c}
  1 & 0 & -1 & -2 & -8 & -3 \\
  0 & 0 & -1 & -2 & -7 & -1 \\
  0 & 0 & 1 & 3 & 9 & 0 \\
\end{block}
\end{blockarray}
\end{equation*}
\begin{equation*}
\xrightarrow{R_3 \rightarrow R_3+R_2}
\begin{blockarray}{ccccccc}
x_1 & x_2 & x_3 & x_4 & x_5 & b \\
\begin{block}{[ccccc|c]c}
  1 & 0 & -1 & -2 & -8 & -3 \\
  0 & 0 & -1 & -2 & -7 & -1 \\
  0 & 0 & 0 & 1 & 2 & -1 \\
\end{block}
\end{blockarray}
\end{equation*}
Algoritmen fortsættes nu med henblik på at opnå reduceret trappeform.
\begin{equation*}
\xrightarrow[R_2 \rightarrow R_2+2R_3]{R_1 \rightarrow R_1+2R_3}
\begin{blockarray}{ccccccc}
x_1 & x_2 & x_3 & x_4 & x_5 & b \\
\begin{block}{[ccccc|c]c}
  1 & 0 & -1 & 0 & -4 & -5 \\
  0 & 0 & -1 & 0 & -3 & -3 \\
  0 & 0 & 0 & 1 & 2 & -1 \\
\end{block}
\end{blockarray}
\end{equation*}
\begin{equation*}
\xrightarrow{R_2 \rightarrow -1R_2}
\begin{blockarray}{ccccccc}
x_1 & x_2 & x_3 & x_4 & x_5 & b \\
\begin{block}{[ccccc|c]c}
  1 & 0 & -1 & 0 & -4 & -5 \\
  0 & 0 & 1 & 0 & 3 & 3 \\
  0 & 0 & 0 & 1 & 2 & -1 \\
\end{block}
\end{blockarray}
\end{equation*}
\begin{equation*}
\xrightarrow{R_1 \rightarrow R_1+R_2}
\begin{blockarray}{ccccccc}
x_1 & x_2 & x_3 & x_4 & x_5 & b \\
\begin{block}{[ccccc|c]c}
  1 & 0 & 0 & 0 & -1 & -2 \\
  0 & 0 & 1 & 0 & 3 & 3 \\
  0 & 0 & 0 & 1 & 2 & -1 \\
\end{block}
\end{blockarray}
\end{equation*}
Herefter kan løsninger opskrives:
\begin{align*}
x_1-x_5&=-2 &\iff x_1&=x_5-2 \\
x_3-3x_5&=3 &\iff x_3&=-3x_5+3 \\
x_4+2x_5&=-1 &\iff x_4&=-2x_5-1 \\
\end{align*}
Parameterfremstillingen for løsningen i $\R^5$ bliver således:
  \begin{align*}
    \mathbf{x} &= \begin{bmatrix}
           x_{1} \\
           x_{2} \\
           x_{3} \\
           x_{4} \\
           x_{5} \\
         \end{bmatrix} 
         = \begin{bmatrix}
           -2 \\
           0 \\
           3 \\
           -1 \\
           0 \\
         \end{bmatrix}
         +x_2 \begin{bmatrix}
           0 \\
           1 \\
           0 \\
           0 \\
           0 \\
         \end{bmatrix}
         +x_5 \begin{bmatrix}
           1 \\
           0 \\
           -3 \\
           -2 \\
           1 \\
         \end{bmatrix}
  \end{align*} 
\label{eks_gauss}
%frie variable
\end{eks}
%
\subsection{Rang og nullitet}
%her skal der stå et eller andet
\begin{defn}{}{}
\textbf{Rangen} af en matrix $A$ er antallet af pivotindgange, noteret $\text{rang}(A)$. \\
\textbf{Nulliteten} af $A$ er antallet af søjler uden pivotindgange, noteret $\text{null}(A)$.
\end{defn}
%
I en $m \times n$-matrix er $\text{rang}(A)+\text{null}(A)=n$

Med udgangspunkt i eksempel \ref{eks_gauss} bliver rang og nullitet i 
$$A=\begin{blockarray}{ccccccc}
x_1 & x_2 & x_3 & x_4 & x_5 & b \\
\begin{block}{[ccccc|c]c}
  1 & 0 & 0 & 0 & -1 & -2 \\
  0 & 0 & 1 & 0 & 3 & 3 \\
  0 & 0 & 0 & 1 & 2 & -1 \\
\end{block}
\end{blockarray}$$ 
således $\text{rang}(A)=3$ og $\text{null}(A)=3$. 
%
% Hvilket er heldigt når der er 6 søjler
\begin{thm}{Konsistens}{konsistens}
Følgende udsagn er ækvivalente:
\begin{enumerate}
\item Matricen $A\mathbf{x}=\mathbf{b}$ er \textbf{konsistent}.
\item Vektoren $\mathbf{b}$ er en linearkombination af søjlerne i $A$.
\item Den reducerede trappeform af totalmatrixen $[A|\mathbf{b}]$ har ingen rækker af formen $[ 0 \ldots 0 | d  ]$, hvor $d \neq 0$.
\end{enumerate}
\end{thm}
%
\begin{proof}
Lad $A$ være en $n \times m$ matrix og lad $\mathbf{b}$ være i $\R^m$ ud fra definition \ref{defn:mvp} følger det, at der eksisterer en vektor 
$$    \mathbf{v} = \begin{bmatrix}
		v_{1} \\
        v_{2} \\
        \vdots \\
        v_{n} 
        \end{bmatrix} $$
i $\R^n$, der opfylder, at $A\mathbf{v}=\mathbf{b}$, hvis og kun hvis $$v_1 \mathbf{a}_1+v_2 \mathbf{a}_2 + \ldots + v_n \mathbf{a}_n$$ hvilket er linearkombinationen af matricen og vektoren. 
Det følger derfor at udsagn $(1)$ og udsagn $(2)$ er ækvivalente.
Herefter bevises at $(1)$ og $(3)$ er ækvivalente. 
Hvis udsagn $(3)$ er falsk følger det at  $$0 x_1+0 x_2 + \ldots + 0x_n =d$$ hvor $d\neq0$, da dette ikke har nogen løsning følger det, at ligningssystemet således bliver inkonsistent. 
Såfremt dette er sandt gælder omvendt, at alle rækker i den reducerede trappeform indeholder en løsning, som også er løsning til $A\mathbf{x}=\mathbf{b}$. 
Således er $(1)$ og $(3)$ ækvivalente.
\end{proof}
\\\\
%
Med udgangspunkt i eksempel \ref{eks_gauss}, ses derfor, at den er konsistent, da $d$ ikke er en pivot-indgang, såfremt ligningssystemet skulle være inkonsistent kunne den have følgende form: 
%
\begin{align*}
A=\begin{blockarray}{ccccccc}
x_1 & x_2 & x_3 & x_4 & x_5 & b \\
\begin{block}{[ccccc|c]c}
  1 & 0 & 0 & 0 & -1 & -2 \\
  0 & 0 & 1 & 0 & 3 & 3 \\
  0 & 0 & 0 & 0 & 0 & -1 \\
\end{block}
\end{blockarray} 
\end{align*}
Hvor $d \neq 0$ er i indgang $a_{3,6}$. 
Det følger derfor også, at man ikke kan opnå dette $b$, da det ikke kan laves som linear kombination af søjlerne. 