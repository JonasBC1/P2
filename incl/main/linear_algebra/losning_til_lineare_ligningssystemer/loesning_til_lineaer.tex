\section{Løsning til lineære ligningssystemer}
%
Disse ligningssystemer løses ved hjælp af forskellige opperationer, og i det følgende vil det introduceres, hvordan dette udføres.
%
\subsection{Elementære rækkeoperationer}
Der eksisterer tre operationer, som kan udføres på rækker i et lineært ligningssystem.
%
\begin{defn}{}{element}
Ved \textbf{ombytning} bytter to rækker plads i ligningssystemet, hvilket noteres
\begin{align*}
A \xrightarrow{R_i \leftrightarrow R_j} B, 
\end{align*}
hvor $A$ er matricen før rækkeoperationen udføres, og $B$ er matricen efter rækkeoperationen er udført. 
$R_j$ og $R_i$ er rækker i $A$.\\\\
Ved \textbf{skalering} multipliceres en række med et tal $c$, hvor $c \neq 0$, hvilket noteres
\begin{align*}
A \xrightarrow{R_i \rightarrow cR_i} B.
\end{align*}
Ved \textbf{udskiftning} udskiftes en række med rækken selv plus en skalering af en anden række, hvilket noteres
\begin{align*}
A \xrightarrow{R_i \rightarrow R_i + cR_h} B.
\end{align*}
%
\end{defn}
\noindent
Rækkeoperationerne kan udføres, når ligningssystemet er skrevet som en matrix. Eksempler på de elementære rækkeoperationer ses i eksempel \ref{eks1}.
\\
%
% Hvis vi bruger tilde skal den kommenteres hær
%
\begin{eks}\label{eks1}
Først laves ombytning mellem $R_1$ og $R_3$ i matricen. 
\begin{align*}
\begin{blockarray}{ccc}
\begin{block}{[cc|c]}
5 & 5 & 3 \\
2 & 2 & 1\\
3 & 4 & 8\\
\end{block}
\end{blockarray}
&\xrightarrow{R_1 \leftrightarrow R_3}
\begin{blockarray}{ccc}
\begin{block}{[cc|c]}
3 & 4 & 8\\
2 & 2 & 1\\
5 & 5 & 3\\
\end{block}
\end{blockarray}
\end{align*}
%
Derefter skaleres $R_2$ med $-2$. 
%
\begin{align*}
\begin{blockarray}{ccc}
\begin{block}{[cc|c]}
3 & 4 & 8\\
2 & 2 & 1\\
5 & 5 & 3\\
\end{block}
\end{blockarray}
&\xrightarrow{R_2 \rightarrow (-2)R_2}
\begin{blockarray}{ccc}
\begin{block}{[cc|c]}
3 & 4 & 8\\
-4 & -4 & -2\\
5 & 5 & 3\\
\end{block}
\end{blockarray}
\end{align*}
%
Slutteligt laves en udskiftning på $R_3$, hvor skaleringen $3R_1$ adderes med $R_3$. 
%
\begin{align*}
\begin{blockarray}{ccc}
\begin{block}{[cc|c]}
3 & 4 & 8\\
-4 & -4 & -2\\
5 & 5 & 3\\
\end{block}
\end{blockarray}
&\xrightarrow{R_3 \rightarrow 3R_1+R_3}
\begin{blockarray}{ccc}
\begin{block}{[cc|c]}
3 & 4 & 8\\
-4 & -4 & -2\\
14 & 17 & 27\\
\end{block}
\end{blockarray}
\end{align*}
\end{eks}
%
% Jeg kunne personligt godt tænke mig mere "matematik" i disse subsections, men jeg ved jo godt, at det er smag og behag. 
%
\subsection{Trappeform og reduceret trappeform}
En række i en matrix kaldes en \textbf{nulrække}, hvis alle indgange er nul, og en \textbf{ikke-nulrække}, hvis mindst én indgang i rækken er forskellig fra nul.
Dette bruges til definitionen af \textbf{trappeform}.
\begin{defn}{}{}
En matrix er på trappeform, hvis den opfylder følgende kriterier:
\begin{enumerate}[label=(\alph*)]
\item Enhver ikke-nulrække ligger over alle nul rækker.
\item Den første ikke-nulindgang i en ikke-nulrække ligger i en søjle til højre for første ikke-nulindgange i forrige række.
\item Hvis en søjle har den første ikke-nulindgang i en række, så er alle indgangene i søjlen under den $0$.
\end{enumerate}
\end{defn}
\begin{eks}
Betragt $A$ og $B$. $A$ er på trappeform, mens $B$ ikke er på trappeform, da $b_{3,2} \neq 0$. 
Hvis der foretages en ombytning på række to og tre, så ville matrix $B$ også være på trappeform
%
\begin{align*}
A=
\begin{blockarray}{cccc}
\begin{block}{[ccc|c]}
2 & 4 & 8 & 2\\
0 & 5 & -2 & 5\\
0 & 0 & 2 & 7\\
0 & 0 & 0 & 0\\
\end{block}
\end{blockarray}
\text{ }
B=
\begin{blockarray}{cccc}
\begin{block}{[ccc|c]}
2 & 4 & 8 & 2\\
0 & 0 & -2 & 5\\
0 & 4 & 2 & 7\\
0 & 0 & 0 & 4\\
\end{block}
\end{blockarray}.
\end{align*}
%
\end{eks}
%
Trappeformen kan reduceres yderligere.
%
\begin{defn}{}{redtrap}
En matrix er på \textbf{reduceret trappeform}, hvis den opfylder følgende kriterier:
\begin{enumerate}[label=(\alph*)]
\item Hvis en søjle har den første ikke-nulindgang i en række, så er alle indgange i søjlen $0$.
\item Den første indgang i hver ikke-nulrække er $1$. 
\end{enumerate}
\end{defn}
\noindent
Den reducerede trappeform af en matrix $A$ noteres  $R$. Et lineært ligningssystem, som ikke står på trappeform, kan laves om til trappeform og derefter reduceret trappeform, hvilket gør løsningen af systemet mere simpel.
Metoden til at gøre dette ses i afsnit \ref{gauss}.
\\
%
\begin{eks}\label{eks:trappe}
Betragt matricen $A$.
\begin{align*}
A=
\begin{blockarray}{ccccc}
\begin{block}{[cccc|c]}
1 & 0 & 5 & 0 & 0\\
0 & 1 & 3 & 0 & 0\\
0 & 0 & 0 & 1 & 0\\
0 & 0 & 0 & 0 & 1\\
\end{block}
\end{blockarray}
\end{align*}
%
Jævnfør definition \ref{defn:redtrap} er $A$ på reduceret trappeform.
\end{eks}
%
%Motiverende tekst
%
\begin{thm}{}{entydig}
Enhver matrix kan, ved hjælp af elementære rækkeoperationer, kun blive omdannet til en entydig matrix på reduceret trappeform.
\end{thm}
%
\begin{proof}
Lav - se Appendix E i bogen
\end{proof}
%
%%%%%%%%%%%%%%%%%%%%%%%%%%%%%%%%%%%%%%%%%%%%%%%%%%%%%%%%%%%%%
%
\subsection{Pivotindgange}
\begin{defn}{}{}
\textbf{Pivotindgange} betegnes som den første ikke-nulindgang i enhver række i en matrix på trappeform. 
\end{defn}
\noindent
\begin{eks}\label{eks:pivot}
Betragt matrix $A$ fra eksempel \ref{eks:trappe}. Pivotindgangene i $A$ er her markeret med blåt.
%
\begin{align*}
A=
\begin{blockarray}{cccc}
\begin{block}{[ccc|c]}
\hlight{2}	& 4			& 8			& 2\\
0			& \hlight{5}& -2		& 5\\
0			& 0			& \hlight{2}& 7\\
0			& 0			& 0			& 0\\
\end{block}
\end{blockarray}.
\end{align*}
%
\end{eks}
Pivot indgange vil fremover blive markeret med blå, som set i eksempel \ref{eks:pivot}.
Dette vil blive gjort første gang en matrice bliver lavet til trappeform eller reduceret trappeform.
%\begin{eks}\label{eks:pivot}
%Betragt eksempel \ref{trappe}. Det ses, at ligningssystemet ikke har nogen løsning, da den fjerde ligning ikke har nogen løsning.\\
%\begin{minipage}{0.5\textwidth}
%%
%\begin{align*}
%A=
%\begin{blockarray}{ccccc}
%x_1 & x_2 & x_3 & x_4 & b \\
%\begin{block}{[cccc|c]}
%\hlight{1} & 0 & 5 & 0 & 0\\
%0 & \hlight{1} & 3 & 0 & 0\\
%0 & 0 & 0 & \hlight{1} & 0\\
%0 & 0 & 0 & 0 & \hlight{1}\\
%\end{block}
%\end{blockarray}.
%\end{align*}
%\end{minipage}
%\begin{minipage}{0.5\textwidth}
%\begin{align*}
%x_1+5x_3&=0\\
%x_2+3x_3&=0\\
%x_4&=0\\
%0&\neq 1
%\end{align*}
%\end{minipage}
%%
%\end{eks}
Hvis der er pivotindgang i den sidste søjle, så har ligningssystemet ingen løsning.
%
De variables søjler, der indeholder pivotindgange, kaldes for \textbf{faste variable}, mens de, som ikke indholder pivotindgange, kaldes for \textbf{frie variable}. 
Ligningssystemet kan løses for de faste variable udtrykt ved de frie variable. 

\begin{thm}{}{pivotu}
Følgende udsagn er sande for enhver matrix $A$: 
\itemize 
\item (a) Pivotsøjlerne i $A$ er lineær uafhængige. 
\item (b) Enhver ikke-pivot søjle i $A$ er en linarkombination af den forrige pivotsøjle i A, hvor koefficienterne i linearkombinationen er indgangene i den tilsvarende søjle i den reducerede trappeform af $A$. 
\end{thm}