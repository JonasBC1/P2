%\newpage
\section{Løsning af lineære ligningssystemer}
%
Disse ligningssystemer kan løses ved hjælp af forskellige operationer,
som introduceres i følgende afsnit.
%
\subsection{Elementære rækkeoperationer}
Der eksisterer tre operationer, som kan udføres på rækkerne i matricen til et lineært ligningssystem.
%
\begin{defn}{}{element}
Ved \textbf{ombytning} bytter rækkerne $R_i$ og $R_j$ plads i ligningssystemet, hvilket noteres
\begin{align*}
A \xrightarrow{R_i \leftrightarrow R_j} B, 
\end{align*}
hvor $A$ er matricen før rækkeoperationen udføres, og $B$ er matricen efter rækkeoperationen er udført.\\\\
Ved \textbf{skalering} multipliceres en række med en skalar $c$, hvor $c \neq 0$, hvilket noteres
\begin{align*}
A \xrightarrow{R_i \rightarrow cR_i} B.
\end{align*}
Ved \textbf{udskiftning} udskiftes en række med rækken selv plus en skalering af en anden række, hvilket noteres
\begin{align*}
A \xrightarrow{R_i \rightarrow R_i + cR_h} B.
\end{align*}
%
\end{defn}
\noindent
Rækkeoperationerne kan udføres, når ligningssystemet er skrevet som en matrix. Brug af de elementære rækkeoperationer kan ses i \ref{eks1}.
\\
%
% Hvis vi bruger tilde skal den kommenteres hær
%
\begin{eks}\label{eks1}
Først ombyttes rækkerne $R_1$ og $R_3$ i matricen. 
\begin{align*}
\begin{blockarray}{ccc}
\begin{block}{[ccc]}
5 & 5 & 3 \\
2 & 2 & 1\\
3 & 4 & 8\\
\end{block}
\end{blockarray}
&\xrightarrow{R_1 \leftrightarrow R_3}
\begin{blockarray}{ccc}
\begin{block}{[ccc]}
3 & 4 & 8\\
2 & 2 & 1\\
5 & 5 & 3\\
\end{block}
\end{blockarray}
\end{align*}
%
Derefter skaleres $R_2$ med $-2$. 
%
\begin{align*}
\begin{blockarray}{ccc}
\begin{block}{[ccc]}
3 & 4 & 8\\
2 & 2 & 1\\
5 & 5 & 3\\
\end{block}
\end{blockarray}
&\xrightarrow{R_2 \rightarrow (-2)R_2}
\begin{blockarray}{ccc}
\begin{block}{[ccc]}
3 & 4 & 8\\
-4 & -4 & -2\\
5 & 5 & 3\\
\end{block}
\end{blockarray}
\end{align*}
%
Slutteligt udskiftes $R_3$ med skaleringen $3R_1$, adderet med $R_3$. 
%
\begin{align*}
\begin{blockarray}{ccc}
\begin{block}{[ccc]}
3 & 4 & 8\\
-4 & -4 & -2\\
5 & 5 & 3\\
\end{block}
\end{blockarray}
&\xrightarrow{R_3 \rightarrow R_3 + 3R_1}
\begin{blockarray}{ccc}
\begin{block}{[ccc]}
3 & 4 & 8\\
-4 & -4 & -2\\
14 & 17 & 27\\
\end{block}
\end{blockarray}
\end{align*}
\end{eks}
%
% Jeg kunne personligt godt tænke mig mere "matematik" i disse subsections, men jeg ved jo godt, at det er smag og behag. 
%
\subsection{Trappeform og reduceret trappeform}
En række i en matrix kaldes en \textit{nulrække}, hvis alle indgange er nul, og en \textit{ikke-nulrække}, hvis mindst én indgang i rækken er forskellig fra nul.
Dette bruges til definitionen af \textit{trappeform}.
%
\begin{defn}{}{trap}
En matrix er på \textbf{trappeform}, hvis den opfylder følgende kriterier:
\begin{enumerate}[label=(\alph*)]
\item Enhver ikke-nulrække ligger over alle nulrækker.
\item Den første ikke-nulindgang i en ikke-nulrække ligger i en søjle til højre for første ikke-nulindgange i forrige række.
%
\item Hvis en søjle har den første ikke-nulindgang i en række, så er alle understående indgange i søjlen nulindgange.
\end{enumerate}
\end{defn}
%
\begin{eks}\label{eks:trappe}
Betragt matricerne
%
\begin{align*}
A=
\begin{blockarray}{cccc}
\begin{block}{[cccc]}
2 & 4 & 8 & 2\\
0 & 5 & -2 & 5\\
0 & 0 & 2 & 7\\
0 & 0 & 0 & 0\\
\end{block}
\end{blockarray}
\text{ og }
B=
\begin{blockarray}{cccc}
\begin{block}{[cccc]}
2 & 4 & 8 & 2\\
0 & 0 & -2 & 5\\
0 & 4 & 2 & 7\\
0 & 0 & 0 & 4\\
\end{block}
\end{blockarray}.
\end{align*}
%
$A$ er på trappeform, mens $B$ ikke er på trappeform. 
Hvis der foretages en ombytning på række to og tre, da vil $B$ ligeledes være på trappeform.
%
\end{eks}
%
\noindent
%
Matricer kan yderligere reduceres til \textit{reduceret trappeform}.
%
\begin{defn}{}{redtrap}
En matrix er på \textbf{reduceret trappeform}, hvis den opfylder følgende kriterier:
\begin{enumerate}[label=(\alph*)]
\item Matricen er på trappeform.
\item Hvis en søjle har den første ikke-nulindgang i en række, så er alle øvrige indgange i søjlen $0$.
\item Den første ikke-nulindgang i hver ikke-nulrække er lig $1$. 
\end{enumerate}
\end{defn}
\noindent
%
Bemærk, at den reducerede trappeform af en matrix $A$ vil blive noteret $A_R$.
\\
\newpage
%
\begin{eks}
Jævnfør \ref{defn:redtrap} er
\begin{align*}
A_R=
\begin{blockarray}{ccccc}
\begin{block}{[ccccc]}
1 & 0 & 5 & 0 & 0\\
0 & 1 & 3 & 0 & 0\\
0 & 0 & 0 & 1 & 0\\
0 & 0 & 0 & 0 & 1\\
\end{block}
\end{blockarray}
\end{align*}
%
på reduceret trappeform.
\end{eks}
%
Et lineært ligningssystem, som ikke står på trappeform, kan omdannes til trappeform og derefter reduceret trappeform, hvilket gør løsningen af systemet mere simpel.
Metoden til at gøre dette ses i afsnit \ref{gauss}.
%
%%%%%%%%%%%%%%%%%%%%%%%%%%%%%%%%%%%%%%%%%%%%%%%%%%%%%%%%%%%%%
%
\subsection{Pivotindgange}
%
\begin{defn}{}{}
\textbf{Pivotindgange} betegnes som den første ikke-nulindgang i enhver række i en matrix på trappeform. 
\end{defn}
\noindent
%
\begin{eks}
\label{eks:pivot}
Betragt matrix $A$ fra \ref{eks:trappe}. Pivotindgangene i $A$ er her markeret med blåt.
%
\begin{align*}
A=
\begin{blockarray}{cccc}
\begin{block}{[ccc|c]}
\hlight{2}	& 4			& 8			& 2\\
0			& \hlight{5}& -2		& 5\\
0			& 0			& \hlight{2}& 7\\
0			& 0			& 0			& 0\\
\end{block}
\end{blockarray}.
\end{align*}
%
\end{eks}
%
Pivotindgange markeres fremover med blå første gang en matrice er på trappeform eller reduceret trappeform, som set i \ref{eks:pivot}.
Hvis der er pivotindgang i den sidste søjle, så har ligningssystemet ingen løsning.
Variable tilhørende en søjle, som indeholder en pivotindgang, betegnes som \textit{faste variable}.
Hvis ikke betegnes disse variable som \textit{frie variable}.
Ligningssystemer kan løses for de faste variable udtrykt ved de frie variable.