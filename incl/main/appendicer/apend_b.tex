%\chapter{Bliver slettet lige om lidt}
%%
%Egenskaber for lineært afhængige og uafhængige mængder.
%%
%\begin{thm}{}{egenskab_lin}
%\begin{enumerate}
%\item En mængde bestående af én ikke-nulvektor er lineært uafhængig, men $\{\textbf{0}\}$ er lineært afhængig.
%\item Ethvert underrum af $\R^n$, der indeholder mere end $n$ vektorer, er lineært afhængigt.
%\item Hvis spannet af $\S$ er et underrum af $\R^n$, og ingen vektorer kan fjernes uden at ændre spannet, så er $\S$ lineært uafhængig.
%\end{enumerate}
%\end{thm}
%%
\chapter{Span og dimension}
%
\ref{thm:spandim}, der omhandler spannet og dimensionen, er vigtig i forhold til visse egenskaber nævnt i rapporten.
%
\begin{thm}{}{spandim}
Antag, at spannet $\S$ af vektorene $\textbf{v}_1,\ldots,\textbf{v}_c$ har dimension $m$.
Så gælder, at
\begin{enumerate}[label=(\alph*)]
\item Der eksisterer en basis af $\S$ bestående af $m$ af vektorene $\mathbf{v}_1,\ldots,\mathbf{v}_c$.
\item Hvis $k\leq m$ og $\mathbf{v}_1,\ldots,\mathbf{v}_k$ er lineært uafhængige, kan der skabes en basis $\S$ ved $\mathbf{v}_1,\ldots,\mathbf{v}_k$, hvor der ligeledes vælges $m-k$ af vektorene $\mathbf{v}_{k+1},\ldots,\textbf{v}_c$.
\end{enumerate}
\end{thm}

\begin{proof}
Da (a) er et tilfælde af (b), hvor $k=0$, er beviset for (b) tilstrækkeligt.
Hvis alle vektorer $\mathbf{v}_{k+1},\ldots,\textbf{v}_c$ kan opnås som linearkombination af $\mathbf{v}_1,\ldots,\mathbf{v}_k$, så er alle vektorer i $\mathbf{v}_{1},\ldots,\textbf{v}_c$ ligeledes en linearkombination af $\mathbf{v}_1,\ldots,\mathbf{v}_k$, hvor $\mathbf{v}_1,\ldots,\mathbf{v}_k$ udgør en basis.
Såfremt dette ikke er tilfældet, gælder det, at en eller flere vektorer i $\mathbf{v}_{k+1},\ldots,\textbf{v}_c$ ikke er linearkombination af $\mathbf{v}_1,\ldots,\mathbf{v}_k$. 
Der udvælges nu en lineært uafhængig vektor blandt $\mathbf{v}_{k+1},\ldots,\textbf{v}_c$.
Derved opnås der en mængde bestående af $k+1$ lineært uafhængige vektorer fra $\mathbf{v}_{1},\ldots,\mathbf{v}_c$.
Den proces gentages $m-k$ gange og derved opnås en basis til $\S$.
\end{proof}
%