\chapter{Entydigheden af en matrix på reduceret trappeform}

\begin{thm}{}{redutrap}
Lad $A$ være en matrix med den reducerede trappeform $A_R$.
Så er følgende sandt:
\begin{enumerate}[label=(\alph*)]
\item  Hvis en søjle $j$ i $A$ er en linearkombination  af andre søjler i $A$, så er søjlen i $A_R$ en linearkombination af de tilsvarende søjler i $A_R$ med de samme koefficienter.
\item  Hvis en søjle $j$ i $A_R$ er en linearkombination  af andre søjler i $A_R$, så er søjlen i $A$ en linearkombination af de tilsvarende søjler i $A$ med de samme koefficienter.
\end{enumerate}
\end{thm}%
\begin{proof}
Jævnfør \ref{thm:eleinv} så gælder det, at der findes en invertibel matrix $P$, så $PA=A_R$.
Deraf gælder $P\textbf{a}_i=\textbf{r}_i$ for alle $i$.
Antag, at søjlen $j$ i $A$ er en linearkombination af de andre søjler i $A$.
Så findes skalarer $c_1,c_2,\ldots,c_k$, så det gælder, at \\
\begin{equation}
\textbf{a}_j=c_1\textbf{a}_1+c_2\textbf{a}_2+\cdots+c_k\textbf{a}_k.
\end{equation}
Derfor 
\begin{align*}
\textbf{r}_j=P\textbf{a}_j&=P(c_1\textbf{a}_1+c_2\textbf{a}_2+\cdots+c_k\textbf{a}_k) \\
&= c_1P\textbf{a}_1+c_2P\textbf{a}_2+\cdots+c_kP\textbf{a}_k\\
&= c_1\textbf{r}_1+c_2\textbf{r}_2+\cdots+c_k\textbf{r}_k.
\end{align*}
Beviset for (b) gør brug af sammen fremgangsmåde, men benytter dog $\textbf{r}_i=P^{-1}\textbf{a}_i.$
\end{proof}\\
%
Heraf kan der udledes en række egenskaber for en matrix på reduceret trappeform.
%
\begin{lem}{}{denderlort}
Lad $A_R$ være en $m \times n$ matrix på reduceret trappeform.
Så gælder, at
\begin{enumerate}[label=(\alph*)]
\item En søjle i $A_R$ er en pivotsøjle, hvis og kun hvis den er ikke-nul og ikke er en linearkombination af de foregående søjler i $A_R$. 
\item Den $j$'te pivotsøjle i $A_R$ er $\textbf{e}_j \in \R^m$ og derfor er pivotsøjlerne i $A_R$ lineært uafhængige. 
\item Antag, at $\textbf{r}_j$ ikke er en pivotsøjle i $A_R$, og at der er $k$ pivotsøjler i $A_R$ før $\textbf{r}_j$.
Så gælder det, at $\textbf{r}_j$ er en linearkombination af de $k$ forudgående pivotsøjler og koefficienterne er de $k$ første indgange i $\textbf{r}_j$. 
\end{enumerate}
\end{lem}
\noindent
%
\ref{lem:denderlort} følger direkte af \ref{defn:trap} og \ref{defn:redtrap}
og kan generaliseres til \ref{lem:pivotu}, som bruges til \ref{thm:entydig}.
%
\begin{lem}{}{pivotu}
Følgende udsagn er sande for enhver matrix $A$: 
\begin{enumerate}[label=(\alph*)]
\item Pivotsøjlerne i $A$ er lineært uafhængige. 
\item Enhver ikke-pivotsøjle i $A$ er en linearkombination af de forrige pivotsøjler i $A$, hvor koefficienterne i linearkombinationen er indgangene i den tilsvarende søjle i den reducerede trappeform af $A$. 
\end{enumerate} 
\end{lem}
%
\begin{thm}{}{entydig}
Enhver matrix kan, ved hjælp af elementære rækkeoperationer, kun blive omdannet til en entydig matrix på reduceret trappeform.
\end{thm}
%
\begin{proof}
%I følgende bevis refereres der til de tidligere egenskaber for en matrix på reduceret trappeform.\\\\
Lad $A$ være en matrix og $A_R$ være den reducerede trappeform af $A$. 
Jævnfør \ref{lem:denderlort}(a) gælder det, at en søjle i $A_R$ er en pivotsøjle i $A_R$, hvis og kun hvis den er en ikke-nulsøjle og ikke er en linearkombination af de forudgående pivotsøjler i $A_R$. 
Jævnfør \ref{thm:redutrap} gælder det, at hvis en søjle i $A_R$ er en pivotsøjle, så er den ikke en linearkombination af de forudgående søjler i $A_R$ og deraf er søjlen i $A$ heller ikke en linearkombination af de forudgående  søjler i $A$, hvilket betyder at positionerne af pivotsøjlerne i $A_R$ er entydigt bestemt af søjlerne i $A$. 
Endvidere gælder det, jævnfør \ref{lem:denderlort}(b), at den $j$'te pivotsøjle i $A_R$ er den $j$'te standardvektor i $\R^m$, og derfor er pivotsøjlerne i $A_R$ endvidere entydigt bestemt af søjlerne i $A$. \\\\
Dernæst bevises, at ikke-pivotsøjlerne i $A_R$ også er bestemt af søjlerne i $A$. 
Antag, at $\textbf{r}_j$ ikke er en pivotsøjle i $A_R$.
Hvis $\textbf{r}_j = \textbf{0}$ medfører det, at $\textbf{a}_j = \textbf{0}$ jævnfør \ref{thm:redutrap}. 
Antag nu, at $\textbf{r}_j \neq \textbf{0}$.
Så gælder det jævnfør \ref{lem:denderlort}(c), at $\textbf{r}_j$ er en linearkombination af de forudgående pivotsøjler i $A_R$, som er lineært uafhængige. 
Det gælder endvidere, at koefficienterne i linearkombinationen er indgangene i $\textbf{r}_j$, hvor indgangene, som ville være rækkens første ikke-nul-indgang, er nulindgange.
Jævnfør \ref{thm:redutrap} er $\textbf{a}_j$ ligeledes en linearkombination af de forudgående pivotsøjler i $A$, som er lineært uafhængige og har de samme koefficienter, der afhænger af $A$. 
Derfor er $\textbf{r}_j$ entydigt bestemt af $A$ og det konkluderes deraf, at $A_R$ er entydig bestemt af $A$. 
\end{proof}