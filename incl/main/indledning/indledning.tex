\chapter{Indledning}
%To til tre sider 
%Vi skal argumentere det er interessant at kigge på lineær programmering
%Kort beskrivelse af hvad vi har lavet 
%Eventuelt 
%   Kapitel 1: bla bla 
%   Kapitel 2: bla bla 
%   m.m.
Optimeringsproblemer er en fundamental del af matematikken, og i den anvendte matematik udgøre disse en vægtig del af matematikkeres virke.
I denne rapport vil lineære optimeringsproblemer undersøges. 
Dele af forskningsfeltet betragter disse som havende ophav ved Danzig og dennes udarbejdelse af simplexmetoden \citep[side 107]{refa} i 1947.
Der er dog tilstadighed diskussion om, hvor lineære programmeringsproblemer som matematisk felt stammer fra og andre peger på Sovjetunionen og behovet for at optimere produktionen i en planøkonomisk kontekst som ophav \citep[side 155]{refb}.
Forudsætningen for, at Danzig arbejdede med lineære programmeringsproblemer var hans ansættelse ved Royal Airforce \citep[side 107]{refa} og en applikation blev i forbindelse med den vestligt kontrollerede sektor af Berlin.
Vestberlin lå dybt inde i den sovjetisk kontrollerede sektor af Tyskland og som følge af spændingerne mellem Vesteuropa og Sovjet besluttede Josef Stalin at lukke for forsyninger via jorden til Berlin.
Det blev derfor nødvendigt for vestmagterne at etablere en luftbro til Berlin, der gjorde det muligt, trods de sovjetiske restriktioner, at indføre basale fornødenheder til de $2,5$ millioner mennesker der levede i byen.
Da en sådan måde at indføre vare på er ressourceintensiv, blev der derfor behov for at gøre dette mest effektivt underlagt de logistiske restriktioner.
Det er derfor et matematisk felt, som i sin brug er nært bundet op på praktiske problemstillinger dette værende økonomiske og logistiske.
Det er derfor også et felt, der har været nært knyttet op til militære problemstillinger som Danzigs ansættelsesforhold også antyder.
Danzigs bidrag til forskningsfeltet, altså simplexmetoden, er dog stadigt en fundamental del af hvordan løsningen af lineære programmeringsproblemer behandles i dag.
\\\\
Projektets primære arbejdsområde vil derfor omhandle simplexmetoden samt den geometriske fremstilling af lineære programmeringsproblemer.
Det skal dog nævnes, at der ligeledes findes optimeringsproblemer, som ikke kan beskrives igennem lineære sammenhænge, en afgrænsning bliver her at disse ikke vil beskrives i dette indeværende projekt.
Problemformuleringen bliver derfor som følger:
\begin{col}{}{}
Hvordan kan lineære programmeringsproblemer anskues geometrisk og hvordan kan disse løses ved hjælp af simplexmetoden?
\end{col}
Med henblik på at besvare denne vil følgende emneområder bliver inddraget:
\begin{itemize}
\item En introduktion til lineær algebra: Afsnittet har til hensigt at indføre passende notation, samt introducere væsentlige metoder og begreber til løsning af lineære programmeringsproblemer.
\item Lineære programmeringsproblemer: Projektets primære emneområde. Det vil her introduceres hvordan disse fremstilles matematiske samt interessante egenskaber herved.
\item Geometrisk fremstilling: her skal der stå et eller andet når vi ved hvad der er med.
\item Simplexmetoden: Beskrivelse og bevis af Simplexmetoden som løsningsalgoritme.
\end{itemize}