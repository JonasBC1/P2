\section{Degenerering}
% idonnotknow
Ud fra definition \ref{defn:basal} vides det at der skal være $n$ aktive betingelser for at $\mathbf{x}$ er en basal løsning. 
Nedenstående defineres tilfælde, hvor der er flere en $n$ aktive betingelser. 
%
\begin{defn}{}{}
En basal løsning $\mathbf{x} \in \R^n$ kaldes \textbf{degenereret}, hvis der er mere end $n$ aktive betingelser for $\mathbf{x}$.
\end{defn}
\noindent
%
En degeneret basal løsning har altså mere end de nødvendige $n$ aktive betingelser.
% 
I $\R^2$ er en degenereret løsning i et skæringspunkt af tre eller flere linjer. 
Ligeså er en degenereret løsning i $\R^3$ i et skæringspunkt af fire eller flere flader. 
På figur \ref{fig:mmmm2} ses $\mathbf{x}$, som er en degenereret basal løsning i $\R^2$, og på figur  \ref{fig:mmmm3} ses $\mathbf{x}$, som er en degenereret basal løsning i $\R^3$.
Generaliseret vil en degenereret løsning dermed opstå i skæringspunktet med mindst $n+1$ hyperplaner i $\R^n$.
%
%%%%%%%%%%%%%%%%%%%%%%%%%%%%%%%%
%%% Flot graf alla Julie     %%%
%%%%%%%%%%%%%%%%%%%%%%%%%%%%%%%%
%
\begin{center}
$
\begin{array}{cc}
\begin{minipage}[b]{0.45\textwidth}
%
%%%%%%%%%%%%%%%%%%%%%%%%%%%%%%%%
%%% Flot graf alla Julie     %%%
%%%%%%%%%%%%%%%%%%%%%%%%%%%%%%%%
%
%
\begin{center}
\begin{tikzpicture}[scale=6]
%
% Koordinater 
% ------------------------------------------------------
\coordinate (a) at (0.1,0.1,-0.1);
\coordinate (b) at (0.5,0.1,-0.1);
\coordinate (aa) at (0.35,0.6,-0.1);
\coordinate (bb) at (0.25,0.6,-0.1);
\coordinate (c) at (0.2,0.5,-0.1);
\coordinate (d) at (0.4,0.5,-0.1);
\coordinate (e) at (0.3,0.5,-0.1);
%
% Polyeden
% -------------------------------------------------------
\filldraw [fill=myblue,opacity=0.5] 
         (a) -- (b) -- (e) -- (a);
%        
% Streger 
% -------------------------------------------------------
  \draw[thick](d)--(c);
  \draw[thick](a)--(b)--(e)--(a);
  \draw[thick](e)--(aa);
  \draw[thick](e)--(bb);
%
%
% Punkt 
% -------------------------------------------------------
\filldraw [black] (e) circle (0.2pt);
\node at (0.3,0.6,-0.1) (){$A$};
% 
%
% Koordinatsystemet 
% -------------------------------------------------------
\draw[thick,->] (0,0,0) -- (0.9,0,0) node[anchor=south east]{$x$};
\draw[thick] (0,0,0) -- (-0.1,0,0);
\draw[thick,->] (0,0,0) -- (0,0.7,0) node[anchor=north west]{$y$};
\draw[thick] (0,0,0) -- (0,-0.1,0);
\draw[thick,->] (0,0,0) -- (0,0,0.5) node[anchor=south east]{$z$};
\draw[thick] (0,0,0) -- (0,0,-0.1);
%
\end{tikzpicture}
  \captionof{figure}{En polyede med en degenereret basal løsning $A$ i $\R^2$.}
  \label{fig:mmmm2}
\end{center}
%
%
\end{minipage}&
\begin{minipage}[b]{0.45\textwidth}
%
%%%%%%%%%%%%%%%%%%%%%%%%%%%%%%%%
%%% Flot graf alla Julie     %%%
%%%%%%%%%%%%%%%%%%%%%%%%%%%%%%%%
%
%
\begin{center}
\begin{tikzpicture}[scale=3]
% Koordinater
% -------------------------------------------------------
\coordinate (c) at (0.7,1.2,1.2);
\coordinate (d) at (1.2,1.2,1.2);
\coordinate (g) at (1.2,0.7,1.2);
\coordinate (h) at (0.7,0.7,1.2);
\coordinate (b) at (2,1.2,1.2);
\coordinate (f) at (1.5,0.7,1.2);
%
% Tegning af vektorer
% ------------------------------------------------------- 
  \draw[thick,->](f)--(b);
  \draw[thick,->](f)--(c);
  \draw[thick,->](f)--(1.7,1.2,1.6);
%
% Punkter og linje 
% -------------------------------------------------------
\filldraw[black] (0.95,1,1.2) circle (0pt) node[left] {$\mathbf{x}$};
\filldraw[black] (1.7,1,1.2) circle (0pt) node[right] {$\mathbf{y}$};
\filldraw[black] (1.4,1,1.2) circle (0pt) node[right] {$\mathbf{z}$};
%
% Planet - Hyperplanet
% -------------------------------------------------------
\filldraw [fill=myblue,opacity=0.3] 
         (b) -- (c) -- (1.7,1.2,1.6) -- cycle;
%
%
%% Koordinatssystem
%% ------------------------------------------------------
%\draw[thick,->] (0,0,-0.5) -- (1.5,0,-0.5) node[anchor=south east]{$x$};
%\draw[thick] (0,0,-0.5) -- (-0.2,0,-0.5);
%\draw[thick,->] (0,0,-0.5) -- (0,0.8,-0.5) node[anchor=north west]{$y$};
%\draw[thick] (0,0,-0.5) -- (0,-0.2,-0.5);
%\draw[thick,->] (0,0,-0.5) -- (0,0,0.3) node[anchor=south east]{$z$};
%\draw[thick] (0,0,-0.5) -- (0,0,-0.8);
%%
%
\end{tikzpicture}
  \captionof{figure}{Nej}
  \label{fig:mmmm3}
\end{center}
%
%
\end{minipage}
\end{array}
$
\end{center}
%
%
Ligeledes findes en definition for en degenereret basal løsning for polyeder på standardform. 
Jævnfør definition \ref{} skal der være $m$ aktive betingelser, og $n-m$ aktive ikke-negativitetsbetingelserne. Dermed skal en polyede på standardform have flere end $m-n$ ikke-negativitetsbetingelserne, for at have en degenereret basal løsning, hvilket er defineret nedenstående:
%
\begin{defn}{}{}
Lad $\mathcal{P}$ være en polyede på standardform
$P=\{ \mathbf{x} \in \R^n \mid A\mathbf{x}=\mathbf{b},x \geq 0 \}$, og lad $m$ være antallet af rækker i $A$.
En basal løsning $\mathbf{x}$ for $\mathcal{P}$ kaldes degenereret, hvis der er mere end $n-m$ af komponenterne i $\mathbf{x}=0$.
\end{defn}
\noindent
%
% Noget overgangstekst og et eksempel xD
%
%
Figur \ref{fig:jegerikkesyg}
%
%%%%%%%%%%%%%%%%%%%%%%%%%%%%%%%%
%%% Flot graf alla Julie     %%%
%%%%%%%%%%%%%%%%%%%%%%%%%%%%%%%%
%
\input{fig/tikz/geometri/degenerering_std}
%











