\section{Degenerering}
% idonnotknow
Af \ref{defn:basal} fremgår det, at der skal være $n$ aktive betingelser, for at $\mathbf{x}$ er en basal løsning. 
Nedenfor defineres tilfælde, hvor der er flere end $n$ aktive betingelser. 
%
\begin{defn}{}{}
En basal løsning $\mathbf{x} \in \R^n$ kaldes \textbf{degenereret}, hvis der er mere end $n$ aktive betingelser for $\mathbf{x}$.
\end{defn}
\noindent
%
En degenereret basal løsning har altså mere end de nødvendige $n$ aktive betingelser.
% 
I $\R^2$ findes en degenereret løsning i et skæringspunkt af tre eller flere linjer. 
Til sammenligning findes en degenereret løsning i $\R^3$ i et skæringspunkt af fire eller flere flader. 
På figur \ref{fig:mmmm2} ses $\mathbf{x}$, som er en degenereret basal løsning i $\R^2$, og på figur  \ref{fig:mmmm3} ses $\mathbf{x}$, som er en degenereret basal løsning i $\R^3$.
Generelt vil en degenereret løsning opstå i skæringspunktet med mindst $n+1$ hyperplaner i $\R^n$.
%
%%%%%%%%%%%%%%%%%%%%%%%%%%%%%%%%
%%% Flot graf alla Julie     %%%
%%%%%%%%%%%%%%%%%%%%%%%%%%%%%%%%
%
\begin{center}
$
\begin{array}{cc}
\begin{minipage}[b]{0.45\textwidth}
%
%%%%%%%%%%%%%%%%%%%%%%%%%%%%%%%%
%%% Flot graf alla Julie     %%%
%%%%%%%%%%%%%%%%%%%%%%%%%%%%%%%%
%
%
\begin{center}
\begin{tikzpicture}[scale=6]
%
% Koordinater 
% ------------------------------------------------------
\coordinate (a) at (0.1,0.1,-0.1);
\coordinate (b) at (0.5,0.1,-0.1);
\coordinate (a1) at (0.075,0.05,-0.1);
\coordinate (b1) at (0.525,0.05,-0.1);
\coordinate (a2) at (0.05,0.1,-0.1);
\coordinate (b2) at (0.55,0.1,-0.1);
\coordinate (aa) at (0.35,0.6,-0.1);
\coordinate (bb) at (0.25,0.6,-0.1);
\coordinate (c) at (0.2,0.5,-0.1);
\coordinate (d) at (0.4,0.5,-0.1);
\coordinate (e) at (0.3,0.5,-0.1);
%
% Polyeden
% -------------------------------------------------------
\filldraw [fill=myblue,opacity=0.5] 
         (a) -- (b) -- (e) -- (a);
%        
% Streger 
% -------------------------------------------------------
  \draw[thick](d)--(c);
  \draw[thick](a)--(b)--(e)--(a);
  \draw[thick](e)--(aa);
  \draw[thick](e)--(bb);
  \draw[thick](b)--(b1);
  \draw[thick](a)--(a1);
  \draw[thick](b)--(b2);
  \draw[thick](a)--(a2);
%
%
% Punkt 
% -------------------------------------------------------
\filldraw [black] (e) circle (0.2pt);
\node at (0.3,0.6,-0.1) (){$\mathbf{x}$};
\filldraw [black] (0,0,0.5) circle (0.pt);
% 
%
% Koordinatsystemet 
% -------------------------------------------------------
\draw[thick,->] (0,0,0) -- (0.9,0,0) node[anchor=south east]{$x$};
\draw[thick] (0,0,0) -- (-0.1,0,0);
\draw[thick,->] (0,0,0) -- (0,0.7,0) node[anchor=north west]{$y$};
\draw[thick] (0,0,0) -- (0,-0.1,0);
%
\end{tikzpicture}
  \captionof{figure}{Et polyeder med en degenereret basal løsning $\mathbf{x}$ i $\R^2$.}
  \label{fig:mmmm2}
\end{center}
%
%
\end{minipage}&
%\begin{minipage}[b]{0.45\textwidth}
%%
%%%%%%%%%%%%%%%%%%%%%%%%%%%%%%%%%
%%%% Flot graf alla Julie     %%%
%%%%%%%%%%%%%%%%%%%%%%%%%%%%%%%%%
%%
%%
%\begin{center}
%\begin{tikzpicture}[scale=6]
%% Koordinater 
%% ------------------------------------------------------
%\coordinate (a) at (0.1,0.1,-0.1);
%\coordinate (b) at (0.5,0.1,-0.1);
%\coordinate (c) at (0.1,0.1,-0.5);
%\coordinate (d) at (0.5,0.1,-0.5);
%\coordinate (e) at (0.3,0.5,-0.3);
%%
%% Planet - Hyperplanet
%% -------------------------------------------------------
%\filldraw [fill=myblue,opacity=0.3] 
%         (a) -- (b) -- (d) -- (c) -- (a);
%\filldraw [fill=myblue,opacity=0.3] 
%         (a) -- (e) -- (b) -- (a);
%\filldraw [fill=myblue,opacity=0.3] 
%         (b) -- (e) -- (d) -- (b);
%\filldraw [fill=myblue,opacity=0.3] 
%         (d) -- (e) -- (c) -- (d);
%\filldraw [fill=myblue,opacity=0.3] 
%         (c) -- (e) -- (a) -- (c);
%%        
%% Streger 
%% -------------------------------------------------------
%  \draw[thick](d)--(e);
%  \draw[dashed, thick](c)--(e);
%  \draw[thick](d)--(b);
%  \draw[dashed,thick](d)--(c);
%  \draw[dashed, thick](a)--(c);
%  \draw[thick](b)--(e);
%  \draw[thick](a)--(e);
%  \draw[thick](a)--(b);
%%
%%
%% Punkt 
%% -------------------------------------------------------
%\filldraw [black] (e) circle (0.2pt) node[above] {$\mathbf{x}$};
%% 
%% Koordinatsystemet 
%% -------------------------------------------------------
%\draw[thick,->] (0,0,0) -- (0.9,0,0) node[anchor=south east]{$x$};
%\draw[thick] (0,0,0) -- (-0.1,0,0);
%\draw[thick,->] (0,0,0) -- (0,0.7,0) node[anchor=north west]{$y$};
%\draw[thick] (0,0,0) -- (0,-0.1,0);
%\draw[thick,->] (0,0,0) -- (0,0,0.5) node[anchor=south east]{$z$};
%\draw[thick] (0,0,0) -- (0,0,-0.1);
%% 
%\end{tikzpicture}
%  \captionof{figure}{Et polyeder med en degenereret basal løsning $\mathbf{x}$ i $\R^3$.}
%  \label{fig:mmmm3}
%\end{center}
%%
%%
%\end{minipage}
%%
\begin{minipage}[b]{0.45\textwidth}
%
%%%%%%%%%%%%%%%%%%%%%%%%%%%%%%%%
%%% Flot graf alla Julie     %%%
%%%%%%%%%%%%%%%%%%%%%%%%%%%%%%%%
%
%
\begin{center}
\begin{tikzpicture}[scale=6]
% Koordinater 
% ------------------------------------------------------
\coordinate (a) at (0.3,0.3,0.5);
\coordinate (b) at (0.7,0.3,0.5);
\coordinate (c) at (0.3,0.3,0.1);
\coordinate (d) at (0.7,0.3,0.1);
\coordinate (e) at (0.5,0.7,0.3);
%
% Planet - Hyperplanet
% -------------------------------------------------------
\filldraw [fill=myblue,opacity=0.3] 
         (a) -- (b) -- (d) -- (c) -- (a);
\filldraw [fill=myblue,opacity=0.3] 
         (a) -- (e) -- (b) -- (a);
\filldraw [fill=myblue,opacity=0.3] 
         (b) -- (e) -- (d) -- (b);
\filldraw [fill=myblue,opacity=0.3] 
         (d) -- (e) -- (c) -- (d);
\filldraw [fill=myblue,opacity=0.3] 
         (c) -- (e) -- (a) -- (c);
%        
% Streger 
% -------------------------------------------------------
  \draw[thick](d)--(e);
  \draw[dashed, thick](c)--(e);
  \draw[thick](d)--(b);
  \draw[dashed,thick](d)--(c);
  \draw[dashed, thick](a)--(c);
  \draw[thick](b)--(e);
  \draw[thick](a)--(e);
  \draw[thick](a)--(b);
%
%
% Punkt 
% -------------------------------------------------------
\filldraw [black] (e) circle (0.2pt) node[above] {$\mathbf{v}$};
% 
% Koordinatsystemet 
% -------------------------------------------------------
\draw[thick,->] (0,0,0) -- (0.9,0,0) node[anchor=south east]{$x$};
\draw[thick] (0,0,0) -- (-0.1,0,0);
\draw[thick,->] (0,0,0) -- (0,0.7,0) node[anchor=north west]{$y$};
\draw[thick] (0,0,0) -- (0,-0.1,0);
\draw[thick,->] (0,0,0) -- (0,0,0.5) node[anchor=south east]{$z$};
\draw[thick] (0,0,0) -- (0,0,-0.1);
% 
\end{tikzpicture}
  \captionof{figure}{Et polyeder med en degenereret basal løsning $\mathbf{v}$ i $\R^3$.}
  \label{fig:mmmm3}
\end{center}
%
%
\end{minipage}
\end{array}
$
\end{center}
%
%
Ligeledes findes en definition for en degenereret basal løsning for polyeder på standardform. 
Jævnfør definition \ref{afsnit:fisk} skal der være $m$ aktive betingelser og $n-m$ aktive ikke-negativitetsbetingelserne. 
% Hedder det negativitetsbetingelser???? 
Dermed skal et polyeder på standardform have flere end $m-n$ ikke-negativitetsbetingelserne for at have en degenereret basal løsning, hvilket er formuleret i \ref{defn:degenenene}. 
%
\begin{defn}{}{degenenene}
Lad $\mathcal{P}$ være et polyeder på standardform
$P=\{ \mathbf{x} \in \R^n \mid A \mathbf{x}=\mathbf{b},x \geq 0 \}$, og lad $m$ være antallet af rækker i $A$.
En basal løsning $\mathbf{x}$ for $\mathcal{P}$ kaldes degenereret, hvis der er mere end $n-m$ af komponenterne i $\mathbf{x}$, der er lig $0$.
\end{defn}
%
\begin{eks}{}{}
%
% Hvis nogen har lyst (MADS? MATHIAS?) så må i gerne lave en figur til dette eksempel :D - Julie som er pro til tikz men ikke til disse former for tikz xD
%
Betragt et polyeder $\mathcal{P}$ på standardform: 
%
\begin{align*}
\mathcal{P} = \{ 
\mathbf{x} \in \R^3 \text{  } | 
\text{  } 2 x_1 + 4 x_2 = 0, 
\text{  } 2 x_1 + 4 x_2 + 4 x_3 = 16, 
\text{  } x_1 , x_2, x_3 \geq 0 \}
\end{align*}
%
For $\mathcal{P}$ er $n=3$ dimensionen og $m=2$ aktive betingelser. 
Løsningen $\mathbf{x}$ kaldes degenereret, hvis flere end $n-m=1$ af komponenterne i $\mathbf{x}$ er lig nul.
Løsningen $\mathbf{x}= [ \text{ } 0 \text{  } 0 \text{  } 4 \text{ } ]^T $ er en degenereret basal løsning, da der er flere end ét komponent i $\mathbf{x}$, som er lig nul. 
Derimod er  $\mathbf{x}= [ \text{ } 4 \text{  } 2 \text{  } 0 \text{ } ]^T $ en basal løsning, da der er kun er ét komponent i $\mathbf{x}$, som er lig $0$. 
%
\end{eks}
% 
%
%
I forbindelse med simplex-metoden har det vist sig at være problematisk med basale løsninger, der er degenererede, hvorfor disse vil forsøges undgået. 
Dette vil blive belyst i afsnit \ref{coronaaaaaaaaaaa}.
Ved en mindre ændring $\epsilon$ i en overflødig aktiv betingelse vil en degenereret basal løsning kunne undgås. 
På figur \ref{fig:jegerikkesyg} ses $\mathbf{a}_1 \mathbf{x} = b_1$, som er en overflødig aktiv betingelse, der medfører en degenereret basal løsning $\mathbf{x}$.
En lille ændring $\epsilon$ vil ændre den degenererede basale løsning til en basal løsning, hvilket kan ses på figur \ref{fig:mmmjegerikkesyg}, hvor 
$\mathbf{a}_1 \mathbf{x} = b_1 - \epsilon$.
%
%%%%%%%%%%%%%%%%%%%%%%%%%%%%%%%%
%%% Flot graf alla Julie     %%%
%%%%%%%%%%%%%%%%%%%%%%%%%%%%%%%%
%
\begin{center}
$
\begin{array}{cc}
\begin{minipage}[b]{0.45\textwidth}
%
%%%%%%%%%%%%%%%%%%%%%%%%%%%%%%%%
%%% Flot graf alla Julie     %%%
%%%%%%%%%%%%%%%%%%%%%%%%%%%%%%%%
%
%
\begin{center}
\begin{tikzpicture}[scale=6]
%
% Koordinater 
% ------------------------------------------------------
\coordinate (a) at (0.1,0.1,-0.1);
\coordinate (b) at (0.5,0.1,-0.1);
\coordinate (a1) at (0.075,0.05,-0.1);
\coordinate (b1) at (0.525,0.05,-0.1);
\coordinate (a2) at (0.05,0.1,-0.1);
\coordinate (b2) at (0.55,0.1,-0.1);
\coordinate (aa) at (0.35,0.6,-0.1);
\coordinate (bb) at (0.25,0.6,-0.1);
\coordinate (c) at (0.17,0.5,-0.1);
\coordinate (d) at (0.43,0.5,-0.1);
\coordinate (e) at (0.3,0.5,-0.1);
%
% Polyeden
% -------------------------------------------------------
\filldraw [fill=myblue,opacity=0.5] 
         (a) -- (b) -- (e) -- (a);
%        
% Streger 
% -------------------------------------------------------
  \draw[thick](d)--(c);
  \draw[thick](a)--(b)--(e)--(a);
  \draw[thick](e)--(aa);
  \draw[thick](e)--(bb);
  \draw[thick](b)--(b1);
  \draw[thick](a)--(a1);
  \draw[thick](b)--(b2);
  \draw[thick](a)--(a2);
%
%
% Punkt 
% -------------------------------------------------------
\filldraw [black] (e) circle (0.2pt);
\node at (0.3,0.6,-0.1) (){$\mathbf{v}$};
\node at (0.6 ,0.5,-0.1) (){$\mathbf{a}_1 \mathbf{x} = b_1$};
\filldraw [black] (0,0,0.5) circle (0.pt);
% 
% 
%
% Koordinatsystemet 
% -------------------------------------------------------
\draw[thick,->] (0,0,0) -- (0.9,0,0) node[anchor=south east]{$x$};
\draw[thick] (0,0,0) -- (-0.1,0,0);
\draw[thick,->] (0,0,0) -- (0,0.7,0) node[anchor=north west]{$y$};
\draw[thick] (0,0,0) -- (0,-0.1,0);
%
\end{tikzpicture}
  \captionof{figure}{En overflødig aktiv betingelse $\mathbf{a}_1 \mathbf{x} = b_1$, som medfører, at $\mathbf{v}$ er en degenereret basal løsning.\\}
  \label{fig:jegerikkesyg}
\end{center}
%
\end{minipage}&
\begin{minipage}[b]{0.45\textwidth}
%
%%%%%%%%%%%%%%%%%%%%%%%%%%%%%%%%
%%% Flot graf alla Julie     %%%
%%%%%%%%%%%%%%%%%%%%%%%%%%%%%%%%
%
%
\begin{center}
\begin{tikzpicture}[scale=6]
%
% Koordinater 
% ------------------------------------------------------
\coordinate (a) at (0.1,0.1,-0.1);
\coordinate (b) at (0.5,0.1,-0.1);
\coordinate (aa) at (0.35,0.6,-0.1);
\coordinate (bb) at (0.25,0.6,-0.1);
\coordinate (a1) at (0.075,0.05,-0.1);
\coordinate (b1) at (0.525,0.05,-0.1);
\coordinate (a2) at (0.05,0.1,-0.1);
\coordinate (b2) at (0.55,0.1,-0.1);
\coordinate (c) at (0.17,0.46,-0.1);
\coordinate (d) at (0.43,0.46,-0.1);
\coordinate (cc) at (0.28,0.46,-0.1);
\coordinate (dd) at (0.32,0.46,-0.1);
\coordinate (e) at (0.3,0.5,-0.1);
%
% Polyeden
% -------------------------------------------------------
\filldraw [fill=myblue,opacity=0.5] 
         (a) -- (b) -- (dd) -- (cc) -- (a);
%        
% Streger 
% -------------------------------------------------------
  \draw[thick](d)--(c);
  \draw[thick](a)--(b)--(e)--(a);
  \draw[thick](e)--(aa);
  \draw[thick](e)--(bb);
  \draw[thick](b)--(b1);
  \draw[thick](a)--(a1);
  \draw[thick](b)--(b2);
  \draw[thick](a)--(a2);
%
%
% Punkt 
% -------------------------------------------------------
\filldraw [black] (e) circle (0.2pt);
\node at (0.3,0.6,-0.1) (){$\mathbf{v}$};
\node at (0.6 ,0.5,-0.1) (){$\mathbf{a}_1 \mathbf{x} = b_1 - \varepsilon$};
\filldraw [black] (0,0,0.5) circle (0.pt);
% 
% 
%
% Koordinatsystemet 
% -------------------------------------------------------
\draw[thick,->] (0,0,0) -- (0.9,0,0) node[anchor=south east]{$x$};
\draw[thick] (0,0,0) -- (-0.1,0,0);
\draw[thick,->] (0,0,0) -- (0,0.7,0) node[anchor=north west]{$y$};
\draw[thick] (0,0,0) -- (0,-0.1,0);
%
\end{tikzpicture}
  \captionof{figure}{En lille ændring $\varepsilon$ i den aktive betingelse sådan, at $\mathbf{a}_1 \mathbf{x} = b_1- \varepsilon$, hvilket medfører, at $\mathbf{v}$ ikke er en degenereret basal løsning.}
  \label{fig:mmmjegerikkesyg}
\end{center}
%
\end{minipage}
\end{array}
$
\end{center}
%
%
I praktiske problemer, hvor $A$ og $\mathbf{b}$ har specifikke værdier , viser det sig dog ofte at der opstår degenererede basale løsninger.
%