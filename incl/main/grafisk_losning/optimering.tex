\section{Optimering af ekstremumspunkter}
%
Så længe en optimal løsning til et lineært optimeringsproblem eksisterer, og løsningsmængden indeholder ét eller flere ekstremumspunkter, så findes en optimal løsning til det lineære programmeringsproblem blandt ekstremumspunkterne i løsningsmængden. 
Sætning \ref{thm:optiekstrem1} fastslår dette for polyedre på standardform og begrænsede polyedre. 
%
\begin{thm}{}{optiekstrem1}
Lad et lineært optimeringsproblem være, at en objektfunktion $\textbf{c}^T \textbf{x}$ skal minimeres over et polyeder $\mathcal{P}$. 
Antag, at der i $\mathcal{P}$ er mindst ét ekstremumspunkt, samt at en optimal løsning eksisterer. 
Så eksisterer der en optimal løsning, som er et ekstremumspunkt i $\mathcal{P}$.
\end{thm}
%
%
\begin{proof}
Lad et lineært optimeringsproblem være, at $\textbf{c}^T \textbf{x}$ skal minimeres over et polyeder $\mathcal{P}$. 
Antag, at der i $\mathcal{P}$ er mindst ét ekstremumspunkt, samt at en optimal løsning eksisterer. 
Lad $Q \neq \emptyset $ være mængden af optimale løsninger, lad $\mathcal{P} = \{\textbf{x} \in \R^n | \textbf{A}\textbf{x} \geq \textbf{b} \}$, og lad $v$ være den optimale værdi for $\textbf{c}^T \textbf{x}$. 
Da $\mathcal{P}$ er løsningsmængden, og $Q \subset \mathcal{P}$, så er $Q = \{\textbf{x} \in \R^n | \textbf{A}\textbf{x} \geq \textbf{b} \text{, } \textbf{c}^T \textbf{x} = v \}$ og dermed også et polyeder. 
Jævnfør sætning \ref{thm:ekstremums1} indeholder $\mathcal{P}$ ingen linjer, hvormed $Q$ ikke indeholder linjer. 
Jævnfør sætning \ref{thm:ekstremums1} indeholder $Q$ dermed et ekstremumspunkt. \\
Lad nu den optimale løsning $\textbf{x}^*$ være et ekstremumspunkt i $Q$. 
Ved hjælp af modstrid vises nu, at $\textbf{x}^*$ også er et ekstremumspunkt i $P$. 
Antag derfor, at $\textbf{x}^*$ \textit{ikke} er et ekstremumspunkt i $\mathcal{P}$. 
Så eksisterer der $\textbf{y} \in \mathcal{P}$, $\textbf{y} \neq \textbf{x}^*$, og $\textbf{z} \in P$, $\textbf{z} \neq \textbf{x}^*$, samt et $\lambda \in [0,1]$, således $ \textbf{x}^* = \lambda \textbf{y} + (1 - \lambda) \textbf{z}$. 
Deraf følger, at $v = \textbf{c}^T \textbf{x}^* =  \lambda \textbf{c}^T \textbf{y} + (1 - \lambda) \textbf{c}^T \textbf{z}$. 
Desuden må det gælde, at $\textbf{c}^T \textbf{y} \geq v$ og $\textbf{c}^T \textbf{z} \geq v$, da $v$ er den optimale værdi. 
Dette medfører dog, at $ \textbf{c}^T \textbf{y} = \textbf{c}^T \textbf{z} = v$, samt at $\textbf{y} \in Q$ og $\textbf{z} \in Q$. 
Derved opstår modstriden, da $\textbf{x}^*$ er et ekstremumspunkt i $Q$. 
Således er det ved modstrid vist, at $\textbf{x}^*$ er et ekstremumspunkt i $\mathcal{P}$, samt at det er en optimal løsning, da $\textbf{x}^* \in Q$. 
\end{proof}\\
%
%
%
Som nævnt gælder sætning \ref{thm:optiekstrem1} kun for polyedre på standard form og begrænsede polyedre. 
Sætning \ref{thm:optiekstrem2} kan betragtes som en udvidelse af sætning \ref{thm:optiekstrem1} og viser, at der findes en optimal løsning, så længe den optimale værdi af objektfunktionen er endelig. 
%
\begin{thm}{}{optiekstrem2}
Lad et lineært optimeringsproblem være, at $\textbf{c}^T \textbf{x}$ skal minimeres over et polyeder $P$. 
Antag, at der i $P$ er mindst ét ekstremumspunkt. 
Så er den optimale værdi af objektfunktionen lig $- \infty$, eller også findes et optimalt ekstremumspunkt. 
\end{thm}
%
\begin{proof}
Bemærk indledningsvist, at et element $\textbf{x}$ i et polyder $\mathcal{P}$ har rang$(\textbf{x})=k$, hvis der findes præcis $k$ lineært uafhængige betingelser, der er aktive ved $\textbf{x}$. 
Antag, at den optimale værdi af objektfunktionen er endelig. 
\\\\
%
Lad et polyeder $\mathcal{P} = \{\textbf{x} \in \R^n \mid \textbf{A}\textbf{x} \geq \textbf{b} \}$ og bemærk, at $\textbf{x} \in \mathcal{P}$, hvor $\textbf{x}$ har rang$(\textbf{x})=k$, $k < n$. 
Lad $i = \{ i \mid \textbf{a}^T_i \textbf{x} = b_i \}$, hvor $\textbf{a}^T_i$ er den $i$'te række af $A$. 
Da $k < n$, ligger vektorerne $\textbf{a}_i$, $i \in I$, i et ægte underrum af $ \R^n$, og der kan vælges en ikke-nul vektor $\textbf{d} \in \R$, som er ortogonal til hver $\textbf{a}_i$. 
Det kan desuden antages, at $\textbf{c}^T \textbf{d} \leq \textbf{0}$, ved muligvis at tage negationen af $\textbf{d}$. 
\\\\
%
Antag nu, at $\textbf{c}^T \textbf{d} < 0$, og lad $\textbf{y} = \textbf{x} + \lambda \textbf{d}$, $ \lambda \in \R^+$, være en halvlinje. 
%Hvad er en halvlinje...?? 
Alle punkter på denne halvlinje opfylder $\textbf{a}^T_i \textbf{y} = b_i$, som i beviset for sætning \ref{thm:ekstremums1}. 
%Er ovenstående rigtigt?? 
Hvis hele halvlinjen var indeholdt i $\mathcal{P}$, havde den optimale værdi for objektfunktionen været $- \infty $; men det antages indledningsvist i dette bevis, at denne værdi er endelig. 
Så er ikke hele halvlinjen indeholdt i $\mathcal{P}$, og den udgår dermed fra $\mathcal{P}$ på et tidspunkt. 
Når halvlinjen er på grænsen af $\mathcal{P}$, lige inden dens udgang derfra, er der et optimalt $ \lambda^* > 0$ og et $j \in I$, således at $\textbf{a}^T_i (\textbf{x} + \lambda^* \textbf{d} ) = b_j $.
\\\\%Se Horias noter fra tirsdag d. 24/3. Det gav så god mening, da han forklarede det, men nu kan jeg ikke huske, hvad han ville have. 
Lad nu $\textbf{y} = \textbf{x} + \lambda^* \textbf{d} $ og bemærk, at $ \textbf{c}^T \textbf{y} < \textbf{c}^T \textbf{x}$, da $ \lambda \in \R^+ $. 
Det vides fra beviset for sætning \ref{thm:ekstremums1}, at $\textbf{a}_j$ er lineært uafhængig af $\textbf{a}_i$, samt at rangen af $\textbf{y}$ minimum er $k + 1$. 
\\\\
%
Antag nu, at $\textbf{c}^T \textbf{d} = 0$, og lad en linje være $\textbf{y} = \textbf{x} + \lambda \textbf{d}$, hvor $ \lambda $ er en arbitrær skalar. 
Eftersom $\mathcal{P}$ ikke indeholder linjer, må linjen nødvendigvis udgå fra $P$ på et tidspunkt. 
Som før haves nu igen en vektor $\textbf{y}$, hvis rang er højere end rangen af $\textbf{x}$. 
Da $\textbf{c}^T \textbf{d} = 0$, haves desuden, at $\textbf{c}^T \textbf{y} = \textbf{c}^T \textbf{x}$. 
Det er nu vist, at der i begge tilfælde findes et punkt $\textbf{y}$ med højere rang end $\textbf{x}$, således $\textbf{c}^T \textbf{y} \leq \textbf{c}^T \textbf{x}$. 
Antag, at denne proces nu fortsættes.
Så findes en vektor $\textbf{w}$ med rang$(\textbf{w})=n$, hvormed den er en basal mulig løsning, således at $\textbf{c}^T \textbf{w} \leq \textbf{c}^T \textbf{x}$. 
Lad $w_1, w_2, \ldots , w_r$ være de basale mulige løsninger i $\mathcal{P}$, og lad $\textbf{w}^*$ være en basal mulig løsning, således at $\textbf{c}^T \textbf{w}^* \leq \textbf{c}^T \textbf{w}_i$ for alle $i$. 
Eftersom der for alle $\textbf{x}$ findes et $i$, således at $\textbf{c}^T \textbf{w}_i \leq \textbf{c}^T \textbf{x}$, følger det, at $\textbf{c}^T \textbf{w}^* \leq \textbf{c}^T \textbf{x}$ for alle $\textbf{x} \in \mathcal{P}$, hvormed vektoren $\textbf{w}^*$ er den optimale løsning. 
\end{proof}\\
%
Eftersom ethvert lineært optimeringsproblem kan omskrives til et tilsvarende problem på standardform, kan sætning \ref{thm:optiekstrem2} siges at gælde generelt. 
Deraf haves korollar \ref{kor:optiekstrem2}. 
%
\begin{kor}{}{optiekstrem2}
Lad et lineært optimeringsproblem være, at $\textbf{c}^T \textbf{x}$ skal minimeres over et ikke-tomt polyeder.
Så er den optimale værdi af objektfunktionen lig $- \infty$, eller også findes en optimal løsning. 
\end{kor}
%
%