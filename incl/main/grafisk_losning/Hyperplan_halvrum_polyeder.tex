I henhold til geometriske fremstillinger af lineære programmeringsproblemer er det nødvendigt at definere en række begreber.
Disse er \textit{polyeder}, \textit{hyperplan} og \textit{halvrum}.
Dette kapitel tager udgangspunkt i \citep[side 42-70]{bert}, hvis ikke andet er angivet.
%
\begin{defn}{}{Polyede}
Lad $A$ være en $m \times n$ matrix, og lad $\mathbf{b}$ være en vektor i  $\R^m$.
Et \textbf{polyeder} $\mathcal{P}$ er en mængde, der kan beskrives som 
$\mathcal{P} = \{\textbf{x} \in \R^n \mid A\mathbf{x}\geq \textbf{b}\}.$
%
\end{defn}
\noindent
%
Som det fremgår af \ref{defn:Polyede}, er et polyeder mængden af mulige løsninger $\mathbf{x}$ til et ligningssystem, hvilket også kaldes værdimængden.
Det gælder endvidere, at en mængde på formen $\mathcal{P} = \{\textbf{x} \in \R^n \mid A\textbf{x} = \textbf{b},\textbf{x} \geq \textbf{0} \}$ kaldes et \textit{polyeder på standardform}, hvilket beskrives yderligere i afsnit \ref{afsnit:fisk}.
%
%
På figur \ref{fig:nej2} ses et polyeder $\mathcal{P}_1$, der er markeret med blå i $\R^2$, og på figur \ref{fig:nej3} ses et polyeder $\mathcal{P}_2$, markeret med blå i $\R^3$.
Bemærk, at løsningsmængden på figur \ref{fig:min_beg}, som er markeret med blå, er et polyeder.
%
%%%%%%%%%%%%%%%%%%%%%%%%%%%%%%%%
%%% Flot graf Af la Julie    %%%
%%%%%%%%%%%%%%%%%%%%%%%%%%%%%%%%
%
\begin{center}
$
\begin{array}{cc}
\begin{minipage}[b]{0.45\textwidth}
%%%%%%%%%%%%%%%%%%%%%%%%%%%%%%%%
%%% Flot graf alla Julie     %%%
%%%%%%%%%%%%%%%%%%%%%%%%%%%%%%%%
%
\begin{center}
%
\begin{tikzpicture}[scale=5]
%
% Koordinater
% -------------------------------------------------------
\coordinate (a) at (1.2,1.2,0.7);
\coordinate (b) at (0.7,1.2,0.7);
\coordinate (c) at (0.7,1.2,1.2);
\coordinate (d) at (1.2,1.2,1.2);
\coordinate (f) at (1.2,0.7,0.7);
\coordinate (g) at (1.2,0.7,1.2);
\coordinate (h) at (0.7,0.7,1.2);
%
% Farvning
% -------------------------------------------------------
\filldraw[fill=myblue,opacity=0.3, thick](c)--(d)--(g)--(h)--(c);
  \draw[thick](d)--(g);
  \draw[thick](d)--(c);
  \draw[thick](g)--(h);
  \draw[thick](h)--(c);
%
% Navngivning og prik til at gøre det pænt 
% -------------------------------------------------------
\draw[black] (0.91,0.96,1.2) circle (0pt) node[anchor=west] {$\mathcal{P}$};
\draw[black] (0,0,0.7) circle (0pt);
% 
% 
% Koordinatssystem 
% -------------------------------------------------------
\draw[thick,->] (0,0,0) -- (1,0,0) node[anchor=south east]{$x$};
\draw[thick] (0,0,0) -- (-0.2,0,0);
\draw[thick,->] (0,0,0) -- (0,1,0) node[anchor=north west]{$y$};
\draw[thick] (0,0,0) -- (0,-0.2,0);
%
%
\end{tikzpicture}
  \captionof{figure}{En polyede $\mathcal{P}$ markeret med blå i $\R^2$.}
  \label{fig:nej2}
\end{center}
%
\end{minipage}&
\begin{minipage}[b]{0.49\textwidth}
\begin{center}
%start tikz picture, and use the tdplot_main_coords style to implement the display 
%coordinate transformation provided by 3dplot
\begin{tikzpicture}[scale=5]%[scale=5,tdplot_main_coords]
%set up some coordinates 
%-----------------------
\coordinate (a) at (1.2,1.2,0.7);
\coordinate (b) at (0.7,1.2,0.7);
\coordinate (c) at (0.7,1.2,1.2);
\coordinate (d) at (1.2,1.2,1.2);
\coordinate (f) at (1.2,0.7,0.7);
\coordinate (g) at (1.2,0.7,1.2);
\coordinate (h) at (0.7,0.7,1.2);
%
%draw figure contents
%--------------------
\filldraw[fill=myblue,opacity=0.3, thick](a)--(b)--(c)--(d)--(a)--(f)--(g)--(h)--(c)--(d)--(a);
  \draw[thick](d)--(g);
  \draw[thick](g)--(f);
  \draw[thick](f)--(a);
  \draw[thick](a)--(d);
  \draw[thick](d)--(c);
  \draw[thick](g)--(h);
  \draw[thick](c)--(b);
  \draw[thick](h)--(c);
  \draw[thick](b)--(a);
  \draw[gray, thick, dashed](f)--(0.7,0.7,0.7)--(0.7,1.2,0.7);
  \draw[gray, thick, dashed](0.7,0.7,0.7)--(0.7,0.7,1.2);
%
\draw[black] (0.91,0.96,1.2) circle (0pt) node[anchor=west] {$\mathcal{P}$};
%draw the main coordinate system axes
\draw[thick,->] (0,0,0) -- (1,0,0) node[anchor=south east]{$x$};
\draw[thick] (0,0,0) -- (-0.2,0,0);
\draw[thick,->] (0,0,0) -- (0,1,0) node[anchor=north west]{$y$};
\draw[thick,dashed] (0,0,0) -- (0,-0.2,0);
\draw[thick,->] (0,0,0) -- (0,0,0.7) node[anchor=south east]{$z$};
\draw[thick] (0,0,0) -- (0,0,-0.2);
\end{tikzpicture}
  \captionof{figure}{En polyede $\mathcal{P}$ markeret med blå i $\R^3$.}
  \label{fig:nej3}
\end{center}
\end{minipage}
\end{array}
$
\end{center}
%
\textit{Randen} er begrænsningerne, som afgrænser polyederet fra resten af rummet. 
Dermed er randen på figur \ref{fig:nej2} linjerne, der afgrænser $\mathcal{P}$ fra resten af $\R^2$.
Randen på figur \ref{fig:nej3} er derimod udsnittet af fladerne, som udgør terningen, der afgrænser $\mathcal{P}$ fra resten af $\R^3$.
\\\\
%
Det gælder endvidere for polyedre, at de enten kan være \textit{begrænsede} eller \textit{ubegrænsede}.
%
\begin{defn}{}{}
En mængde $\S \subset \R^n$ er \textbf{begrænset}, såfremt der eksisterer en konstant $c$, således den absolutte værdi af alle komponenter i alle elementer i $\S$ er mindre eller lig $c$. 
Såfremt en sådan konstant ikke eksisterer, er mængden \textbf{ubegrænset}.
\end{defn}
\noindent
% 
%
%alternativt værdien af ..... $\leq k$ eller $\geq$
%
I henhold til lineære programmeringsproblemer er løsningsmængden ofte begrænset.
Endvidere er det en fordel at definere polyedre, der er begrænset af én lineær betingelse. 
%
\begin{defn}{}{}
Lad $\mathbf{a}$ være en vektor i $\R^n$, hvor $\mathbf{a} \neq \mathbf{0}$ og lad $b$ være en skalar.
Mængden $$\{\textbf{x} \in \R^n \mid \mathbf{a}^T \mathbf{x}=b\}$$ kaldes et \textbf{hyperplan}.
%
Mængden $$\{\textbf{x} \in \R^n \mid \mathbf{a}^T \mathbf{x} \geq b\}$$ kaldes det \textbf{øvre halvrum}, og
mængden $$\{\textbf{x} \in \R^n \mid \mathbf{a}^T \mathbf{x} \leq b\}$$ kaldes det \textbf{nedre halvrum}.
\end{defn}
\noindent
%
Det gælder her, at hyperplanet er grænsen mellem et øvre og et nedre halvrum.
I $\R^2$ er hyperplanet således en ret linje, som deler rummet. 
Der er dermed et halvrum på hver side af hyperplanet, hvilket er illustreret på figur \ref{fig:Graf123}, hvor det øvre halvrum er markeret med blå, og det nedre halvrum er markeret med rød. 
%
%%%%%%%%%%%%%%%%%%%%%%%%%%%%%%%%
%%% Flot graf alla Julie     %%%
%%%%%%%%%%%%%%%%%%%%%%%%%%%%%%%%
%
\input{fig/tikz/geometri/hyperplan_r2}
%
Ligeledes ses et hyperplan i $\R^3$ på figur \ref{fig:nej6}, der afskærer det øvre og nedre halvrum.
%
%%%%%%%%%%%%%%%%%%%%%%%%%%%%%%%%
%%% Flot graf halla Julie     %%%
%%%%%%%%%%%%%%%%%%%%%%%%%%%%%%%%
%
\input{fig/tikz/geometri/hyperplan_r3}
%
Betragt polyederet $\mathcal{P}$ på figur \ref{fig:nej2}. 
Randen består af fire hyperplaner.
Fællesmængden for de øvre halvrum udgør polyederet.
På figur \ref{fig:nej5} er de fire hyperplaner markeret. 
%
%%%%%%%%%%%%%%%%%%%%%%%%%%%%%%%%
%%% Flot graf balla Julie     %%%
%%%%%%%%%%%%%%%%%%%%%%%%%%%%%%%%
%
\begin{center}
\begin{tikzpicture}[scale=6]%[scale=5,tdplot_main_coords]
%set up some coordinates 
%-----------------------
\coordinate (a) at (1.2,1.2,0.7);
\coordinate (b) at (0.7,1.2,0.7);
\coordinate (c) at (0.7,1.2,1.2);
\coordinate (d) at (1.2,1.2,1.2);
\coordinate (f) at (1.2,0.7,0.7);
\coordinate (g) at (1.2,0.7,1.2);
\coordinate (h) at (0.7,0.7,1.2);
%
%draw figure contents
%--------------------
\filldraw[fill=myblue,opacity=0.3, thick](c)--(d)--(g)--(h)--(c);
  \draw[thick](d)--(g);
  \draw[thick,->](1.2,0.95,1.2)--(1.1,0.95,1.2) node[anchor=south east]{$a_4$};
  \draw[thick](d)--(c);
  \draw[thick,->](0.95,1.2,1.2)--(0.95,1.1,1.2) node[anchor=north east]{$a_3$};
  \draw[thick](g)--(h);
  \draw[thick,->](0.95,0.7,1.2)--(0.95,0.8,1.2) node[anchor=south west]{$a_2$};
  \draw[thick](h)--(c);
  \draw[thick,->](0.7,0.95,1.2)--(0.8,0.95,1.2) node[anchor=north west]{$a_1$};
%
\draw[black] (0.91,0.96,1.2) circle (0pt) node[anchor=west] {$\mathcal{P}$};
%draw the main coordinate system axes
\draw[thick,->] (0,0,0) -- (1,0,0) node[anchor=south east]{$x$};
\draw[thick] (0,0,0) -- (-0.2,0,0);
\draw[thick,->] (0,0,0) -- (0,1,0) node[anchor=north west]{$y$};
\draw[thick] (0,0,0) -- (0,-0.2,0);
\end{tikzpicture}
  \captionof{figure}{En polyede $\mathcal{P}$ markeret med blå i $\R^2$.}
  \label{fig:nej5}
\end{center}
%
Hyperplanerne på figur \ref{fig:nej3} danner dermed randen, og fællesmængden for det øvre halvrum udgør polyederet.
\\\\
%
%
%Et polyhedron er derfor en samling af polyeder der til sammen skaber en geometrisk figur i et givet vektorrum.
%
Vektoren $\textbf{a}$ er ortogonal med hele hyperplanet $\{\textbf{x} \in \R^n \mid \mathbf{a}^T \mathbf{x}=b\}$, hvilket vises i \ref{thm:juliexd}. 
%
\begin{thm}{}{juliexd}
Lad $\mathbf{a}$ være en vektor i $\R^n$, hvor 
$\mathbf{a} \neq \mathbf{0}.$
For hyperplanet 
$\{\textbf{x} \in \R^n \mid \mathbf{a}^T \mathbf{x}=b\},$ 
vil $\mathbf{a}$ være ortogonal med hyperplanet.
\end{thm}
%
\begin{proof}
Lad $\mathbf{u}$ og $\mathbf{v}$ tilhøre det samme hyperplan. 
Så er $\mathbf{a}^T\textbf{u}=\mathbf{a}^T\textbf{v}.$
Dermed er $\mathbf{a}^T(\textbf{u}-\textbf{v})=0$, derfor er $\mathbf{a}$ ortogonal med alle vektorer i hyperplanet. 
\end{proof}