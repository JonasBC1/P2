I henhold til geometriske fremstillinger af lineære programmeringsproblemer, er det nødvendigt at definere en række begreber.
Disse er \textit{hyperplan}, \textit{halvrum} og \textit{polyeder}.
%
\begin{defn}{}{Polyede}
Lad $A$ være en $m \times n$ matrix, og lad $\mathbf{b}$ være en vektor i  $\R^m$.
En \textbf{polyede} er en mængde, der kan beskrives som 
$\{x\in \R^n \mid A\mathbf{x}\geq b\}.$
%
\end{defn}
\noindent
%
Som det fremgår fra definitionen, er en polyede således mængden af mulige løsninger $\mathbf{x}$ af et ligningssystem, altså værdimængden.
Det gælder endvidere, at en mængde af formen $ \{x \in \R^n \mid A\textbf{x}=b,x \geq 0 \}$, kaldet en \textit{polyede på standardform}, hvilket vil blive beskrevet dybere i afsnit \ref{afsnit:fisk}.
% Kommentar til at der her er derfor det er rellevant og at det vil blive uddybet senere 
% Eventuelt med et eksempel 
%lidt forskellige muligheder her, enten kan det relateres til simplex eller blive introduceret her
%
Med udgangspunkt i eksempel \ref{eks:min_lin} er polyeden markeret med det blå område på figur \ref{fig:min_beg}. 
Ligeledes ses på figur \ref{fig:nej2} en polyede $\mathcal{P}$, der er markeret med blå i $\R^2$. 
På figur \ref{fig:nej3} ses en polyede $\mathcal{P}$ markeret med blå i $\R^3$. 
%
%%%%%%%%%%%%%%%%%%%%%%%%%%%%%%%%
%%% Flot graf alla Julie     %%%
%%%%%%%%%%%%%%%%%%%%%%%%%%%%%%%%
%
\begin{center}
$
\begin{array}{cc}
\begin{minipage}[b]{0.45\textwidth}
%%%%%%%%%%%%%%%%%%%%%%%%%%%%%%%%
%%% Flot graf alla Julie     %%%
%%%%%%%%%%%%%%%%%%%%%%%%%%%%%%%%
%
\begin{center}
%start tikz picture, and use the tdplot_main_coords style to implement the display 
%coordinate transformation provided by 3dplot
\begin{tikzpicture}[scale=5]%[scale=5,tdplot_main_coords]
%set up some coordinates 
%-----------------------
\coordinate (a) at (1.2,1.2,0.7);
\coordinate (b) at (0.7,1.2,0.7);
\coordinate (c) at (0.7,1.2,1.2);
\coordinate (d) at (1.2,1.2,1.2);
\coordinate (f) at (1.2,0.7,0.7);
\coordinate (g) at (1.2,0.7,1.2);
\coordinate (h) at (0.7,0.7,1.2);
%
%draw figure contents
%--------------------
\filldraw[fill=myblue,opacity=0.3, thick](c)--(d)--(g)--(h)--(c);
  \draw[thick](d)--(g);
  \draw[thick](d)--(c);
  \draw[thick](g)--(h);
  \draw[thick](h)--(c);
%
\draw[black] (0.91,0.96,1.2) circle (0pt) node[anchor=west] {$\mathcal{P}$};
\draw[black] (0,0,0.7) circle (0pt);
%draw the main coordinate system axes
\draw[thick,->] (0,0,0) -- (1,0,0) node[anchor=south east]{$x$};
\draw[thick] (0,0,0) -- (-0.2,0,0);
\draw[thick,->] (0,0,0) -- (0,1,0) node[anchor=north west]{$y$};
\draw[thick] (0,0,0) -- (0,-0.2,0);
\end{tikzpicture}
  \captionof{figure}{En polyede $\mathcal{P}$ markeret med blå i $\R^2$.}
  \label{fig:nej2}
\end{center}
\end{minipage}&
\begin{minipage}[b]{0.45\textwidth}
\begin{center}
%
\begin{tikzpicture}[scale=5]
% Koordinater 
% -------------------------------------------------------
\coordinate (a) at (1.2,1.2,0.7);
\coordinate (b) at (0.7,1.2,0.7);
\coordinate (c) at (0.7,1.2,1.2);
\coordinate (d) at (1.2,1.2,1.2);
\coordinate (f) at (1.2,0.7,0.7);
\coordinate (g) at (1.2,0.7,1.2);
\coordinate (h) at (0.7,0.7,1.2);
%
% Figur
% -------------------------------------------------------
\filldraw[fill=myblue,opacity=0.3, thick](a)--(b)--(c)--(d)--(a)--(f)--(g)--(h)--(c)--(d)--(a);
  \draw[thick](d)--(g);
  \draw[thick](g)--(f);
  \draw[thick](f)--(a);
  \draw[thick](a)--(d);
  \draw[thick](d)--(c);
  \draw[thick](g)--(h);
  \draw[thick](c)--(b);
  \draw[thick](h)--(c);
  \draw[thick](b)--(a);
  \draw[gray, thick, dashed](f)--(0.7,0.7,0.7)--(0.7,1.2,0.7);
  \draw[gray, thick, dashed](0.7,0.7,0.7)--(0.7,0.7,1.2);
%
%
% Navngivning og prik til at gøre det pænt 
% -------------------------------------------------------
\draw[black] (0.91,0.96,1.2) circle (0pt) node[anchor=west] {$\mathcal{P}$};
% 
% 
% Koordinatssystem 
% -------------------------------------------------------
\draw[thick,->] (0,0,0) -- (1,0,0) node[anchor=south east]{$x$};
\draw[thick] (0,0,0) -- (-0.2,0,0);
\draw[thick,->] (0,0,0) -- (0,1,0) node[anchor=north west]{$y$};
\draw[thick,dashed] (0,0,0) -- (0,-0.2,0);
\draw[thick,->] (0,0,0) -- (0,0,0.7) node[anchor=south east]{$z$};
\draw[thick] (0,0,0) -- (0,0,-0.2);
%
%
\end{tikzpicture}
  \captionof{figure}{En polyede $\mathcal{P}$ markeret med blå i $\R^3$.}
  \label{fig:nej3}
\end{center}
%
\end{minipage}
\end{array}
$
\end{center}
%
\textit{Polygonet} er begrænsningerne, som afgrænser polyeden fra resten af rummet. 
Dermed er polygonet på figur \ref{fig:nej2} linjerne, der afgrænser $\mathcal{P}$ fra resten af $\R^2$.
Polygonet på figur \ref{fig:nej4} er derimod udsnittet af fladerne, der udgør kuben, som afgrænser $\mathcal{P}$ fra resten af $\R^3$.
%
Det gælder endvidere for polyeder at disse både kan være \textit{begrænsede} og \textit{ubegrænsede}.
%
\begin{defn}{}{}
En mængde $S \subset \R^n$ er \textbf{begrænset} såfremt der eksister en konstant $c$, hvorom det gælder, at den absolutte værdi af alle komponenter i alle elementer i $S$ er $\leq c$. 
Såfremt en sådan konstant ikke eksisterer er mængden \textbf{ubegrænset}. 
\end{defn}
\noindent
%
%alternativt værdien af ..... $\leq k$ eller $\geq$
%
I henhold til lineære programmeringsproblemer vil dette ofte være begrænset.
% Dette er eksempelvis tilfældet, hvis problemet er af en sådan karakter, at ingen af variabelene i karakterligningen kan have negative værdier.
%er det rigtigt det jeg skriver i ovenstående?
Ligeledes er det en fordel at definere polyeder, der er begrænset af kun én lineær betingelse. 
%
%skal der laves en definition på et polyede?
%
\begin{defn}{}{}
Lad $\mathbf{a}$ være en vektor i $\R^n$, hvor $\mathbf{a} \neq \mathbf{0}$ og lad $b$ være en skalar.
Mængden $\{x \in \R^n \mid \mathbf{a}^T \mathbf{x}=b\}$ kaldes et \textbf{hyperplan}.
%
Mængden $\{x \in \R^n \mid \mathbf{a}^T \mathbf{x} \geq b\}$ kaldes det \textbf{øvre halvrum}, og
mængden $\{x \in \R^n \mid \mathbf{a}^T \mathbf{x} \leq b\}$ kaldes det \textbf{nedre halvrum}.
\end{defn}
\noindent
%
Det gælder her, at hyperplanet er grænsen mellem de tilsvarende øvre og nedre halvrum.
I $\R^2$ vil hyperplanet således være en ret linje, som afskære en del af rummet, og der vil dermed være et halvrum på hver side af hyperplanet.
På figur \ref{fig:Graf123} ses et udsnit af hyperplanet, som afskærer det øvre halvrum markeret med blå og det nedre halvrum markeret med rød. 

%%%%%%%%%%%%%%%%%%%%%%%%%%%%%%%%
%%% Flot graf alla Julie     %%%
%%%%%%%%%%%%%%%%%%%%%%%%%%%%%%%%
\begin{center}
\begin{tikzpicture}
% Den øverste linje
\draw[name path=b,-, white, thick] (0,3.5) -- (6,3.5);
%
% Den nederste linje
\draw[name path=c,-, white, thick] (0,-1) -- (6,-1);
%
% Den mellemste linje
\draw[name path=a,-, white, thick] (0,0) -- (6,3);
%
% Farvning 
\tikzfillbetween [of=a and b]{myblue!5}
\tikzfillbetween [of=a and c]{myred!5}
%
% Linjen mellem 
\draw[-, black, very thick] (0,0) -- (6,3);
\draw[->, black, thick] (2.4,1.2) -- (2,2);
\filldraw[black] (0.2,-0.3) circle (0pt) node[anchor=west] {$\mathbf{a}^T \mathbf{x}=b$};
\filldraw[black] (3,2.6) circle (0pt) node[anchor=west] {$\mathbf{a}^T \mathbf{x} > b$};
\filldraw[black] (3,1.3) circle (0pt) node[anchor=west] {$\mathbf{a}^T \mathbf{x} < b$};
\filldraw[black] (1.6,2.2) circle (0pt) node[anchor=west] {$\mathbf{a}$};
\end{tikzpicture}
  \captionof{figure}{Et hyperplan og to halvrum, markeret med henholdsvis blå for den øvre halvrum $\mathbf{a}^T \mathbf{x} < b$ og rød for den nedre halvrum $\mathbf{a}^T \mathbf{x} > b$.}
  \label{fig:Graf123}
\end{center}
%
Med udgangspunkt i polyeden $\mathcal{P}$ på figur \ref{fig:nej2} består polyeden af fire hyperplaner. 
Fællesmængden for de øvre halvrum udgør polygonet.
På figur \ref{fig:nej5} er de fire hyperplaner markeret. 
%
%%%%%%%%%%%%%%%%%%%%%%%%%%%%%%%%
%%% Flot graf alla Julie     %%%
%%%%%%%%%%%%%%%%%%%%%%%%%%%%%%%%
%
\begin{center}
\begin{tikzpicture}[scale=8]
% Koordinater
% -------------------------------------------------------
\coordinate (c) at (0.7,1.2,1.2);
\coordinate (d) at (1.2,1.2,1.2);
\coordinate (g) at (1.2,0.7,1.2);
\coordinate (h) at (0.7,0.7,1.2);
%
% Tegning af figur og kommentarer
% -------------------------------------------------------
%
% Farve 
% -------------------------------------------------------
\filldraw[fill=myblue,opacity=0.3, thick](c)--(d)--(g)--(h)--(c);
%
% Linjer og navne
% -------------------------------------------------------
% a_1
\draw[gray,thin,dashed](0.7,0.5,1.2)--(0.7,1.41,1.2);
\draw[thick](h)--node[left]{$\mathbf{a}_1^T \mathbf{x}=b_1$} (c);
\draw[thick,->](0.7,0.95,1.2)--(0.8,0.95,1.2) node[right]{$a_1$};
%
% a_2
\draw[gray,thin,dashed](1.4,1.2,1.2)--(0.5,1.2,1.2);
\draw[thick](d)--node[above]{$\mathbf{a}_2^T \mathbf{x}=b_2$} (c);
\draw[thick,->](0.95,1.2,1.2)--(0.95,1.1,1.2) node[below]{$a_2$};
%
% a_3
\draw[gray,thin,dashed](1.2,1.4,1.2)--(1.2,0.5,1.2);
\draw[thick](d)--node[right] {$\mathbf{a}_3^T \mathbf{x}=b_3$} (g);
\draw[thick,->](1.2,0.95,1.2)--(1.1,0.95,1.2) node[left]{$a_3$};
%
% a_4  
\draw[gray,thin,dashed](1.4,0.7,1.2)--(0.5,0.7,1.2);
\draw[thick](g)--node[below]{$\mathbf{a}_4^T \mathbf{x}=b_4$} (h);
\draw[thick,->](0.95,0.7,1.2)--(0.95,0.8,1.2) node[above]{$a_4$};
%
% Navn til polyeden
% -------------------------------------------------------
\draw[black] (0.91,0.96,1.2) circle (0pt) node[anchor=west] {$\mathcal{P}$};
%
\end{tikzpicture}
  \captionof{figure}{Fællesmængden af de fire hyperplaners øvre halvrum udgør polyederet $\mathcal{P} \in \R^2$ markeret med blå.}
  \label{fig:nej5}
\end{center}
%
På figur \ref{fig:min_beg} udgør begrænsningerne hver sin hypeplan, hvor fællesmængden for de øvre halvrum udgør polyeden.

%
%\begin{center}
%\begin{tikzpicture}
%\begin{axis}
%\addplot3[
%    surf,
%]
%{0.5*x};
%\end{axis}
%\end{tikzpicture}
%  \captionof{figure}{FUCK DET HER LORT XD}
%  \label{fig:NEJ}
%\end{center}
% 
%%%%%%%%%%%%%%%%%%%%%%%%%%%%%%%%
%%% Flot graf alla Julie     %%%
%%%%%%%%%%%%%%%%%%%%%%%%%%%%%%%%
%
\begin{center}
%start tikz picture, and use the tdplot_main_coords style to implement the display 
%coordinate transformation provided by 3dplot
\begin{tikzpicture}[scale=5]%[scale=5,tdplot_main_coords]
%set up some coordinates 
%-----------------------
\coordinate (O) at (0,0,0);
\coordinate (XZ) at (1,0,1);
\coordinate (X) at (1,0,-0.3);
\coordinate (Z) at (-0.3,0,1);
\coordinate (Y) at (-0.3,0,-0.3);
%
\coordinate (XZa) at (1,0.5,1);
\coordinate (Xa) at (1,0.5,-0.3);
\coordinate (Za) at (-0.3,0.5,1);
\coordinate (Ya) at (-0.3,0.5,-0.3);
%
\coordinate (Na) at (0.8,0.5,0.5);
\coordinate (Pa) at (0.8,0.7,0.5);
%
\coordinate (N) at (0.5,0,0.5);
\coordinate (P) at (0.5,0.5,0.5);
%
\draw[black](0.3,0.5,0.5) circle (0pt) node[anchor=west] {$\mathbf{a}^T \mathbf{x}=b$};
\draw[black] (0.3,1,0.5) circle (0pt) node[anchor=west] {$\mathbf{a}^T \mathbf{x} > b$};
\draw[black] (0.3,0.1,0.5) circle (0pt) node[anchor=west] {$\mathbf{a}^T \mathbf{x} < b$};
%
%draw figure contents
%--------------------
\draw [thick,->] (Na) -- (Pa) node[anchor=south west]{$a$};
\draw [thick, dashed] (Na) -- (0.8,0.25,0.5);
%
\filldraw [fill=myblue,opacity=0.3] 
         (Ya) -- (Xa) -- (XZa) -- (Za) -- cycle;
%
%draw the main coordinate system axes
\draw[thick,->] (0,0,0) -- (1,0,0) node[anchor=south east]{$x$};
\draw[thick] (0,0,0) -- (-0.3,0,0);
\draw[thick,->] (0,0,0) -- (0,1,0) node[anchor=north west]{$y$};
\draw[thick,dashed] (0,0,0) -- (0,-0.3,0);
\draw[thick,->] (0,0,0) -- (0,0,1) node[anchor=south east]{$z$};
\draw[thick] (0,0,0) -- (0,0,-0.3);
\end{tikzpicture}
  \captionof{figure}{Et hyperplan i $\R^3$ og to halvrum.}
  \label{fig:mm}
\end{center}
%
%Et polyhedron er derfor en samling af polyeder der til sammen skaber en geometrisk figur i et givet vektorrum.
%
\begin{thm}{}{}
Lad $\mathbf{a}$ være en vektor i $\R^n$, hvor 
$\mathbf{a} \neq \mathbf{0}.$
For hyperplanet 
$\{x \in \R^n \mid \mathbf{a}^T \mathbf{x}=b\},$ 
vil $\mathbf{a}$ være ortogonal med hyperplanet.
\end{thm}
% 
\begin{proof}
Lad $\mathbf{x}$ og $\mathbf{y}$ tilhører det samme hyperplan, så er $\mathbf{a}^Tx=\mathbf{a}^Ty.$
Dermed er $\mathbf{a}^T(x-y)=0$, og så er $\mathbf{a}$ ortogonal til alle vektorer i hyperplanet. 
\end{proof}