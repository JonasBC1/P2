\section{Ekstremer, hjørnepunkter og basale løsninger}
%
%\textit{Vis at løsningerne findes i hjørnene af en polytede.}
Som det fremgår af afsnit \ref{heeeeejjulle}, vil en optimal løsning være tilbøjelige til at ligge i hjørnerne af en polyede.
Der er derfor behov for måder at definere disse hjørner på.
%
\begin{defn}{}{ekstrema}
Lad $\mathcal{P}$ være en polyede. 
En vektor $\mathbf{u} \in \mathcal{P}$ kaldes et \textbf{ekstremumspunkt} i $\mathcal{P}$, hvis der ikke eksisterer to vektorer $\mathbf{v},\mathbf{w} \in \mathcal{P}$, $\mathbf{v} \land \mathbf{w} \neq \mathbf{u}$, 
samt en skalar $\lambda \in [0,1]$, hvorom det gælder, at $\mathbf{u}=\lambda\mathbf{v}+(1-\lambda)\textbf{w}$.
\end{defn}
\noindent
%
%
Et hjørne kan beskrives ud fra denne definition, idet der, såfremt $\mathbf{u}=\lambda\mathbf{v}+(1-\lambda) \mathbf{w}$, og alle vektorene findes i $P$, gælder, at $\mathbf{u}$ er en konveks kombination af $\mathbf{v}$ og $\mathbf{w}$.
Hvis $\mathbf{u}=\lambda\mathbf{v}+(1-\lambda) \textbf{w}$ og $\mathbf{u}$ er et ekstremumspunkt, må det derfor gælde, at $\mathbf{v}\notin P$ eller $\mathbf{w}\notin P$ eller $\mathbf{u}=\mathbf{w}$ eller $\mathbf{u}=\mathbf{v}$.
Det skal her nævnes, at denne definition er strengt geometrisk. 
% Kun gælder som geometrisk definition gælder ikke som algebrarisk definition xD
På figur \ref{fig:ekstrema} ses en polyede $\mathcal{P}$, hvor $\textbf{x}$ er et ekstremumspunkt, da der ikke findes vektorer $\textbf{y}$ og $\textbf{z}$ sådan at $\mathbf{x}=\lambda\mathbf{y}+(1-\lambda) \mathbf{z}$. Vektor $\textbf{w}$ er i modsætning ikke et ekstremumspunkt, da der findes $\textbf{v}$ og $\textbf{u}$, sådan at $\mathbf{w}=\lambda\mathbf{v}+(1-\lambda)u$.
%
\begin{figure}[h!]
  \centering
  \begin{tikzpicture}[scale=0.7]
    \tikzset{punkt/.style={point, draw=black}}
%
%
% Koordinater
% -------------------------------------------------------
\coordinate (y) at (1,3);
\coordinate (x) at (3,3);
\coordinate (z) at (5,3);
\coordinate (v) at (-3,2);
\coordinate (w) at (-3,0);
\coordinate (u) at (-3,-2);
%    
% Punkter
% -------------------------------------------------------
	\node at (3,3)	(1){};
	\node at (-3,3)	(4){};
	\node at (3,-3)	(2){};
	\node at (-3,-3) (3){};
% 
% Firkant 
% -------------------------------------------------------
	\filldraw[mygrey, fill=myblue!15] (2) rectangle (4);
	\node at (0,0) (P){$\mathcal{P}$};
%	
% Punkter og tilhørende tekst
% -------------------------------------------------------
	\filldraw [black] (y) circle (2pt) node[above] {$\mathbf{y}$};
	\filldraw [black] (x) circle (2pt) node[above] {$\mathbf{x}$};
	\filldraw [black] (z) circle (2pt) node[above] {$\mathbf{z}$};
	\filldraw [black] (v) circle (2pt) node[left] {$\mathbf{v}$};
	\filldraw [black] (w) circle (2pt) node[left] {$\mathbf{w}$};
	\filldraw [black] (u) circle (2pt) node[left] {$\mathbf{u}$};
	\filldraw [black] (-5,3) circle (0pt) node[left] {};
%
% Streger mellem punkterne 
% -------------------------------------------------------
	\draw[-,black, thick] (v) -- (w) -- (u);
	\draw[-,black, thick] (y) -- (x);
	\draw[-, dashed,black,ultra thick] (x) -- (z);
%
%
  \end{tikzpicture}
  \caption{Et polyeder $\mathcal{P}$, hvor $\textbf{x}$ er et ekstremumspunkt, da der ikke findes vektorer $\textbf{y}$ og $\textbf{z}$, sådan at $\mathbf{x}=\lambda\mathbf{y}+(1-\lambda) \mathbf{z}$.
Vektor $\textbf{w}$ er i modsætning ikke et ekstremumspunkt, da der findes $\textbf{v}$ og $\textbf{u}$, sådan at $\mathbf{w}=\lambda\mathbf{v}+(1-\lambda) \mathbf{u}$.}
  \label{fig:ekstrema}
\end{figure}
%
\\\\
%
En alternativ geometrisk definition relaterer sig til \textit{hjørne punkter}, som er den entydige optimale løsning til et givet lineært programmeringsproblem med den mulige løsningsmængde $\mathcal{P}$.
%
\begin{defn}{}{hjoerner}
Lad $\mathcal{P}$ være en polyede. 
En vektor $\mathbf{u}\in \mathcal{P}$ siges at være et \textbf{hjørne punkt}, hvis der eksisterer en vektor $\mathbf{c}$, hvorom det gælder, at $\mathbf{c}^T\mathbf{u}<\mathbf{c}^T\mathbf{v}$ for alle $\mathbf{v}$, som opfylder $\mathbf{v} \in \mathcal{P}$ samt $\mathbf{v}\neq\mathbf{u}$.
\end{defn}
\noindent
%
På figur \ref{fig:julieermegaseeeeeeeeeej} ses et eksempel på en polyede $\mathcal{P}$, hvor $\textbf{x}$ er et hjørnepunkt, da hyperplanet kun rammer $\mathcal{P}$ i $\mathbf{x}$. 
Vektoren $\textbf{w}$ er i modsætning ikke hjørnepunkt, da hyperplanet rammer $\mathcal{P}$ i flere punkter end $\mathbf{w}$.
%
\begin{figure}[h!]
  \centering
  \begin{tikzpicture}[scale=0.7]
    \tikzset{punkt/.style={point, draw=black}}
%
%
% Koordinater
% -------------------------------------------------------
\coordinate (y) at (1,4);
\coordinate (x) at (3,3);
\coordinate (z) at (5,2);
\coordinate (v) at (-3,4);
\coordinate (w) at (-3,2);
\coordinate (u) at (-3,0);
%    
% Punkter
% -------------------------------------------------------
	\node at (3,3)	(1){};
	\node at (-3,3)	(4){};
	\node at (3,-1)	(2){};
	\node at (-3,-1) (3){};
% 
% Firkant 
% -------------------------------------------------------
	\filldraw[mygrey, fill=myblue!30] (2) rectangle (4);
	\node at (0,1) (P){$\mathcal{P}$};
%	
% Punkter og tilhørende tekst
% -------------------------------------------------------
	\filldraw [black] (5,2) circle (0pt) node[right] {$ \{ \mathbf{y} \text{  } | \text{  } \mathbf{c}^T \mathbf{y} = \mathbf{c}^T \mathbf{x} \}$};
	\filldraw [black] (x) circle (2pt) node[above] {$\mathbf{x}$};
	\filldraw [black] (-3,3.8) circle (0pt) node[left] {$ \{ \mathbf{y} \text{  } | \text{  } \mathbf{c}^T \mathbf{y} = \mathbf{c}^T \mathbf{w} \}$};
	\filldraw [black] (w) circle (2pt) node[left] {$\mathbf{w}$};
%
% Streger mellem punkterne 
% -------------------------------------------------------
	\draw[-,black, thick] (v) -- (w) -- (u);
	\draw[-,black, thick] (y) -- (x);
	\draw[-,black, thick] (x) -- (z);
	\draw[->,black, thick] (-3,1.5) -- (-4.3,1.5) node[left]  {$\mathbf{a}_w$};
	\draw[->,black, thick] (4,2.5) -- (4.5,3.5) node[above] {$\mathbf{a}_x$};
%
\filldraw [black] (-9,2) circle (0pt);
%
  \end{tikzpicture}
  \caption{En polyede $\mathcal{P}$, hvor $\textbf{x}$ er et hjørnepunkt, da hyperplanet kun rammer $\mathcal{P}$ i $\mathbf{x}$. Vektoren $\textbf{w}$ er i modsætning ikke hjørnepunkt, da hyperplanet rammer $\mathcal{P}$ i flere punkter end $\mathbf{w}$.}
  \label{fig:julieermegaseeeeeeeeeej}
\end{figure}
%
\\\\
%
Dette kan ligeledes beskrives som, at $\mathbf{x}$ er et hjørne i $\mathcal{P}$, såfremt $\mathcal{P}$ er på den ene side af et hyperplan, der skærer $\mathcal{P}$ i $\mathbf{x}$. 
Jævnfør \ref{defn:ekstrema} har dette hyperplan ligningen $y \mid \mathbf{c}^T\mathbf{x}=\mathbf{c}^T\mathbf{y}.$