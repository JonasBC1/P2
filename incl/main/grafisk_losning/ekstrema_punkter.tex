\section{Ekstremer, hjørnepunkter og basale løsninger}
%
%\textit{Vis at løsningerne findes i hjørnene af en polytede.}
Som det fremgår af afsnit \ref{heeeeejjulle}, vil en optimal løsning være tilbøjelige til at ligge i hjørnerne af en polyede.
Der er derfor behov for måder at definere disse hjørner på.
%
\begin{defn}{}{ekstrema}
Lad $\mathcal{P}$ være en polyede. 
En vektor $\mathbf{x} \in \mathcal{P}$ kaldes et \textbf{ekstremumspunkt} i $\mathcal{P}$, hvis der ikke eksisterer to vektorer $\mathbf{y},\mathbf{z} \in \mathcal{P}$, $\mathbf{y} \land \mathbf{z} \neq \mathbf{x}$, 
samt en skalar $\lambda \in [0,1]$, hvorom det gælder, at $\mathbf{x}=\lambda\mathbf{y}+(1-\lambda)\textbf{z}$.
\end{defn}
\noindent
%
%
Et hjørne kan beskrives ud fra denne definition, idet der, såfremt $\mathbf{x}=\lambda\mathbf{y}+(1-\lambda)z$, og alle vektorene findes i $P$, gælder, at $\mathbf{x}$ er en konveks kombination af $\mathbf{y}$ og $\mathbf{z}$.
Hvis $\mathbf{x}=\lambda\mathbf{y}+(1-\lambda)\textbf{z}$ og $\mathbf{x}$ er et ekstremumspunkt, må det derfor gælde, at $\mathbf{y}\notin P$ eller $\mathbf{z}\notin P$ eller $\mathbf{x}=\mathbf{z}$ eller $\mathbf{x}=\mathbf{y}$. 
Det skal her nævnes, at denne definition er strengt geometrisk.
%

\begin{figure}[h!]
  \centering
  \begin{tikzpicture}
    \tikzset{punkt/.style={point, draw=black}}\draw[color=black](1,-2) circle (2.5);
	\draw[color=black](7,-2) circle (2.5);  
	\draw[color=black, fill=myblue!15](7,-2) circle (1.5);  
%Punkter
	\draw  node[fill,circle,inner sep=0pt,minimum size=3pt] at (1,-2.65)  (v1) {};
	\draw  node[fill,circle,inner sep=0pt,minimum size=3pt] at (1,-1.35)  (v2) {};
	\draw  node[fill,circle,inner sep=0pt,minimum size=3pt] at (7,-2)  (v3) {};

%Labels
	\draw  node at (0.75,-2.55)  {$\textbf{u}$};
	\draw  node at (0.75,-1.25)  {$\textbf{v}$};
	\draw  node at (7.25,-1.85)  {$\textbf{w}$};
	
	
	
	%Navn på cirklerne 
	\draw[black, fill=black] (1.25,1) circle (0pt) node[anchor=west] {$\S_1$};    
	\draw[black, fill=black] (6.75,1) circle (0pt) node[anchor=west] {$\S_2$};
	%Navn på punkter 

	\draw[black, fill=black] (4,1.5) circle (0pt) node[anchor=west] {$f$};
	
	\draw [->,thick, draw=black] (2,1) -- (6.5,1);

	 \draw [->, thick, draw=red] ([xshift=5pt, yshift=-1pt]v1.north) -- ([xshift=-5pt, yshift=1pt]v3.south);
     \draw [->, thick, draw=black] ([xshift=5pt, yshift=1pt]v2.south) -- ([xshift=-5pt, yshift=-1pt]v3.north);


	\draw[black, fill=black] (0,-5) circle (0pt) node[anchor=west] {Domæne};
	\draw[black, fill=black] (6,-5) circle (0pt) node[anchor=west] {Codomæne};
	\draw[black, fill=black] (5.5,-2.5) circle (0pt) node[anchor=west] {Værdimængde};    
	%Streger mellem punkterne
  \end{tikzpicture}
  \caption{En funktion fra $\S_1$ til $\S_2$, hvor $f(\textbf{v})=\textbf{w}$ og f$(\textbf{u})=\textbf{w}$. }
  \label{fig:afbild}
\end{figure}
%
%
%    \node[punkt] at (-4,0.5)      (v1){$v_1$};
%    \node[punkt] at (-2,0.5)      (v2){$v_2$};
%    \node[punkt] at (-4,-1.5)     (v3){$v_3$};
%    \node[punkt] at (-2,-1.5)     (v4){$v_4$};
%    \node at (-3,2)     (v){$K_{4}$};
%
%
%    \node[punkt] at (4.6,-0.2)      (k1){$v_3$};
%    \node[punkt] at (1.4,-0.2)      (k2){$v_2$};
%    \node[punkt] at (2,-2)     (k3){$v_4$};
%    \node[punkt] at (4,-2)     (k4){$v_5$};
%    \node[punkt] at (3,1)      (k5){$v_1$};
%    \node at (3,2)      (k){$K_{5}$};
%
%
%
%
%
%    \draw [-, thick, draw=black] (v1) -- (v2);
%    \draw [-, thick, draw=black] (v1) -- (v3);
%    \draw [-, thick, draw=black] (v1) -- (v4);
%    \draw [-, thick, draw=black] (v2) -- (v3);
%    \draw [-, thick, draw=black] (v2) -- (v4);
%    \draw [-, thick, draw=black] (v3) -- (v4);
\\\\
%
En alternativ geometrisk definition relaterer sig til \textit{hjørne punkter}, som er den entydige optimale løsning til et givet lineært programmeringsproblem med den mulige løsningsmængde $\mathcal{P}$.
%
\begin{defn}{}{hjoerner}
Lad $P$ være en polyede. 
En vektor $\mathbf{x}\in P$ siges at være et \textbf{hjørne punkt}, hvis der eksisterer en vektor $\mathbf{c}$, hvorom det gælder, at $\mathbf{c}^T\mathbf{x}<\mathbf{c}^T\mathbf{y}$ for alle $\mathbf{y}$, som opfylder $\mathbf{y} \in P$ samt $\mathbf{y}\neq\mathbf{x}$.
\end{defn}
\noindent
%
%
Dette kan ligeledes beskrives som, at $\mathbf{x}$ er et hjørne i $P$, såfremt $P$ er på den ene side af et hyperplan, der skærer $P$ i $\mathbf{x}$. 
Jævnfør definition \ref{defn:ekstrema} har dette hyperplan ligningen $y \mid \mathbf{c}^T\mathbf{x}=\mathbf{c}^T\mathbf{y}.$