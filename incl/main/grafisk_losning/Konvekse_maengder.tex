\section{Konvekse mængder}
%
I forbindelse med mængder af vektorer er det vigtigt at definere \textit{konvekse mængder}.
\begin{defn}{}{konveks}
<<<<<<< HEAD
En mængde $S \subset \R^n$ er konveks, hvis $ \lambda \textbf{x} + (1-\lambda )\textbf{y} \in S$ for ethvert $\textbf{x}, \textbf{y} \in S$ og ethvert $ \lambda \in [0,1]$. 
=======
En mængde $S \subset \R^n$ er \textbf{konveks}, hvis $\lambda \textbf{x} + (1- \lambda ) \textbf{y} \in S$ for ethvert $\textbf{x}, \textbf{y} \in S$ og ethvert $\lambda \in [0,1]$. 
>>>>>>> bed6d0f15fb865e1928f1eb8c2b30c730cdc1599
\end{defn}
\noindent
%
<<<<<<< HEAD
En mængde er med andre ord konveks, hvis ethvert element på en ret linje mellem to vilkårlige punkter $x$ og $y$ i mængden også tilhører mængden. 
På figur !REF HER! ses eksempler på konvekse og ikke-konvekse mængder. 
%
\begin{figure}[H]
  \centering
  \begin{tikzpicture}
    \tikzset{punkt/.style={point, draw=black}}
=======
En mængde er med andre ord konveks, hvis ethvert element på en ret linje mellem to vilkårlige punkter $x$ og $y$ i mængden også tilhører mængden. På figur \ref{asdf} ses eksempler på konvekse og ikke-konvekse mængder. 
%
\begin{figure}[h!]
  \centering
  \begin{tikzpicture}
    \tikzset{punkt/.style={point, draw=black}} 
%   
% Punkter
% -------------------------------------------------
	\node at (4,4)	(1){};
	\node at (-4,4)	(4){};
	\node at (4,-4)	(2){};
	\node at (-4,-4) (3){};
	\node at (-4,-4) (3){};
	\node at (-4,-4) (3){};
% 
% 
	\filldraw[black, fill=blue!5] (2) rectangle (4);
	\node at (0,0) (P){$P$};
	
	\filldraw [black] (2,4) circle (2pt);
	\filldraw [black] (4,4) circle (2pt);
	\filldraw [black] (6,4) circle (2pt);
	\filldraw [black] (-4,2) circle (2pt);
	\filldraw [black] (-4,0) circle (2pt);
	\filldraw [black] (-4,-2) circle (2pt);	
	\node at (2,4.25)	 (y){$\textbf{y}$};
	\node at (4,4.25)	 (x){$\textbf{x}$};
	\node at (6,4.25)	 (z){$\textbf{z}$};
	\node at (-3.75,2)  (v){$\textbf{v}$};
	\node at (-3.75,0)	 (w){$\textbf{w}$};
	\node at (-3.75,-2) (u){$\textbf{u}$};


	\draw[-, dashed,black,ultra thick] (6,4) -- (2,4);
	\draw[-, dashed,black,ultra thick] (-4,2) -- (-4,-2);
>>>>>>> bed6d0f15fb865e1928f1eb8c2b30c730cdc1599

  \end{tikzpicture}
  \caption{}
  \label{fig:konveks}
\end{figure}
%
\begin{defn}{}{konvekskombiskrog}
Lad $\textbf{x}^1, \textbf{x}^2, \ldots, \textbf{x}^k \in \R^n$, og $\lambda_1, \lambda_2, \ldots, \lambda_k$ være ikke-negative skalarer, hvis sum er $1$. 
\begin{enumerate}[label=(\alph*)]
	\item Vektoren $$\sum_{i=1}^{k} \lambda_i \textbf{x}^i$$ kaldes en \textbf{konveks kombination} af vektorerne $\textbf{x}^1, \textbf{x}^2, \ldots, \textbf{x}^k$. 
	\item Det \textbf{konvekse skrog} af vektorerne $\textbf{x}^1, \textbf{x}^2, \ldots, \textbf{x}^k$ er mængden af alle konvekse kombinationer af disse vektorer. 
%Det hedder vel ikke konvekse skrog, hvad kalder vi det? HORIA HJÆLP JENS
\end{enumerate}
\end{defn}
%
%Evt. noget metatekst
%
\begin{thm}{}{konveks}
\begin{enumerate}[label=(\alph*)]
	\item Skæringspunktet for konvekse mængder er konvekst. 
	%Det hedder ikke skæringspunktet, det er det mellemliggende, ved ikke hvad det hedder
	\item Enhver polyhedron er en konveks mængde.
	\item En konveks kombination af et endeligt antal elementer fra en konveks mængde tilhører også den mængde. 
	\item Det konvekse skrog af et endeligt antal vektorer er en konveks mængde. 
\end{enumerate}
\end{thm}
%
%
\begin{proof}
Jaja
\end{proof}