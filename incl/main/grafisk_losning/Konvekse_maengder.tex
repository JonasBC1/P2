\section{Konvekse mængder}
\label{julieerlakker}
%
I forbindelse med mængder af vektorer er det vigtigt at definere \textit{konvekse mængder}.
%
\begin{defn}{}{konveks}
En mængde $S \subset \R^n$ er \textbf{konveks}, hvis $\lambda \textbf{u} + (1- \lambda ) \textbf{v} \in S$ for ethvert $\textbf{u}, \textbf{v} \in S$ og ethvert $\lambda \in [0,1]$. 
\end{defn}
\noindent
%
En mængde er med andre ord konveks, hvis ethvert element på en ret linje mellem to vilkårlige elementer i mængden også tilhører mængden. 
På figur \ref{fig:neej1} ses et eksempel på en konveks mængde, markeret med blå, og ikke-konveks mængde, markeret med rød.
Den blå mængde er en konveks mængde grundet, at alle elementer mellem et vilkårligt $\textbf{u}$ og $\textbf{v}$ er indeholdt i mængden.
Tilsvarende er den røde mængde ikke en konveks mængde grundet, at ikke alle elementer mellem et vilkårligt $\textbf{u}$ og $\textbf{v}$ er indeholdt i mængden.\\
%
%%%%%%%%%%%%%%%%%%%%%%%%%%%%%%%%
%%% Flot graf alla Julie     %%%
%%%%%%%%%%%%%%%%%%%%%%%%%%%%%%%%
%
\begin{center}
% 
\begin{tikzpicture}[scale=5]
% Koordinater
% -------------------------------------------------------
% 
% Konvekse mængde
\coordinate (c) at (0.7,1.2,1.2);
\coordinate (d) at (1.2,1.2,1.2);
\coordinate (g) at (1.2,0.7,1.2);
\coordinate (h) at (0.7,0.7,1.2);
% 
% Ikke-konveks mængde 
\coordinate (a) at (1.5,1.2,1.2);
\coordinate (b) at (2,1.2,1.2);
\coordinate (e) at (2,0.7,1.2);
\coordinate (f) at (1.5,0.7,1.2);
\coordinate (m) at (1.75,0.95,1.2);
%
% Favning
% -------------------------------------------------------
%
% Konveks mængde farvning
\filldraw[fill=myblue,opacity=0.3, thick](c)--(d)--(g)--(h)--(c);
  \draw[thick](d)--(g);
  \draw[thick](d)--(c);
  \draw[thick](g)--(h);
  \draw[thick](h)--(c);
%
% Ikke-konveks mængde farvning
\filldraw[fill=myred,opacity=0.3, thick](a)--(b)--(e)--(m)--(f)--(a);
  \draw[thick](a)--(b);
  \draw[thick](b)--(e);
  \draw[thick](e)--(m);
  \draw[thick](m)--(f);
  \draw[thick](f)--(a);
%
% Punkter og linje 
% -------------------------------------------------------
% Blå figur
\draw[thick,-] (0.8,0.8,1.2) -- (1.1,0.8,1.2);
%
\filldraw[black] (0.8,0.8,1.2) circle (0.2pt) node[left] {$\mathbf{x}$};
\filldraw[black] (1.1,0.8,1.2) circle (0.2pt) node[right] {$\mathbf{y}$};
%
% Rød figur 
\draw[thick,-] (1.6,0.8,1.2) -- (1.9,0.8,1.2);
%
\filldraw[black] (1.6,0.8,1.2) circle (0.2pt) node[left] {$\mathbf{x}$};
\filldraw[black] (1.9,0.8,1.2) circle (0.2pt) node[right] {$\mathbf{y}$};
%
%
\end{tikzpicture}
  \captionof{figure}{En konveks mængde markeret med blå og en ikke-konveks mængde markeret med rød.}
  \label{fig:neej}
\end{center}
%
\begin{defn}{}{konvekskombihylster}
Lad $\textbf{v}_1, \textbf{v}_2, \ldots, \textbf{v}_k \in \R^n$, og $\lambda_1, \lambda_2, \ldots, \lambda_k$ være ikke-negative skalarer, hvis sum er $1$. Da gælder:
%
\begin{enumerate}[label=(\alph*)]
	\item Vektoren $$\sum_{i=1}^{k} \lambda_i \textbf{v}_i$$ kaldes en \textbf{konveks kombination} af vektorerne $\textbf{v}_1, \textbf{v}_2, \ldots, \textbf{v}_k$. 
	\item Det \textbf{konvekse hylster} af vektorerne $\textbf{v}_1, \textbf{v}_2, \ldots, \textbf{v}_k$ er mængden af alle konvekse kombinationer af disse vektorer. 
\end{enumerate}
%
% Det hedder vel ikke konvekse skrog, hvad kalder vi det?
% HORIA HJÆLP JENS
%
\end{defn}
%
På figur \ref{fig:neej2} ses to vektorer $\mathbf{u}$ og $\mathbf{v}$ i $\R^n$ og den konvekse kombination $\frac{1}{2} \mathbf{u}+\frac{1}{2} \mathbf{v}$, hvor $\lambda_\textbf{u} = \lambda_\textbf{v} = \frac{1}{2}$. 
Det konvekse hylster af vektorerne er en ret linje, der er markeret med blå. 
\\\\
%
%%%%%%%%%%%%%%%%%%%%%%%%%%%%%%%%
%%% Flot graf alla Julie     %%%
%%%%%%%%%%%%%%%%%%%%%%%%%%%%%%%%
%
\begin{center}
% 
\begin{tikzpicture}[scale=6.5]
% Koordinater
% -------------------------------------------------------
\coordinate (c) at (0.7,1.2,1.2);
\coordinate (d) at (1.2,1.2,1.2);
\coordinate (g) at (1.2,0.7,1.2);
\coordinate (h) at (0.7,0.7,1.2);
\coordinate (b) at (2,1.2,1.2);
\coordinate (f) at (1.5,0.7,1.2);
%
% Tegning af vektorer
% -------------------------------------------------------
  \draw[very thick,color=myblue!50](b)--(c);  
  \draw[thick,->](f)--(b);
  \draw[thick,->](f)--(c);
  \draw[thick,->, dashed](f)--(1.35,1.2,1.2);
%
% Punkter og linje 
% -------------------------------------------------------
\filldraw[black] (0.95,1,1.2) circle (0pt) node[left] {$\mathbf{u}$};
\filldraw[black] (1.7,1,1.2) circle (0pt) node[right] {$\mathbf{v}$};
\filldraw[black] (1.1,1,1.2) circle (0pt) node[right] {$\frac{1}{2} \mathbf{u}+\frac{1}{2} \mathbf{v}$};
%
%% Koordinatssystem
%% ------------------------------------------------------
%\draw[thick,->] (0,0,-0.5) -- (1.5,0,-0.5) node[anchor=south east]{$x$};
%\draw[thick] (0,0,-0.5) -- (-0.2,0,-0.5);
%\draw[thick,->] (0,0,-0.5) -- (0,0.8,-0.5) node[anchor=north west]{$y$};
%\draw[thick] (0,0,-0.5) -- (0,-0.2,-0.5);
%%
%
\end{tikzpicture}
  \captionof{figure}{Vektorerne $\mathbf{u}$ og $ \mathbf{v}$, samt den konvekse kombination af dem, hvor $\lambda_{\textbf{u}} = \lambda_{\textbf{v}} = \frac{1}{2}$, og det konvekse hylster af vektorerne, markeret med blå.}
  \label{fig:neej2}
\end{center}
%
%
På figur \ref{fig:neeej} ses derimod tre vektorer $\mathbf{u}$, $\mathbf{v}$ og $\mathbf{w}$ i $\R^3$. 
Det konvekse hylster af vektorerne er på figuren markeret med blå.
%
%
%%%%%%%%%%%%%%%%%%%%%%%%%%%%%%%%
%%% Flot graf alla Julie     %%%
%%%%%%%%%%%%%%%%%%%%%%%%%%%%%%%%
%
%
\begin{center}
\begin{tikzpicture}[scale=6]
% Koordinater
% -------------------------------------------------------
\coordinate (c) at (0.7,1.2,1.2);
\coordinate (d) at (1.2,1.2,1.2);
\coordinate (g) at (1.2,0.7,1.2);
\coordinate (h) at (0.7,0.7,1.2);
\coordinate (b) at (2,1.2,1.2);
\coordinate (f) at (1.5,0.7,1.2);
%
% Tegning af vektorer
% ------------------------------------------------------- 
  \draw[thick,->](f)--(b);
  \draw[thick,->](f)--(c);
  \draw[thick,->](f)--(1.7,1.2,1.6);
%
% Punkter og linje 
% -------------------------------------------------------
\filldraw[black] (0.95,1,1.2) circle (0pt) node[left] {$\mathbf{x}$};
\filldraw[black] (1.7,1,1.2) circle (0pt) node[right] {$\mathbf{y}$};
\filldraw[black] (1.4,1,1.2) circle (0pt) node[right] {$\mathbf{z}$};
%
% Planet - Hyperplanet
% -------------------------------------------------------
\filldraw [fill=myblue,opacity=0.3] 
         (b) -- (c) -- (1.7,1.2,1.6) -- cycle;
\filldraw [fill=myblue,opacity=0.3] 
         (b) -- (1.7,1.2,1.6) -- (f);
\filldraw [fill=myblue,opacity=0.3] 
         (c) -- (1.7,1.2,1.6) -- (f)--(c);
%
%
% Koordinatssystem
% ------------------------------------------------------
\draw[thick,->] (0,0,-0.5) -- (1.5,0,-0.5) node[anchor=south east]{$x$};
\draw[thick] (0,0,-0.5) -- (-0.2,0,-0.5);
\draw[thick,->] (0,0,-0.5) -- (0,0.8,-0.5) node[anchor=north west]{$y$};
\draw[thick] (0,0,-0.5) -- (0,-0.2,-0.5);
\draw[thick,->] (0,0,-0.5) -- (0,0,0.3) node[anchor=south east]{$z$};
\draw[thick] (0,0,-0.5) -- (0,0,-0.8);
%
%
\end{tikzpicture}
  \captionof{figure}{Vektorerne $\mathbf{x}$, $ \mathbf{y}$ og $ \mathbf{z}$ i $\R^3$, og det konvekse hylster af vektorerne, der er markeret med blå.}
  \label{fig:neeej}
\end{center}
%
%
%
Af \ref{defn:konvekskombihylster} følger \ref{thm:konveks}.
%
\begin{thm}{}{konveks}
\begin{enumerate}[label=(\alph*)]
	\item Fællesmængden for konvekse mængder er konveks. 
	\item Ethvert polyeder er en konveks mængde.
	\item En konveks kombination af et endeligt antal elementer fra en konveks mængde tilhører også den mængde. 
	\item Det konvekse hylster af et endeligt antal vektorer er en konveks mængde. 
\end{enumerate}
\end{thm}
%
%
\begin{proof}
\begin{enumerate}[label=(\alph*)]
\item Lad $I$ være en indeksmængde, lad $\S_i$, $i \in I$, være konvekse mængder, og lad $ \lambda \in [0,1]$.
Antag nu, at vektorerne $\textbf{u}$ og $\textbf{v}$ tilhører fællesmængden $ \bigcap_{i \in I} \S_i$. 
Det haves, at $ \lambda \textbf{u} + (1 - \lambda )\textbf{v} \in \S_i$, eftersom enhver $\S_i$ er konveks og  indeholder $\textbf{u}$ og $\textbf{v}$. Derfor er $ \bigcap_{i \in I} S_i$ konveks, da $ \lambda \textbf{u} + (1 - \lambda )\textbf{v} \in  \bigcap_{i \in I} \S_i$, hvilket beviser (a).
%
\item Lad $\textbf{a}$ være en vektor, lad $b$ være en skalar, og lad $ \lambda \in [0,1]$. 
Antag, at $\textbf{u}$ og $\textbf{v}$ henholdsvis opfylder, at $\textbf{a}^T \textbf{u} \geq b$ og $\textbf{a}^T \textbf{v} \geq b$, således de tilhører samme halvrum. 
Så er $\textbf{a}^T (\lambda \textbf{u} + (1 - \lambda) \textbf{v} ) \geq \lambda b + (1 - \lambda ) b = b$, hvilket beviser, at $ \lambda \textbf{u} + (1 - \lambda )\textbf{v}$ tilhører samme halvrum, hvormed halvrummet er konvekst.
Bemærk, at et polyeder er fællesmængden af et endeligt antal halvrum, og at (b) dermed er bevist jævnfør (a).
%
\item  Dette bevises ved induktion.
Jævnfør \ref{defn:konvekskombihylster} ligger en konveks kombination af to elementer i en konveks mængde i mængden.
Som induktionsantagelse antages det, at en konveks kombination af $k$ elementer i en konveks mængde også er i mængden.
Dernæst undersøges om en konveks kombination af $k+1$ elementer $\textbf{u}_1,\textbf{u}_2,\ldots,\textbf{u}_{k+1}$ i en konveks mængde $\S$ også er i mængden.
Lad $\lambda_1,\ldots,\lambda_{k+1}$ være ikke-negative skalarer, hvis sum er $1$. 
Det antages, at $\lambda_{k+1}\neq1$. 
Så haves, at
%
$$\sum^{k+1}_{i=1}\lambda_i\textbf{u}_i=\lambda_{k+1}\textbf{u}_{k+1}+(1-\lambda_{k+1})\sum^{k}_{i=1}\dfrac{\lambda_i}{1-\lambda_{k+1}}\textbf{u}_i.$$
%
Koefficienterne 
%
$$\frac{\lambda_i}{1-\lambda_{k+1}}, \phantom{i} \text{ for } i=1,\ldots,k,$$
%
er ikke-negative og summerer til $1$. 
Ved induktionsantagelsen gælder, at
%
$$\sum^{k}_{i=1}\frac{\lambda_i\textbf{u}_i}{1-\lambda_{k+1}}\in \S.$$
%
Da $\S$ er konveks, fås 
$\sum^{k+1}_{i=1}\lambda_i\textbf{u}_i\in \S$, hvormed (c) er bevist. 
%
\item Lad $\S$ være et konvekst hylster af $\textbf{u}_1, \textbf{u}_2, \ldots, \textbf{u}_k$, og lad
%
$$\textbf{v} = \sum_{i=1}^{k} \zeta_i \textbf{u}^i \text{ og } \textbf{w} = \sum_{i=1}^{k} \theta_i \textbf{u}^i$$
%
være to elementer i $\S$, hvor $ \zeta_i \geq 0$, $ \theta_i \geq 0$, og $ \sum_{i=1}^{k} \zeta_i = \sum_{i=1}^{k} \theta_i = 1$. 
Lad $ \lambda = [0,1]$. 
Så er $$\lambda \textbf{v} + (1 - \lambda ) \textbf{w} = \lambda \sum_{i=1}^k \zeta_i \textbf{u}^i + (1 - \lambda) \sum_{i=1}^k \theta_i \textbf{u}^i = \sum_{i=1}^k (\lambda \zeta_i + (1-\lambda )\theta_i ) \textbf{u}^i.$$
Bemærk, at koefficienterne $ \lambda \zeta_i + (1 - \lambda) \theta_i$, $i = 1, \ldots, k$, summerer til $1$ og ikke er negative. 
Dermed er $ \lambda \textbf{v} + (1 - \lambda ) \textbf{w}$ en konveks kombination af vektorerne $\textbf{u}_1, \ldots, \textbf{u}_k$, og tilhører derfor mængden $S$, hvormed (d) er bevist. 
\end{enumerate}
\end{proof}