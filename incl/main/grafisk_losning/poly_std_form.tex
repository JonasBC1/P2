\section{Polyeder på standardform}
\label{afsnit:fisk}
%
Som beskrevet i afsnit \ref{sec:standard}, kan optimeringsproblemer opskrives på standardform.
Dette korresponderer med polyeder, der ligeledes kan opskrives på standardform: 
$\mathcal{P}=\{ \mathbf{x} \in \R^n  \mid  A\mathbf{x}=\mathbf{b}, \mathbf{x} \geq \mathbf{0} \}$, hvor er $A$ er en $m \times n$ matrix.
$\mathcal{P}$ er her et \textit{polyeder på standardform}.
I de fleste tilfælde er det en fordel at antage, at de $m$ rækker i $A$ er lineært uafhængige.
I sætning \ref{thm:julieerfantalastisk} vises endvidere, at antagelsen om, at rækkerne er uafhængige, kan foretages, da de svarer til overflødige betingelser, som kan udelades.
Som det ses i definition \ref{defn:basal}, skal der være $n$ aktive begrænsninger i spil for at finde en basal mulig løsning.
Såfremt $m \neq n$, skal der derfor vælges $n-m$ variable $x_i$ med henblik på, at sætte disse $x_i=0$, hvilket gør begrænsningerne $x_i \geq 0$ aktive.
%ovenstående 53 i bogen har behov for andre læser så vi kan diskutere hvad det betyder.
Det er dog ikke uvæsentligt, hvilke af disse variable der omdannes til $0$ hvilket belyses af sætning \ref{thm:polystd}.
%
\begin{thm}{}{polystd}
Ved begrænsningerne $A\mathbf{x}=\mathbf{b}$ og $\mathbf{x}\geq 0$.
Antag, at $A$ er en $m \times n$ matrix og har lineært uafhængige rækker.
Vektoren $\mathbf{x} \in \R^n$ er en basal løsning, hvis og kun hvis $A\mathbf{x}=\mathbf{b}$, og der eksisterer indekser $B(1),\ldots,B(m)$, hvorom der gælder følgende:
\begin{enumerate}[label=(\alph*)]
\item Søjlerne $\mathbf{A}_{B(1)},\ldots,\mathbf{A}_{B(m)}$ er lineært uafhængige.
\item Hvis $i \neq B(1),\ldots, B(m)$, så er $x_i=0$.
\end{enumerate}
\end{thm}
\begin{proof}
Betragt et $x \in \R^n$ og antag, at der eksisterer indekser $\mathbf{A}_{B(1)},\ldots,\mathbf{A}_{B(m)}$, der opfylder (a) og (b).
%
Det gælder således, at de aktive begrænsninger, at $x_i=0$, når $i\neq B(1),\ldots,B(m)$, samt at $A\mathbf{x}=\mathbf{b}$.
Dette implicerer, da det forrige gælder,  at 
%
$$\sum_{i=1}^{m}\textbf{A}_{B(i)}x_{B(i)}=\sum_{i=1}^{n}\textbf{A}_ix_i=A\textbf{x}=\textbf{b}.$$
%
Da søjlerne $\textbf{A}_{B(i)}x_{B(i)}$ for $i=1,\ldots,m$ er lineært uafhængige, kan $x_{B(1)},\ldots,x_{B(m)}$ bestemmes entydigt. 
Altså har ligningssystemet skabt af de aktive begrænsninger en entydig løsning.
Jævnfør sætning \ref{thm:betingi},
% Ved ikke om denne reference er rigtig xD
følger det, at der er $n$ aktive begrænsninger, hvoraf $\mathbf{x}$ er en basal løsning. 
\\\\
%
%
Antag nu, at $\mathbf{x}$ er en basal løsning. 
Det skal nu vises, at (a) og (b) da er opfyldt.
Lad $x_{B_1},\ldots,x_{B_k}$ være ikke-nul komponenter i $\textbf{x}$.
Eftersom $\mathbf{x}$ er en basal løsning, følger nu, at ligningssysemet givet ved de aktive begrænsninger $x_i=0$, når $i\neq B(1),\ldots,B(k)$, samt  $\sum_{i=1}^{n}\mathbf{A}_ix_i=\mathbf{b}$, har en entydig løsning. 
Det samme må derfor gøre sig gældende for $\sum_{i=1}^{k}\mathbf{A}_{B(i)}x_{B(i)}=\mathbf{b}$.
Det følger derfor, at søjlerne i $A_{B(1)},\ldots,A_{B(k)}$ er lineært uafhængige.
%
Hvis dette ikke var tilfældet, ville der findes løsninger til $\sum_{i=1}^{k}\mathbf{A}_{B(i)} \lambda_i=\mathbf{0}$ udover den trivielle, hvor $\lambda_i=0$ for $i=1,2,\ldots,k$, hvilket ville betyder, at løsningen $\mathbf{x}$ ikke er entydig. 
Dette er i modstrid til at denne er en basal løsning.
$\mathbf{A}_{B(1)},\ldots ,\mathbf{A}_{B(k)}$ er således lineært uafhængige og $k \leq m$.
Da $A$ har $m$ lineært uafhængige rækker, er der ligeledes $m$ lineært uafhængige søjler.
%her kommer noget der følger fra sætning 1.3 i sektion 1.5 tror ikke vi har noget tilsvarende.
Der kan derfor findes $m-k$ søjler $A_{B(k+1)},\ldots,A_{B(m)}$, hvorom det gælder, at søjlerne $\mathbf{A}_{B(i)}$ med $i=1,\ldots,m$, er lineært uafhængige.
Da $k \leq m$ gælder det, hvis $i \neq B(1),\ldots,B(m)$, at $i \neq B(1),\ldots,B(k)$ og $x_i=0$.
%
\end{proof}
%der skal flere igennem det her bevis, tænker det er svært for andre end mig.
\\
\noindent
Som følge af sætning \ref{thm:polystd} kan basale løsninger konstrueres ved hjælp af følgende procedure.
\begin{enumerate}
\item Vælg $m$ lineært uafhængige søjler $\textbf{A}_{B(1)},\ldots,\textbf{A}_{B(m)}.$
\item Lad $x_i=0$ for $i \neq B(1),\ldots,B(m).$
\item Løs ligningssystemet med de $m$ ligninger, $A\textbf{x}=\textbf{b}$ for $x_{B(1)}, \ldots , x_{B(m)}.$
\end{enumerate}
Hvis løsningen, der er konstrueret af denne procedure, er ikke-negativ, så er løsningen en basal mulig løsning.
\\\\
\noindent
Det bevises nu, at antagelsen om at rækkerne var lineært uafhængige, kunne fortages.
\begin{thm}{}{julieerfantalastisk}
Lad $P=\{\textbf{x} \mid  A\textbf{x}=\textbf{b},x\geq 0\}$ være et ikke-tom polyeder, hvor $A$ er en $m \times n$ matrix med rækker $\textbf{a}^{T}_{1},\ldots,\textbf{a}^{T}_{m}$.
Antag, at rang$(A)=k<m$ og at rækkerne $\textbf{a}^T_{i1},\ldots,\textbf{a}^T_{ik}$ er lineært uafhængige. Givet polyederet 
$$Q=\{x \mid \textbf{a}^T_{i1}x=b_{i1},\ldots,\textbf{a}^T_{ik}x=b_{ik}, x \geq 0  \}$$ 
så er $Q=P$.
\end{thm}
\begin{proof}
Det bevises i tilfældet af $i_1=1,\ldots,i_k=k$, som er tilfældet, hvor de første $k$ rækker er lineært uafhængige. 
Alle andre tilfælde kan omskrives til dette tilfælde ved rækkeombytning. 
Det gælder, at $\mathcal{P}$ er en delmængde af $Q$, da alle elementer i $\mathcal{P}$ opfylder $Q$'s betingelser. 
Det skal dernæst vises, at $Q$ er en delmængde af $\mathcal{P}$.
Da rang$(A)=k$, har rækkerummet af $A$ dimension $k$ og et basis bestående af søjlerne $\textbf{a}_1,\ldots,\textbf{a}_k$. 
Derfor kan kan alle rækker $\textbf{a}_i$ af $A$ blive udtrykt som linearkombination af de andre rækker ved $\textbf{a}_i=\sum^{k}_{j=1}\lambda_{ij}\textbf{a}_j$ for skalarer $\lambda_{ij}$. 
Lad $x$ være et element i $\mathcal{P}$. 
Så gælder
$$b_i=\textbf{a}_ix=\sum^{k}_{j=1}\lambda_{ij}\textbf{a}_jx=\sum^{k}_{j=1}\lambda_{ij}b_j,$$
for $i=1,\ldots,m.$
Betragt nu et element $y$ i $Q$. 
Dette vil også være tilhøre $\mathcal{P}$, da
$$ \textbf{a}_iy=\sum^{k}_{j=1}\lambda_{ij}\textbf{a}_jy=\sum^{k}_{j=1}\lambda_{ij}b_j=b_i.$$
Dermed er det vist, at $Q$ er en delmængde til $\mathcal{P}$. 
Dermed vist at $Q=\mathcal{P}$, da $\mathcal{P}$ ligeledes er en delmængde af $Q$.
\end{proof} \\