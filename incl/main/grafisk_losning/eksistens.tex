\section{Eksistens af ekstremumspunkter}
%
Dette afsnit belyser tilstrækkelige og nødvendige betingelser for, at et polyeder har mindst ét ekstremumspunkt, da ikke alle polyedre har nogen.
Dette gøres ved at undersøge, om polyederet indeholder en \textit{uendelig linje}.
%
\begin{defn}{}{}
Et polyeder $\mathcal{P} \in \R^n$ indeholder en \textbf{uendelig linje}, hvis der eksisterer en vektor $\textbf{x} \in \mathcal{P}$ og en ikke-nulvektor $\textbf{d} \in \mathcal{P}$, således at $\textbf{x} + \lambda \textbf{d} \in \mathcal{P}$ for alle skalarer $\lambda$.
\end{defn}
\noindent
%
Af denne definition kommer følgende sætning.
%
\begin{thm}{}{ekstremums1}
Lad $\mathcal{P} = \{\textbf{x} \in \R^n | \textbf{a}_i^T\textbf{x} \geq b_i, i = 1,2,\ldots,m  \}$ være et ikke-tomt polyeder.
Så er følgende udsagn ækvivalente.
%
\begin{enumerate}[label = (\alph*)]
\item $\mathcal{P}$ har mindst ét ekstremumspunkt.
\item $\mathcal{P}$ indeholder ikke en uendelig linje.
\item Der eksisterer $n$ vektorer i mængden $\{a_1, a_2, \ldots ,a_m \}$, som er lineært uafhængige.
\end{enumerate}
\end{thm}
%
\begin{proof}
Lad $\textbf{x}$ være et element i $\mathcal{P}$ og lad $I = \{ i| \textbf{a}_i^T\textbf{x} = b_i \}$.
Hvis $n$ af vektorerne  $\textbf{a}_i, i \in I$, svarende til de bindende betingelser, er lineært uafhængige, så er $\textbf{x}$ en basal mulig løsning, og der eksisterer et ekstremumspunkt.
Hvis dette ikke er tilfældet, ligger alle vektorerne $\textbf{a}_i, i \in I$, i et ægte underrum af $\R^n$, og der eksisterer en ikke-nulvektor $\textbf{d} \in \R^n$, således at $\textbf{a}_i^T\textbf{d} = 0$ for alle $i \in I$.
Betragt linjen bestående af alle punkter på formen $\textbf{y} = \textbf{x} + \lambda \textbf{d}$, hvor $\lambda$ er en vilkårlig skalar.
For $i \in I$ haves, at $\textbf{a}_i^T\textbf{y} = \textbf{a}_i^T\textbf{x} + \lambda \textbf{a}_i^T\textbf{d} = \textbf{a}_i^T\textbf{x} = b_i$.
De betingelser, der er bindende i $\textbf{x}$, er således også bindende i alle punkter på linjen.
Antages det, at $\mathcal{P}$ ikke indeholder en uendelig linje, må der eksistrer et $\lambda$, hvor en af betingelser ikke længere er opfyldt.
I punktet, hvor betingelsen er ved at blive brudt, må en ny betingelse blive bindende.
Derfor må der eksistere et $\lambda_j$ og et $j \notin I$, således at $\textbf{a}_j^T (\textbf{x} + \lambda_j\textbf{d}) = b_j$.\\\\
%
Det haves, at $\textbf{a}_j^T\textbf{x} \neq b_j$, da $j \notin I$, og $\textbf{a}_i^T (\textbf{x} + \lambda_j\textbf{d}) = b_j$.
Derfor er $\textbf{a}_j^T\textbf{d} \neq 0$ og $\textbf{a}_i^T\textbf{d} = 0$ for alle $i \in I$.
Derfor er $\textbf{d}$ ortogonal med alle linearkombinationer af vektorerne $\textbf{a}_i, i\in I$.
Da $\textbf{d}$ ikke er ortogonal med $\textbf{a}_j$, konkluderes det, at $\textbf{a}_j$ ikke er en linearkombination af vektorerne $\textbf{a}_i, i \in I$.
Det er derfor muligt at forøge antallet af bindende lineært uafhængige betingelser med mindst én ved at gå fra $\textbf{x}$ til $\textbf{x} + \lambda_j\textbf{d}$.
Dette argument gentages indtil et punkt er nået, hvor der er $n$ lineært uafhængige betingelser, som er bindende.
Dette punkt vil således være en basal mulig løsning, da der er $n$ bindende betingelser og punktet er inde for $\mathcal{P}$, hvilket viser, at (b) $\rightarrow$ (a).\\\\
%
Hvis $\mathcal{P}$ har et ekstremumspunkt $\textbf{x}$, så er $\textbf{x}$ en basal mulig løsning og der eksisterer $n$ betingelser, som er bindende ved $\textbf{x}$.
Betingelsernes tilsvarende vektorer $\textbf{a}_i$ er lineært uafhængige, hvilket giver, at (a) $\rightarrow$ (b).\\\\
%
Lad $n$ af vektorerne $\textbf{a}_i$ være lineært uafhængige og antag, at det er $\textbf{a}_1, \textbf{a}_2, \ldots , \textbf{a}_n$, som er lineært uafhængige.
Antag, at $\mathcal{P}$ indeholder en linje $\textbf{x} + \lambda \textbf{d}$, hvor $\textbf{d}$ ikke er en nulvektor.
Så haves, at $\textbf{a}_i^T (\textbf{x} + \lambda \textbf{d}) \geq b_i$ for alle $i$ og alle $\lambda$.
Det konkluderes, at $\textbf{a}_i^T \textbf{d} = 0$ for alle $i$, da både  $\textbf{a}_i^T \textbf{d} < 0$ og  $\textbf{a}_i^T \textbf{d} > 0$ medfører, at der eksisterer et $\lambda$, som bryder betingelsen.
Da vektorerne $\textbf{a}_i, i = 1, 2, \ldots, n,$ er lineært uafhængige, medfører det, at $\textbf{d}=\textbf{0}$.
Dette giver en modstrid og derfor må (c) $\rightarrow$ (b).
\end{proof}\\
%
Bemærk, at begrænsede polyeder ikke indeholder en uendelig linje.
Ligeledes indeholder den positive kvadrant $\{ \textbf{x}|\textbf{x} \geq \textbf{0}\}$ ikke uendelige linjer og da polyeder på standardform befinder sig i den positive kvadrant, indeholder disse heller ikke uendelige linjer.
%
\begin{kor}{}{}
Ethvert begrænset ikke-tomt polyeder og ethvert ikke-tomt polyeder på standardform har mindst én basal mulig løsning.
\end{kor}
%
\begin{proof}
Bevis udelades.
\end{proof}