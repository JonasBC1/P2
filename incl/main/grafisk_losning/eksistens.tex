\section{Eksistens af ekstremumspunkter}
%
Dette afsnit belyser tilstrækkelige og nødvendige betingelser for, at et polyeder har mindst ét ekstremumspunkt, da dette ikke er tilfældet for alle polyedre. 
Dette gøres ved at undersøge, om polyederet indeholder en \textit{linje}.
%
\begin{defn}{}{klogemads}
Et polyeder $\mathcal{P} \in \R^n$ indeholder en \textbf{linje}, hvis der eksisterer en vektor $\textbf{v} \in \mathcal{P}$ og en ikke-nulvektor $\textbf{d} \in \mathcal{P}$, således at $\textbf{v} + \lambda \textbf{d} \in \mathcal{P}$ for alle skalarer $\lambda$.
\end{defn}
\noindent
%
\textcolor{red}{
På figur \ref{fig:julieerovergud1} ses et polyeder $\mathcal{P}_1$, hvor $\mathbf{v} + \lambda \mathbf{d}$ ikke tilhører $\mathcal{P}_1$, hvormed polyederet ikke indeholder en linje. 
På figur \ref{fig:julieerovergud2} ses et polyeder $\mathcal{P}_2$, hvor $\mathbf{v} + \lambda \mathbf{d}$ derimod tilhører $\mathcal{P}_2$, således polyederet indeholder en linje.
}
\\\\
\begin{center}
%
%%%%%%%%%%%%%%%%%%%%%%%%%%%%%%%%
%%% Flot graf alla Julie     %%%
%%%%%%%%%%%%%%%%%%%%%%%%%%%%%%%%
%
\begin{minipage}[b]{0.45\textwidth}
%
\begin{center}
%
\begin{tikzpicture}[scale=6]
% Koordinater 
% ------------------------------------------------------
\coordinate (a) at (0.5,0,0);
\coordinate (b) at (0,0,0);
\coordinate (c) at (0,0.5,0);
\coordinate (d) at (0,0,0);
\coordinate (e) at (0.25,0.25,0);
%
% Planet - Hyperplanet
% -------------------------------------------------------
\filldraw [fill=myblue,opacity=0.3] 
         (a) -- (b) -- (c) -- (a);
%        
% Streger 
% -------------------------------------------------------
  \draw[thick](-0.1,0.6,0)--(0.6,-0.1,0);
  \draw[thick](c)--(a);
  \draw[thick, color=myred](-0.05,0.55,0)--(0.1,0.4,0);
%
%
% Punkt 
% -------------------------------------------------------
%\filldraw [black] (e) circle (0.2pt) node[anchor=south west] {$\mathbf{v}$};
%\filldraw [black] (b) circle (0.2pt) node[left] {$\mathbf{u}$};
\filldraw [black] (0.15,0.15,0) circle (0pt) node[above] {$\mathcal{P}_1$};

\filldraw [black] (-0.05,0.55,0) circle (0.2pt) node[anchor=north east] {$\mathbf{v} + \lambda \mathbf{d} $};
\filldraw [black] (0.1,0.4,0) circle (0.2pt) node[anchor=north east] {$\mathbf{v}$};
% 
% Koordinatsystemet 
% -------------------------------------------------------
\draw[thick,->] (0,0,0) -- (0.7,0,0) node[anchor=south east]{$x_1$};
\draw[thick] (0,0,0) -- (-0.1,0,0);
\draw[thick,->] (0,0,0) -- (0,0.7,0) node[anchor=north west]{$x_2$};
\draw[thick] (0,0,0) -- (0,-0.1,0);
% 
\end{tikzpicture}
  \captionof{figure}{Illustration af et polyeder $\mathcal{P}_1$, hvor $\mathbf{v} + \lambda \mathbf{d}$ ikke tilhører $\mathcal{P}_1$, og dermed indeholder polyederet ikke en linje.}
  \label{fig:julieerovergud1}
\end{center}
\end{minipage}
%
\phantom{..}
%
\begin{minipage}[b]{0.45\textwidth}
\begin{center}
%
\begin{tikzpicture}[scale=6]
% Koordinater 
% ------------------------------------------------------
\coordinate (a) at (0.5,0,0);
\coordinate (b) at (0,0,0);
\coordinate (c) at (0,0.5,0);
\coordinate (d) at (0,0,0);
\coordinate (e) at (0.25,0.25,0);
%
% Planet - Hyperplanet
% -------------------------------------------------------
\filldraw [fill=myblue,opacity=0.3] 
         (-0.1,0.6,0) -- (0.6,-0.1,0) -- (0.7,-0.1,0) -- (0.7,0.7,0) -- (-0.1,0.7,0)-- (-0.1,0.6,0);
%        
% Streger 
% -------------------------------------------------------
  \draw[thick](-0.1,0.6,0)--(0.6,-0.1,0);
  \draw[thick](c)--(a);
   \draw[thick, color=myred](-0.05,0.55,0)--(0.1,0.4,0);
%
%
% Punkt 
% -------------------------------------------------------
%\filldraw [black] (e) circle (0.2pt) node[anchor=south west] {$\mathbf{v}$};
%\filldraw [black] (b) circle (0.2pt) node[left] {$\mathbf{u}$};
\filldraw [black] (0.4,0.4,0) circle (0pt) node[above] {$\mathcal{P}_2$};

\filldraw [black] (-0.05,0.55,0) circle (0.2pt) node[anchor=north east] {$\mathbf{v} + \lambda \mathbf{d} $};
\filldraw [black] (0.1,0.4,0) circle (0.2pt) node[anchor=north east] {$\mathbf{v}$};
% 
% Koordinatsystemet 
% -------------------------------------------------------
\draw[thick,->] (0,0,0) -- (0.7,0,0) node[anchor=south east]{$x_1$};
\draw[thick] (0,0,0) -- (-0.1,0,0);
\draw[thick,->] (0,0,0) -- (0,0.7,0) node[anchor=north west]{$x_2$};
\draw[thick] (0,0,0) -- (0,-0.1,0);
% 
\end{tikzpicture}
  \captionof{figure}{Illustration af et polyeder $\mathcal{P}_2$, hvor $\mathbf{v} + \lambda \mathbf{d}$ tilhører $\mathcal{P}_2$, og dermed indeholder polyederet en linje.}
  \label{fig:julieerovergud2}
\end{center}
\end{minipage}
\end{center}
%
Af \ref{defn:klogemads} kan \ref{thm:ekstremums1} udledes.
%
\begin{thm}{}{ekstremums1}
Lad $\mathcal{P} = \{\textbf{x} \in \R^n \mid \textbf{a}_i^T\textbf{x} \geq b_i, i = 1,2,\ldots,m  \}$ være et ikke-tomt polyeder, som ikke er på standardform.
Så er følgende udsagn ækvivalente.
%
\begin{enumerate}[label = (\alph*)]
\item $\mathcal{P}$ har mindst ét ekstremumspunkt.
\item $\mathcal{P}$ indeholder ikke en linje.
\item Der eksisterer $n$ vektorer i mængden $\{\mathbf{a}_1, \mathbf{a}_2, \ldots , \mathbf{a}_m \}$, som er lineært uafhængige.
\end{enumerate}
\end{thm}
%
%
\begin{proof}
Lad $\textbf{v}$ være et element i $\mathcal{P}$ og lad $I = \{ i \mid \textbf{a}_i^T\textbf{v} = b_i \}$.
Hvis $n$ af vektorerne  $\textbf{a}_i, i \in I$, svarende til de aktive betingelser, er lineært uafhængige, så er $\textbf{v}$ en basal mulig løsning, og der eksisterer et ekstremumspunkt.
Hvis dette ikke er tilfældet, ligger alle vektorerne $\textbf{a}_i, i \in I$, i et ægte underrum af $\R^n$, og der eksisterer en ikke-nulvektor $\textbf{d} \in \R^n$, således at $\textbf{a}_i^T\textbf{d} = 0$ for alle $i \in I$.
Betragt linjen bestående af alle punkter på formen $\textbf{u} = \textbf{v} + \lambda \textbf{d}$, hvor $\lambda$ er en vilkårlig skalar.
For $i \in I$ haves, at $$\textbf{a}_i^T\textbf{u} = \textbf{a}_i^T\textbf{v} + \lambda \textbf{a}_i^T\textbf{d} = \textbf{a}_i^T\textbf{v} = b_i.$$
De betingelser, der er aktive i $\textbf{v}$, er således også aktive i alle punkter på linjen.
Antages det, at $\mathcal{P}$ ikke indeholder en linje, så må der eksistere et $\lambda$, hvor en af betingelserne ikke længere er opfyldt.
I punktet, hvor betingelsen er ved at blive brudt, må en ny betingelse blive aktiv.
Derfor må der eksistere et $\lambda_j$, $j \notin I$, således at $\textbf{a}_j^T (\textbf{v} + \lambda_j\textbf{d}) = b_j$.\\\\
%
Det haves, at $\textbf{a}_j^T\textbf{v} \neq b_j$, da $j \notin I$, og $\textbf{a}_i^T (\textbf{v} + \lambda_j\textbf{d}) = b_j$.
Derfor er $\textbf{a}_j^T\textbf{d} \neq 0$ og $\textbf{a}_i^T\textbf{d} = 0$ for alle $i \in I$.
Derfor er $\textbf{d}$ ortogonal med alle linearkombinationer af vektorerne $\textbf{a}_i, i\in I$.
Da $\textbf{d}$ ikke er ortogonal med $\textbf{a}_j$, konkluderes det, at $\textbf{a}_j$ ikke er en linearkombination af vektorerne $\textbf{a}_i, i \in I$.
Det er derfor muligt at forøge antallet af aktive lineært uafhængige betingelser med mindst én ved at gå fra $\textbf{v}$ til $\textbf{v} + \lambda_j\textbf{d}$.
Dette argument gentages indtil et punkt er nået, hvor der er $n$ lineært uafhængige betingelser, som er aktive.
Dette punkt vil således være en basal mulig løsning, da der er $n$ aktive betingelser og punktet er inde for $\mathcal{P}$, hvilket viser, at (b) medfører (a).\\\\
%
Hvis $\mathcal{P}$ har et ekstremumspunkt $\textbf{v}$, så er $\textbf{v}$ en basal mulig løsning og der eksisterer $n$ betingelser, som er aktive ved $\textbf{v}$.
Betingelsernes tilsvarende vektorer $\textbf{a}_i$ er lineært uafhængige, hvilket giver, at (a) medfører (c).
\\\\
%
Lad $n$ af vektorerne $\textbf{a}_i$ være lineært uafhængige og antag, at det er $\textbf{a}_1, \textbf{a}_2, \ldots , \textbf{a}_n$, som er lineært uafhængige.
Antag, at $\mathcal{P}$ indeholder en linje $\textbf{v} + \lambda \textbf{d}$, hvor $\textbf{d}$ ikke er en nulvektor.
Så haves, at $\textbf{a}_i^T (\textbf{v} + \lambda \textbf{d}) \geq b_i$ for alle $i$ og alle $\lambda$.
Det konkluderes, at $\textbf{a}_i^T \textbf{d} = 0$ for alle $i$, da både  $\textbf{a}_i^T \textbf{d} < 0$ og  $\textbf{a}_i^T \textbf{d} > 0$ medfører, at der eksisterer et $\lambda$, som bryder betingelsen.
Da vektorerne $\textbf{a}_i, i = 1, 2, \ldots, n,$ er lineært uafhængige, medfører det, at $\textbf{d}=\textbf{0}$.
Dette giver en modstrid og derfor må (c) medfører (b).
\end{proof}\\
%
Bemærk, at begrænsede polyedre ikke indeholder en linje.
Ligeledes indeholder den positive kvadrant $\{ \textbf{x} \mid \textbf{x} \geq \textbf{0}\}$ ikke uendelige linjer og da polyedre på standardform befinder sig i den positive kvadrant, indeholder disse heller ikke uendelige linjer.
Dette giver \ref{kor:adenstatistiker}.
%Det følger af sætning 3.7 på grund af det oversåtende
\begin{kor}{}{adenstatistiker}
Ethvert begrænset ikke-tomt polyeder og ethvert ikke-tomt polyeder på standardform har mindst én basal mulig løsning.
\end{kor}
%%
%\begin{proof}
%Bevis udelades.
%\end{proof}