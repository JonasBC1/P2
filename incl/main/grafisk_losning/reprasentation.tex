\section{Repræsentation af begrænsede polyedre}
%
Et polyeder har indtil videre i nærværende rapport været defineret ved uligheder. 
I dette afsnit introduceres et alternativ til dette. 
Det vises, at et begrænset polyeder kan repræsenteres som et konvekst hylster af dets ekstremumspunkter. 
Dette er givet ved \ref{thm:begraenspoly}. \\
%%%%
\begin{thm}{}{begraenspoly}
Et ikke-tomt og begrænset polyeder er givet ved det konvekse hylster af dets ekstremumspunkter.
\end{thm}
\begin{proof}
Enhver konveks kombination af ekstremumspunkterne er et element i polyederet, eftersom polyederet er en konveks mængde. 
Derfor ønskes at bevise det omvendte og dermed bevise, at ethvert element i et begrænset polyeder kan repræsenteres ved en konveks kombination af ekstremumspunkterne. \\
%%%%%
For at bevise dette defineres dimensionen af et polyeder $P \subset \R^n$ som det mindste heltal $k$, så polyederet $P$ er betinget af $k$-dimensionelle tilgrænsede underrum af $\R^n$. 
Beviset fortsættes dernæst ved at bevise dimensionen af $P$ ved induktion. 
Antag, at $P$ er nul-dimensionel, så polyederet indeholder et enkelt punkt. 
Punktet er et ekstremumspunkt i $P$ og resultatet er derfor sandt. \\\\
%%%%%
Induktionsskridtet er dernæst at bevise, at det er sandt for alle polyedre med færre dimensioner end $k$. Lad derfor $P = \{\textbf{x} \in \R^n \mid \mathbf{a}_{i}^{'} \textbf{x} \geq b_i, i=1,\ldots,m\}$ være et ikke-tom begrænset $k$-dimensionelt polyeder. 
Så er $P$ indrammmet i et tilgrænset underrum $S$ i $\R^n$, som er givet ved følgende: \\
%%%%%
$$S = \{\textbf{x}^0+\lambda_1\textbf{x}^1+\cdots+\lambda_k\textbf{x}^k \mid \lambda_1,\ldots,\lambda_k \in \R\},$$
hvor $\textbf{x}^1,\ldots,\textbf{x}^k$ er vektorer i $\R^n$. 
Lad $\textbf{f}_1,\ldots,\textbf{f}_n-k$ være lineært uafhængige vektorer, som er ortogonale med vektorerne $\textbf{x}^1,\ldots,\textbf{x}^k$. 
Lad dernæst $g_i=\mathbf{f}_{i}^{'}\textbf{x}^0$, gældende for $i=1,\ldots,n-k$, således at ethvert element $\textbf{x}$ i $S$ tilfredsstiller følgende: \\
%%%%%
$$\mathbf{f}_{i}^{'}\textbf{x}=g_i, \text{     } \text{for} \text{     } i=1,\ldots,n-k.$$ \\
Siden $P \subset S$, så skal det være sandt for ethvert element i $P$. \\
%%%%%
Lad $\textbf{z}$ være et arbitrært element i $P$. 
Hvis det gælder for $\textbf{z}$, at punktet er et ekstremumspunkt i $P$, så er $\textbf{z}$ en triviel konveks kombination af ekstremumspunkterne i $P$, og beviset er derfor færdigt. 
Hvis $\textbf{z}$ derimod ikke er et ekstremumspunkt i $P$, skal det bevises, at $\textbf{z}$ er en konveks kombination af ekstremumspunkterne i $P$. \\
%%%%%
Lad derfor $\textbf{y}$ være et arbitrært ekstremumspunkt og dan derfor en halvlinje indeholdende alle punkter af formen $\textbf{z}+\lambda(\textbf{z}-\textbf{y})$, hvor $\lambda$ angiver en ikke-negativ skalar. 
Da det gælder, at $P$ er et begrænset polyeder, vil denne halvlinje nødvendigvis forlade $P$ og derfor overskride en vilkårlig begrænsning, $\mathbf{a}_{i^*}^{'} \textbf{x} \geq b_{i^*}$. 
Ved at anskue hvad det vil resultere i at bryde denne begrænsning, vil der findes et $\lambda^* \geq 0$ og $\textbf{u} \in P$, så følgende gælder: \\ 
%%%%%
$$\textbf{u}=\textbf{z}+\lambda*(\textbf{z}-\textbf{y}), \text{     } \text{og} \text{      } \mathbf{a}_{i^*}^{'} \textbf{u} = b_{i^*} $$
Eftersom begrænsningen $\mathbf{a}_{i}^{'} \textbf{x} \geq b_i$ overskrides, hvis $lambda$ vokser større end $\lambda^*$, så følger det, at $\mathbf{a}_{i^*}^{'}(\textbf{z}-\textbf{y}) <0.$ \\
%%%%%
Da dette gælder, lad derfor $Q$ være et polyeder defineret ved 
$$Q=\{\textbf{x} \in P \mid \mathbf{a}_{i^*}^{'} \textbf{x} =b_{i^*}\} = \{\textbf{x} \in \R^n \mid \mathbf{a}_{i}^{'} \textbf{x} \geq b_{i}\}, i=1,\ldots,m, \text{   } \mathbf{a}_{i^*}^{'} \textbf{x} =b_{i^*}\}.$$
Da $\textbf{z},\textbf{y} \in P$, så medfølger $\mathbf{f}_{i}^{'}\textbf{z}=g_i=\mathbf{f}_{i}^{'}\textbf{y}$, hvilket viser, at $\textbf{z}-\textbf{y}$ er ortogonal med enhver vektor $\mathbf{f}_{i}$, for alle $i=1,\ldots,n-k$. 
På den anden side er det vist, at $\mathbf{a}_{i^*}^{'}(\textbf{z}-\textbf{y}) <0$, så det gælder, at vektoren  $\mathbf{a}_{i^*}$ ikke er en linear kombination af vektorerne $\mathbf{f}_{i}$, og derfor ikke er lineært uafhængige. 
Bemærk derfor, at \\
%%%%%
$$Q \subset \{\textbf{x} \in \R^n \mid \mathbf{a}_{i^*}^{'} \textbf{x}=b_{i^*}, \text{  } \mathbf{f}_{i}^{'}\textbf{x}=g_i, \text{  } i=1,\ldots,n-k\},$$ 
gældende for ethvert element i $P$. 
Mængden på højresiden er defineret ved $n-k+1$ lineært uafhængige lighedsbegrænsninger. 
Dette betyder derfor, at det tilgrænsede underrum har dimensionen givet ved $k$-1, hvormed polyederet $Q$ højst har dimensionen $k-1$. \\
%%%%%
Ved afslutningsvis at benytte induktionshypotesen på $Q$ og $\textbf{u}$ kan $\textbf{u}$ udtrykkes ved den konvekse kombination
$$\textbf{u}=\sum_{i}\lambda_{i}\textbf{v}^i$$
af ekstremumspunkterne $\textbf{v}^i$ i polyederet $Q$, hvor $\lambda_i$ er ikke-negative skalarer, som summerer til én. 
Bemærk dernæst, at i et ekstremumspunkt $\textbf{v}$ skal $\mathbf{a}_{i}^{'} \textbf{v}=b_{i}$ for $n$ antal lineært uafhængige vektorer $\textbf{a}_i$. 
Derfor er  $\textbf{v}$ også et ekstremumspunkt i $P$. 
Ved brug af definitionen for $\lambda^*$ haves derfor, at 
$$ \textbf{z}=\frac{\textbf{u}+\lambda^*\textbf{y}}{1+\lambda^*}.$$
Dermed er det bevist, at $\textbf{z}$ er en konveks kombination af ekstremumspunkter i $P$, da det haves, at 
$$\textbf{z}=\frac{\lambda^*\textbf{y}}{1+\lambda^*}+\sum_{}i\frac{\lambda_i}{1+\lambda_i}\textbf{v}^i.$$
\end{proof}
