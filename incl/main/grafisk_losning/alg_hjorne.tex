%
\begin{defn}{}{algh}
Hvis en vektor $\textbf{x}$ opfylder $\textbf{a}^T_i\textbf{x}= b_i$ for $i \in M_1, M_2 \text{ eller } M_3$, så siges den tilsvarende betingelse at være \textbf{bindende} ved $\textbf{x}$.
\end{defn}\noindent
%
Følgende sætning giver anledning til hvordan disse bindende betingelser kan føres i relation til løsninger af lineære optimeringsproblemer.
%
\begin{thm}{}{betingi}

Lad $\textbf{x}$ være et element i $\R^n$ og lad $I=\{i|\textbf{a}^T_i\textbf{x}=b_i\}$ være en mængde af indexer på betingelser, der er bindende ved $\textbf{x}$.
Så er følgende udsagn ækvivalente.
%
\begin{enumerate}[label=(\alph*)]
\item Der eksisterer $n$ vektorer i mængden $\{\textbf{a}_i|i \in I \}$, som er lineært uafhængige.
\item Spannet af vektorerne $\textbf{a}_i$, for $i \in I$, dækker hele $\R^n$.
\item Ligningssystemet $\textbf{a}^T_i\textbf{x}= b_i$, for $i \in I$, har en entydig løsning.
\end{enumerate}
\end{thm}
%
\begin{proof}
Antag, at $\text{span}\{a_i|i \in I\}= \R^n$.
Dette span har dimensionen $n$ og jævnfør sætning (1.3(a) i bertsimas) kan et basis for spannet dannes med $n$ af vektorerne, som vil være lineært uafhængige.
%
Hvis der ligeledes eksisterer $n$ vektorer i $\{a_i|i \in I\}$, som er lineært uafhængige, må disse udspænde hele $\R^n$.
Dette viser, at (a) og (b) er ækvivalente.\\\\
%
Antag at ligningssystemet $\textbf{a}^T_i\textbf{x}=b_i$, $i \in I$, har løsningerne $\textbf{x}_1$ og $\textbf{x}_2$.
Så skal ikke-nulvektoren $\textbf{d} = \textbf{x}_1 - \textbf{x}_2$ opfylde, at $\textbf{a}^T_i\textbf{d}=0$ for alle $i \in I$.
Dette kræver, at $\textbf{d}$ er ortogonal med alle $\textbf{a}_i$, $i \in I$ og dermed ikke er en linearkombination af disse vektorer.
Dette medfører, at $\text{span}\{a_i|i \in I\} \neq \R^n$.
%
Antag nu, at vektorerne $\textbf{a}_i$, $i \in I$ ikke spænder over $\R^n$.
Så er det muligt , at vælge en ikke-nulvektor $\textbf{d}$, som er ortogonal med underrummet, som vektorerne spænder over.
Hvis $\textbf{x}$ opfylder  $\textbf{a}^T_i\textbf{x}= b_i$, for alle $i \in I$, så haves, at $\textbf{a}^T_i(\textbf{x}+\textbf{d})= b_i$, for alle $i \in I$, og der er dermed adskillige løsninger.
Dermed er det vist, at (b) og (c) er ækvivalente.
\end{proof}\\
%
Med udgangspunkt i betingelsers tilsvarende vektorer kan givne betingelser siges at være lineært uafhængige, hvis de tilsvarende vektorer er lineært uafhængige.
Med dette og sætning \ref{thm:betingi}(a) kan hjørnepunkterne defineres som mulige løsninger, hvor $n$ lineært uafhængige betingelser er bindende.
Bemærk, at alle betingelser skal være opfyldt, for at et givent punkt kan være en mulig løsning og, at et punkt i ikke behøver at opfylde alle betingelser for at kunne have $n$ bindende betingelser.
%
\begin{defn}{}{}
Lad $P$ være en polyede defineret med lineære lighed- og ulighedsbetingelser og lad $\textbf{x}$ være en vektorer i $\R^n$.
%
\begin{enumerate}[label=(\alph*)]
\item Vektoren $\textbf{x}$ er en \textbf{basal løsning}, hvis:
%
\begin{enumerate}[label=(\roman*)]
\item Alle lighedsbetingelser er opfyld;
\item Af de bindende betingelser er der $n$ af dem, som er lineært uafhængige.
\end{enumerate}
%
\item Vektoren $\textbf{x}$ er en \textbf{basal mulig løsning}, hvis $\textbf{x}$ er en \textit{basal løsning} og alle betingelser er opfyldt.
\end{enumerate}
\end{defn}
%
Bemærk, at hvis en polyede er defineret med $m$ lineære betingelser og $m<n$, så eksisterer der ikke et punkt med $n$ bindende betingelser.
Dette medfører, at der hverken er \textit{bassale løsninger} eller \textit{basale mulige løsninger}.\\\\
%
Tre definitioner, som ønsker at beskrive det samme, er blevet givet.
Det er muligt at skifte mellem definitionerne efter ønske, da de er ækvivalente.
%
\begin{thm}{}{hjorneeq}
Lad $P$ være en ikke tom polyede og lad $\textbf{x}\in P$.
Så er følgende udsagn ækvivalente:
%
\begin{enumerate}[label=(\alph*)]
\item $\textbf{x}$ er et \textit{ekstremumpunkt}.
\item $\textbf{x}$ er et \textit{hjørnepunkt}.
\item $\textbf{x}$ er en \textit{basal mulig løsning}.
\end{enumerate}
%
\end{thm}
%
\begin{proof}
To be continued...
\end{proof}