%
\begin{defn}{}{algh}
Hvis en vektor $\textbf{v}$ opfylder $\textbf{a}^T_i\textbf{v}= b_i$ for $i \in M_1, M_2$ eller $M_3$, så er den tilsvarende betingelse \textbf{bindende} eller \textbf{aktive} i $\textbf{v}$.
\end{defn}\noindent
%
Bemærk, at $M_1, M_2$ og $M_3$ er indeksmængder for henholdsvis lighedsbetingelser og ulighedsbetingelser.
På figur \ref{fig:julieermegadeeeeeejliiiiig} ses et polyeder $\mathcal{P}= \{ \mathbf{x} \in \R^3 \mid x_1 + x_2 + x_3 = 1, \text{  } x_1 , x_2, x_3 \geq 0  \}$. 
Punktet $\textbf{u}$ har tre aktive betingelser $x_1 + x_2 + x_3 = 1$, $x_1=0$ og $x_3=0$, og $\mathbf{v}$ har to aktive betingelser $x_1 + x_2 + x_3 = 1$ og $x_3=0$.
%
\begin{figure}[h!]
  \centering
%
%%%%%%%%%%%%%%%%%%%%%%%%%%%%%%%%
%%% Flot graf alla Julie     %%%
%%%%%%%%%%%%%%%%%%%%%%%%%%%%%%%%
%
%
\begin{tikzpicture}[scale=6]
% Koordinater 
% ------------------------------------------------------
\coordinate (a) at (0.5,0,0);
\coordinate (b) at (0,0.5,0);
\coordinate (c) at (0,0,0.5);
\coordinate (d) at (0,0,0);
\coordinate (e) at (0.25,0.25,0);
%
% Planet - Hyperplanet
% -------------------------------------------------------
\filldraw [fill=myblue,opacity=0.6] 
         (b) -- (a) -- (c) -- (b);
%        
% Streger 
% -------------------------------------------------------
  \draw[thick](b)--(a);
  \draw[thick](a)--(c);
  \draw[thick](c)--(b);
%
%
% Punkt 
% -------------------------------------------------------
\filldraw [black] (e) circle (0.2pt) node[anchor=south west] {$\mathbf{v}$};
\filldraw [black] (b) circle (0.2pt) node[left] {$\mathbf{u}$};
\filldraw [black] (0.15,0.15,0.15) circle (0pt) node[above] {$\mathcal{P}$};
% 
% Koordinatsystemet 
% -------------------------------------------------------
\draw[thick,->] (0,0,0) -- (0.7,0,0) node[anchor=south east]{$x_1$};
\draw[thick] (0,0,0) -- (-0.1,0,0);
\draw[thick,->] (0,0,0) -- (0,0.7,0) node[anchor=north west]{$x_2$};
\draw[thick] (0,0,0) -- (0,-0.1,0);
\draw[thick,->] (0,0,0) -- (0,0,0.7) node[anchor=south east]{$x_3$};
\draw[thick] (0,0,0) -- (0,0,-0.1);
% 
\end{tikzpicture}
  \caption{Et polyeder $\mathcal{P}= \{ \mathbf{x} \in \R^3 \mid x_1 + x_2 + x_3 = 1, \text{  } x_1 , x_2, x_3 \geq 0  \}$. Punktet $\textbf{u}$ har tre aktive betingelser $x_1 + x_2 + x_3 = 1$, $x_1=0$ og $x_3=0$, og $\mathbf{v}$ har to aktive betingelser $x_1 + x_2 + x_3 = 1$ og $x_3=0$.}
  \label{fig:julieermegadeeeeeejliiiiig}
\end{figure}
%
\\\\
%
\ref{thm:betingi} giver anledning til, at disse bindende betingelser kan føres i relation til løsninger af lineære optimeringsproblemer.
%
%
\begin{thm}{}{betingi}
Lad $\textbf{x}$ være et element i $\R^n$ og lad $I=\{i \mid \textbf{a}^T_i\textbf{x}=b_i\}$ være en mængde af indekser på betingelser, der er bindende ved $\textbf{x}$.
Så er følgende udsagn ækvivalente:
%
\begin{enumerate}[label=(\alph*)]
\item Der eksisterer $n$ vektorer i mængden $\{\textbf{a}_i \mid i \in I \}$, som er lineært uafhængige.
\item Spannet af vektorerne $\textbf{a}_i$, for $i \in I$, dækker hele $\R^n$.
\item Ligningssystemet $\textbf{a}^T_i\textbf{x}= b_i$, for $i \in I$, har en entydig løsning.
\end{enumerate}
\end{thm}
%
\begin{proof}
Antag, at $\text{span}\{\textbf{a}_i \mid i \in I\}= \R^n$.
Dette span har dimension $n$, og jævnfør sætning \ref{} %(1.3(a) i bertsimas) 
kan en basis for spannet dannes med $n$ af vektorerne, som er lineært uafhængige.
%
Hvis der ligeledes eksisterer $n$ vektorer i $\{\textbf{a}_i \mid i \in I\}$, som er lineært uafhængige, må disse udspænde hele $\R^n$.
Dette viser, at (a) og (b) er ækvivalente.\\\\
%
Antag, at ligningssystemet $\textbf{a}^T_i\textbf{x}=b_i$, $i \in I$, har løsningerne $\textbf{u}$ og $\textbf{v}$.
Så skal ikke-nulvektoren $\textbf{d} = \textbf{u} - \textbf{v}$ opfylde, at $\textbf{a}^T_i\textbf{d}=0$ for alle $i \in I$.
Dette kræver, at $\textbf{d}$ er ortogonal med alle $\textbf{a}_i$, $i \in I$ og dermed ikke er en linearkombination af disse vektorer.
Dette medfører, at $\text{span}\{\textbf{a}_i \mid i \in I\} \neq \R^n$.
%
Antag nu, at vektorerne $\textbf{a}_i$, $i \in I$, ikke spænder over $\R^n$.
Så er det muligt at vælge en ikke-nulvektor $\textbf{d}$, der er ortogonal med underrummet, som vektorerne spænder over.
Hvis $\textbf{x}$ opfylder  $\textbf{a}^T_i\textbf{x}= b_i$, for alle $i \in I$, så haves, at $\textbf{a}^T_i(\textbf{x}+\textbf{d})= b_i$, for alle $i \in I$, og der er dermed adskillige løsninger.
Dermed er det vist, at (b) og (c) er ækvivalente.
\end{proof}\\
%
Hvis $\textbf{a}_i$ for en given mængde betingelser er lineært uafhængige, siges betingelserne at være lineært uafhængige.
Med dette og \ref{thm:betingi}(a) kan hjørnepunkterne defineres som \textit{basale mulige løsninger}, hvor $n$ lineært uafhængige betingelser er bindende.
%Bemærk, at alle betingelser skal være opfyldt, for at et givent punkt kan være en basal mulig løsning, samt at et punkt ikke behøver at opfylde alle betingelser for at kunne have $n$ bindende betingelser.
%
\begin{defn}{}{basal}
Lad $\mathcal{P}$ være et polyeder defineret ved lineære ligheds- og ulighedsbetingelser og lad $\textbf{v}$ være en vektor i $\R^n$.
%
\begin{enumerate}[label=(\alph*)]
\item Vektoren $\textbf{v}$ er en \textbf{basal løsning}, hvis:
%
\begin{enumerate}[label=(\roman*)]
\item Alle lighedsbetingelser er opfyldt.
\item Der er $n$ af de bindende betingelser, som er lineært uafhængige.
\end{enumerate}
%
\item Vektoren $\textbf{v}$ er en \textbf{basal mulig løsning}, hvis $\textbf{v}$ er en \textit{basal løsning}, og alle betingelser er opfyldt.
\end{enumerate}
\end{defn}\noindent
%
Bemærk, at hvis et polyeder er defineret med $m$ lineære betingelser, og $m<n$, så eksisterer der ikke et punkt med $n$ bindende betingelser.
Dette medfører, at der hverken er basale løsninger eller basale mulige løsninger.
På figur \ref{fig:basale} ses punkterne $A$ til $H$, som er basale løsninger, hvor $A,B,C,E$ og $G$ også er basale mulige løsninger.
\\\\
%
\begin{center}
%
\begin{tikzpicture}[scale=5]
%
% Koordinater
% -------------------------------------------------------
\coordinate (a) at (0,0,0); 
\coordinate (b) at (0.7,0,0); 
\coordinate (c) at (0.393,0.55,0); 
\coordinate (d) at (0.252,0.805,0);
\coordinate (e) at (0.066,0.434,0);
\coordinate (f) at (0,0.41,0);
\coordinate (g) at (0,0.305,0);
\coordinate (h) at (-0.15,0,0); 
%
% Farvning
% -------------------------------------------------------
\filldraw[fill=myblue,opacity=0.6](a)--(b)--(c)--(e)--(g)--(a);
  \draw[thick](-0.2,-0.1,0)--(0.3,0.9,0); % n -> d  
  \draw[thick](0.2,0.9,0)--(0.757,-0.1,0); % d -> b
  \draw[thick](-0.3,0.3,0)--(0.8,0.7,0); % f -> c
% 
%
% Punkterne 
% -------------------------------------------------------
\filldraw [black] (a) circle (0.2pt) node[anchor=north west] {$A$};
\filldraw [black] (b) circle (0.2pt) node[anchor=north east] {$B$};
\filldraw [black] (c) circle (0.2pt) node[anchor=south west] {$C$};
\filldraw [black] (d) circle (0.2pt) node[anchor=west] {$D$};
\filldraw [black] (e) circle (0.2pt) node[anchor=north west] {$E$};
\filldraw [black] (f) circle (0.2pt) node[anchor=south east] {$F$};
\filldraw [black] (g) circle (0.2pt) node[anchor=east] {$G$};
\filldraw [black] (h) circle (0.2pt) node[anchor=north west] {$H$};
%
% 
\filldraw [black] (0.27,0.17,0) circle (0pt) node[above] {$\mathcal{P}$};
% 
% Koordinatssystem 
% -------------------------------------------------------
\draw[thick] (0,0,0) -- (0.9,0,0);
\draw[thick] (0,0,0) -- (-0.2,0,0);
\draw[thick] (0,0,0) -- (0,0.9,0);
\draw[thick] (0,0,0) -- (0,-0.2,0);
%
\end{tikzpicture}
  \captionof{figure}{Punkterne $A,B,C,D,E,F,G$ og $H$ er basal løsninger, hvoraf $A,B,C,E$ og $G$ alle er basal mulige løsninger.}
  \label{fig:fig:basale}
\end{center}
%
%
Tre definitioner, som ønsker at beskrive det samme, er blevet givet.
Det er muligt at skifte mellem definitionerne efter behov, da de er ækvivalente.
%
\begin{thm}{}{hjorneeq}
Lad $\mathcal{P}$ være et ikke-tomt polyeder og lad vektoren $\textbf{v}\in \mathcal{P}$.
Så er følgende udsagn ækvivalente:
%
\begin{enumerate}[label=(\alph*)]
\item $\textbf{v}$ er et ekstremumspunkt.
\item $\textbf{v}$ er et hjørnepunkt.
\item $\textbf{v}$ er en basal mulig løsning.
\end{enumerate}
%
\end{thm}
%
\begin{proof}
Lad $\mathcal{P}$ være et polyeder defineret ved betingelser på formen $\textbf{a}_i^T\textbf{x} \geq b_i$ og $\textbf{a}_i^T\textbf{x} = b_i$.\\\\
%
Lad $\textbf{u} \in \mathcal{P}$ være et hjørnepunkt. Jævnfør \ref{defn:hjoerner} eksisterer der så et $\textbf{c} \in \R^n$, således at $\textbf{c}^T\textbf{u} < \textbf{c}^T\textbf{v}$ for alle $\textbf{v}$, som opfylder $\textbf{v} \in \mathcal{P}$ og $\textbf{v} \neq \textbf{u}$.
Hvis $\textbf{v} \in \mathcal{P}$, $\textbf{w} \in \mathcal{P}$, $\textbf{v} \neq \textbf{u}$, $\textbf{w} \neq \textbf{u}$ og $0 \leq \lambda \leq 1$, så er $\textbf{c}^T\textbf{u} < \textbf{c}^T\textbf{v}$ og $\textbf{c}^T\textbf{u} < \textbf{c}^T\textbf{w}$.
Dette medfører, at $\textbf{c}^T\textbf{u} < \textbf{c}^T(\lambda \textbf{v} + (1-\lambda)\textbf{w})$, som betyder, at $\textbf{u} \neq \lambda \textbf{v} + (1 - \lambda)\textbf{w}$.
Derfor kan $\textbf{u}$ ikke udtrykes som en kombination af to andre elementer i $\mathcal{P}$, hvilket betyder, at $\textbf{u}$ er et ekstremumspunkt ifølge \ref{defn:ekstrema}.
(b) medfører hermed (a).
\\\\
%
Antag, at $\textbf{u} \in \mathcal{P}$ ikke er en basal mulig løsning og lad $I = \{ i \mid \textbf{a}_i^T \textbf{u} = b_i\}$.
Da $\textbf{u}$ ikke er en basal mulig løsning, må der ikke eksistere $n$ lineært uafhængige vektorer af typen $\textbf{a}_i$, $i \in I$.
Derfor må vektorerne $\textbf{a}_i$, $i \in I$, ligge i et ægte underrum af $\R^n$ og der eksisterer derfor en ikke-nulvektor $\textbf{d} \in \R^n$, således at $\textbf{a}_i^T\textbf{d} = 0$ for alle $i \in I$.
Lad $\varepsilon$ være et lille, positivt tal, og betragt vektorerne $\textbf{v} = \textbf{u} + \varepsilon\textbf{d}$ og $\textbf{w} = \textbf{u} - \varepsilon\textbf{d}$.
Bemærk, at $\textbf{a}_i^T \textbf{v} = \textbf{a}_i^T \textbf{u} = b_i$, for $i \in I$.
Endvidere haves, at $\textbf{a}_i^T \textbf{u} > b_i$ for $i \notin I$ og givet, at $\varepsilon$ er småt nok, haves $\textbf{a}_i^T \textbf{v} > b_i$ for $i \notin I$.
Det er her tilstrækkeligt at vælge $\varepsilon$, således at $\varepsilon \mid \textbf{a}_i^T\textbf{d} \mid < \textbf{a}_i^T\textbf{u} - b_i$ for alle $i \notin I$.
Når $\varepsilon$ er lille nok, haves det, at $\textbf{v} \in \mathcal{P}$, og med samme argumentation haves, at $\textbf{w} \in \mathcal{P}$.
Dette betyder, at $\textbf{u}$ ikke er et ekstremumspunkt, da $\textbf{u} = \frac{\textbf{v} + \textbf{w}}{2}$.
Ved kontraponering ses (a) medfører (c).
\\\\
%
Lad $\textbf{u}$ være en basal mulig løsning, lad $I = \{i \mid \textbf{a}_i^T\textbf{u} = b_i \}$ og lad $\textbf{c}=\sum_{i \in I} \textbf{a}_i$.
Så haves, at
%
\begin{align*}
\textbf{c}^T\textbf{u} = \sum_{i\in I}\textbf{a}_i^T\textbf{u} = \sum_{i\in I}b_i.
\end{align*}
%
Yderligere haves, at $\textbf{a}_i^T\textbf{u} \geq b_i$ for ethvert $\textbf{u} \in \mathcal{P}$ og ethvert $i$, og 
%
\begin{align}\label{eq:ulig}
\textbf{c}^T\textbf{u} = \sum_{i\in I}\textbf{a}_i^T\textbf{u} \geq \sum_{i\in I}b_i.
\end{align}
%
Dette viser, at $\textbf{u}$ er en optimal løsning til minimumsproblemet $\textbf{c}^T\textbf{u}$ over $\mathcal{P}$.
Yderligere holder uligheden i ligning \ref{eq:ulig}, hvis og kun hvis $\textbf{a}_i^T\textbf{u} = b_i$ for alle $i \in I$.
Da $\textbf{u}$ er en basal mulig løsning, er der $n$ lineært uafhængige betingelser, som er bindende i $x$, og $x$ er en entydig løsning til ligningssystemet $\textbf{a}_i^T\textbf{u} = b_i$, $i \in I$.
Det følger derfor, at $\textbf{u}$ er entydigt minimum til $\textbf{c}^T\textbf{u}$ over $\mathcal{P}$ og dermed et hjørnepunkt til $\mathcal{P}$.
(c) medfører hermed (b).
%
\end{proof}\\
%
Eftersom en vektor er en basal mulig løsning, hvis og kun hvis det er et ekstremumspunkt og da definitionen på et ekstremumspunkt ikke refererer til en specifik repræsentation af polyederet, kan det konkluderes, at basale mulige løsninger er uafhængige af den valgte repræsentation.
%
\begin{kor}{}{}
Der kan kun være et endeligt antal af basale løsninger og basale mulige løsninger, givet der er et endeligt antal af lineære ulighedsbetingelser.
\end{kor}
%
\begin{proof}
Betragt et system af $m$ lineære ulighedsbetingelser for vektoren $\textbf{v} \in \R^n$.
Ved enhver basal løsning er der $n$ lineære ulighedsbetingelser, som er bindende.
Da $n$ bindende lineære ulighedsbetingelser definerer et entydigt punkt, følger det, at en anden basal løsning må have en anden mængde af $n$ bindende lineære ulighedsbetingelser.
Dette betyder, at antallet af basale løsninger har en øvre begrænsning, som afhænger af mængden af forskellige måder, hvorpå $n$ betingelser kan udvælges af de $m$ betingelser.
\end{proof}\\
%
To forskellige basale løsninger til samme mængde af lineære betingelser i $\R^n$ siges at være \textit{tilstødende}, hvis der findes $n-1$ lineært uafhængige betingelser, som er bindende ved begge. 
Hvis de to basale løsninger er basale mulige løsninger, så er det forbindende linjestykke en af løsningsmængdens kanter.
På figur \ref{fig:basale} er $A$ og $B$ et eksempel på tilstødende basale mulige løsninger.