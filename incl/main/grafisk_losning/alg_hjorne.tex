\section{Ekstremer, hjørnepunkter og basale løsninger}
\begin{defn}{}{algh}
Hvis en vektor $\textbf{x}$ opfylder $\textbf{a}^T_i\textbf{x}= b_i$ for $i \in M_1, M_2 \text{ eller } M_3$, så siges den tilsvarende betingelse at være \textbf{bindende} ved $\textbf{x}$.
\end{defn}\noindent
%
Følgende sætning giver anledning til hvordan disse bindende betingelser kan føres i relation til løsninger af lineære optimeringsproblemer.
%
\begin{thm}{}{}

Lad $\textbf{x}$ være et element i $\R^n$ og lad $I=\{i|\textbf{a}^T_i\textbf{x}=b_i\}$ være en mængde af indexer på betingelser, der er bindende ved $\textbf{x}$.
Så er følgende udsagn ækvivalente.
%
\begin{enumerate}[label=(\alph*)]
\item Der eksisterer $n$ vektorer i mængden $\{\textbf{a}_i|i \in I \}$, som er lineært uafhængige.
\item Spannet af vektorerne $\textbf{a}_i$, for $i \in I$, dækker hele $\R^n$.
\item Ligningssystemet $\textbf{a}^T_i\textbf{x}= b_i$, for $i \in I$, har en entydig løsning.
\end{enumerate}
\end{thm}
%
\begin{proof}
Smukt bevis INC.
\end{proof}
\noindent