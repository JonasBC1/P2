\section{Løsninger til lineære programmeringsproblemer}
%
% vil bare lige sige, at titlen før var "Løsninger til lineære løsningsproblemer", lol
%
\label{heeeeejjulle}
%
% FUCK EN PÆN LABEL XD
% I forhold til grafisk at løse lignende lineære programmeringsproblemer overstående, så vil følgende eksempel illustrere hvordan. 
%
I følgende afsnit uddybes, hvordan et lineært programmeringsproblem kan løses grafisk med udgangspunkt i \ref{eks:min_loes}.
\\\\
\begin{eks}
\label{eks:min_loes}
Betragt maksimumsproblemet
%
\begin{align*}
\begin{array}{lrrlr}
\text{Maksimér}		&	\multicolumn{2}{c}{x_1+x_2} &= &z\\
\text{begrænset af}	&x_1& + 2x_2			&\leq 	& 6,\\
					&2x_1& + x_2			&\leq	& 6,\\
					&x_1&    				&\geq	& 0,\\
					&x_2&    				&\geq	& 0.\\
\end{array}
\end{align*}
%
På figur \ref{fig:julieerikkedum} kan løsningsmængden, markeret med blå, for maksimumsproblemet ses. 
%
\begin{figure}[H]
\centering
%
\begin{tikzpicture}
%
%
% -------------------------------------------------------
\begin{axis}[
	 axis x line=center,
  axis y line=center,
  xtick={-5,-4,...,5},
  ytick={-5,-4,...,5},
  xlabel={$x_1$},
  ylabel={$x_2$},
  xlabel style={below right},
  ylabel style={above left},
  xmin=-1,
  xmax=4,
  ymin=-1,
  ymax=4]
	legend style={at={(0.45,0.025)},anchor=south west},
]
\addplot[name path=c, domain=0:3,samples=2,color=black, style=very thin]{0};
%%%
%%%
\addplot[name path=b, domain={0:6}, color=myblue, style=very thick,samples=2]{6-2*x};
\addlegendentry{$2x_1+x_2 \leq 6$}
\addplot[name path=a, domain={0:6}, color=myred, style=very thick, samples=2]{3-0.5*x};
\addlegendentry{$x_1+2x_2 \leq 6$}




\addplot[fill=myblue!15,opacity=0.4] fill between [of=a and c, soft clip={domain=0:2}];
\addplot[fill=myblue!15,opacity=0.4] fill between [of=b and c, soft clip={domain=2:3}];
 

%

%	
\addplot[domain=-1:6, samples=2, color=black, dotted, thick]{-x};
\addplot[domain=-1:6, samples=2, color=black, dotted, thick]{2-x};
\addplot[domain=0:6, samples=2, color=black, dotted, thick]{4-x};
\addplot[mark=*] coordinates {(2,2)};
\end{axis}
\draw[->] (1.4,1.2) -- (2.75,2.35);
\end{tikzpicture}
\caption{Grafisk illustration med én entydig løsning til problemet i \ref{eks:min_loes}.}
%
\label{fig:opti_loes}
\end{figure}
%
\end{eks}
%
%
% For at finde den optimale løsning findes enhver given skalar $k$ og løsningsmængden med alle punkter overvejes, hvor objektfunktionen $\textbf{c}^T\textbf{x}$ er lig med $k$. 
%
Ethvert punkt i løsningsmængden, hvor objektfunktionen $\textbf{c}^T\textbf{x}$ er lig med $k$, undersøges for at finde den optimale løsning.
Dette illustreres ved en linje, der er beskrevet af ligningen $x_1+x_2=k$ for maksimumsproblemet i \ref{eks:min_loes}. 
Dette betyder, at den optimale løsning, som enten maksimerer eller minimerer omkostningen kan findes grafisk ved at indtegne niveaukurver $f(x_1, x_2)=k$, og forskyde disse niveaukurver i den retning, der optimerer omkostningen, indtil den optimale løsning opnås.  
Med udgangspunkt i \ref{eks:min_loes}, er denne linje vinkelret på vektoren 
%
$$
\textbf{x}=
\begin{bmatrix}
1\\1
\end{bmatrix}.
$$
%
Forskellige værdier for $k$ fører til forskellige linjer, hvor alle linjer er parallelle med hinanden samt vinkelret på vektoren $\textbf{v}$. 
Hvis værdien af $k$ øges, svarer det til at bevæge niveaukurven $x_1+x_2=k$ i retning af $\textbf{v}$.
Eftersom \ref{eks:min_loes} er et maksimumsproblem, bevæges linjen i retningen af $\textbf{v}$ uden at forlade løsningsmængden. 
Den optimale niveaukurve er derfor $k = 4$, hvormed vektor
%
$$
\textbf{x}=
\begin{bmatrix}
2\\2
\end{bmatrix}
$$
%
er den optimale løsning i løsningsmængden, og dermed den entydige optimale løsning. 
På figur \ref{fig:opti_loes} er enkelte niveaukurver, vektoren $\mathbf{v}$, og den optimale løsning markeret.
%
\begin{figure}[H]
\centering
%
\begin{tikzpicture}
%
%
% -------------------------------------------------------
\begin{axis}[
axis lines = left,
	xlabel = $x_1$,
	ylabel = \rotatebox{-90}{$x_2$},
  xmax=4,
  ymax=4,
	legend style={at={(1,1)},anchor=north east},
]
%%%
%%%
\addplot[name path=b, domain={0:3}, color=myblue, style=very thick,samples=2]{6-2*x};
\addlegendentry{$2x_1+x_2 \leq 6$}
\addplot[name path=a, domain={0:6}, color=myred, style=very thick, samples=2]{3-0.5*x};
\addlegendentry{$x_1+2x_2 \leq 6$}
%
\addplot[name path=c, domain=0:3,samples=2,color=black, style=very thin]{0};
%
\addplot[fill=myblue!15] fill between [of=a and c, soft clip={domain=0:2}];
\addplot[fill=myblue!15] fill between [of=b and c, soft clip={domain=2:3}];
%
%
%	
\addplot[domain=0:6, samples=2, color=black, dotted, thick]{-x};
\addplot[domain=0:6, samples=2, color=black, dotted, thick]{2-x};
\addplot[domain=0:6, samples=2, color=black, dotted, thick]{4-x};
\addplot[mark=*] coordinates {(2,2)};
\addplot[mark=none] coordinates {(0,0) (0,3)};
\addplot[mark=none, thick, ->] coordinates {(0,0) (1,1)} node[above right] {$\mathbf{v}$};
\end{axis}
\end{tikzpicture}
\caption{Løsningsmængden til \ref{eks:min_loes}, markeret med blå, hvor $\mathbf{v}$, og enkelte niveaukurver er market, samt den entydige løsning til problemet.}
%
\label{fig:opti_loes}
\end{figure}
\noindent
%
Betragt figur \ref{fig:opti_loes}. 
I det hjørne, hvor den optimale løsning findes, maksimeres omkostningen, der repræsenterer præcis én løsning til maksimumsproblemet. 
Antag nu, at den ene koefficient i objektfunktion er rykket, eller at den ene begrænsning er skiftet, så den ligger parallelt med niveaukurven, og vinkelret på vektoren $\textbf{v}$.
Dette medfører, at der findes uendeligt mange optimale løsninger til maksimumsproblemet, eftersom den optimale niveaukurve skærer løsningsmængden i uendeligt mange punkter.
På figur \ref{fig:opti_loes2} introduceres et eksempel, hvor der findes et uendeligt antal løsninger, der
maksimerer omkostningen. 
%
\newline
\phantom{Hej}
\\\\
\phantom{Hej}
\\\\
\phantom{Hej}
\\\\
\phantom{Hej}
\\\\
%
\begin{figure}[H]
\centering
%
\begin{tikzpicture}
%
%
% -------------------------------------------------------
\begin{axis}[
	 axis x line=center,
  axis y line=center,
  xtick={-5,-4,...,5},
  ytick={-5,-4,...,5},
  xlabel={$x_1$},
  ylabel={$x_2$},
  xlabel style={below right},
  ylabel style={above left},
  xmin=-1,
  xmax=4,
  ymin=-1,
  ymax=4]
	legend style={at={(0.45,0.025)},anchor=south west},
]
\addplot[name path=c, domain=0:3,samples=2,color=black, style=very thin]{0};
%%%
%%%
\addplot[name path=b, domain={0:6}, color=myblue, style=very thick,samples=2]{4-x};
\addplot[name path=a, domain={0:6}, color=myred, style=very thick, samples=2]{3-0.5*x};

\addplot[fill=myblue!15,opacity=0.4] fill between [of=a and c, soft clip={domain=0:2}];
\addplot[fill=myblue!15,opacity=0.4] fill between [of=b and c, soft clip={domain=2:4}];
 
%

%	
\addplot[domain=-1:6, samples=2, color=black, dotted, thick]{-x};
\addplot[domain=-1:6, samples=2, color=black, dotted, thick]{2-x};
\addplot[domain=0:6, samples=2, color=black, dotted, ultra thick]{4-x};
\end{axis}
\draw[->] (1.4,1.2) -- (2.75,2.35);
\end{tikzpicture}
\caption{Grafisk illustration med uendeligt antal løsninger til problemet i \ref{eks:min_loes}.}
%
\label{fig:opti_loes2}
\end{figure}
%
\noindent
%
Betragt figur \ref{fig:opti_loes3}.
Dette scenarie har ikke nogle løsninger, som opfylder betingelserne, og dermed er der ingen optimal løsning til det lineære programmeringsproblem. 
%
\begin{figure}[H]
\centering
%
\begin{tikzpicture}
%
%
% -------------------------------------------------------
\begin{axis}[
	 axis x line=center,
  axis y line=center,
  xtick={-5,-4,...,5},
  ytick={-5,-4,...,5},
  xlabel={$x_1$},
  ylabel={$x_2$},
  xlabel style={below right},
  ylabel style={above left},
  xmin=-1,
  xmax=4,
  ymin=-1,
  ymax=4]
	legend style={at={(0.45,0.025)},anchor=south west},
]
\addplot[name path=d, domain=0:4,samples=2,color=white, style=ultra thin]{4};
\addplot[name path=c, domain=0:4,samples=2,color=black, style=ultra thin]{0};
%%%
%%%
\addplot[name path=b, domain={0:3}, color=myblue, style=very thick,samples=2]{3-1*x};
\addplot[name path=a, domain={0:2}, color=myred, style=very thick, samples=2]{2-1*x};


\addplot[fill=myblue!15,opacity=0.4] fill between [of=a and c, soft clip={domain=0:2}];
\addplot[fill=myblue!15,opacity=0.4] fill between [of=b and d, soft clip={domain=0:4}];
 

%

%	
\addplot[domain=-1:6, samples=2, color=black, dotted, thick]{-x};
\addplot[domain=-1:6, samples=2, color=black, dotted, thick]{2-x};
\addplot[domain=0:6, samples=2, color=black, dotted, thick]{4-x};
\end{axis}
\draw[->] (1.4,1.2) -- (2.75,2.35);
\end{tikzpicture}
\caption{Grafisk illustration, hvor løsningsmængden er tom til problemet i \ref{eks:min_loes3}.}
%
\label{fig:opti_loes3}
\end{figure}
%
\noindent
%
Antag nu, at det lineære programmeringsproblem i stedet er ubegrænset opadtil.
Dette scenarie er illustreret på figur \ref{fig:opti_loes4}.
%
Dette betyder, at betingelserne ikke begrænser, hvor stor
omkostningen kan blive. 
Derfor er den optimale løsning $\infty$.
%
\phantom{Hej}
\\\\
\phantom{Hej}
\\\\
\phantom{Hej}
\\\\
%
\begin{figure}[H]
\centering
%
\begin{tikzpicture}
%
%
% -------------------------------------------------------
\begin{axis}[
	 axis x line=center,
  axis y line=center,
  xtick={-5,-4,...,5},
  ytick={-5,-4,...,5},
  xlabel={$x_1$},
  ylabel={$x_2$},
  xlabel style={below right},
  ylabel style={above left},
  xmin=-1,
  xmax=4,
  ymin=-1,
  ymax=4]
	legend style={at={(0.45,0.025)},anchor=south west},
]
%
%
%
%
%
\addplot[name path=b, domain={0:3}, color=myblue, style=very thick,samples=2]{6-2*x};
\addlegendentry{$2x_1+x_2 \geq 6$}
\addplot[name path=a, domain={0:6}, color=myred, style=very thick, samples=2]{3-0.5*x};
\addlegendentry{$x_1+2x_2 \geq 6$}
%%%
%%%
\addplot[name path=d, domain=0:6,samples=2,color=white, style=ultra thin]{6};
\addplot[name path=c, domain=0:3,samples=2,color=black, style=very thin]{0};



\addplot[fill=myblue!15] fill between [of=d and b, soft clip={domain=0:2}];
\addplot[fill=myblue!15] fill between [of=d and a, soft clip={domain=2:4}];
 

%

%	
\addplot[domain=-1:6, samples=2, color=black, dotted, thick]{-x};
\addplot[domain=-1:6, samples=2, color=black, dotted, thick]{2-x};
\addplot[domain=0:6, samples=2, color=black, dotted, thick]{4-x};
\addplot[mark=*] coordinates {(2,2)};
\end{axis}
\draw[thick,->] (1.4,1.2) -- (2.75,2.35) node[above right] {$\mathbf{v}$};
\end{tikzpicture}
\caption{En begrænset løsningsmængde, markeret med blå, afgrænset af lineære betingelser, hvor den optimale løsning er $\infty$ til problemet.}
%
\label{fig:opti_loes4}
\end{figure}
%
\noindent
%
Som gennemgået i de forudgående eksempler, så findes følgende fire løsningsmuligheder, der er betinget af det lineære optimeringsproblems form.
%
\begin{itemize}
\item Der findes en entydig optimal løsning.
\item Der findes flere optimale løsninger i løsningsmængden.
\item Løsningmængden for det lineære optimeringsproblem er tom. 
\item Den optimale løsning er enten $\infty$ eller $-\infty$.
\end{itemize}
%