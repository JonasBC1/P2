\section{Mathias}
% God arbejdslyst - tak julle :))

I forhold til grafisk at løse lignende lineære programmeringsproblemer forhenstående, så vil følgende eksempel illustrere hvordan. 
%Mit forslag til ovenstående: Eksempel \ref{eks:min_loes} illustrerer, hvordan lineære programmeringsproblemer løses i nærværende rapport.

\begin{eks}
\label{eks:min_loes}
Betragt minimumsproblemet
%
\begin{align*}
\begin{array}{lrrlr}
\text{Minimér}		&	\multicolumn{2}{c}{x_1-x_2}  &\\
\text{begrænset af}	&x_1& + 2x_2			&\leq 	& 6,\\
					&2x_1& + x_2			&\leq	& 6,\\
					&x_1&    				&\geq	& 0,\\
					&x_2&    				&\geq	& 0.\\
\end{array}
\end{align*}
%
På figur \ref{eks:min_loes} kan løsningsområdet - markeret med gråblå - for minimumsproblemet ses.
%
\begin{figure}[H]
\centering
%
\begin{tikzpicture}
%
%
% -------------------------------------------------------
\begin{axis}[
	 axis x line=center,
  axis y line=center,
  xtick={-5,-4,...,5},
  ytick={-5,-4,...,5},
  xlabel={$x_1$},
  ylabel={$x_2$},
  xlabel style={below right},
  ylabel style={above left},
  xmin=-1,
  xmax=4,
  ymin=-1,
  ymax=4]
	legend style={at={(0.45,0.025)},anchor=south west},
]
%
%
\addplot[name path=b, domain={0:6}, color=myblue, style=very thick,samples=2]{6-2*x};
	\addlegendentry{$2x_1+x_2 \leq 6$}
	\addplot[name path=a, domain={0:6}, color=myred, style=very thick, samples=2]{3-0.5*x};
	\addlegendentry{$x_1+2x_2 \leq 6$}
%
%
	\addplot[name path=c, domain=0:6, samples=2, color=myred, style=very thick]{((6-2*x};
	\addplot[name path=d, domain=0:6,samples=2,color=myblue, style=very thick]{3-0.5*x};
%
	
	%\tikzfillbetween [of=d and c]{myblue!15}
	%\tikzfillbetween [of=c and a]{myblue!15}
%	
\addplot[domain=-1:6, samples=2, color=black, dotted, thick]{-x};
\addplot[domain=-1:6, samples=2, color=black, dotted, thick]{2-x};
\addplot[domain=0:6, samples=2, color=black, dotted, thick]{4-x};
\addplot[mark=*] coordinates {(2,2)};
\end{axis}
\end{tikzpicture}
\caption{Grafisk løsning til problemet i \ref{eks:min_loes}.}
%
\label{fig:min_loes}
\end{figure}
%
\end{eks}
\noindent
Løsningsmængden er det skraverede område på figur \ref{fig:min_loes}. For at finde den optimale løsning findes enhver given skalar $k$ og løsningsmængden med alle punkter overvejes, hvor objektfunktionen $\textbf{c}^T\textbf{x}$ er lig med $k$. Dette illustreres ved den linje, der er beskrevet af ligningen $-x_l-x_2=k$. Dette betyder, at den optimale løsning som enten maksimerer eller minimerer værdien
af objektfunktionen kan findes grafisk ved at indtegne niveaukurver $f(x1, x2)=k$ og forskyde disse niveaukurver i den retning, der optimerer værdien af objektfunktionen, indtil den optimale løsning opnås.  Bemærk, i det følgende eksempel, at denne linje er vinkelret på vektoren $\textbf{v}$ med koordinaterne $\textbf{v}=(-1,-1)$.
Forskellige værdier for $k$ fører til forskellige linjer, hvor  alle linjer er parallelle med hinanden samt vinkelret med vektoren $\textbf{v}$. Hvis værdien af $k$ øges, svarer det til at bevæge linjen $z=x_l-x_2$ langs retningen af vektoren $\textbf{v}$. For at finde den optimale løsning for dette eksempel, er det interessante at minimere $k$, og dermed bevæge linjen så meget som muligt i retning af $-\textbf{v}$ uden at forlade løsningsmængden. Det mest optimale er derfor $k = -4$, hvormed vektor $\textbf{x}=(2,2)$ er den mest optimale løsning i løsningsmængden. 