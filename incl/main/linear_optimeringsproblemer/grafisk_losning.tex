\section{Løsninger til lineære programmeringsproblemer}
%
% vil bare lige sige, at titlen før var "Løsninger til lineære løsningsproblemer", lol
%
\label{heeeeejjulle}
%
% FUCK EN PÆN LABEL XD
% I forhold til grafisk at løse lignende lineære programmeringsproblemer overstående, så vil følgende eksempel illustrere hvordan. 
%
I følgende afsnit uddybes, hvordan et lineæt programmeringsproblem kan løses grafisk, samt introducerende teori, med udgangspunkt i \ref{eks:min_loes}.
\\\\
\begin{eks}
\label{eks:min_loes}
Betragt maksimumsproblemet
%
\begin{align*}
\begin{array}{lrrlr}
\text{Maksimér}		&	\multicolumn{2}{c}{x_1+x_2}  &\\
\text{begrænset af}	&x_1& + 2x_2			&\leq 	& 6,\\
					&2x_1& + x_2			&\leq	& 6,\\
					&x_1&    				&\geq	& 0,\\
					&x_2&    				&\geq	& 0.\\
\end{array}
\end{align*}
%
På figur \ref{fig:julieerikkedum} kan løsningsmængden, markeret med blå, for maksimumsproblemet ses. 
%
\begin{figure}[H]
\centering
%
\begin{tikzpicture}
%
%
% -------------------------------------------------------
\begin{axis}[
	 axis x line=center,
  axis y line=center,
  xtick={-5,-4,...,5},
  ytick={-5,-4,...,5},
  xlabel={$x_1$},
  ylabel={$x_2$},
  xlabel style={below right},
  ylabel style={above left},
  xmin=-1,
  xmax=4,
  ymin=-1,
  ymax=4]
	legend style={at={(0.45,0.025)},anchor=south west},
]
\addplot[name path=c, domain=0:3,samples=2,color=black, style=very thin]{0};
%%%
%%%
\addplot[name path=b, domain={0:6}, color=myblue, style=very thick,samples=2]{6-2*x};
\addlegendentry{$2x_1+x_2 \leq 6$}
\addplot[name path=a, domain={0:6}, color=myred, style=very thick, samples=2]{3-0.5*x};
\addlegendentry{$x_1+2x_2 \leq 6$}




\addplot[fill=myblue!15,opacity=0.4] fill between [of=a and c, soft clip={domain=0:2}];
\addplot[fill=myblue!15,opacity=0.4] fill between [of=b and c, soft clip={domain=2:3}];
 

%

%	
\addplot[domain=-1:6, samples=2, color=black, dotted, thick]{-x};
\addplot[domain=-1:6, samples=2, color=black, dotted, thick]{2-x};
\addplot[domain=0:6, samples=2, color=black, dotted, thick]{4-x};
\addplot[mark=*] coordinates {(2,2)};
\end{axis}
\draw[->] (1.4,1.2) -- (2.75,2.35);
\end{tikzpicture}
\caption{Grafisk illustration med én entydig løsning til problemet i \ref{eks:min_loes}.}
%
\label{fig:opti_loes}
\end{figure}
%
\end{eks}
%
%
% For at finde den optimale løsning findes enhver given skalar $k$ og løsningsmængden med alle punkter overvejes, hvor objektfunktionen $\textbf{c}^T\textbf{x}$ er lig med $k$. 
%
Ethvert punkt i en løsningsmængden, hvor objektfunktionen $\textbf{c}^T\textbf{x}$ er lig med $k$, undersøges for at finde den optimale løsning.
Dette illustreres ved en linje, der er beskrevet af ligningen $-x_l-x_2=k$ for maksimumsproblemet i \ref{eks:min_loes}. 
Dette betyder, at den optimale løsning, som enten maksimerer eller minimerer værdien af objektfunktionen kan findes grafisk ved at indtegne niveaukurver $f(x_1, x_2)=k$, og forskyde disse niveaukurver i den retning, der optimerer værdien af objektfunktionen, indtil den optimale løsning opnås.  
Med udgangspunkt i \ref{eks:min_loes}, er denne linje vinkelret på vektoren $\textbf{v}$ med koordinaterne $\textbf{v}=(1,1)$.
Forskellige værdier for $k$ fører til forskellige linjer, hvor alle linjer er parallelle med hinanden samt vinkelret med vektoren $\textbf{v}$. 
Hvis værdien af $k$ øges, svarer det til at bevæge linjen $-x_l-x_2=k$ langs retningen af $\textbf{v}$. 
Eftersom \ref{eks:min_loes} er et maksimumsproblem, bevæges linjen i retningen af $\textbf{v}$ uden at forlade løsningsmængden. 
Den optimale niveaukurve er derfor $k = -4$, hvormed vektor $\textbf{x}=(2,2)$ er den optimale løsning i løsningsmængden, og dermed en entydig optimale løsning. 
På figur \ref{fig:opti_loes} ses enkelte niveaukurverne markert, vektoren $\mathbf{v}$, og den optimale løsning.
%
\begin{figure}[H]
\centering
%
\begin{tikzpicture}
%
%
% -------------------------------------------------------
\begin{axis}[
	 axis x line=center,
  axis y line=center,
  xtick={-5,-4,...,5},
  ytick={-5,-4,...,5},
  xlabel={$x_1$},
  ylabel={$x_2$},
  xlabel style={right},
  ylabel style={above},
  xmin=-1,
  xmax=4,
  ymin=-1,
  ymax=4]
	legend style={at={(0.45,0.025)},anchor=south west},
]
%%%
%%%
\addplot[name path=b, domain={0:3}, color=myblue, style=very thick,samples=2]{6-2*x};
\addlegendentry{$2x_1+x_2 \leq 6$}
\addplot[name path=a, domain={0:6}, color=myred, style=very thick, samples=2]{3-0.5*x};
\addlegendentry{$x_1+2x_2 \leq 6$}
%
\addplot[name path=c, domain=0:3,samples=2,color=black, style=very thin]{0};
%
\addplot[fill=myblue!15] fill between [of=a and c, soft clip={domain=0:2}];
\addplot[fill=myblue!15] fill between [of=b and c, soft clip={domain=2:3}];
%
%
%	
\addplot[domain=-1:6, samples=2, color=black, dotted, thick]{-x};
\addplot[domain=-1:6, samples=2, color=black, dotted, thick]{2-x};
\addplot[domain=0:6, samples=2, color=black, dotted, thick]{4-x};
\addplot[mark=*] coordinates {(2,2)};
\end{axis}
\draw[thick,->] (1.4,1.2) -- (2.75,2.35) node[above right] {$\mathbf{v}$};
\end{tikzpicture}
\caption{Den begrænset løsningsmængde, markeret med blå til \ref{eks:min_loes}, hvor $\mathbf{v}$, og enkelte niveaukurver er market, samt den entydige løsning til problemet.}
%
\label{fig:opti_loes}
\end{figure}
\noindent
%
Betragt figur \ref{fig:opti_loes}. 
I det hjørne, hvor den optimale løsning findes, maksimeres værdien af objektfunktionen, der repræsenterer præcis én løsning til maksimumsproblemet. 
Antag nu, at den ene koefficient i objektfunktion er rykket, eller at den ene begrænsning er skiftet, så den ligger parallelt med niveaukurven, som ligger vinkelret med vektoren $\textbf{v}$. Dette medfører at der findes uendeligt mange optimale løsninger til maksimumsproblemet, eftersom der er mange vektorer, som opfylder betingelserne ved niveaukurven.
På figur \ref{fig:opti_loes2} introduceres sådan et eksempel, hvor der matematisk set findes et uendeligt antal løsninger, der
maksimerer værdien af objektfunktionen. 
%
\begin{figure}[H]
\centering
%
\begin{tikzpicture}
%
%
% -------------------------------------------------------
\begin{axis}[
	 axis x line=center,
  axis y line=center,
  xtick={-5,-4,...,5},
  ytick={-5,-4,...,5},
  xlabel={$x_1$},
  ylabel={$x_2$},
  xlabel style={below right},
  ylabel style={above left},
  xmin=-1,
  xmax=4,
  ymin=-1,
  ymax=4]
	legend style={at={(0.45,0.025)},anchor=south west},
]
\addplot[name path=c, domain=0:3,samples=2,color=black, style=very thin]{0};
%%%
%%%
\addplot[name path=b, domain={0:6}, color=myblue, style=very thick,samples=2]{4-x};
\addplot[name path=a, domain={0:6}, color=myred, style=very thick, samples=2]{3-0.5*x};

\addplot[fill=myblue!15,opacity=0.4] fill between [of=a and c, soft clip={domain=0:2}];
\addplot[fill=myblue!15,opacity=0.4] fill between [of=b and c, soft clip={domain=2:4}];
 
%

%	
\addplot[domain=-1:6, samples=2, color=black, dotted, thick]{-x};
\addplot[domain=-1:6, samples=2, color=black, dotted, thick]{2-x};
\addplot[domain=0:6, samples=2, color=black, dotted, ultra thick]{4-x};
\end{axis}
\draw[->] (1.4,1.2) -- (2.75,2.35);
\end{tikzpicture}
\caption{Grafisk illustration med uendeligt antal løsninger til problemet i \ref{eks:min_loes}.}
%
\label{fig:opti_loes2}
\end{figure}
%
\noindent
%
Betragt figur \ref{fig:opti_loes3}.
Dette senarie, har ikke nogle løsninger, som opfylder betingelserne, og dermed er der ingen optimal løsning til det lineære programmeringsproblem. 
%
\begin{figure}[H]
\centering
%
\begin{tikzpicture}
%
%
% -------------------------------------------------------
\begin{axis}[
	 axis x line=center,
  axis y line=center,
  xtick={-5,-4,...,5},
  ytick={-5,-4,...,5},
  xlabel={$x_1$},
  ylabel={$x_2$},
  xlabel style={below right},
  ylabel style={above left},
  xmin=-1,
  xmax=4,
  ymin=-1,
  ymax=4]
	legend style={at={(0.45,0.025)},anchor=south west},
]
\addplot[name path=d, domain=0:4,samples=2,color=white, style=ultra thin]{4};
\addplot[name path=c, domain=0:4,samples=2,color=black, style=ultra thin]{0};
%%%
%%%
\addplot[name path=b, domain={0:3}, color=myblue, style=very thick,samples=2]{3-1*x};
\addplot[name path=a, domain={0:2}, color=myred, style=very thick, samples=2]{2-1*x};


\addplot[fill=myblue!15,opacity=0.4] fill between [of=a and c, soft clip={domain=0:2}];
\addplot[fill=myblue!15,opacity=0.4] fill between [of=b and d, soft clip={domain=0:4}];
 

%

%	
\addplot[domain=-1:6, samples=2, color=black, dotted, thick]{-x};
\addplot[domain=-1:6, samples=2, color=black, dotted, thick]{2-x};
\addplot[domain=0:6, samples=2, color=black, dotted, thick]{4-x};
\end{axis}
\draw[->] (1.4,1.2) -- (2.75,2.35);
\end{tikzpicture}
\caption{Grafisk illustration, hvor løsningsmængden er tom til problemet i \ref{eks:min_loes3}.}
%
\label{fig:opti_loes3}
\end{figure}
%
\noindent
%
Antag nu, at det lineære programmeringsproblem i stedet er ubegrænset opadtil.
Dette senarie er illustreret på figur \ref{fig:opti_loes4}.
%
Dette betyder, at betingelserne ikke begrænser, hvor stor
værdien af objektfunktionen kan blive. 
Derfor er den optimale maksimale løsning $\infty$, hvilket betyder, at ingen optimal løsning ligger i løsningsområdet, som var givet af de to betingelser. 
% Jeg vil gerne lige have forklaret det sidste tak xD 
%
\begin{figure}[H]
\centering
%
\begin{tikzpicture}
%
%
% -------------------------------------------------------
\begin{axis}[
	 axis x line=center,
  axis y line=center,
  xtick={-5,-4,...,5},
  ytick={-5,-4,...,5},
  xlabel={$x_1$},
  ylabel={$x_2$},
  xlabel style={below right},
  ylabel style={above left},
  xmin=-1,
  xmax=4,
  ymin=-1,
  ymax=4]
	legend style={at={(0.45,0.025)},anchor=south west},
]
%
%
%
%
%
\addplot[name path=b, domain={0:3}, color=myblue, style=very thick,samples=2]{6-2*x};
\addlegendentry{$2x_1+x_2 \geq 6$}
\addplot[name path=a, domain={0:6}, color=myred, style=very thick, samples=2]{3-0.5*x};
\addlegendentry{$x_1+2x_2 \geq 6$}
%%%
%%%
\addplot[name path=d, domain=0:6,samples=2,color=white, style=ultra thin]{6};
\addplot[name path=c, domain=0:3,samples=2,color=black, style=very thin]{0};



\addplot[fill=myblue!15] fill between [of=d and b, soft clip={domain=0:2}];
\addplot[fill=myblue!15] fill between [of=d and a, soft clip={domain=2:4}];
 

%

%	
\addplot[domain=-1:6, samples=2, color=black, dotted, thick]{-x};
\addplot[domain=-1:6, samples=2, color=black, dotted, thick]{2-x};
\addplot[domain=0:6, samples=2, color=black, dotted, thick]{4-x};
\addplot[mark=*] coordinates {(2,2)};
\end{axis}
\draw[thick,->] (1.4,1.2) -- (2.75,2.35) node[above right] {$\mathbf{v}$};
\end{tikzpicture}
\caption{En begrænset løsningsmængde, markeret med blå, afgrænset af lineære betingelser, hvor den optimale løsning er $\infty$ til problemet.}
%
\label{fig:opti_loes4}
\end{figure}
%
\noindent
%
Som gennemgået i de forudgående eksempler, så findes følgende fire løsningsmuligheder, der er betinget af det lineære optimeringsproblems form.
%
\begin{enumerate}
\item Der findes én entydig optimal løsning.
\item Der findes flere optimale løsninger i løsningsmængden.
\item Løsningmængden for det lineære optimeringsproblem er tom. 
\item De optimale løsninger er $\infty$, og ingen af de optimale løsninger ligger i løsningsmængden.
\end{enumerate}
%