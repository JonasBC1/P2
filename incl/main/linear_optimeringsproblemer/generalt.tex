\chapter{Lineære optimeringsproblemer}
%
I lineær programmering er målet at optimere en lineær funktion afgrænset af \textit{lineære betingelser}.
Dette kapitel tager udgangspunkt i \citep[side 2-6, 21-24 og 139-146]{bert}, hvis ikke andet er angivet.
Funktionen $z$, som optimeres, kaldes \textit{objektfunktionen}.
Objektfunktionen kan enten maksimeres eller minimeres, og er begrænset af de lineære betingelser.
Findes henholdsvis et maksimum eller minimum inden for begrænsningerne, er dette problemets \textit{optimale værdi}.
\\\\
%
Et lineært maksimumsproblem kan opstilles som følgende:
%
\begin{align*}
\begin{array}{lrll}
\text{Maksimér}		&\textbf{c}^T\textbf{x}	& = z		&\\
\text{begrænset af}	&\textbf{a}^T\textbf{x}	&\geq b_i,	&i \in M_1,\\
					&\textbf{a}^T\textbf{x}	&\leq b_i,	&i \in M_2,\\
					&\textbf{a}^T\textbf{x}	& = b_i,	&i \in M_3,\\
					&x_j					&\geq 0,	&j \in N_1,\\
					&x_j					&\leq 0,	&j \in N_2,
\end{array}
\end{align*}
%
hvor $\textbf{c}^T\textbf{x}$ er objektfunktionen og $\textbf{a}^T\textbf{x}$ er de lineære betingelser, samt $M_1, M_2, M_3, N_1, N_2$ er endelige mængder med indekser.
$\textbf{c}$ er \textit{omkostningsvektoren} og $z$ er \textit{omkostningen}.
\textit{Løsningsmængden} betegner mængden af vektorer, som opfylder alle betingelserne.
Vektorer indeholdt i løsningsmængden kaldes for \textit{løsningsvektorer}.
Hvis værdiområdet er uendeligt stort, er optimeringsproblemet \textit{ubegrænset}; ellers er det \textit{begrænset}. 
\\\\
%
\begin{eks}
\label{eks:min_lin}
Betragt minimumsproblemet
%
\begin{align*}
\begin{array}{lrrlr}
\text{Minimér}		&	\multicolumn{2}{c}{4x_1-3x_2}  &=	&z\\
\text{begrænset af}	&2x_1& - x_2			&\leq 	&-10,\\
					&x_1& - x_2				&\geq	& 7,\\
					&x_1& + x_2				&\leq	& 20,\\
					&x_1&					&\geq	& 0,\\
					& &x_2					&\leq	& 12.
\end{array}
\end{align*}
%
På figur \ref{fig:min_beg} kan løsningsmængden, markeret med blå, for minimumsproblemet ses.
%
\begin{figure}[H]
\centering
\begin{tikzpicture}
\begin{axis}[
	axis lines = left,
	xlabel = $x_1$,
	ylabel = {$x_2$},
	legend style={at={(0.45,0.025)},anchor=south west},
]
	\addplot[name path=b, domain={0:1}, color=myblue, style=very thick,samples=2]{2*x+10};
	\addlegendentry{$2x_1-x_2 \leq -10$}
	\addplot[name path=a, domain={0:13.5}, color=mygreen, style=very thick, samples=2]{x-7};
	\addlegendentry{$x_1-x_2 \geq 7$}
	\addplot[domain=0:15,samples=2,color=myred, style=very thick]{-x+20};
	\addlegendentry{$x_1 + x_2 \leq 20$}
	\addplot[name path=c, domain={1:8}, color=black, style=very thick,samples=2]{12};
	\addlegendentry{$x_2 \leq 12$}
	
	\tikzfillbetween [of=a and b]{myblue!15}
	\tikzfillbetween [of=c and a]{myblue!15}
	
	\addplot[domain=0:15,samples=2,color=myred, style=very thick]{-x+20};
	\addplot[domain=0:5,samples=2,color=myblue, style=very thick]{2*x+10};
	\addplot[domain=0:15,samples=2,color=mygreen, style=very thick]{x-7};
	\addplot[domain=0:15,samples=2,color=black, style=very thick]{12};
	
	\addplot[domain=0:15, samples=2, color=black, dotted, thick]{0};

\end{axis}
\end{tikzpicture}
\caption{En begrænset løsningsmængde, markeret med gråblåt, afgrænset af lineære betingelser.}
\label{fig:min_beg}
\end{figure}

\noindent
%
Med udgangspunkt i figur \ref{fig:min_beg} kan det ses, at løsningsmængden er begrænset.
% 
Løsningsmængden for minimumsproblemet
%
\begin{align*}
\begin{array}{lrrlr}
\text{Minimer}		&\multicolumn{2}{c}{4x_1-3x_2}&=  &z\\
\text{begrænset af}	&2x_1& - x_2			&\leq 	&-10,\\
					&x_1& - x_2				&\geq	& 7,\\
					&x_1&					&\geq	& 0,
\end{array}
\end{align*}
er derimod ubegrænset, da $x_1$ og $x_2$ kan blive uendeligt store. 
Et udsnit af løsningsområdet for minimumsproblemet ses på figur \ref{fig:min_ubeg}, hvor det er markeret med blå.
%
\begin{figure}[H]
\centering
\begin{tikzpicture}
\begin{axis}[
	axis lines = left,
	xlabel = $x_1$,
	ylabel = {$x_2$},
	legend style={at={(0.45,0.025)},anchor=south west},
]
	\addplot[name path=b, domain={0:5}, color=myblue, style=very thick,samples=2]{2*x+10};
	\addlegendentry{$2x_1-x_2 \leq -10$}
	\addplot[name path=a, domain={0:15}, color=mygreen, style=very thick, samples=2]{x-7};
	\addlegendentry{$x_1-x_2 \geq 7$}
	\addplot[name path=c, domain={5:15}, color=white, style=very thick,samples=2]{20};
	
	\tikzfillbetween [of=a and b]{myblue!15}
	\tikzfillbetween [of=c and a]{myblue!15}
	
	\addplot[domain=0:5,samples=2,color=myblue, style=very thick]{2*x+10};
	\addplot[domain=0:15,samples=2,color=mygreen, style=very thick]{x-7};
	
	\addplot[domain=0:15, samples=2, color=black, dotted, thick]{0};

\end{axis}
\end{tikzpicture}
\caption{En ubegrænset løsningsmængde, markeret med blå, afgrænset af lineære betingelser.}
\label{fig:min_ubeg}
\end{figure}

%
\end{eks}