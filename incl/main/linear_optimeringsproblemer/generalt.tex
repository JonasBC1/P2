\chapter{Lineær optimeringsproblemer}
%
Lineære programmering er en problemstilling, hvor det ønskes at optimere et problem bestående af en lineære funktion og \textbf{lineære betingelser}.
Funktionen som optimeres for kaldes \textbf{objektfunktionen}.
Objektfunktionen kan enten optimeres for maksimum eller minimum og er begrænset af de lineære betingelser.
Findes der et maksimum eller minimum inde for begrænsningerne kaldes dette for den problemets optimale værdi.
\\\\
Et generelt lineært maksimeringsproblem kan opstilles som følgende:
%
\begin{align*}
\begin{array}{lrll}
\text{Maksimer}		&\textbf{c}^T\textbf{x}	&			&\\
\text{begrænset af}	&\textbf{a}^T\textbf{x}	&\geq b_i,	&i \in M_1,\\
					&\textbf{a}^T\textbf{x}	&\leq b_i,	&i \in M_2,\\
					&\textbf{a}^T\textbf{x}	& = b_i,	&i \in M_3,\\
					&x_j					&\geq 0,	&j \in N_1,\\
					&x_j					&\leq 0,	&j \in N_2,
\end{array}
\end{align*}
%
hvor $M_1, M_2, M_3, N_1, N_2$ er endelige mængder med indexer.
\textbf{Mulighedsområdet} betegner mængden af vektorer, som opfylder alle betingelserne.
Vektorer indeholdt i mulighedsområdet kaldes for \textbf{løsningsvektorer}.
Hvis mulighedsområdet er uendeligt stor kaldes optimeringsproblemet for \textbf{ubegrænset}.
Ellers kaldes det \textbf{begrænset}.\\\\
%
\begin{eks}
Betragt minimeringsproblem
%
\begin{align*}
\begin{array}{lrrlr}
\text{Minimer}		&	\multicolumn{2}{c}{4x_1-3x_2}  &\\
\text{begrænset af}	&2x_1& - x_2			&\leq 	&-10,\\
					&x_1& - x_2				&\geq	& 7,\\
					&x_1& + x_2				&\leq	& 20,\\
					&x_1&					&\geq	& 0,\\
					& &x_2					&\leq	& 12.
\end{array}
\end{align*}
%
På figur \ref{} kan mulighedsområdet, markeret med blå, for minimeringsproblemet ses.
%
\begin{figure}[H]
\centering
\begin{tikzpicture}
% 
%
\begin{axis}[
	axis lines = left,
	xlabel = $x_1$,
	ylabel = \rotatebox{-90}{$x_2$},
	legend style={at={(1,0.05)},anchor=south east},
]
	\addplot[name path=b, domain={0:1}, color=myblue, style=very thick,samples=2]{2*x+10};
	\addlegendentry{$2x_1-x_2 \leq -10$}
	\addplot[name path=a, domain={0:13.5}, color=mygreen, style=very thick, samples=2]{x-7};
	\addlegendentry{$\phantom{.}x_1-x_2 \geq \phantom{....}7$}
	\addplot[domain=0:15,samples=2,color=myred, style=very thick]{-x+20};
	\addlegendentry{$\phantom{.}x_1 + x_2 \leq \phantom{..}20$}
	\addplot[name path=c, domain={1:8}, color=black, style=very thick,samples=2]{12};
	\addlegendentry{$\phantom{.........}x_2 \leq \phantom{..}12$}
	
	\tikzfillbetween [of=a and b]{myblue!15}
	\tikzfillbetween [of=c and a]{myblue!15}
	
	\addplot[domain=0:15,samples=2,color=myred, style=very thick]{-x+20};
	\addplot[domain=0:5,samples=2,color=myblue, style=very thick]{2*x+10};
	\addplot[domain=0:15,samples=2,color=mygreen, style=very thick]{x-7};
	\addplot[domain=0:15,samples=2,color=black, style=very thick]{12};
	
	\addplot[domain=0:15, samples=2, color=black, dotted, thick]{0};
	\addplot[domain=0:15, samples=2, color=black, dotted, thick]{0};
	\addplot[name path=c, domain=0:0.00003,samples=2,color=black, style=very thin]{10000*x};
	\addplot[mark=none] coordinates {(0,-7) (0,10)};
	
%
\end{axis}
\end{tikzpicture}
\caption{En begrænset løsningsmængde, markeret med blå, afgrænset af lineære betingelser.}
\label{fig:min_beg}
\end{figure}
% 
Havde minimeringsproblemet istedet set ud som følgende: 

\begin{align*}
\begin{array}{lrrlr}
\text{Minimer}		&	\multicolumn{2}{c}{4x_1-3x_2}  &\\
\text{begrænset af}	&2x_1& - x_2			&\leq 	&-10,\\
					&x_1& - x_2				&\geq	& 7,\\
					&x_1&					&\geq	& 0,
\end{array}
\end{align*}
havde minimeringsproblemet været ubegrænset, da funktionsværdierne kunne blive uendelig store. 
Mulighedsområdet for minimeringsproblemet ses på figur \ref{}, hvor det er markeret med blå.
%
\begin{figure}[H]
\centering
\begin{tikzpicture}
% 
%
\begin{axis}[
	axis lines = left,
	xlabel = $x_1$,
	ylabel = \rotatebox{-90}{$x_2$},
	legend style={at={(1,0.05)},anchor=south east},
]
	\addplot[name path=b, domain={0:1}, color=myblue, style=very thick,samples=2]{2*x+10};
	\addlegendentry{$2x_1-x_2 \leq -10$}
	\addplot[name path=a, domain={0:13.5}, color=mygreen, style=very thick, samples=2]{x-7};
	\addlegendentry{$\phantom{.}x_1-x_2 \geq \phantom{....}7$}
	\addplot[domain=0:15,samples=2,color=myred, style=very thick]{-x+20};
	\addlegendentry{$\phantom{.}x_1 + x_2 \leq \phantom{..}20$}
	\addplot[name path=c, domain={1:8}, color=black, style=very thick,samples=2]{12};
	\addlegendentry{$\phantom{.........}x_2 \leq \phantom{..}12$}
	
	\tikzfillbetween [of=a and b]{myblue!15}
	\tikzfillbetween [of=c and a]{myblue!15}
	
	\addplot[domain=0:15,samples=2,color=myred, style=very thick]{-x+20};
	\addplot[domain=0:5,samples=2,color=myblue, style=very thick]{2*x+10};
	\addplot[domain=0:15,samples=2,color=mygreen, style=very thick]{x-7};
	\addplot[domain=0:15,samples=2,color=black, style=very thick]{12};
	
	\addplot[domain=0:15, samples=2, color=black, dotted, thick]{0};
	\addplot[domain=0:15, samples=2, color=black, dotted, thick]{0};
	\addplot[name path=c, domain=0:0.00003,samples=2,color=black, style=very thin]{10000*x};
	\addplot[mark=none] coordinates {(0,-7) (0,10)};
	
%
\end{axis}
\end{tikzpicture}
\caption{En begrænset løsningsmængde, markeret med blå, afgrænset af lineære betingelser.}
\label{fig:min_beg}
\end{figure} % Skal lige ændres til den rigtige 
%
\end{eks}