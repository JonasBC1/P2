\newpage
\section{Dualitet}
% Vi kan vælge at tage det med lagrange multiplier med også men tænker vi snakker om det jeg er personligt i tvivl og ville nødigt spilde tiden på det.
	% - Julie: Jeg synes ikke vi skal have det med 
% Jeg er iøvrigt bange for jeg er kommet til at skyde mig selv i foden, men umiddelbart giver det mening for mig, det snakker vi om.
	% - Julie: Jeg tror du har styr på det xD
%
Et vigtigt element af teorien vedrørende lineære optimeringsproblemer og simplexmetodens brug omhandler \textit{dualproblemer}.
%
\begin{defn}{}{dualproblemer}
Givet maksimumsproblemet
%
\begin{align*}
\begin{array}{lrll}
\text{Maksimér}		&\textbf{c}^T\textbf{x}	&=z			&\\
\text{begrænset af}	&\textbf{a}_i^T\textbf{x}	&\geq b_i,	&i \in M_1,\\
					&\textbf{a}_i^T\textbf{x}	&\leq b_i,	&i \in M_2,\\
					&\textbf{a}_i^T\textbf{x}	& = b_i,	&i \in M_3,\\
					&x_j					&\geq 0,	&j \in N_1,\\
					&x_j					&\leq 0,	&j \in N_2,\\							&x_j					&\text{fri},	&j \in N_3.
\end{array}
\end{align*}
%
Da vil \textbf{dual minimumsproblemet} være
% Minimer $v=\mathbf{b}^T \mathbf{p}$ begrænset af $A^T \mathbf{p} \geq c$ og $\mathbf{p} \geq 0$
%
\begin{align*}
\begin{array}{lrll}
\text{Minimér}		&\textbf{p}^T\textbf{b}	&=v			&\\
\text{begrænset af}	&p_i					&\geq 0,	&i \in M_1,\\
					&p_i					&\leq 0,	&i \in M_2,\\
					&p_i					&\text{fri},	&i \in M_3,\\
					&\textbf{p}^T\textbf{A}_j	&\leq c_j,	&j \in N_1,\\
					&\textbf{p}^T\textbf{A}_j	&\geq c_j,	&j \in N_2,\\
					&\textbf{p}^T\textbf{A}_j	& = c_j,	&j \in N_3.
\end{array}
\end{align*}
%
\end{defn}
\noindent
%
Problemet
%
\begin{align*}
\begin{array}{lrl}
\text{Maksimér}		&\textbf{c}^T\textbf{x}	& = z		\\
\text{begrænset af}	&A\textbf{x}	&\leq \mathbf{b},	\\
					&\mathbf{x}				&\geq \mathbf{0},
\end{array}
\end{align*}
%
vil således have
%
\begin{align*}
\begin{array}{lrl}
\text{Minimér}		&\textbf{p}^T\textbf{b}	& = v			\\
\text{begrænset af}	&\textbf{p}^TA	&\leq \mathbf{c},	\\
					&\mathbf{b}				&\leq \mathbf{0}
\end{array}
\end{align*}
%
som dualproblem.
%Ideer til en federe opsætning
\\\\
%
For hver begrænsning i det oprindelige problem indføres således en variabel i dualproblemet.
Tilsvarende indføres en begrænsning for hver variabel. 
Derudover skiftes fra et maksimums- til minimumsproblem.
I tabel \ref{juliedum} ses en oversigt over egenskaber ved primærproblemet og dettes dualproblem.\\
%
\begin{table}[H]
\begin{center}
\begin{tabular}{llll}
Primær  & Maksimér   & Minimér    & Dual         \\
\hline
Begræsninger & $\leq b_i$ & $\geq 0$   & Variable     \\
             & $\geq b_i$ & $\leq 0$   &              \\
             & $=b_i$     & fri        &            \\ 
\hline             
Variable     & $\geq 0$   & $\geq c_j$ & Begræsninger \\
             & $\leq 0$   & $\leq c_j$ &              \\
             & fri        & $=c_j$     &  
\end{tabular}
\end{center}
\captionof{table}{Oversigt over egenskaber ved primærproblemet og dettes dual.}
\label{juliedum}
%\captionof{tabular}{Fisk finder i på noget.}
\end{table}
\noindent
%
Dualproblemet kan således opstilles som i \ref{dual}. \\
%
\begin{eks}
\label{dual}
%
Betragt primærproblemet
%
\begin{align*}
\begin{array}{lrrlr}
\text{Maksimér}		&	\multicolumn{2}{c}{5x_1+4x_2}  & = & z\\
\text{begrænset af}	&3x_1& +5x_2			&\leq 	&78,\\
					&4x_1& + x_2				&\leq	& 36,\\
					&x_1,&  x_2				&\geq	& 0.
\end{array}
\end{align*}
\\
Dualproblemet vil således have formen
%hvilket bogstave foretrækker i istedet for z? hvis i vil ha z i standard notationen 
	% Julie: Tænker v? var det ikke det du skrev i den forrige?
\begin{align*}
\begin{array}{lrrlr}
\text{Minimér}		&	\multicolumn{2}{c}{78p_1+36p_2}  & =  & v\\
\text{begrænset af}	&3p_1& +4p_2			&\geq 	&5,\\
					&5p_1& + p_2				&\geq	& 4,\\
					&p_1,&  p_2				&\geq	& 0.
\end{array}
\end{align*}
% 
%I afsnit \ref{coronaaaaaaaaaaa} vil dette eksempel blive belyst. Dualproblemet vil istedet minimere indholdet af sukker og kulhydrater begrænset af forholdet mellem to typer slik i en slikpose.
%
\end{eks}
%
%Som det fremgår i eksemplet har løsningen på dualproblemet samme optimale løsning, hvor slackvariable her er løsningen på det oprindelige problem.
%
Det gælder endvidere, at der fra dualproblemet kan vendes tilbage til det oprindelige problem, hvilket udtrykkes ved \ref{thm:dualitetsaet}.
%
\begin{thm}{}{dualitetsaet}
Hvis dualproblemet omdannes til et ækvivalent minimumsproblem og dualproblemet af minimumsproblemet dannes, opnås primærproblemet.
\end{thm}
%
\newpage
%
\begin{proof}
Betragt problemet
\begin{align*}
\begin{array}{lrll}
\text{Minimér}		&\textbf{p}^T\textbf{b}		&= z		&\\
\text{begrænset af}	&p_i						&\geq 0,	&i \in M_1,\\
					&p_i						&\leq 0,	&i \in M_2,\\
					&p_i						&\text{fri},&i \in M_3,\\
					&\textbf{p}^T\textbf{A}_j	&\leq c_j,	&j \in N_1,\\
					&\textbf{p}^T\textbf{A}_j	&\geq c_j,	&j \in N_2,\\
					&\textbf{p}^T\textbf{A}_j	& = c_j,	&j \in N_3,
\end{array}
\end{align*}
og dets dualproblem
\begin{align*}
\begin{array}{lrll}
\text{Maksimér}		&\textbf{c}^T\textbf{x}		&= v		&\\
\text{begrænset af}	&\textbf{a}_i^T\textbf{x}	&\geq b_i,	&i \in M_1,\\
					&\textbf{a}_i^T\textbf{x}	&\leq b_i,	&i \in M_2,\\
					&\textbf{a}_i^T\textbf{x}	& = b_i,	&i \in M_3,\\
					&x_j						&\geq 0,	&j \in N_1,\\
					&x_j						&\leq 0,	&j \in N_2,\\
					&x_j						&\text{fri},&j \in N_3.
\end{array}
\end{align*}
Dualproblemet omdannes til det ækvivalente minimumsproblem
\begin{align*}
\begin{array}{lrll}
\text{Minimér}		&-\textbf{c}^T\textbf{x}	&=-v		&\\
\text{begrænset af}	&-\textbf{a}_i^T\textbf{x}	&\leq -b_i,	&i \in M_1,\\
					&-\textbf{a}_i^T\textbf{x}	&\geq -b_i,	&i \in M_2,\\
					&-\textbf{a}_i^T\textbf{x}	& = -b_i,	&i \in M_3,\\
					&x_j						&\geq 0,	&j \in N_1,\\
					&x_j						&\leq 0,	&j \in N_2,\\
					&x_j						&\text{fri},&j \in N_3.
\end{array}
\end{align*}
Dualproblemet til dette er
\begin{align*}
\begin{array}{lrll}
\text{Maksimér}		&-\textbf{p}^T\textbf{b}	&=-z		&\\
\text{begrænset af}	&p_i					&\geq 0,	&i \in M_1,\\
					&p_i					&\leq 0,	&i \in M_2,\\
					&p_i					&\text{fri},	&i \in M_3,\\
					&-\textbf{p}^T\textbf{A}_j	&\geq -c_j,	&j \in N_1,\\
					&-\textbf{p}^T\textbf{A}_j	&\leq -c_j,	&j \in N_2,\\
					&-\textbf{p}^T\textbf{A}_j	& = -c_j,	&j \in N_3,
\end{array}
\end{align*}
%
hvilket har det ækvivalente minimumsproblem
%
\begin{align*}
\begin{array}{lrll}
\text{Minimér}		&\textbf{p}^T\textbf{b}	&=z			&\\
\text{begrænset af}	&p_i					&\geq 0,	&i \in M_1,\\
					&p_i					&\leq 0,	&i \in M_2,\\
					&p_i					&\text{fri},	&i \in M_3,\\
					&\textbf{p}^T\textbf{A}_j	&\leq c_j,	&j \in N_1,\\
					&\textbf{p}^T\textbf{A}_j	&\geq c_j,	&j \in N_2,\\
					&\textbf{p}^T\textbf{A}_j	& = c_j,	&j \in N_3,
\end{array}
\end{align*}
%
som er identisk med primærproblemet. 
%
\end{proof}
%
%42 min https://youtu.be/0TRxEvMRE7s definition