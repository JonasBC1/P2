\section{Dualitet}
% Vi kan vælge at tage det med lagrange multiplier med også men tænker vi snakker om det jeg er personligt i tvivl og ville nødigt spilde tiden på det.
	% - Julie: Jeg synes ikke vi skal have det med 
% Jeg er iøvrigt bange for jeg er kommet til at skyde mig selv i foden, men umiddelbart giver det mening for mig, det snakker vi om.
	% - Julie: Jeg tror du har styr på det xD
%
Et vigtigt element af teorien vedrørende lineære optimerings problemer og Simplex metodens brug, omhandler \textit{dualproblemer}.
%
\begin{defn}{}{dualproblemer}
Givet maksimumsproblemet: 
%
\begin{align*}
\begin{array}{lrl}
\text{Maksimér}		&\textbf{c}^T\textbf{x}	& = z		\\
\text{begrænset af}	&A\textbf{x}	&\leq \mathbf{b},	\\
					&\mathbf{x}				&\geq \mathbf{0},
\end{array}
\end{align*}
%
Da vil \textbf{dual minimumsproblemet} være: \\
% Minimer $v=\mathbf{b}^T \mathbf{p}$ begrænset af $A^T \mathbf{p} \geq c$ og $\mathbf{p} \geq 0$
%
\begin{align*}
\begin{array}{lrl}
\text{Minimer}		&\textbf{b}^T\textbf{p}	& = v			\\
\text{begrænset af}	&A^T \textbf{p}	&\leq \mathbf{c},	\\
					&\mathbf{p}				&\geq \mathbf{0},
\end{array}
\end{align*}
%
\end{defn}
%
Problemet:
%
\begin{align*}
\begin{array}{lrll}
\text{Maksimér}		&\textbf{c}^T\textbf{x}	&			&\\
\text{begrænset af}	&\textbf{a}_i^T\textbf{x}	&\geq b_i,	&i \in M_1,\\
					&\textbf{a}_i^T\textbf{x}	&\leq b_i,	&i \in M_2,\\
					&\textbf{a}_i^T\textbf{x}	& = b_i,	&i \in M_3,\\
					&x_j					&\geq 0,	&j \in N_1,\\
					&x_j					&\leq 0,	&j \in N_2,\\							&x_j					&\text{fri},	&j \in N_3.
\end{array}
\end{align*}

Vil således have:


\begin{align*}
\begin{array}{lrll}
\text{minimer}		&\textbf{p}^T\textbf{b}	&			&\\
\text{begrænset af}	&p_i					&\geq 0,	&i \in M_1,\\
					&p_i					&\leq 0,	&i \in M_2,\\
					&p_i					&\text{fri},	&i \in M_3,\\
					&\textbf{p}^T\textbf{A}_j	&\leq c_j,	&j \in N_1,\\
					&\textbf{p}^T\textbf{A}_j	&\geq c_j,	&j \in N_2,\\
					&\textbf{p}^T\textbf{A}_j	& = b_i,	&i \in N_3,
\end{array}
\end{align*}
%
som sit dualproblem.
%Ideer til en federe opsætning
\\\\
%
Det sker således at der for hver begrænsning i det oprindelige problem indføres en variable i dualproblemet, tilsvarende indføres der en begrænsning for hver variable. 
Derudover skiftes som nævnt i definitionen fra et maksimerings til minimeringsproblem. 
Dette kan opsummeres i følgende tabel: \\
\begin{table}[H]
\begin{center}
\begin{tabular}{llll}
Oprindelige  & maksimer   & minimer    & Dual         \\
\hline
Variable     & $\geq 0$   & $\geq b_i$ & Begræsninger \\
             & $\leq 0$   & $\leq b_i$ &              \\
             & fri        & $=b_i$     &              \\
\hline
Begræsninger & $\leq c_j$ & $\geq 0$   & Variable     \\
             & $\geq c_j$ & $\leq 0$   &              \\
             & $=c_j$     & fri        &             
\end{tabular}
\end{center}
\captionof{table}{Oversigt over egenskaber ved primærproblemet og dettes dual.} 
%\captionof{tabular}{Fisk finder i på noget.}
\end{table}
\noindent
%
Dual problemet kan således opstilles som i \ref{dual} \\
%
\begin{eks}
\label{dual}
%
Betragt diæt problemet fra afsnittet om simplexmetoden:
%
\begin{align*}
\begin{array}{lrrlr}
\text{Maksimer}		&	\multicolumn{2}{c}{z=5x_1+4x_2}  &\\
\text{begrænset af}	&3x_1& +5x_2			&\leq 	&78,\\
					&4x_1& + x_2				&\leq	& 36,\\
					&x_1& , x_2				&\geq	& 0.
\end{array}
\end{align*}
\\
Dualproblemet vil således have denne form:
%hvilket bogstave foretrækker i istedet for z? hvis i vil ha z i standard notationen 
	% Julie: Tænker v? var det ikke det du skrev i den forrige?
\begin{align*}
\begin{array}{lrrlr}
\text{Minimer}		&	\multicolumn{2}{c}{z=78p_1+36p_2}  &\\
\text{begrænset af}	&3p_1& +4p_2			&\geq 	&5,\\
					&5p_1& + p_2				&\geq	& 4,\\
					&p_1& , p_2				&\geq	& 0.
\end{array}
\end{align*}
Hvor problemet her istedet er at minimere indholdet af sukker og kulhydrater begrænset af forholdet mellem to typer slik i en slikpose.
%?? Julie skal have lidt mere forklaring af dine begrænsninger for at komme i mål her.
%
%Løses dette ved hjælp af simplex-algoritmen opnås følgende tabel:
%\begin{align*}
%\begin{blockarray}{cccccc}
%p_1 & p_2 & \textcolor{blue}{s_1} & \textcolor{blue}{s_2} & z & b \\
%\begin{block}{[cc|ccc|c]}
%3 & 4 & 1 & 0 & 0 & 5 \\
%5 & 1 & 0 & 1 & 0 & 4 \\
%\cline{1-6}
%78 & 36 & 0 & 0 & 1 & 0\\
%\end{block}
%\end{blockarray}
%\end{align*}
%skal have lidt hjælp til at omskrive problemet til standardform simplex(det er bare fortegn jeg er i tvivl om, har et program der udføre algoritmen, så er good to go bagefter)
%
%her skal løst simplex smides ind
%
%Som det fremgår findes minimum i samme punkt:
%$z=78$ hvor $p_1$ og $p_2$ slackvariable $s_1=6$ og $s_2=12$
%
\end{eks}
%hvis nedenstående skal med skal simplex udføres på tabellen
%Som det fremgår i eksemplet har løsningen på dualproblemet samme optimale løsning, hvor slackvariable her er løsningen på det oprindelige problem.
Det gælder endvidere at der fra dualproblemet kan vendes tilbage til det oprindelige problem hvilket udtrykkes ved følgende sætning:
\begin{thm}{}{dualitetsaet}
Hvis dualproblemet omdannes til et ækvivalent minimumsproblem og dualproblemet af dette dannes, opnås det oprindelige problem.
\end{thm}
\begin{proof}
Betragt problemet:
\begin{align*}
\begin{array}{lrll}
\text{minimer}		&\textbf{p}^T\textbf{b}	&			&\\
\text{begrænset af}	&p_i					&\geq 0,	&i \in M_1,\\
					&p_i					&\leq 0,	&i \in M_2,\\
					&p_i					&\text{fri},	&i \in M_3,\\
					&\textbf{p}^T\textbf{A}_j	&\leq c_j,	&j \in N_1,\\
					&\textbf{p}^T\textbf{A}_j	&\geq c_j,	&j \in N_2,\\
					&\textbf{p}^T\textbf{A}_j	& = b_i,	&i \in N_3,
\end{array}
\end{align*}
og dets dualproblem:
\begin{align*}
\begin{array}{lrll}
\text{Maksimér}		&\textbf{c}^T\textbf{x}	&			&\\
\text{begrænset af}	&\textbf{a}_i^T\textbf{x}	&\geq b_i,	&i \in M_1,\\
					&\textbf{a}_i^T\textbf{x}	&\leq b_i,	&i \in M_2,\\
					&\textbf{a}_i^T\textbf{x}	& = b_i,	&i \in M_3,\\
					&x_j					&\geq 0,	&j \in N_1,\\
					&x_j					&\leq 0,	&j \in N_2,\\							&x_j					&\text{fri},	&j \in N_3,
\end{array}
\end{align*}
Dualproblemet omdannes til et ækvivalent minimumsproblem:
\begin{align*}
\begin{array}{lrll}
\text{minimer}		&-\textbf{c}^T\textbf{x}	&			&\\
\text{begrænset af}	&-\textbf{a}_i^T\textbf{x}	&\leq -b_i,	&i \in M_1,\\
					&-\textbf{a}_i^T\textbf{x}	&\geq -b_i,	&i \in M_2,\\
					&-\textbf{a}_i^T\textbf{x}	& = -b_i,	&i \in M_3,\\
					&x_j					&\geq 0,	&j \in N_1,\\
					&x_j					&\leq 0,	&j \in N_2,\\							&x_j					&\text{fri},	&j \in N_3,
\end{array}
\end{align*}
Dualen til dette problem findes nu:
\begin{align*}
\begin{array}{lrll}
\text{maksimer}		&-\textbf{p}^T\textbf{b}	&			&\\
\text{begrænset af}	&p_i					&\geq 0,	&i \in M_1,\\
					&p_i					&\leq 0,	&i \in M_2,\\
					&p_i					&\text{fri},	&i \in M_3,\\
					&-\textbf{p}^T\textbf{A}_j	&\geq -c_j,	&j \in N_1,\\
					&-\textbf{p}^T\textbf{A}_j	&\leq -c_j,	&j \in N_2,\\
					&-\textbf{p}^T\textbf{A}_j	& = -b_i,	&i \in N_3,
\end{array}
\end{align*}
Hvilket har et ækvivalent minimumsproblem:
\begin{align*}
\begin{array}{lrll}
\text{minimer}		&\textbf{p}^T\textbf{b}	&			&\\
\text{begrænset af}	&p_i					&\geq 0,	&i \in M_1,\\
					&p_i					&\leq 0,	&i \in M_2,\\
					&p_i					&\text{fri},	&i \in M_3,\\
					&\textbf{p}^T\textbf{A}_j	&\leq c_j,	&j \in N_1,\\
					&\textbf{p}^T\textbf{A}_j	&\geq c_j,	&j \in N_2,\\
					&\textbf{p}^T\textbf{A}_j	& = b_i,	&i \in N_3,
\end{array}
\end{align*}
\end{proof}
%
%42 min https://youtu.be/0TRxEvMRE7s definition